\documentclass[a4paper,12pt,oneside]{book}
\usepackage[T1]{fontenc}                                      
\usepackage[utf8]{inputenc}                               
\usepackage[italian]{babel}
\usepackage{amsfonts}
\usepackage{amsthm}
\usepackage{amsmath,amssymb}
\usepackage{array}
\usepackage{arydshln}
\usepackage{braket}
\usepackage{blindtext}
\usepackage{calc}
\usepackage{cancel}
\usepackage{caption}
\usepackage{epsfig}
\usepackage{eucal}
\usepackage{fancyhdr}
\usepackage{geometry}
\usepackage{graphicx}
\usepackage{indentfirst}
\usepackage{hhline}
\usepackage{hyperref}
\hypersetup{
			colorlinks=true,
			linkcolor=black,
			anchorcolor=black,
			citecolor=black,
			urlcolor=black,
			pdftitle={Appunti di Meccanica Quantistica},
			pdfauthor={Vittorio Lubicz}
}

\usepackage{latexsym}
\usepackage{listings} 
\usepackage{longtable}
\usepackage{makeidx}
\usepackage{mathrsfs}
\usepackage{mathdots}
\usepackage{multirow}
\usepackage{nicefrac}
\usepackage{pdfpages}
\usepackage{physics}
\usepackage{setspace}
\usepackage[many]{tcolorbox}
\usepackage{tikz}
\usepackage{tikz-3dplot}
\usepackage{textcomp}
\usepackage{titlesec,color}
\usepackage{vmargin}
\setpapersize{A4}
\setmarginsrb{35mm}{30mm}{35mm}{30mm}%
             {0mm}{10mm}{0mm}{10mm}



\definecolor{gray75}{gray}{0.75}
\newcommand{\hsp}{\hspace{20pt}}

\titleformat{\chapter}[hang]{\huge\bfseries}{\myfont{\textit{\large{\chaptername\hspace{1pt} \thechapter\hspace{3pt}}}}\textcolor{gray75}{$\mid$}\hspace{0.4cm}}{0pt}{\myfont{\huge\bfseries}}

\titleformat{\section}[hang]{\large\bfseries}{\myfont{\textit{\normalsize{\thesection\hspace{2pt}}}}\hspace{0.4cm}}{0pt}{\myfont{\Huge\bfseries}}

\titleformat{\subsection}[hang]{\large\bfseries}{\myfont{\textit{\small{\thesubsection\hspace{2pt}}}}\hspace{0.4cm}}{0pt}{\myfont{\huge\bfseries}}

\renewcommand{\chaptermark}[1]{\markboth{#1}{}}
\renewcommand{\sectionmark}[1]{\markright{#1}}
\newcommand*{\myfont}{\fontfamily{ppl}\selectfont}

\begin{document}

%*****************LAYOUT PAGINE**************************
\fancypagestyle{plain}{%
\fancyhf{} % cancella tutti i campi di  intestazione e pi\`e di pagina
\fancyfoot[C]{\bfseries \myfont{\thepage}} % tranne il centro
\renewcommand{\headrulewidth}{0pt}
\renewcommand{\footrulewidth}{0pt}}

\fancypagestyle{VS}{
\headheight = 15pt
\lhead[\myfont{\textit{\textbf{\thechapter\nouppercase{\leftmark}}}}]{\myfont{\textit{\textbf{\nouppercase{\leftmark}}}}}
\chead[]{}
\rhead[\myfont{\textbf{\thepage}}]{\myfont{\textbf{\thepage}}}

\lfoot[]{}
\cfoot[]{}
\rfoot[]{}
}
%*******************************************************



\pagestyle{VS}
\setcounter{chapter}{20}
\setcounter{page}{206}
\chapter{Atomo di idrogeno}
\section{Il problema dei due corpi e il moto in un campo centrale }
Il problema del moto di due particelle interagenti nella meccanica quantistica può essere ridotto al problema di una sola particella, in modo analogo a come può essere fatto in meccanica classica.\\

L'hamiltoniano di due particelle (con massa $m_1$ ed $m_2$) che interagiscono secondo la legge $V\left(r\right)$, dove $r$ è la distanza tra le particelle, ha la forma
	\begin{equation}
		\tcboxmath[sharp corners=downhill, colback=white, colframe=black]{
			H=\frac{\vec{p}_1^{\ 2}}{2m_1}+\frac{\vec{p}_2^{\ 2}}{2m_2}+V\left(|\vec{r}_1-\vec{r}_2|\right),
			}
	\end{equation} 
dove
	\begin{equation}
		\tcboxmath[enhanced, sharp corners=downhill, colframe=black, colback=white, borderline={2pt}{-3pt}{black}]{
			V(r) =-\frac{Ze^2}{r},
			}
	\end{equation}
per l'atomo idrogenoide.\\

Introduciamo in luogo dei raggi vettori delle particelle,$\vec{r_1}$ ed $\vec{r_2}$, le nuove variabili
	\begin{equation}
		\tcboxmath[sharp corners=downhill, colback=white, colframe=black]{
			\vec{R}=\frac{m_1\vec{r}_1+m_2\vec{r}_2}{m_1+m_2},
			} \qquad
		\tcboxmath[sharp corners=downhill, colback=white, colframe=black]{
			\vec{r}=\vec{r}_1-\vec{r}_2 ,
			}
	\end{equation}
dove \textbf{$\vec{R}$ è il raggio vettore del centro di massa della particella ed $\vec{r}$ è il vettore della distanza mutua}.\\

Con un semplice calcolo possiamo ottenere le espressioni dell'operatore di energia cinetica in termini degli impulsi coniugati alle variabili $\vec{R}$ ed $\vec{r}$. Si ha:
	\begin{align} 
		\frac{\partial}{\partial x_1} & = \frac{\partial R_x}{\partial x_1} \frac{\partial}{\partial R_x}+\frac{\partial R_y}{\partial x_1} \frac{\partial}{\partial R_y}+\frac{\partial R_z}{\partial x_1} \frac{\partial}{\partial R_z}+\frac{\partial r_x}{\partial x_1} \frac{\partial}{\partial r_x}+\frac{\partial r_y}{\partial x_1} \frac{\partial}{\partial r_y}+\frac{\partial r_z}{\partial x_1} \frac{\partial}{\partial r_z}= \nonumber\\
 		& = \frac{m_1}{m_1+m_2}\frac{\partial}{\partial R_1}+\frac{\partial}{\partial r_1} ,
	\end{align}
e dunque
	\begin{equation}
		\vec{\nabla}_1=\frac{m_1}{m_1+m_2}\vec{\nabla}_R+\vec{\nabla}_r;
	\end{equation}
analogamente si trova
	\begin{equation}
		\vec{\nabla}_2=\frac{m_2}{m_1+m_2}\vec{\nabla}_R-\vec{\nabla}_r.
	\end{equation}
Prendendo il quadrato di queste espressioni otteniamo per i laplaciani:
	\begin{equation}
		\vec{\nabla}_1^2=\frac{m_1^2}{\left(m_1+m_2\right)^2}\vec{\nabla}_R^2+\vec{\nabla}_r^2+\frac{2m_1}{m_1+m_2}\vec{\nabla}_R\cdot\vec{\nabla}_r
	\end{equation}
\begin{equation}
\vec{\nabla}_2^2=\frac{m_2^2}{\left(m_1+m_2\right)^2}\vec{\nabla}_R^2+\vec{\nabla}_r^2-\frac{2m_2}{m_1+m_2}\vec{\nabla}_R\cdot\vec{\nabla}_r.
\end{equation}
Allora
\begin{equation}
\frac{1}{m_1}\nabla_1^2+\frac{1}{m_2}\nabla_2^2=\frac{1}{m_1+m_2}\nabla_R^2+\left(\frac{1}{m_1}+\frac{1}{m_2}\right)\nabla_r^2 .
\end{equation}\\

\textbf{L'hamiltoniana delle due particelle prende} allora, in termini delle variabili del centro di massa e del moto relativo, la forma:
	\begin{equation}
		\tcboxmath[enhanced, sharp corners=downhill, colframe=black, colback=white, borderline={2pt}{-3pt}{black}]{
			H=\frac{\vec{P}^{2}}{2M}+\frac{\vec{p}^{\ 2}}{2m}+V\left(r\right) , 
			}
	\end{equation}
dove
	\begin{equation}
		\tcboxmath[sharp corners=downhill, colback=white, colframe=black]{
			\vec{P}=-i\hbar\vec{\nabla}_R,
			} \qquad
		\tcboxmath[sharp corners=downhill, colback=white, colframe=black]{
			\vec{p}=-i\hbar\vec{\nabla}_r ,
			}
\end{equation}
ed abbiamo introdotto la \textbf{massa totale} del sistema
	\begin{equation}
		\tcboxmath[sharp corners=downhill, colback=white, colframe=black]{
			M=m_1+m_2
			}
	\end{equation}
e la \textbf{massa ridotta}\footnote{Nel caso di protone ed elettrone, per i quali $m_p \gg m_e$ ($m_p \simeq$ 938MeV, $m_e\simeq$ 0.51MeV) si ha $M\simeq m_p$ ed $m\simeq m_e$.}
	\begin{equation}
		\tcboxmath[sharp corners=downhill, colback=white, colframe=black]{
			m=\left(\frac{1}{m_1}+\frac{1}{m_2}\right)^{-1}=\frac{m_1m_2}{m_1+m_2}.
			}
	\end{equation}\\
	
L'hamiltoniano si scompone quindi nella somma di due parti indipendenti. Partendo da questo fatto \textbf{si può cercare la soluzione dell'equazione di Schr\"{o}dinger del sistema nella forma}:
	\begin{equation}
		\tcboxmath[enhanced, sharp corners=downhill, colframe=black, colback=white, borderline={2pt}{-3pt}{black}]{
			\psi\left(\vec{r}_1,\vec{r}_2\right)=\phi(\vec{R})\psi(\vec{r}) ,
			}
	\end{equation}
\textbf{dove la funzione $\psi (\vec{R} )$ descrive il moto del centro di massa, come moto libero di una particella di massa $M=m_1+m_2$, e $\psi\left(\vec{r}\right)$ descrive il moto relativo delle particelle come moto di una particella di massa $m$ in un campo a simmetria centrale $V=V\left(r\right)$}.\\

L'equazione di Schr\"{o}dinger del moto di una particella nel campo a simmetria centrale ha la forma
	\begin{equation}
		\tcboxmath[enhanced, sharp corners=downhill, colframe=black, colback=white, borderline={2pt}{-3pt}{black}]{
			\left[-\frac{\hbar^2}{2m}\nabla^2+V\left( r \right) \right]\psi\left(r\right)=E\psi\left(r\right).
			}
	\label{21.1}
	\end{equation} 
In questa equazione \textbf{$E=E_{tot}-\frac{\vec{P}^2}{2M}$} è l'energia interna del sistema delle due particelle, ossia l'energia restante a seguito della sottrazione dell'energia cinetica associata al moto traslatorio del sistema nel suo insieme.\\ 

Risulta conveniente studiare l'equazione (\ref{21.1}) in coordinate polari. A tale scopo deriviamo in primo luogo una \textbf{relazione tra l'operatore laplaciano ed il quadrato $L^2$ del momento angolare orbitale}. Si ha:
	\begin{align} 
	\label{21.2}
		L^2 & = \left(\vec{r}\wedge\vec{p}\right)^2=-\hbar^2\left(\vec{r}\wedge\vec{\nabla}\right)^2=-\hbar^2\left(\vec{r}\wedge\vec{\nabla}\right)_i\left(\vec{r}\wedge\vec{\nabla}\right)_i= \nonumber \\
		 & =  -\hbar^2\varepsilon_{ijk}\,\varepsilon_{ilm}\, r_j\partial_k r_l\partial_m= -\hbar^2\left(\delta_{jl}\delta_{km}-\delta_{jm}\delta_{kl}\right)r_j\partial_k r_l\partial_m=\nonumber \\
		 & =-\hbar^2\left(r_j\partial_kr_j\partial_k-r_j\partial_kr_k\partial_j\right) ,
\end{align}
facendo uso della seguente relazione
	\begin{equation}
		\partial_kr_j=r_j\partial_k+\delta_{kj}
	\end{equation}
è possibile scrivere la (\ref{21.2}) come:
	\begin{align}
	\label{21.3}
		L^2&=-\hbar^2\left[r_j\left(r_j\partial_k+\delta_{kj}\right)\partial_k-\left(\partial_kr_j-\delta_{kj}\right)r_k\partial_j\right]= \nonumber\\
		&=-\hbar^2\left[r_jr_j\partial_k\partial_k+r_k\partial_j-\partial_kr_kr_j\partial_j+r_j\partial_j\right]= \nonumber \\
		&=-\hbar^2\left[r^2\nabla^2+2\vec{r}\cdot\vec{\nabla}-\left(r_k\partial_k+\delta_{kk}\right)r_j\partial_j\right]= \nonumber\\
		&=-\hbar^2\left[r^2\nabla^2-\left(\vec{r}\cdot\vec{\nabla}\right)^2-\vec{r}\cdot\vec{\nabla}\right].
	\end{align}\\
%Ossia
%\begin{equation}
%L^2=r^2p^2-\left(\vec{r}\cdot\vec{p}\right)^2+i\hbar\vec{r}\cdot\vec{p} ,
%\end{equation}
%o, equivalentemente
%\begin{equation}
%\label{21.3}
%p^2=\frac{\left(\vec{r}\cdot\vec{p}\right)^2}{r^2}-i\hbar\frac{\vec{r}\cdot\vec{p}}{r^2}+\frac{L^2}{r^2}.
%\end{equation}
%Si noti che in meccanica classica vale la stessa relazione con $\hbar=0$ ($\left[p_i,r_j\right]\rightarrow0$)\\
D'altra parte il prodotto scalare $\vec{r}\cdot\vec{p}$ coincide, a meno di un fattore moltiplicativo, con la proiezione dell'operatore gradiente nella direzione del raggio vettore $\vec{r}$, ossia con la derivata rispetto ad $r$. Si ha infatti esplicitamente:
	\begin{align} 
		\frac{\partial}{\partial r}&= \frac{\partial x}{\partial r} \frac{\partial}{\partial x}+\frac{\partial y}{\partial r} \frac{\partial}{\partial y}+\frac{\partial z}{\partial r} \frac{\partial}{\partial z}= \nonumber \\
		&=  \frac{1}{r}\left(x\frac{\partial}{\partial x}+y\frac{\partial}{\partial y}+z\frac{\partial}{\partial z}\right)=\frac{1}{r}\vec{r}\cdot\vec{\nabla} ,
	\end{align}
ossia
	\begin{equation}
		\tcboxmath[sharp corners=downhill, colback=white, colframe=black]{
			\left(\vec{r}\cdot\vec{p}\right)=-i\hbar\left(\vec{r}\cdot\vec{\nabla}\right)=-i\hbar r \frac{\partial}{\partial r} .
			}
	\end{equation}\\
	
Sostituendo questa equazione nella relazione (\ref{21.3}) giungiamo infine all'espressione del laplaciano, in coordinate polari, in termini dell'operatore $L^2$
	\begin{align}
		\vec{p}^{\ 2}&= -\hbar^2\nabla^2= -\frac{\hbar ^2}{r^2}\left[ \left( \vec{r}\cdot \vec{\nabla} \right) ^2 - \left( \vec{r}\cdot \vec{\nabla} \right)\right] +\frac{L^2}{r^2} = \nonumber \\
		&= -\frac{\hbar^2}{r^2}\left[\left(r\frac{\partial}{\partial r}\right)\left(r\frac{\partial}{\partial r}\right)+\left(r\frac{\partial}{\partial r}\right)\right]+\frac{L^2}{r^2}= -\frac{\hbar}{r^2}\left[r^2\frac{\partial^2}{\partial r^2}+2r\frac{\partial}{\partial r}\right]+\frac{L^2}{r^2} ,
	\end{align}
o, equivalentemente,
	\begin{equation}
		\tcboxmath[sharp corners=downhill, colback=white, colframe=black]{
			\vec{p}^{\ 2}=-\hbar^2\nabla^2=\frac{\hbar^2}{r^2}\frac{\partial}{\partial r}\left(r^2 \frac{\partial}{\partial r}\right)+\frac{L^2}{r^2} .
			}
	\end{equation}\\
	
\textbf{L'equazione di Schr\"{o}dinger del moto di una particella nel campo a simmetria centrale si scrive allora nella forma:}
	\begin{equation}
		\tcboxmath[enhanced, sharp corners=downhill, colframe=black, colback=white, borderline={2pt}{-3pt}{black}]{
			-\frac{\hbar^2}{2mr^2}\frac{\partial}{\partial r}\left(r^2\frac{\partial}{\partial r}\right)\psi+\left[\frac{L^2}{2mr^2}+V\left(r\right)\right]\psi=E\psi .
			}
	\end{equation}
Tutta la dipendenza dagli angoli delle coordinate polari, in questa equazione, è \textbf{contenuta nell'operatore $L^2$}, che dunque \textbf{commuta con l'hamiltoniano}. Ne segue pertanto che\textbf{ nel moto in un campo a simmetria centrale il momento angolare orbitale si conserva}. \textbf{L'hamiltoniano commuta anche con le componenti $L_x$, $L_y$ , $L_z$ che dunque sono separatamente conservate.}\\

Consideriamo ora le autofunzioni simultanee dell'hamiltoniano e dell'operatore $L^2$, ossia le f.d.o.~degli stati stazionari del sistema con valori determinati del momento angolare $l$ e della sua proiezione lungo l'asse $z$. Queste funzioni sono della forma
	\begin{equation}
		\tcboxmath[enhanced, sharp corners=downhill, colframe=black, colback=white, borderline={2pt}{-3pt}{black}]{
			\psi\left(\vec{r}\right)=R\left(r\right)Y_{lm}\left(\theta,\phi\right) ,
			}
	\end{equation}
dove $Y_{lm}$ sono le funzioni armoniche sferiche.\\

Poiché $L^2Y_{lm}=\hbar^2l\left(l+1\right)Y_{lm}$, per la funzione d'onda radiale $R\left(r\right)$ si ottiene l'equazione:
	\begin{equation}
		\tcboxmath[enhanced, sharp corners=downhill, colframe=black, colback=white, borderline={2pt}{-3pt}{black}]{
			\left[-\frac{\hbar^2}{2mr^2}\frac{\partial}{\partial r}\left(r^2\frac{\partial}{\partial r}\right)+\frac{\hbar^2l\left(l+1\right)}{2mr^2}+V\left(r\right)-E\right]R\left(r\right)=0 .
			}
	\end{equation}\\
	
Questa equazione non contiene affatto il valore di $L_z=m$, da cui segue che \textbf{i livelli di energia sono $\left(2l+1\right)$ volte degeneri rispetto alle direzioni del momento angolare}.\\

Effettuiamo nella equazione d'onda per il moto radiale la sostituzione:
	\begin{equation}
		\tcboxmath[sharp corners=downhill, colback=white, colframe=black]{
			R\left(r\right)=\frac{1}{r}\chi\left(r\right) .
			}
	\end{equation}
Poiché
	\begin{align}
		\frac{1}{r^2}\frac{\partial}{\partial r}\left(r^2\frac{\partial R}{\partial r}\right)&=\frac{1}{r^2}\frac{\partial}{\partial r}\left(r^2\frac{\partial}{\partial r}\frac{\chi}{r}\right)=\frac{1}{r^2}\frac{\partial}{\partial r}\left(r \frac{\partial\chi}{\partial r}-\chi\right)= \nonumber \\
		&= \frac{1}{r^2}\left(r\frac{\partial^2\chi}{\partial r^2}+\frac{\partial\chi}{\partial r}-\frac{\partial\chi}{\partial r}\right) = \frac{1}{r}\frac{\partial^2 \chi}{\partial r^2}
\end{align}
l'equazione radiale si riduce a:
	\begin{equation}
		\tcboxmath[sharp corners=downhill, colback=white, colframe=black]{
			\left[-\frac{\hbar^2}{2m}\frac{\partial^2}{\partial r^2}+\frac{\hbar^2l\left(l+1\right)}{2mr^2}+V\left(r\right)-E\right]\chi\left(r\right)=0 .
			}
	\end{equation}\\

Questa equazione coincide formalmente con l'\textbf{equazione di Schr\"{o}dinger unidimensionale} per il moto in un campo con energia potenziale
	\begin{equation}
		\tcboxmath[sharp corners=downhill, colback=white, colframe=black]{
			V_l\left(r\right)=V\left(r\right)+\frac{\hbar l\left(l+1\right)}{2mr^2} ,
			}
	\end{equation}
uguale alla somma dell'energia $V\left(r\right)$ e del termine
\begin{equation}
\frac{\hbar l\left(l+1\right)}{2mr^2}=\frac{L^2}{2mr^2} ,
\end{equation}
che si chiama energia centrifuga. IL problema del moto in un campo a simmetria centrale si riduce, quindi, al problema unidimensionale in una regione semilimitata, $r>0$.\\

Carattere unidimensionale ha anche la condizione di normalizzazione della funzione $\chi\left(r\right)$, che è definita dall'integrale:
	\begin{equation}
		\tcboxmath[sharp corners=downhill, colback=white, colframe=black]{
			\int_0^\infty dr\ r^2|R\left(r\right)|^2=\int_0^\infty dr\ |\chi\left(r\right)|^2 .
			}
	\end{equation}\\

É possibile dimostrare che nel moto unidimensionale in una regione semilimitata \textbf{i livelli energetici non sono degeneri}. Si può allora affermare che l'assegnazione del valore dell'energia determina completamente la parte radiale della funzione d'onda. Tenendo anche conto che la parte angolare della funzione d'onda è data completamente dai valori di l ed m, concludiamo che \textbf{nel moto in un campo a simmetrica centrale la f.d.o.~è definita completamente da valori i $E$, $l$, $m$}. In altri termini l'energia, il quadrato del momento angolare e la sua proiezione costituiscono un sistema completo di grandezze fisiche per tale moto.

\section{Campo Coulombiano}
Un caso molto importante di  moto in un campo a simmetria centrale è quello del moto in un campo coulombiano
	\begin{equation}
		\tcboxmath[enhanced, sharp corners=downhill, colframe=black, colback=white, borderline={2pt}{-3pt}{black}]{
			V\left(r\right)=-\frac{Ze^2}{r} 
			}
	\end{equation}
($Z=1$ per l'atomo di idrogeno).\\

Dalle considerazioni fatte sappiamo che il moto si riduce formalmente ad un moto unidimensionale con energia potenziale efficace
	\begin{equation}
		\tcboxmath[sharp corners=downhill, colback=white, colframe=black]{
V_l\left(R\right)=-\frac{Xe^2}{r}+\frac{\hbar^2l\left(l+1\right)}{2mr^2} .
}
	\end{equation}
Riportiamo qui di seguito un grafico di tale potenziale\\
\begin{figure}[!htbp]
\begin{center}
\includegraphics[width=8.5cm]{immagini/cap_21/fig_21_1.png}
\end{center}
\end{figure}\\
Si vede allora che lo \textbf{spettro degli autovalori negativi dell'energia è discreto e corrisponde agli stati legati del sistema, mentre quello delle energie positive è continuo ed il moto corrispondente si estende da zero all'infinito}.\\

Consideriamo qui in particolare il caso dello \textbf{spettro discreto} ossia degli stati legati degli atomi idrogenoidi.\\

L'equazione di Schr\"{o}dinger per le funzioni radiali si scrive:
	\begin{equation}
		\begin{aligned}
		\tcboxmath[enhanced, sharp corners=downhill, colframe=black, colback=white, borderline={2pt}{-3pt}{black}]{
			\left[-\frac{\hbar^2}{2m}\frac{1}{r^2}\frac{d}{d r}\left(r^2\frac{d}{d r}\right)+\frac{\hbar^2l\left(l+1\right)}{2mr^2}\right]R\left(r\right)-\left(\frac{Ze^2}{r}+E\right)R\left(r\right)=0 .
		}\\[-0.3cm]			
			\quad
		\end{aligned}
	\label{21.4}
	\end{equation}\\

Per risolvere questa equazione risulta conveniente introdurre in primo luogo delle \textbf{variabili adimensionali}. Dalla precedente equazione risultano evidenti le seguenti uguaglianze "dimensionali":
	\begin{equation}
		\left[E\right]=\left[\frac{Ze^2}{r}\right]=\left[\frac{\hbar^2}{mr^2}\right] ,
	\end{equation}
ossia
	\begin{equation}
		\tcboxmath[sharp corners=downhill, colback=white, colframe=black]{
			\left[r\right]=\left[\frac{\hbar^2}{mZe^2}\right] ,
			} \qquad
		\tcboxmath[sharp corners=downhill, colback=white, colframe=black]{
			\left[E\right]=\left[\frac{m\left(Ze^2\right)^2}{\hbar^2}\right] .
			}
	\end{equation}

Introduciamo allora, in luogo dell'energia, una nuova variabile definita da
	\begin{equation}
		\tcboxmath[enhanced, sharp corners=downhill, colframe=black, colback=white, borderline={2pt}{-3pt}{black}]{
			E=-\frac{1}{2n^2}\frac{m\left(Ze^2\right)^2}{\hbar^2} ;
			}
	\label{21.5}
	\end{equation}
per energie negative $n$ è un numero reale positivo.\\

In luogo del raggio $r$ introduciamo poi la variabile adimensionale $\rho$ definita da 
	\begin{equation}
		\tcboxmath[enhanced, sharp corners=downhill, colframe=black, colback=white, borderline={2pt}{-3pt}{black}]{
			r=\frac{n}{2}\frac{\hbar^2}{mze^2}\, \rho .
			}
	\label{21.6}
	\end{equation}
Sostituendovi le (\ref{21.5}) e (\ref{21.6}), l'equazione (\ref{21.4}) diventa:
	\begin{align}
		\frac{\hbar^2}{2m}\left(\frac{2mZe^2}{n\hbar^2}\right) &\left[\frac{1}{\rho^2}\frac{d}{d\rho}\left(\rho^2\frac{d}{d\rho}\right)   -\frac{l\left(l+1\right)}{\rho^2}\right]R +\nonumber \\
		& \qquad + \left[\frac{2m\left(Ze^2\right)^2}{n\hbar^2}\frac{1}{\rho}-\frac{m\left(Ze^2\right)^2}{2n^2\hbar^2}\right]R=0 ,
	\end{align}
ossia, dividendo per $2m\left(Ze^2\right)^2/\left(n^2\hbar^2\right)$ ed esplicitando le derivate:
	\begin{equation}
		\tcboxmath[enhanced, sharp corners=downhill, colframe=black, colback=white, borderline={2pt}{-3pt}{black}]{
			\frac{d^2R}{d\rho^2}+\frac{2}{\rho}\frac{dR}{d\rho}+\left[-\frac{l\left(l+1\right)}{\rho^2}+\frac{n}{\rho}-\frac{1}{4}\right]R=0 .
			}
	\label{21.7}
	\end{equation}\\

Consideriamo dapprima le \textbf{soluzioni asintotiche} dell'equazione (\ref{21.7}) valide per $\rho\rightarrow0$ (che è un punto singolare) e $\rho\rightarrow\infty$.\\

Nel limite \textbf{$\rho\rightarrow0$} l'equazione (\ref{21.7}) si riduce a:
	\begin{equation}
		\tcboxmath[sharp corners=downhill, colback=white, colframe=black]{
			\frac{d^2R}{d\rho^2}+\frac{2}{\rho}\frac{dR}{d\rho}-\frac{l\left(l+1\right)}{\rho^2}R=0 .
			}
	\label{21.8}
	\end{equation}\\
	
Cerchiamo una soluzione di questa equazione della forma
	\begin{equation}
		\tcboxmath[sharp corners=downhill, colback=white, colframe=black]{
			R\left(\rho\right)=\mbox{costante}\cdot \rho^s .
			}
	\end{equation}\\

Sostituendo questa espressione nell'equazione (\ref{21.8}) si ottiene:
	\begin{equation}
		s\left(s-1\right)+2s-l\left(l+1\right)=0 ,
	\end{equation}
ossia
	\begin{equation}
		\tcboxmath[sharp corners=downhill, colback=white, colframe=black]{
		s\left(s+1\right)=l\left(l+1\right) ,
		}
	\end{equation}
che ha come soluzione
\begin{equation}
s=\frac{-1\pm\sqrt{1+4l\left(l+1\right)}}{2}=\frac{-1\pm\left(2l+1\right)}{2}=\begin{cases}l, \\-\left(l+1\right) . \end{cases}
\end{equation}\\

La soluzione $s=-\left(l+1\right)$ deve essere scartata perché  conduce ad una f.d.o.~divergente nell'origine. \textbf{Nell'intorno dell'origine} si ha quindi
	\begin{equation}
		\tcboxmath[enhanced, sharp corners=downhill, colframe=black, colback=white, borderline={2pt}{-3pt}{black}]{
			R\left(\rho\right)\simeq\rho^l ,
			} \qquad \left(\rho\rightarrow0\right) .
	\end{equation}\\

Osserviamo che \textbf{questo risultato rimane valido per ogni potenziale che nell'origine diverge più lentamente del potenziale centrifugo}, ossia più lentamente di $1/r^2$. Il suo significato è che quanto più grande è il valore del momento angolare, tanto più piccolo è la probabilità di trovare la particella nell'origine.
Questo risultato è in accordo anche con le previsioni classiche.\\

Studiando ora l'equazione per grandi $\rho$, ossia nel limite \textbf{$\rho\rightarrow\infty$}. In tale approssimazione l'equazione (\ref{21.7}) si riduce a:
	\begin{equation}
		\tcboxmath[sharp corners=downhill, colback=white, colframe=black]{
			\frac{d^2R}{d\rho^2}-\frac{1}{4}R=0 .
			}
	\end{equation}
La cui soluzione è
	\begin{equation}
		R\left(\rho\right)=e^{\pm\rho/2} . 
	\end{equation}
La soluzione che si annulla all'infinito, la sola fisicamente accettabile, è:
	\begin{equation}
		\tcboxmath[enhanced, sharp corners=downhill, colframe=black, colback=white, borderline={2pt}{-3pt}{black}]{
			R\left(\rho\right)=e^{-\rho/2} .
			}
	\end{equation}\\
	
In definitiva concludiamo che la soluzione cercata è della forma
	\begin{equation}
		\tcboxmath[enhanced, sharp corners=downhill, colframe=black, colback=white, borderline={2pt}{-3pt}{black}]{
			R\left(\rho\right)=\rho^le^{-\rho/2}w\left(\rho\right) ,
			}
	\label{21.9}
	\end{equation}
dove $w$ è una funzione da determinare che deve divergere all'infinito non più rapidamente di una potenza finita di $\rho$ e deve essere finita per $\rho=0$.\\

Sostituendo la f.d.o.~(\ref{21.9}) nell'equazione radiale (\ref{21.7}), e considerando che 
	\begin{equation}
		\frac{dR}{d\rho}=\rho^{l-1}e^{-\rho/2}[lw-\frac{1}{2}\rho w+\rho w^\prime] ,
	\end{equation}
e
	\begin{equation}
		\tcboxmath[sharp corners=downhill, colback=white, colframe=black]{
			\frac{d^2R}{d\rho^2}=\rho^{l-2}e^{-\rho/2}\left[\rho^2w^{\prime\prime}+\left(2l-\rho\right)\rho w^\prime+\left(l\left(l-1\right)-l\rho+\frac{1}{4}\rho^2\right)w\right] ,
			}
	\end{equation}
otteniamo per $w$ l'equazione
	\begin{equation}
		\tcboxmath[enhanced, sharp corners=downhill, colframe=black, colback=white, borderline={2pt}{-3pt}{black}]{
			\rho w^{\prime\prime}+\left(2l+2-\rho\right)w^\prime+\left(n-l-1\right)w=0 .
			}
	\end{equation}\\
	
Cerchiamo per la soluzione $w\left(\rho\right)$ un'espressione per serie, poniamo cioè 
	\begin{equation}
		\tcboxmath[enhanced, sharp corners=downhill, colframe=black, colback=white, borderline={2pt}{-3pt}{black}]{
			w\left(\rho\right)=\sum_{k=0}^\infty a_k \rho^k .
			}
	\end{equation}
Sostituendo nell'equazione di $w$ otteniamo:
	\begin{equation}
		\tcboxmath[sharp corners=downhill, colback=white, colframe=black]{
			\sum_{k=0}^\infty\left[a_{k+1}k\left(k+1\right)+\left(2l+2\right)\left(k+1\right)a_{k+1}-ka_k+\left(n-l-1\right)a_k\right]\rho^k=0 .
			}
	\end{equation}
Poiché la serie sia nulla per ogni valore di $\rho$ devono essere separatamente nulli i coefficienti di ogni potenza di $\rho$, si deve cioè avere
	\begin{equation}
		\tcboxmath[enhanced, sharp corners=downhill, colframe=black, colback=white, borderline={2pt}{-3pt}{black}]{
			a_{k+1}=\frac{k-n+l+1}{\left(k+1\right)\left(k+2l+2\right)}a_k .
			}
	\end{equation}
L'andamento asintotico dei coefficienti della serie, per grandi valori di $k$, risulta tale che
	\begin{equation}
		\frac{a_{k+1}}{a_k}\simeq \frac{1}{k},
	\end{equation}
pertanto
	\begin{equation}
		a_{k+1}=\frac{a_{k+1}}{a_{k}}\frac{a_{k}}{a_{k-1}}\frac{a_{k-1}}{a_{k-2}}\dots \frac{a_{1}}{a_{0}}a_0  \simeq \frac{1}{k} \frac{1}{k-1}\frac{1}{k-2}\dots a_0\simeq \frac{a_0}{k!}  .
	\end{equation}
Ma in questo caso
	\begin{equation}
		w(\rho) =\sum _{k=0} ^{\infty} a_n\rho ^k \approx a_0 \sum _k \frac{\rho ^k}{k!} \approx a_0\ e^{\rho}.
	\end{equation}\\

La funzione $w (\rho )$ così trovata non soddisfa la condizione al contorno all'infinito. Perché $w(\rho )$ diverga all'infinito come una potenza finita di $\rho$ deve essere $(n-l-1)$ un numero intero positivo o nullo. In tal caso al serie viene interrotta e $w (\rho )$ si riduce ad un polinomio di grado $(n-l-1)$. Siamo così giunti alla conclusione che \textbf{il numero $n$ deve essere un intero positivo e, per $l$ dato, si deve avere:}
	\begin{equation}
		\tcboxmath[enhanced, sharp corners=downhill, colframe=black, colback=white, borderline={2pt}{-3pt}{black}]{
			n \geq l+1.
			}
	\end{equation}\\
	
Ricordando la definizione del parametro $n$ troviamo
	\begin{equation}
		\tcboxmath[enhanced, sharp corners=downhill, colframe=black, colback=white, borderline={2pt}{-3pt}{black}]{
			E_n= -\frac{m\left(Ze^2\right) ^2}{2\hbar ^2 n^2}
			}\qquad n=1,2,\dots
	\end{equation}
Lo spettro discreto in un campo coulombiano è costituito dunque da un'infinità di livelli compresi tra il livello fondamentale
	\begin{equation}
		\tcboxmath[enhanced, sharp corners=downhill, colframe=black, colback=white, borderline={2pt}{-3pt}{black}]{
			E_1-\frac{m\left(Ze^2\right) ^2}{2\hbar ^2 }\simeq Z^2\cdot \left(13.6 \textrm{eV}\right)
			}
	\end{equation}
e zero. Gli intervalli tra due livelli consecutivi diminuiscono al crescere di $n$. I livelli si infittiscono man mano che ci si avvicina al valore $E=0$, dove lo spettro discreto si connette con quello continuo.\\

Il numero intero $n$ è detto \textbf{numero quantico principale}. Per un numero quantico principale dato, il numero $l$ può assumere i valori
	\begin{equation}
		\tcboxmath[enhanced, sharp corners=downhill, colframe=black, colback=white, borderline={2pt}{-3pt}{black}]{
			l=0,1,\dots,n-1,
			}
	\end{equation}
in totale $n$ valori diversi.\\

Nell'espressione dell'energia entra solo il numero $n$. pertanto tutti gli stati con $l$ diversi, ma con uguali $n$, hanno la stessa energia. Ogni autovalore è quindi degenere non soltanto rispetto al numero quantico $m$ (come per qualsiasi moto in un campo a simmetria centrale) ma anche rispetto al numero $l$. Quest'ultima \textbf{degenerazione}, detta \textbf{accidentale}, è specifica del campo coulombiano. Ad ogni valore di $l$ dato corrispondono $2l+1$ valori differenti di $m$. pertanto \textbf{l'ordine di degenerazione dell' n-esimo livello energetico è}, considerando anche la degenerazione di spin:
	\begin{equation}
		2 \left[ \sum _{l=0} ^{n-1} \left(2l+1 \right)\right]=2 \left[ 2\frac{\left( n-1 \right) n}{2}+n\right],
	\end{equation}
ossia
	\begin{equation}
		\tcboxmath[enhanced, sharp corners=downhill, colframe=black, colback=white, borderline={2pt}{-3pt}{black}]{
			2 \left[ \sum _{l=0} ^{n-1} \left(2l+1 \right)\right]= 2n^2.
			}
	\end{equation}\\
	
I polinomi che si ottengono interrompendo la serie che esprime $w(\rho)$ sono i cosiddetti \textbf{polinomi generalizzati di Laguerre.} Le f.d.o.~radiali complete normalizzate con la condizione
	\begin{equation}
		\tcboxmath[enhanced, sharp corners=downhill, colframe=black, colback=white]{
			\int _0 ^{\infty} dr\, r^2 {R_{nl} (r)}^2=1,
			}
	\end{equation}
hanno la forma
	\begin{equation}
		\tcboxmath[enhanced, sharp corners=downhill, colframe=black, colback=white, borderline={2pt}{-3pt}{black}]{
			R_{nl}(r) =\frac{2}{n^2}\sqrt{\frac{\left(n-l-1\right) !}{\left(n+l\right) !}} e^{-\frac{\bar{r}}{n}}\left(\frac{2\bar{r}}{n}\right) ^l L_{n-l-1} ^{2l+1} \left(\frac{2\bar{r}}{n}\right) ,
			}
	\end{equation}
dove $L_k ^a (x)$ sono i polinomi generalizzati di Laguerre e
	\begin{align}
		&\tcboxmath[enhanced, sharp corners=downhill, colframe=black, colback=white, borderline={2pt}{-3pt}{black}]{
			\bar{r}= Z\frac{r}{a_0},
			} \\[0.3cm]
		&\tcboxmath[enhanced, sharp corners=downhill, colframe=black, colback=white, borderline={2pt}{-3pt}{black}]{
			a_0= \frac{\hbar ^2}{me^2}.
			}
	\end{align}\\
	
La grandezza $a_0$ è detta \textbf{raggio di Bohr} e vale
	\begin{equation}
		\tcboxmath[sharp corners=downhill, colback=white, colframe=black]{
			a_0 \simeq 0.529 \times 10^{-8} \textrm{cm}.
			}
	\end{equation}
La decrescita esponenziale delle f.d.o. radiali indica che $n a_0/Z$ rappresenta la tipica dimensione radiale delle orbite stazionarie per un dato valore del numero quantico principale $n$. Inoltre le orbite sono tanto più vicine al nucleo quanto più è alta la carica elettrica $Z$ di quest'ultimo.\\

A partire dalla dimensione tipica del raggio delle orbite, $r \backsim a_0$, deriviamo la tipica \textbf{velocità del moto dell'elettrone}, utilizzando il principio di indeterminazione. Si ha:
	\begin{equation}
		r   \backsim   a_0 =\frac{\hbar ^2}{me^2}  \Rightarrow   v=\frac{p}{m} \backsim \frac{1}{m} \frac{\hbar}{r}\backsim \frac{\hbar}{m}\frac{me^2}{\hbar ^2}=\frac{e^2}{\hbar}, 
	\end{equation}
da cui
	\begin{equation}
		\tcboxmath[sharp corners=downhill, colback=white, colframe=black]{
			\frac{v}{c}\backsim \frac{e^2}{\hbar c}=\alpha \simeq \frac{1}{137}.
			}
	\end{equation}
%\section{Modello di Bohr dell'atomo di idrogeno (1913)}
%Tre ipotesi:
%\begin{enumerate}
%\item L'elettrone ruota attorno al nucleo su orbita stabile (senza emettere radiazione);
%\item Le sole orbite consentite sono quelle per le quali il momento angolare risulta essere un multiplo intero di $\hbar$:
%\begin{equation}
%L=mvr=h\hbar;
%\label{eq:21.10}
%\end{equation}
%\item L'elettrone può effettuare transizioni discontinue tra due orbite consentite. Quando ciò accade viene emessa o assorbita radiazione di frequenza
%\begin{equation}
%\hbar \omega = E-E'
%\end{equation}
%dove $\Delta E = E-E'$ è la variazione di energia dell'elettrone tra le due orbite.
%\end{enumerate}
%\textbf{Conseguenze:} la stabilità di un'orbita è determinata dall'equilibrio tra la forza coulombiana e la forza centrifuga:
%\begin{eqnarray}
%&\displaystyle{\frac{e^2}{r^2}=\frac{mv^2}{r}.}&\\
%&\textrm{N.B. Forza centrifuga: } F=m\omega ^2 r = \frac{mv^2}{r}. \qquad \qquad (v=\omega r)& \nonumber
%\end{eqnarray}
%Questa condizione, unita alla (\ref{eq:21.10}), fornisce
%\begin{equation}
%\begin{cases}
%mv^2r=e^2\\
%mvr=n\hbar
%\end{cases}
%\Rightarrow \ v=\frac{e^2}{n\hbar}, \quad r=\frac{n\hbar}{mv}=\frac{n^2\hbar ^2}{me^2},
%\end{equation}
%dunque
%\begin{eqnarray}
%&r=n^2\frac{\hbar^2}{me^2}=n^2a_0, \qquad \left[\textrm{In meccanica quantistica } \langle \frac{1}{r}\rangle =\frac{1}{n^2 a_0}\right]& \\
%&\frac{v}{c}=\frac{1}{n}\frac{e^2}{\hbar c}=\frac{\alpha}{n}. \qquad \qquad \left[\textrm{In meccanica quantistica } \langle \frac{v^2}{c^2}\rangle =\frac{\alpha ^2}{n^2}\right]&
%\end{eqnarray}
%Calcoliamo l'energia delle orbite:
%\begin{eqnarray}
%&\displaystyle{\frac{p^2}{2m}=\frac{mv^2}{2}=\frac{me^4}{2n^2\hbar ^2}=\frac{mc^2 \alpha ^2}{2n^2}} ,& \\
%\nonumber \\
%&\displaystyle{\frac{e^2}{r}=\frac{me^4}{n^2\hbar ^2}=\frac{mc^2 \alpha ^2}{n^2}} ,&\\
%& \left[\textrm{N.B. In M.Q. valgono questi stessi risultati per i valori medi di T e V.}\right]& \nonumber
%\end{eqnarray}
%da cui
%\begin{equation}
%E=\frac{p^2}{2m}-\frac{e^2}{r}=-\frac{mc^2\alpha ^2}{2n^2}.
%\end{equation}
%Le righe di emissione e assorbimento dell'atomo hanno pertanto frequenze della forma
%\begin{equation}
%\hbar \omega =E-E' = \frac{mc^2\alpha ^2}{2} \left( \frac{1}{n'^2}-\frac{1}{n^2}\right).
%\end{equation}
\section{Trucchi con le costanti fondamentali}
In luogo di $c$, $\hbar$, $e$, $m_e$ utilizzare:
\begin{itemize}
\item $\displaystyle{\alpha =\frac{e^2}{\hbar c}}= \frac{1}{137}$, costante di struttura fine;
\item $\displaystyle{r_0 =\frac{e^2}{m_e c^2}=2.82 \times 10^{-13}}$ cm ($=2.82$ fm), raggio classico dell'elettrone;
\item $\displaystyle{m_ec^2=0.511}$ MeV, massa a riposo dell'elettrone;
\item $\displaystyle{c=3\times 10^{10}}$ cm/s, velocità della luce nel vuoto.
\end{itemize}
Esempi:
\begin{itemize}
\item[-]Raggio di Bohr:
\begin{eqnarray}
a_o &=&\frac{\hbar ^2}{me^2}=\left( \frac{\hbar c}{e^2}\right)\frac{e ^2}{mc^2}=\frac{r_0}{\alpha ^2} \nonumber \\
&\simeq &(137)^2\cdot 2.82 \cdot 10^{-13}=0.529\cdot10^{-10} \textrm{ cm}\ (=0.529 \textrm{ Å})\nonumber
\end{eqnarray}
\item[-]Lunghezza d'onda Compton dell'elettrone:
\begin{eqnarray}
\lambda _e &=&\frac{\hbar}{mc}=\frac{\hbar}{e^2}\frac{e^2}{mc}= \frac{r_0}{\alpha}= \nonumber \\
&\simeq & 137\cdot 2.82\cdot 10^{-13} \textrm{ cm}= 3.86 \cdot 10^{-11} \textrm{ cm} \nonumber
\end{eqnarray}
\item[ ]Livelli di energia dell'atomo di idrogeno:
\begin{eqnarray}
E_n &=& -\frac{me^4}{2n^2\hbar^2}=-\frac{1}{2n^2}mc^2 \left(\frac{e^2}{\hbar c^2}\right)^2= -\frac{mc^2 \alpha ^2}{2n^2}= \nonumber \\
&\simeq & -\frac{1}{n^2}\frac{0.511 \textrm{ MeV}}{2\cdot (137)^2}=-\frac{1}{n^2}13.6 \textrm{ eV}. \nonumber
\end{eqnarray}
\item[-]Costante di Planck:
\begin{eqnarray}
\hbar =\frac{\hbar c}{e^2}\frac{e^2}{mc^2}\frac{mc^2}{c}=\frac{1}{\alpha}\frac{r_0 mc^2}{c}\simeq 6.58\cdot10^{-22} \textrm{ MeV}\cdot \textrm{s} \nonumber 
\end{eqnarray}
\end{itemize}
\section[Raggio classico dell'elettrone]{Raggio classico dell'elettrone: significato fisico}
Schematizziamo l'elettrone come una sfera di raggio $a$ sulla cui superficie è distribuita una carica $e$.\\

Il campo elettrico generato dalla carica è:
\begin{equation}
E=\begin{cases}
0 \qquad \textrm{per }r<a;\\
\\
\displaystyle{\frac{e}{r^2}} \qquad \textrm{per }r>a.
\end{cases}
\end{equation}
La densità associata al campo è:
\begin{equation}
U=\frac{E^2}{8\pi}=\begin{cases}
0 \qquad \textrm{per }r<a;\\
\\
\displaystyle{\frac{e^2}{8\pi r^4}} \qquad \textrm{per }r>a.
\end{cases}
\end{equation}
L'energia totale è dunque
\begin{equation}
E_{el}=\int U dV = \int _a ^{\infty} \frac{e^2}{8\pi r^4} 4\pi r^2 dr = \frac{e^2}{2} \int _a ^{\infty} \frac{dr}{r^2}=\frac{e^2}{2a}. 
\end{equation}
Uguagliando questa energia all'energia di riposo dell'elettrone, $E=m_e c^2$, ricaviamo il raggio dell'elettrone:
\begin{equation}
a=\frac{1}{2}\frac{e^2}{mc^2}=\frac{1}{2} r_0,
\end{equation}
dove $r_0$ è il raggio classico dell'elettrone. Il fattore $\frac{1}{2}$, ottenuto nella precedente espressione, è una conseguenza dei dettagli del modello e segue in particolare dall'aver scelto la carica distribuita solo sulla superficie della sfera (e non, ad esempio, all'interno della sfera stessa). Notiamo anche che l'energia associata ad una carica puntiforme ($a=0$) risulta infinita.
\begin{figure}[!htbp]
\begin{center}
\includegraphics[width= \textwidth]{immagini/cap_21/fig_21_2.png}\\
\end{center}
\end{figure}
\begin{figure}[!htbp]
\begin{center}
\includegraphics[width= \textwidth]{immagini/cap_21/fig_21_3.png}\\
\end{center}
\end{figure}
\begin{figure}[!htbp]
\begin{center}
\includegraphics[width= \textwidth]{immagini/cap_21/fig_21_4.png}\\
\end{center}
\end{figure}
\begin{figure}[!htbp]
\begin{center}
\includegraphics[width= \textwidth]{immagini/cap_21/fig_21_5.png}
\end{center}
\end{figure}
\end{document}
