\documentclass[a4paper,12pt,oneside]{book}
\usepackage[T1]{fontenc}                                      
\usepackage[utf8]{inputenc}                               
\usepackage[italian]{babel}
\usepackage{color}
\usepackage{fancyhdr}
\usepackage{graphicx}
\usepackage{hyperref}
\hypersetup{
			colorlinks=true,
			linkcolor=black,
			anchorcolor=black,
			citecolor=black,
			urlcolor=black,
			pdftitle={Appunti di Meccanica Quantistica},
			pdfauthor={Vittorio Lubicz}
}
\usepackage{titlesec}
\usepackage{vmargin}
\setpapersize{A4}
\setmarginsrb{35mm}{30mm}{35mm}{30mm}%
             {0mm}{10mm}{0mm}{10mm}



\definecolor{gray75}{gray}{0.75}
\newcommand{\hsp}{\hspace{20pt}}

\titleformat{\chapter}[hang]{\huge\bfseries}{\myfont{\textit{\large{\chaptername\hspace{1pt} \thechapter\hspace{3pt}}}}\textcolor{gray75}{$\mid$}\hspace{0.4cm}}{0pt}{\myfont{\huge\bfseries}}

\titleformat{\section}[hang]{\large\bfseries}{\myfont{\textit{\normalsize{\thesection\hspace{2pt}}}}\hspace{0.4cm}}{0pt}{\myfont{\Huge\bfseries}}

\titleformat{\subsection}[hang]{\large\bfseries}{\myfont{\textit{\small{\thesubsection\hspace{2pt}}}}\hspace{0.4cm}}{0pt}{\myfont{\huge\bfseries}}

\renewcommand{\chaptermark}[1]{\markboth{#1}{}}
\renewcommand{\sectionmark}[1]{\markright{#1}}
\newcommand*{\myfont}{\fontfamily{ppl}\selectfont}

\begin{document}

%*****************LAYOUT PAGINE**************************
\fancypagestyle{plain}{%
\fancyhf{} % cancella tutti i campi di  intestazione e pi\`e di pagina
\fancyfoot[C]{\bfseries \myfont{\thepage}} % tranne il centro
\renewcommand{\headrulewidth}{0pt}
\renewcommand{\footrulewidth}{0pt}}

\fancypagestyle{VS}{
\headheight = 15pt
\lhead[\myfont{\textit{\textbf{\thechapter\nouppercase{\leftmark}}}}]{\myfont{\textit{\textbf{\nouppercase{\leftmark}}}}}
\chead[]{}
\rhead[\myfont{\textbf{\thepage}}]{\myfont{\textbf{\thepage}}}

\lfoot[]{}
\cfoot[]{}
\rfoot[]{}
}
%*******************************************************



\pagestyle{VS}
\chapter*{Elenco degli argomenti trattati}
\begin{small}
\textbf{Nota}: all'interno del'elenco degli argomenti trattati viene citata la seguente bibliografia:
\begin{itemize}
\item[ ] J.J. Sakurai, Jim Napolitano, \textit{Meccanica Quantistica Moderna}, Seconda edizione, Zanichelli.
\item[ ] R.P. Feynman et al., \textit{La Fisica di Feynman}, Volume III, Masson;
\item[ ] L. Landau e E. Lifschitz, \textit{Meccanica Quantistica}, Editori Riuniti;
\item[ ] S. Gasiorowicz, \textit{Quantum Physics}, J.Wiley \& Sons.
\end{itemize}
Nell'elenco si fa esplicito riferimento ai testi indicando, ad esempio, con S1.6 il paragrafo
1.6 del libro di Sakurai, con F1.1 il paragrafo 1.1 del libro di Feynman, eccetera.
\end{small}\\

\begin{itemize}
\item \textbf{Crisi della fisica classica} (G1). Spettro del corpo nero. Effetto fotoelettrico. Effetto Compton. Diffrazione degli elettroni e relazione di de Broglie. Esperimento di Rutherford, spettri atomici. Modello di Bohr dell'atomo di idrogeno.

\item \textbf{Onde e particelle} (F1.1-1.8; LL1). Principi fondamentali della meccanica quantistica. Esperimenti di interferenza con pallottole, onde ed elettroni. Probabilità ed ampiezze di probabilità. Principio di indeterminazione

\item \textbf{Vettori di stato ed operatori} (F5.1-5.8,3.1-3.2; S1.1-1.3, 1.5). Formalismo generale della meccanica quantistica Esperimento di Stern e Gerlach ed esperimenti di Stern e Gerlach ripetuti. Stati di base. Vettori di stato bra e ket. Principio di sovrapposizione. Operatori lineari. Rappresentazioni matriciali. Relazione di completezza. Prodotto di operatori. Cambiamenti di base e trasformazioni unitarie.

\item \textbf{Misure, osservabili e relazione di indeterminazione} (S1.4). Valori di aspettazione di osservabili fisiche. Autovalori ed autovettori di osservabili. Autovettori di osservabili come vettori di base. Osservabili compatibili e incompatibili. Relazione di indeterminazione.

\item \textbf{Operatore di posizione} (S1.6-1.7). Autostati dell'operatore di posizione. Funzioni d'onda. Normalizzazione degli autostati della posizione e funzione $\delta$ di Dirac (S1.6-1.7; LL5). Operatori nella rappresentazione delle coordinate. Regole di commutazione per gli operatori di posizione. 

\item \textbf{Richiami di meccanica classica}. Parentesi di Poisson. Trasformazioni canoniche. Trasformazioni canoniche infinitesime. Impulso, energia e momento angolare come generatori delle traslazioni spaziali, traslazioni temporali e rotazioni in meccanica classica.

\item \textbf{Traslazioni e impulso} (S1.6; LL15). Operatore impulso in meccanica quantistica come generatore delle traslazioni. Regole di commutazione canoniche (S1.6; LL16). Operatore impulso nella rappresentazione delle coordinate, autofunzioni dell'impulso (S1.7; LL15). Funzione d'onda nella rappresentazione degli impulsi (S1.7; LL5). Operatore posizione nella rappresentazione degli impulsi. Pacchetti d'onda gaussiani (S1.7).

\item \textbf{Evoluzione temporale ed equazione di Schr\"{o}dinger} (S2.1; LL8). Dinamica quantistica. Evoluzione temporale degli stati. Operatore hamiltoniano. Stati stazionari. Equazione d'onda di Schr\"{o}dinger (S2.4; LL17). La particella libera. Densità di corrente ed equazione di continuità (S2.4).

\item \textbf{Parità} (G4; S4.2). Autovalori ed autostati dell'operatore di parità. Simmetria per inversione spaziale e conservazione della parità.

\item \textbf{Problemi unidimensionali} (LL18). Proprietà generali dell'equazione di Schr\"{o}dinger. Buca di potenziale infinita e buca di potenziale finita (G4-5; LL22). Gradino di potenziale (G5; LL27). Coefficienti di trasmissione e riflessione. Barriera di potenziale. Effetto tunnel (G5; LL25).

\item \textbf{Oscillatore armonico} (S2.3; LL23; G5). Metodo operatoriale di Dirac. Autovalori e autofunzioni dell'energia. 

\item \textbf{Simmetrie e leggi di conservazione} (S2.2, 4.1). Derivata di un operatore rispetto al tempo. Grandezze conservative. Teorema di Ehrenfest (S2.2; LL19).

\item \textbf{Rappresentazioni di Schr\"{o}dinger e di Heisenberg} (S2.2; LL13). Forma di Heisenberg per le equazioni del moto.

\item \textbf{Teoria delle perturbazioni indipendenti dal tempo} (S5.1-5.2; LL38-39, G16). Caso non degenere e caso degenere.

\item \textbf{Teoria delle perturbazioni dipendenti dal tempo} (S5.5-5.6; LL40,42-44, G21). Probabilità di transizione al primo ordine. Transizioni per effetto di una perturbazione costante. Relazione di indeterminazione tempo-energia. Regola d'oro di Fermi. Transizioni per effetto di una perturbazione periodica. Assorbimento ed emissione stimolata.

\item \textbf{Rotazioni e momento angolare} (S3.1; LL26). Regole di commutazione per gli operatori del momento angolare. Autovalori ed elementi di matrice degli operatori di momento angolare (S3.5, LL27).

\item \textbf{Momento angolare orbitale} (S3.6; L26; G11). Autovalori del momento angolare e armoniche sferiche (S3.6; LL27-28; G11). 

\item \textbf{Spin} (S3.2; LL54-55). Operatori di spin e formalismo di Pauli per spin 1/2.

\item \textbf{Composizione di momenti angolari} (S3.7, LL31). Coefficienti di Clebsch-Gordan (S3.7). Stati di tripletto e singoletto per due particelle di spin 1/2 (S3.7). Composizione dei momenti angolari orbitale e di spin per una particella di spin 1/2.

\item \textbf{Particelle identiche} (S6.1-6.3; LL61-62; G8). Principio di indistinguibilità delle particelle identiche. Operatore di scambio. Bosoni e fermioni. Sistemi composti da due particelle identiche. Interazione di scambio.

\item \textbf{Atomo di idrogeno}. Problema dei due corpi, moto in un campo centrale (LL32; G10). Campo coulombiano (LL36; G12). Autovalori ed autofunzioni dello spettro discreto.

\item \textbf{Hamiltoniana di una particella in campo elettromagnetico esterno}. Momento magnetico orbitale ed intrinseco.

\item \textbf{Atomo in un campo elettrico} (LL11,113; S2.6, 5.3). Effetto Stark quadratico e lineare (S5.1-5.2; G16).

\item \textbf{Atomo in un campo magnetico} (S5.3; LL113; G17). Effetto Zeeman ed effetto Paschen-Back.

\item \textbf{Correzioni relativistiche all'Hamiltoniano dell'atomo di idrogeno} (S5.3; G17). Termine cinetico e accoppiamento spin-orbita. Calcolo perturbativo delle correzioni.
\end{itemize}
\end{document}