\chapter[Operatore di parità]{Operatore di parità}
Risulta spesso utile considerare l'operatore di parità P, definito come l'operatore unitario che effettua una \textbf{trasformazione di inversione spaziale}: $\va*{x}\to-\va*{x}$. L'azione dell'operatore di parità può dunque essere definita convenientemente a partire dagli autostati della posizione $\ket{\vec{x}^{\, \prime}}$, sui quali l'operatore agisce nel modo seguente:
	\begin{equation}
	\label{eq:cap9_1}
		\tcboxmath[sharp corners=downhill, colback=white, colframe=black]{
			P\ket{\vec{x}^{\, \prime}} = \ket{-\vec{x}^{\, \prime}} ,
			} \qquad
		\tcboxmath[sharp corners=downhill, colback=white, colframe=black]{
			PP^+ = 1.
			}
	\end{equation}\\
	
Poiché gli autostati della posizione costituiscono un insieme completo di stati di base, l'equazione (\ref{eq:cap9_1}) definisce \textbf{l'azione dell'operatore di parità su uno stato arbitrario $\ket{\alpha}$}:
	\begin{align}
	  \ket{\alpha} &= \int\dd \vec{x}^{\, \prime} \ket{\vec{x}^{\, \prime}}\braket{\vec{x}^{\, \prime}}{\alpha},\\
	  \nonumber \\
	  P\ket{\alpha} &= \int\dd \vec{x}^{\, \prime} P\ket{ \vec{x}^{\, \prime}}\braket{\vec{x}^{\, \prime}}{\alpha} = \int\dd \vec{x}^{\, \prime} \ket{-\vec{x}^{\, \prime}}\braket{\vec{x}^{\, \prime}}{\alpha} =\nonumber\\
	  &= \int\dd \vec{x}^{\, \prime} \ket{\vec{x}^{\, \prime}}\braket{-\vec{x}^{\, \prime}}{\alpha} .
	\end{align}
Da questa equazione vediamo tra l'altro che se $\psi_{\alpha}\qty(\vec{x}^{\, \prime}) = \braket{\vec{x}^{\, \prime}}{\alpha}$ è la f.d.o.~dello stato $\ket{\alpha}$, la f.d.o.~dello stato trasformato per parità, $\ket{\alpha'} = P\ket{\alpha}$, è:
	\begin{equation}
	  \tcboxmath[sharp corners=downhill, colback=white, colframe=black]{
		  \psi_{\alpha'}\qty(\vec{x}^{\, \prime}) = \braket{\vec{x}^{\, \prime}}{\alpha'} = \mel{\vec{x}^{\, \prime}}{P}{\alpha} = \braket{-\vec{x}^{\, \prime}}{\alpha} = \psi_{\alpha}\qty(-\vec{x}^{\, \prime})  ,
		  }
	\end{equation}
o anche più brevemente, nella rappresentazione delle coordinate
	\begin{equation}
	  \tcboxmath[sharp corners=downhill, colback=white, colframe=black]{
	  P\psi_{\alpha}\qty(\vec{x}^{\, \prime}) = \psi_{\alpha}\qty(-\vec{x}^{\, \prime}) .
	  }
	\end{equation}
Applicando due volte consecutive l'operatore di parità si deve ottenere la trasformazione identità, giacche una doppia inversione spaziale non può produrre alcun cambiamento. Pertanto
	\begin{equation}
	  \label{eq:cap9_2}
	 \tcboxmath[sharp corners=downhill, colback=white, colframe=black]{
		  P^2 = 1 .
		  }
	\end{equation}
Poiché l'operatore di parità è per definizione anche unitario, $P^{+}P=1$, da confronto con la precedente equazione segue che
	\begin{equation}
		\tcboxmath[sharp corners=downhill, colback=white, colframe=black]{
			P= P^{+} ,
			}
	\end{equation}
ossia \textbf{l'operatore di parità} è un operatore \textbf{hermitiano}.\\

Indichiamo con $\ket{\alpha_{\lambda}}$ un autostato dell'operatore parità corrispondente all'autovalore $\lambda$:
	\begin{equation}
	  P\ket{\alpha_{\lambda}} = \lambda\ket{\alpha_{\lambda}} .
	\end{equation}
Applicando $P^2$ allo stato $\ket{\alpha_{\lambda}}$ e considerando l'eq.(\ref{eq:cap9_2}) troviamo
	\begin{equation}
	  P^2 = \ket{\alpha_{\lambda}} = P\lambda\ket{\alpha_{\lambda}} = \lambda^2\ket{\alpha_{\lambda}} = \ket{\alpha_{\lambda}} ,
	\end{equation}
ossia
	\begin{equation}
		\tcboxmath[sharp corners=downhill, colback=white, colframe=black]{
			\lambda^2 = 1 .
			}
	\end{equation}
\textbf{Gli autovalori dell'operatore di parità possono dunque valere solo $\pm1$}:
	\begin{equation}
		\tcboxmath[sharp corners=downhill, colback=white, colframe=black]{
			P\ket{\alpha_{\pm}} = \pm\ket{\alpha_{\pm}} .
			}
	\end{equation}
Da questo segue anche per le corrispondenti autofunzioni dell'operatore di parità che
	\begin{equation}
		\tcboxmath[sharp corners=downhill, colback=white, colframe=black]{
			\psi_{\pm}\qty(\vec{x}^{\, \prime}) = \braket{\vec{x}^{\, \prime}}{\alpha_{\pm}} = \pm\mel{\vec{x}^{\, \prime}}{P}{\alpha_{\pm}} = \pm\psi_{\pm}\qty(-\vec{x}^{\, \prime}) ,
			}
	\end{equation}
ossia \textbf{le autofunzioni dell'operatore di parità corrispondenti agli autovalori $\pm1$ sono funzioni rispettivamente pari o dispari}.\\

Dimostriamo ora che \textbf{se una particella è soggetta ad un campo di forze esterne il cui potenziale è una funzione pari delle coordinate, allora l'Hamiltoniano che descrive la particella commuta con l'operatore di parità}:
	\begin{equation}
		\tcboxmath[enhanced, sharp corners=downhill, colframe=black, colback=white, borderline={2pt}{-3pt}{black}]{
			\comm{H}{P} = 0 \qquad \text{se } V\qty(-\va*{x})= V\qty(\va*{x}) .
			}
	\end{equation}
Per dimostrarlo utilizziamo la rappresentazione delle coordinate. Data una qualunque funzione $\psi\qty(\va*{x}')$ si ha:
	\begin{align}
		PH\psi\qty(\vec{x}^{\, \prime}) &= P\qty(-\frac{\hbar^2}{2m}\grad^2+V\qty(x'))\psi\qty(\vec{x}^{\, \prime}) =\nonumber\\
		&=\qty(-\frac{\hbar^2}{2m}\grad^2+V\qty(-x'))\psi\qty(-\vec{x}^{\, \prime}) \overbrace{=}^{ V\qty(-\vec{x}^{\, \prime})=V\qty(\vec{x}^{\, \prime})} HP\psi(\vec{x})  ,
	\end{align}
avendo utilizzato per il termine cinetico (con notazione per semplicità unidimensionale)
	\begin{align}
	  &P\dv{x}\psi(x)=P\psi'(x)= \psi'(-x)=\dv{(-x)}\psi(-x)=-\dv{x}P\psi(x),\\
	  &\Rightarrow P\dv[2]{x}\psi(x)= P\dv{x}\psi'(x) = -\dv{x}P\psi'(x)= \dv[2]{x}P\psi(x) .
	\end{align}\\
	
Sappiamo anche che se l'operatore di parità commuta con l'Hamiltoniano del sistema, allora P ed H ammettono una base di autostati in comune. Si conclude pertanto che \textbf{se il potenziale è una funzione pari delle coordinate, allora gli autostasti dell'Hamiltoniano possono sempre essere scelti come autostati anche dell'operatore di parità, e le corrispondenti autofunzioni risultano essere funzioni pari o dispari delle coordinate}.
	\begin{equation}
		\tcboxmath[sharp corners=downhill, colback=white, colframe=black]{
			H\psi_n\qty(\va*{x})=E_n\psi_n\qty(\va*{x}),
			}\qquad
		\tcboxmath[sharp corners=downhill, colback=white, colframe=black]{
			\psi_n\qty(-\va*{x})=\pm\psi_n\qty(\va*{x}) .
			}
	\end{equation}
\subsection*{Esempio}
Come esempio di quanto discusso consideriamo il caso di una buca di potenziale infinita unidimensionale centrata nell'origine delle coordinate.\\
\begin{minipage}{.4\textwidth}
\includegraphics[width=\textwidth]{immagini/cap_9/fig_9_1.png}	
\end{minipage}
\begin{minipage}{.55\textwidth}
\begin{align}
V(x)= 
\begin{cases}
\infty \quad x<0,\\
0 \quad -a/2>x>a/2, \\
\infty \quad \textrm{fuori}.
\end{cases}
\\
\textrm{(buca di potenziale infinita)} \nonumber
\end{align}
\end{minipage}\\[0.5cm]

In questo caso, infatti, il potenziale è una funzione pari delle coordinate e l'operatore hamiltoniano commuta con l'operatore di parità (1-dimensionale)
	\begin{equation}
	\tcboxmath[sharp corners=downhill, colback=white, colframe=black]{
		\comm {H}{P}= 0 .
		}
	\end{equation}\\
	
Le autofunzioni di questa hamiltoniana si possono ottenere dalle autofunzioni calcolate nel caso della buca di potenziale infinita tra $x=0$ ed $x=a$ semplicemente traslando di $-a/2$ il valore della coordinata. Pertanto si trova
	\begin{align}
		\psi_n(x) = \sqrt{\frac{2}{a}}\sin(\frac{n\pi}{a})\qty(x+\frac{a}{2})= \sqrt{\frac{2}{a}}\ Im\{e^{\frac{in\pi x}{a}}e^{-\frac{in\pi}{2}}\}=\nonumber\\
		=\sqrt{\frac{2}{a}}\qty[\sin(\frac{n\pi x}{a})\cos(\frac{n\pi}{2})+\cos(\frac{n\pi x}{a})\sin(\frac{n\pi}{2})] .
	\end{align}
Si presentano allora due casi, a seconda che \emph{n} sia pari o dispari:
\begin{equation}
  \begin{cases}
    \psi_n(x)= (-1)^{n/2}\sqrt{\frac{2}{a}}\sin(\frac{n\pi x}{a}),\quad n=2,4,6,\;\dots\\
    \psi_n(x)= (-1)^{\frac{n-1}{2}}\sqrt{\frac{2}{a}}\cos(\frac{n\pi x}{a}),\quad n=1,3,5,\;\dots
  \end{cases}
\end{equation}
Senza perdere di generalità possiamo omettere in queste espressioni i fattori di fase irrilevanti e scrivere dunque:
	\begin{equation}
		\tcboxmath[sharp corners=downhill, colback=white, colframe=black]{
		  \begin{cases}
		    \psi_n(x)= \sqrt{\frac{2}{a}}\sin(\frac{n\pi x}{a}),\quad n=2,4,6,\;\dots\\
		    \psi_n(x)= \sqrt{\frac{2}{a}}\cos(\frac{n\pi x}{a}),\quad n=1,3,5,\;\dots
		  \end{cases}
		  }
	\end{equation}
e, ricordiamo, $\psi _n (x)=0$ per $\abs{x} >a/2$.\\

Vediamo dunque che \textbf{le autofunzioni dell'hamiltoniana} sono funzioni pari o dispari delle coordinate, ossia \textbf{risultano simultaneamente autofunzioni dell'operatore di parità}.
