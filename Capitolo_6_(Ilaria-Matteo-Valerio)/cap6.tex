\documentclass[a4paper,12pt,oneside]{book}
\usepackage[T1]{fontenc}                                      
\usepackage[utf8]{inputenc}                               
\usepackage[italian]{babel}
\usepackage{amsfonts}
\usepackage{amsthm}
\usepackage{amsmath,amssymb}
\usepackage{array}
\usepackage{arydshln}
\usepackage{braket}
\usepackage{blindtext}
\usepackage{calc}
\usepackage{cancel}
\usepackage{caption}
\usepackage{epsfig}
\usepackage{eucal}
\usepackage{fancyhdr}
\usepackage{geometry}
\usepackage{graphicx}
\usepackage{indentfirst}
\usepackage{hhline}
\usepackage{hyperref}
\hypersetup{
			colorlinks=true,
			linkcolor=black,
			anchorcolor=black,
			citecolor=black,
			urlcolor=black,
			pdftitle={Appunti di Meccanica Quantistica},
			pdfauthor={Vittorio Lubicz}
}

\usepackage{latexsym}
\usepackage{listings} 
\usepackage{longtable}
\usepackage{makeidx}
\usepackage{mathrsfs}
\usepackage{mathdots}
\usepackage{multirow}
\usepackage{nicefrac}
\usepackage{pdfpages}
\usepackage{physics}
\usepackage{setspace}
\usepackage{tikz}
\usepackage{tikz-3dplot}
\usepackage{textcomp}
\usepackage{titlesec,color}
\usepackage{vmargin}
\setpapersize{A4}
\setmarginsrb{35mm}{30mm}{35mm}{30mm}%
             {0mm}{10mm}{0mm}{10mm}



\definecolor{gray75}{gray}{0.75}
\newcommand{\hsp}{\hspace{20pt}}

\titleformat{\chapter}[hang]{\huge\bfseries}{\myfont{\textit{\large{\chaptername\hspace{1pt} \thechapter\hspace{3pt}}}}\textcolor{gray75}{$\mid$}\hspace{0.4cm}}{0pt}{\myfont{\huge\bfseries}}

\titleformat{\section}[hang]{\large\bfseries}{\myfont{\textit{\normalsize{\thesection\hspace{2pt}}}}\hspace{0.4cm}}{0pt}{\myfont{\Huge\bfseries}}

\titleformat{\subsection}[hang]{\large\bfseries}{\myfont{\textit{\small{\thesubsection\hspace{2pt}}}}\hspace{0.4cm}}{0pt}{\myfont{\huge\bfseries}}

\renewcommand{\chaptermark}[1]{\markboth{#1}{}}
\renewcommand{\sectionmark}[1]{\markright{#1}}
\newcommand*{\myfont}{\fontfamily{ppl}\selectfont}

\begin{document}

%*****************LAYOUT PAGINE**************************
\fancypagestyle{plain}{%
\fancyhf{} % cancella tutti i campi di  intestazione e pi\`e di pagina
\fancyfoot[C]{\bfseries \myfont{\thepage}} % tranne il centro
\renewcommand{\headrulewidth}{0pt}
\renewcommand{\footrulewidth}{0pt}}

\fancypagestyle{VS}{
\headheight = 15pt
\lhead[\myfont{\textit{\textbf{\thechapter\nouppercase{\leftmark}}}}]{\myfont{\textit{\textbf{\nouppercase{\leftmark}}}}}
\chead[]{}
\rhead[\myfont{\textbf{\thepage}}]{\myfont{\textbf{\thepage}}}

\lfoot[]{}
\cfoot[]{}
\rfoot[]{}
}
%*******************************************************



\pagestyle{VS}
\setcounter{chapter}{5}

\chapter{Richiami di Meccanica Classica}

\section{Parentesi di Poisson}
Sia $f(\mathbf{p},\mathbf{q},t)$ una funzione delle coordinate degli impulsi e del tempo e calcoliamone la sua derivata totale rispetto al tempo:

\begin{equation}
\frac{df}{dt}= \frac{\partial f}{\partial t} + \sum_i \left( \frac{\partial f}{\partial q_i} \dot {q_i} + \frac{\partial f}{\partial p_i} \dot {p_i} \right).
\end{equation}

Utilizzando le equazioni di Hamilton:

\begin{equation}
\begin{matrix}


\dot{q_i}=\frac{\partial H}{\partial p_i}, & 
\dot{p_i}=-\frac{\partial H}{\partial q_i}


\end{matrix}
\end{equation}

possiamo scrivere:

\begin{equation}
\frac{df}{dt}=	\frac{\partial f}{\partial t} + \sum_i \left( \frac{\partial H}{\partial p_i}\frac{\partial f}{\partial q_i}-\frac{\partial H}{\partial q_i}\frac{\partial f}{\partial p_i}\right) \equiv \frac{\partial f}{\partial t} + \left\lbrace H,f \right\rbrace
\end{equation}

dove si \`e introdotta la notazione:

\begin{equation}
\left\lbrace H,f \right\rbrace \equiv \sum_i \left(\frac{\partial H}{\partial p_i}\frac{\partial f}{\partial q_i}-\frac{\partial H}{\partial q_i}\frac{\partial f}{\partial p_i}\right) 
\end{equation}.

La quantit\`a $\left\lbrace H,f \right\rbrace$ \`e detta \textbf{parentesi di Poisson} di H con f. In generale:

\begin{equation}
\left\lbrace f,g \right\rbrace \equiv \sum_i \left(\frac{\partial f}{\partial p_i}\frac{\partial g}{\partial q_i}-\frac{\partial f}{\partial q_i}\frac{\partial g}{\partial p_i}\right) 
\end{equation}.

Se $f(\mathbf{p},\mathbf{q},t)$ \`e una costante del moto, allora

\begin{equation}
\begin{matrix}

\frac{\partial f}{\partial t}+\left\lbrace H,f \right\rbrace  \equiv 0 & & &
\left( \frac{df}{dt}=0 \right)

\end{matrix}
\end{equation}.

Se l'integrale del moto non dipende esplicitamente dal tempo, allora

\begin{equation}
\left\lbrace H,f \right\rbrace  \equiv 0
\end{equation}

ossia la sua parentesi di Poisson con l'Hamiltoniana deve annullarsi.
Come esempio specifico consideriamo il caso in cui la funzione $f$ \`e la stessa Hamiltoniana $H$ del sistema. Si ha

\begin{equation}
\frac{d H}{d t} = \frac{\partial H}{\partial t} + \left\lbrace H,H \right\rbrace = \frac{\partial H}{\partial t}
\end{equation}

Per un sistema isolato, l'Hamiltoniana $H$, in virt\`u dell'omogeneit\`a del tempo, non dipende esplicitamente dal tempo: $`partial H/partial t=0$ In questo caso risulta allora

\begin{equation}
\begin{vmatrix}
\frac{d H}{d t = 0 & \text{sui sistemi isolati}
\end{vmatrix}
\end{equation}

che esprime la legge di conservazione dell'energia.
Consideriamo la parentesi di Poisson di una funzione $f$ con una componente delle coordinate e degli impulsi:

\begin{equation}
\begin{matrix}
\left\lbrace f,q_k \right\rbrace = \sum_i \left( \frac{\partial f}{\partial p_i} \frac{\partial q_k}{\partial q_i} - \frac{\partial f}{\partial q_i}\frac{\partial q_k}{\partial p_i} = \frac{\partial f}{\partial p_k} \\


\left\lbrace f,p_k \right\rbrace = \sum_i \left( \frac{\partial f}{\partial p_i} \frac{\partial p_k}{\partial p_i} - \frac{\partial f}{\partial q_i}\frac{\partial p_k}{\partial p_i} = - \frac{\partial f}{\partial q_k}
\end{matrix}
\end{equation}.

Da queste segue allora anche:

\begin{equation}
\begin{matrix}
\left\lbrace q_i,q_k \right\rbrace =0 &  \left\lbrace p_i,p_k \right\rbrace =0 & \left\lbrace p_i,q_k \right\rbrace = \delta_{i,k}
\end{matrix}
\end{equation}.

\section{Trasformazioni Canoniche}

Nel formalismo lagrangiano la scelta delle coordinate non \`e limitata da alcuna condizione: la loro funzione pu\`o essere determinata da $N$ grandezze qualsiasi che definiscono univocamente la posizione del sistema nello spazio. La forma delle equazioni di Lagrange non dipende da questa scelta. In altri termini, in luogo di un insieme di coordinate $q$ che soddisfano le equazioni:

\begin{equation}
\frac{d}{dt}  \left( \sum_i\frac{\partial L}{\partial \dot{q}_i} \right) - \sum_i frac{\partial L}{\partial q_i} = 0
\end{equation}

posso scegliere un differente insieme $Q$, dove in generale:

\begin{equation}
Q_i = Q_i(q,t)
\label{Q_i}
\end{equation}.

La nuova lagranciana $L$ \`e definita dall'equazione

\begin{equation}
L'(Q,\dot{Q},t) = L'\left(Q(q,t), \dot {Q}(q, \dot{q}, t), t \right) = L(q,\dot{q},t)
\end{equation}

e valgono le equazioni di Eulero-Lagrange\footnote{DIM: Si utilizza $\dot{Q} = \frac{\partial Q}{\partial q}\dot{q} + \frac{\partial Q}{\partial t} \Rightarrow \frac{\partial \dot{Q}}{\partial \dot{q}} = \frac{\partial Q}{\partial q}$ e si impone ....**illeggibile***}

\begin{equation}
\frac{d}{dt} \left( \sum_i \frac{\partial L}{\partial \dot{Q}_i} \right) - \frac{\partial L'}{\partial Q_i = 0
\end{equation}.

Queste trasformazioni lasciano inoltre evidentemente invariata anche la forma delle equazioni di Hamilton giacch\'e queste possono essere dedotte a partire dalle equazioni di Eulero-Lagrange.
Nel formalismo hamiltoniano, tuttavia, le coordinate e gli impulsi sono variabili indipendenti. \`E possibile allora considerare, in luogo delle trasformazioni (\ref{Q_i}), una classe di trasformazioni pi\`u ampia, della forma:

\begin{equation}
\begin{matrix}
Q_i = Qi(p,q,t) & P_i = Pi(p,q,t)
\end{matrix}
\label{**}
\end{equation}.

La possibilit\`a di allargare la classe delle trasformazioni ammissibili rappresenta uno dei vantaggi sostanziali della formulazione hamiltoniana della meccanica.
In generale, tuttavia, trasformazioni della forma (\ref{**}) non conducono ad equazioni di Hamilton nella forma canonica, ossia

\begin{equation}
\begin{matrix}
\dot{Q}_i = \frac{\partial H'}{\partial \mathbf{P}_i} & \dot{P}_i = -\frac{\partial H'}{\partial \mathbf{Q}_i} 
\end{matrix}
\end{equation}

con una nuova $H'$.
La classe di trasformazioni (\ref{**}) per le quali ci\`o invece accade sono dette \textit{trasformazioni canoniche}. Stabiliamo le condizioni per le quali le trasformazioni (\ref{**}) risultano essere trasformazioni canoniche.
Poich\`e $ L = \sum_i p_i \dot{q}_i -H $, le equazioni di Hamilton possono essere derivate dal principio di minima azione scritto nella forma:


\begin{equation}
\partial S = \delta \int_{t_1}^{t_2} \left[ \sum_i p_i \dot{q}_i - H \right] dt = \delta \int \left[ \sum_i p_i dq_i - Hdt \right] = 0
\label{***}
\end{equation}.

Considerando le variazioni $p_i \rightarrow p_i + \delta p_i$ , $q_i \rightarrow q_i + \delta q_i$ si ha infatti:

\begin{equation}
\begin{matrix}

\delta S = \int_{t_1}^{t_2} \sum_i \left[ \delta p_i \dot{q}_i + p_i \delta \dot{q}_i - \frac{\partial H}{\partial p_i} \delta p_i - \frac{\partial H}{\partial q_i}\delta q_i \right] dt = \\

= \int_{t_1}^{t_2} \sum_i \left[\delta p_i \dot{q}_i - \dot{p}_i \delta q_i - \frac{\partial H}{\partial p_i} \delta p_i - \frac{\partial H }{\partial q_i} \delta q_i \right] dt + \sum_i pi \delta q_i \vbar_{t_1}^{t_2} = \\

= \int_{t_1}^{t_2} \sum_i \left[ \left( \dot{q}_i - \frac{\partial H}{\partial p_i} \right) \delta p_i - \left( \dot{p}_i + \frac{\partial H}{\partial q_i}\right)\delta q_i \right] dt = 0

\end{matrix}
\end{equation}

da cui

\begin{equation}
\begin{matrix}
\dot{q}_i = \frac{\partial H}{\partial p_i} & \dot{p}_i = - \frac{\partial H}{\partial q_i}
\end{matrix}
\end{equation}

Se le nuove variabili $P_i$ e $Q_i$ sono definite da trasformazioni canoniche, ossia soddisfano le equazioni di Hamilton, allora devono soddisfare anche un principio di minima azione della forma:

\begin{equation}
\delta \int \left( \sum_i P_i dQ_i - H'dt\right] = 0
\label{****}
\end{equation}

Le eq. (\ref{***}) e (\ref{****}) possono risultare simultaneamente soddisfatte solo se le espressioni integrande differiscono al pi\`u per il differenziale totale di una funzione $F$ degli impulsi, delle coordinate e del tempo. In tal caso la differenza tra i due integrali, ossia la differenza dei valori della funzione $F$ nei limiti di integrazione, sar\`a una costante ininfluente ai fini della variazione. Si ha pertanto:

\begin{equation}
\sum_i p_i dq_i - Hdt = \sum_i P_i d Q_i - H'dt + dF
\end{equation}.

Ogni trasformazione canonica \`e caratterizzata da una sua funzione $F$, detta \textit{funzione generatrice della trasformazione}.
Riscrivendo la precedente equazione nella forma

\begin{equation}
dF = \sum_i p_i d q_i - \sum_i P_i dQ_i + (H'-H)dt
\label{star}
\end{equation}

vediamo che

\begin{equation}
\begin{matrix}

p_i = \frac{\partial F}{\partial q_i} & P_i = - \frac{\partial F}{\partial Q_i} & H' = H + \frac{\partial F}{\partial t}

\end{matrix}
\label{o}
\end{equation}

dove si \`e posto che la funzione generatrice \`e data come funzione delle vecchie e delle nuove coordinate del tempo:

\begin{equation}
F = F(q_i, Q_i,t)
\end{equation}.

Pu\`o essere utile esprimere la funzione generatrice non in termini delle variabili $q$ e $Q$ ma in termini delle coordinate $q$ e dei nuovi impulsi $P$. Questo si ottiene effettuando una trasformata di Legendre nella (\ref{star}):

\begin{equation}
d(F+\sum_i P_iQ_i) = \sum_i p_i dq_i + \sum_i Q_i dP_i + (H'-H)dt
\end{equation}.

Indicando con $\Phi$ la funzione $F+\sum_iP_i Q_i $ si ha in questo caso:

\begin{equation}
\begin{matrix}

\frac{\partial \Phi}{\partial q_i} = p_i & \frac{\partial \Phi}{\partial P_i} = Q_i & \frac{\partial \Phi}{\partial t} = H'- H

\end{matrix}
\label{oo}
\end{equation}

con
\begin{equation}
\Phi = \Phi (q,P,t)
\end{equation}.

Analogamente si possono definire funzioni generatrici dipendenti dalle variabili $(p,Q)$ o $(p,P)$.
Osserviamo che se la funzione generatrice non dipende esplicitamente dal tempo,

\begin{equation}
\begin{matrix}

\frac{\partial F}{\partial t} = 0 & \rightarrow & H' = H

\end{matrix}
\end{equation}

ossia la nuova hamiltoniana $H'$ si ottiene semplicemente da $H$ sostituendo le vecchie variabili $p$ e $q$ in termini delle nuove variabili $P$ e $Q$.
Come esempio particolare di trasformazione canonica, consideriamo il caso in cui la funzione generatrice \`e:

\begin{equation}
F(q,Q) = \sum_i q_i Q_i
\end{equation}.

Si ha allora

\begin{equation}
\begin{matrix}

p_i = \frac{\partial F}{\partial q_i} = Q_i & P_i = - \frac{\partial F}{\partial Q_i} = -q_i

\end{matrix}
\end{equation}.

La trasformazione

\begin{equation}
\begin{matrix}
Q_i = p_i & P_i = -q_i
\end{matrix}
\end{equation}

\`e dunque una trasformazione canonica, che corrisponde essenzialmente ad invertire tra loro il ruolo delle coordinate e degli impulsi.

\section{Trasformazioni infinitesime e corrispondenti generatori}
\end{document}





























%%%%%%%%%%%%%%%%%%%%%%%%%%%%%%%%%%%%%%%%%%%%%%%%%%%%%%%%%%%%%%%%%%%%%



\end{document} 
