\documentclass[a4paper,12pt,oneside]{book}
\usepackage[T1]{fontenc}                                      
\usepackage[utf8]{inputenc}                               
\usepackage[italian]{babel}
\usepackage{amsfonts}
\usepackage{amsthm}
\usepackage{amsmath,amssymb}
\usepackage{array}
\usepackage{arydshln}
\usepackage{braket}
\usepackage{blindtext}
\usepackage{calc}
\usepackage{cancel}
\usepackage{caption}
\usepackage{epsfig}
\usepackage{eucal}
\usepackage{fancyhdr}
\usepackage{geometry}
\usepackage{graphicx}
\usepackage{indentfirst}
\usepackage{hhline}
\usepackage{hyperref}
\hypersetup{
			colorlinks=true,
			linkcolor=black,
			anchorcolor=black,
			citecolor=black,
			urlcolor=black,
			pdftitle={Appunti di Meccanica Quantistica},
			pdfauthor={Vittorio Lubicz}
}

\usepackage{latexsym}
\usepackage{listings} 
\usepackage{longtable}
\usepackage{makeidx}
\usepackage{mathrsfs}
\usepackage{mathdots}
\usepackage{multirow}
\usepackage{nicefrac}
\usepackage{pdfpages}
\usepackage{physics}
\usepackage{setspace}
\usepackage{tikz}
\usepackage{tikz-3dplot}
\usepackage{textcomp}
\usepackage{titlesec,color}
\usepackage{vmargin}
\setpapersize{A4}
\setmarginsrb{35mm}{30mm}{35mm}{30mm}%
             {0mm}{10mm}{0mm}{10mm}



\definecolor{gray75}{gray}{0.75}
\newcommand{\hsp}{\hspace{20pt}}

\titleformat{\chapter}[hang]{\huge\bfseries}{\myfont{\textit{\large{\chaptername\hspace{1pt} \thechapter\hspace{3pt}}}}\textcolor{gray75}{$\mid$}\hspace{0.4cm}}{0pt}{\myfont{\huge\bfseries}}

\titleformat{\section}[hang]{\large\bfseries}{\myfont{\textit{\normalsize{\thesection\hspace{2pt}}}}\hspace{0.4cm}}{0pt}{\myfont{\Huge\bfseries}}

\titleformat{\subsection}[hang]{\large\bfseries}{\myfont{\textit{\small{\thesubsection\hspace{2pt}}}}\hspace{0.4cm}}{0pt}{\myfont{\huge\bfseries}}

\renewcommand{\chaptermark}[1]{\markboth{#1}{}}
\renewcommand{\sectionmark}[1]{\markright{#1}}
\newcommand*{\myfont}{\fontfamily{ppl}\selectfont}

\newcommand{\facciatabianca}{\newpage\shipout\null}

\newcommand\Sfondo[1]{\AddToShipoutPicture*{\AtPageCenter{\makebox (0,0){\includegraphics[width=\paperwidth, height=\paperheight]{#1}}}}}
\begin{document}

%*****************LAYOUT PAGINE**************************
\fancypagestyle{plain}{%
\fancyhf{} % cancella tutti i campi di  intestazione e pi\`e di pagina
\fancyfoot[C]{\bfseries \myfont{\thepage}} % tranne il centro
\renewcommand{\headrulewidth}{0pt}
\renewcommand{\footrulewidth}{0pt}}

\fancypagestyle{VS}{
\headheight = 15pt
\lhead[\myfont{\textit{\textbf{\thechapter\nouppercase{\leftmark}}}}]{\myfont{\textit{\textbf{\nouppercase{\leftmark}}}}}
\chead[]{}
\rhead[\myfont{\textbf{\thepage}}]{\myfont{\textbf{\thepage}}}

\lfoot[]{}
\cfoot[]{}
\rfoot[]{}
}
%*******************************************************



\pagestyle{VS}

\pagenumbering{Alph}
\thispagestyle{empty}
\setcounter{page}{3}
 \begin{tikzpicture}[remember picture,overlay]
   \node at (+6.75cm,-9.5cm) {\includegraphics[width=\pdfpagewidth,height=\pdfpageheight]{immagini/copertina/Sfondo.png}};
 \end{tikzpicture}
\clearpage
\setcounter{page}{3}
\facciatabianca
\begin{titlepage}
\setcounter{page}{3}
\begin{center}
% Upper part of the page. The '~' is needed because \\
% only works if a paragraph has started.
\includegraphics[width=0.40\textwidth]{immagini/copertina/logo.pdf}~\\[1cm]

\textsc{\LARGE \bfseries Università degli studi  ``Roma Tre''}\\[0.4cm]

\textsc{\large Dipartimento di Matematica e Fisica}\\[1.5cm]


% Title
\hrule 
\vspace{0.1cm}
\hrule 
\vspace{0.4cm}
{ \Huge \bfseries \textsc{\myfont{Appunti di Meccanica Quantistica}} \\[0.4cm] }
\hrule 
\vspace{0.1cm}
\hrule 
\vspace{0.5cm}
\hfill
\begin{minipage}[t]{0.47\textwidth}\raggedleft
{\Large{\myfont{\textit{Prof. ~Vittorio Lubicz}}}}\\

\vspace{55mm}
\par
\noindent





\end{minipage}

\vspace{1mm}

\begin{center}
\hrule height 0.6mm
{\large{\textbf \quad}\\
\vspace{3mm}}
\includegraphics[width=0.20\textwidth]{immagini/copertina/cc.png}\\
\small{Quest'opera è stata rilasciata sotto la licenza Creative Commons Attribuzione-Non Commerciale-Non opere derivate 2.5 Italia, vedi http://creativecommons.org//licenses/by-nc-nd/2.5/it/}
\end{center}


\end{center}
\end{titlepage}
\setcounter{page}{3}
\facciatabianca
\thispagestyle{empty}
\setcounter{page}{3}
\begin{center}
\myfont{\Large{\textbf{\textit{Premessa}}}}
\end{center}
Gli \textit{Appunti di Meccanica Quantistica} sono da anni il testo di riferimento per il corso di Istituzioni di Fisica Teorica del CdL in Fisica a Roma Tre e rappresentano un ottimo punto di partenza per chi si affaccia, per la prima volta, al vasto e complesso mondo della Meccanica Quantistica. Tutti noi ne abbiamo apprezzato la semplicità, la chiarezza e l'esaustività con cui gli argomenti del corso venivano affrontati, eppure sapevamo che la chiarezza e la fruibilità degli \textit{Appunti} potevano  essere, in qualche modo, affinate. L'idea di base era quella di realizzare una sorta di ``nuova edizione'',  in cui venissero curati maggiormente gli aspetti grafici (coinvolgendo l'indice, le figure e l'impaginazione), il tutto senza perdere il minimo dettaglio per ciò che riguarda le dimostrazioni e, in generale, gli argomenti trattati, che sono rimasti immutati.\\
Questa idea, dato il vasto numero di argomenti trattati negli \textit{Appunti}, ha richiesto parecchio tempo nella realizzazione ed ha coinvolto un team molto ampio di persone. Speriamo che il risultato vi sia utile e vi possa piacere: è il nostro piccolo regalo. Da studenti, per gli studenti.\\
\begin{table}[!htbp]
\begin{center}
\begin{tabular}{cc}
Matteo Altorio & Ilaria Carlomagno \\
Paolo Costantini & Marco De Cicco \\
Eleonora Diociaiuti & Emilia Margoni \\
Carmen Monaco & Emanuele Navisse \\
Irene Schiesaro & Valerio Serpente \\
Giulio Settanta & Carlotta Trigila \\
Valentina Vecchio & Antonio Vigilante
\end{tabular}
\end{center}
\end{table}
\begin{table}[!htbp]
\begin{tabular}{c}
Un ringraziamento particolare a Marco Petrucci per la copertina.
\end{tabular}
\end{table}
\newpage
\setcounter{page}{3}
\facciatabianca
\thispagestyle{empty}
\setcounter{page}{3}
\begin{center}
\myfont{\Large{\textbf{\textit{Nota per il lettore}}}}
\end{center}
All'interno del testo viene citata la seguente bibliografia:
\begin{itemize}
\item J.J. Sakurai, Jim Napolitano, \textit{Meccanica Quantistica Moderna}, Seconda edizione, Zanichelli.
\item R.P. Feynman et al., \textit{La Fisica di Feynman}, Volume III, Masson;
\item L. Landau e E. Lifschitz, \textit{Meccanica Quantistica}, Editori Riuniti;
\item S. Gasiorowicz, \textit{Quantum Physics}, J.Wiley \& Sons.
\end{itemize}
Nel proseguo dei capitoli si fa esplicito riferimento ai testi indicando, ad esempio, con S1.6 il paragrafo
1.6 del libro di Sakurai, con F1.1 il paragrafo 1.1 del libro di Feynman, eccetera.
\newpage
\setcounter{page}{3}
\facciatabianca

\frontmatter
\tableofcontents
\mainmatter
\input{Capitolo_1_(Valerio)/capitolo1.tex}%DEFINITIVO
\pagestyle{VS}
\chapter[Principi fondamentali della M.Q.]{Principi fondamentali\\ della Meccanica \\ Quantistica. Dualismo\\ onda-particella,\\ probabilità ed ampiezze\\ di probabilità, onde ed\\ elettroni}
La \textbf{meccanica quantistica} è la descrizione del comportamento della materia e della luce in tutti i suoi dettagli ed in particolare di ciò che avviene su scala atomica.\\
\textbf{Gli oggetti su scala molto piccola non si comportano come nessuna cosa di cui si possa avere diretta esperienza.} Sotto alcuni aspetti si comportano come onde, sotto altri come particelle, ma in effetti non si comportano come né l'una né l'altra cosa.\\
D'altra parte il \textbf{comportamento quantistico degli oggetti atomici} (elettroni, protoni, neutroni, fotoni e così via) \textbf{è lo stesso per tutti}: sono tutti ``onde-particelle'', o qualunque altro nome gli si voglia dare. \textbf{Una descrizione coerente del comportamento della materia su scala microscopica venne dato, negli anni 1926-1927, principalmente da Schr\"{o}dinger, Heisenberg e Born.}\\

Consideriamo qui le principali caratteristiche di tale descrizione, descrivendo un \textbf{``esperimento ideale''}, che mette a confronto in una particolare situazione sperimentale, il comportamento quantistico degli elettroni con il comportamento di particelle classiche, quali pallottole, ed onde classiche, del tipo di quelle che si formano nell'acqua.
\newpage
\section*{Un esperimento con pallottole}
\begin{figure}[!htbp]
\begin{center}
\includegraphics[width=.6\textwidth]{immagini/cap_2/fig_2_1.png}
\end{center}
\end{figure}
\begin{itemize}
\item[$P_1=$ ]probabilità che il proiettile giunga in $x$ passando per il foro 1;
\item[$P_2=$ ]probabilità che il proiettile giunga in $x$ passando per il foro 2;
\item[$P_{12}=$ ]probabilità che il proiettile giunga in $x$ passando per il foro 1 o per il foro 2.
\end{itemize}
\textbf{Risultato dell'esperimento:} I proiettili arrivano sempre a blocchi identici e distinti. L'effetto con entrambi i fori aperti è la somma degli effetti che si hanno quando è aperto ciascuno dei due fori da solo. Le probabilità vanno sommate:
	\begin{equation}
	\boxed{
		P_{12}=P_1+P_2.
		}
\end{equation}
\textbf{Non si osserva interferenza.}
\newpage
\section*{Un esperimento con onde (prodotte in acqua)}
\begin{figure}[!htbp]
\begin{center}
\includegraphics[width=.6\textwidth]{immagini/cap_2/fig_2_2.png}
\end{center}
\end{figure}
\begin{itemize}
\item[$I_1=$ ]intensità misurata lasciando aperto solo il foro 1;
\item[$I_2=$ ]intensità misurata lasciando aperto solo il foro 2;
\item[$I_{12}=$ ]intensità misurata lasciando aperti entrambi i fori.
\end{itemize}
\textbf{Risultato dell'esperimento:} l'intensità  può assumere qualsiasi valore; non possiede una struttura \textit{a blocchi}. \textbf{L'intensità} misurata quando entrambi i fori sono aperti \textbf{non è la somma} di I$_1$ e I$_2$: \textbf{si ha interferenza tra le due onde.}
\subsection*{Matematica dell'interferenza (formalismo complesso)}
\begin{itemize}
\item $\Re{\left(h_1 e^{i\omega t} \right)}=$ altezza istantanea al rivelatore dell'onda proveniente dal foro 1;
\item $\Re{\left(h_2 e^{i\omega t} \right)}=$ altezza istantanea al rivelatore dell'onda proveniente dal foro 2;
\item $\Re{\left( \left(h_1 +h_2 \right) e^{i\omega t} \right)}=$ altezza istantanea al rivelatore dell'onda che arriva quando entrambi i fori sono aperti.
\end{itemize}
L'intensità è proporzionale all'ampiezza quadratica media, cioè, con il famoso formalismo complesso, al modulo quadro dell'ampiezza. Tralasciando la costante di proporzionalità:
\begin{eqnarray}
& &I_1= \lvert {h_1} \rvert ^2, \nonumber \\
& &I_2 \lvert {h_1} \rvert ^2, \\
& &I_{12}= \lvert {h_1+h_2} \rvert ^2= \lvert {h_1} \rvert ^2+ \lvert {h_2} \rvert ^2 + 2 \lvert {h_1} \rvert \lvert {h_2} \rvert \cos \delta ,\nonumber  \\ 
& &\left( \delta = \textrm{ differenza di fase tra } h_1 \textrm{ e } h_2 \textrm{, funzione di }x \right)\nonumber 
\end{eqnarray}
allora, in termini di intensità:
	\begin{equation}
		\boxed{
			I_{12}= I_1 + I_2 + 2 \sqrt{I_1 I_2}\cos \delta.
			}
	\end{equation}
L'ultimo termine in questa espressione è il \textbf{termine di interferenza.}
\section*{Un esperimento con elettroni}
\begin{figure}[!htbp]
\begin{center}
\includegraphics[width=.6\textwidth]{immagini/cap_2/fig_2_3.png}
\end{center}
\end{figure}
\begin{itemize}
\item[\textbf{P$_1=$} ]probabilità che l'elettrone giunga in $x$ passando per il foro 1 (con il foro 2 chiuso);
\item[\textbf{P$_2=$} ]probabilità che l'elettrone giunga in $x$ passando per il foro 2 (con il foro 1 chiuso);
\item[\textbf{P$_{12}=$} ]probabilità che il proiettile giunga in $x$ con entrambi i fori aperti.
\end{itemize}
\textbf{Risultato dell'esperimento:} I proiettili arrivano sempre in granuli, tutti identici tra loro (come i proiettili). La \textbf{probabilità} $P_{12}$ ottenuta con entrambi i fori aperti \textbf{non è la somma} di $P_1$ e $P_2$:
	\begin{equation}
		\boxed{
			P_{12}\neq P_1+P_2.
			}
	\end{equation}
Se fosse vero che ciascun elettrone o attraversa il foro 1 oppure attraversa il foro 2 allora la probabilità $P_{12}$ dovrebbe essere la somma di $P_1$ e $P_2$. Si potrebbe pensare che gli elettroni seguano percorsi complicati, passando magari più volte per ciascun foro. Ma nemmeno questo è possibile:
\begin{itemize}
\item \textbf{vi sono punti n cui} arrivano meno elettroni quando sono aperti entrambi i fori, ossia \textbf{la chiusura di un foro aumenta il numero di elettroni provenienti dall'altro;}
\item \textbf{al centro della curva}, $P_{12}$ è maggiore della somma $P_1 + P_2$,  \textbf{è come se la chiusura di un foro diminuisce il numero di elettroni che escono dall'altro}.
\end{itemize}

Sebbene questi risultati siano incomprensibili, la loro descrizione matematica è estremamente semplice: \textbf{la curva $P_{12}$ è proprio una curva di interferenza come $I_{12}$ e la matematica è dunque quella dell'interferenza.}\\
I risultati dell'esperimento possono essere descritti introducendo due numei complessi, funzione di $\chi$: $\phi _1$ e $\phi _2$. Si ha poi:
\begin{equation}
\boxed{
  \begin{aligned}
& P_1= \lvert \phi _1 \rvert ^2, & \\
& P_2= \lvert \phi _2 \rvert ^2, &\\
& P_{12}= \lvert \phi _1 + \phi _2 \rvert ^2. 
\end{aligned}
}
\end{equation}
\section*{Osservazione di elettroni}
Poiché il numero di elettroni che arriva in un particolare punto \textbf{non} è uguale al numero di elettroni che arrivano passando dal foro 1 più quelli che passano dal foro 2 ci porta a concludere che \textbf{non è vero che gli elettroni passano attraverso l'uno o l'altro dei fori 1 e 2}. Verifichiamo questa conclusione con un esperimento.\\
Aggiungiamo nell'apparato sperimentale una sorgente di luce, posa dietro allo schermo, a metà tra i due fori. Poiché le cariche elettriche diffondono la luce, quando un elettrone riesce ad attraversare lo schermo devierà verso il nostro occhio della luce e potremo \textit{vedere} il cammino dell'elettrone stesso.
\newpage
\begin{figure}[!htbp]
\begin{center}
\includegraphics[width=.6\textwidth]{immagini/cap_2/fig_2_4.png}
\end{center}
\end{figure}
\textbf{Risultato dell'esperimento:} Gli elettroni che vengono osservati risultano essere passati o dal foro 1 o dal foro 2 ma non da tutti e due insieme. L'andamento di $P'_1$ e $P'_2$, costruiti lasciando entrambi i fori aperti ma osservando da quale foro sia passato l'elettrone è uguale all'andamento di $P_1$ e $P_2$ osservato nel precedente esperimento chiudendo uno dei due fori. Quindi \textbf{gli elettroni che vediamo arrivare attraverso il foro 1 sono distribuiti nello stesso modo, indipendentemente dalla situazione del foro 2.}\\
La probabilità totale risulta dunque essere la somma delle probabilità
\begin{equation}
P'_{12}= P'_1 + P'_2.
\end{equation}
\textbf{La distribuzione degli elettroni sullo schermo quando li osserviamo è differente da quella quando non li osserviamo}.\\
Evidentemente la luce, nell'essere diffusa dagli elettroni, dà loro un colpo che ne fa mutare il movimento. Si può tentare allora di modificare l'esperimento in modo da osservare gli elettroni senza disturbarli troppo. Ma questo non risulta essere possibile. \textbf{È impossibile costruire un apparecchio per determinare da quale foro è passato l'elettrone che allo stesso tempo non perturbi l'elettrone sufficientemente da distruggere l'interferenza. se un apparecchio è capace di determinare da quale foro è passato l'elettrone  non può essere così delicato da non alterarne in modo essenziale la distribuzione.} Questo risultato è una conseguenza particolare del \textbf{principio di indeterminazione.}
\section[Probabilità e ampiezza di probabilità]{Principi base della Meccanica quantistica: probabilità e ampiezza di probabilità}
Riassumiamo ora, in forma generale le principali conclusioni dell'esperimento sopra descritto:
\begin{enumerate}
\item La probabilità di un evento in un esperimento ideale è data dal quadrato del modulo d un numero complesso $\phi$ che viene detto \textbf{ampiezza di probabilità:}
	\begin{equation}
		\boxed{
			\begin{aligned}
			&P= \textrm{ probabilità}, \\
			&\phi = \textrm{ ampiezza di probabilità}, \\
			&P = \lvert \phi \rvert ^2 .
			\end{aligned}
			}
	\end{equation}
\item Quando un evento può avvenire secondo varie alternative, l'ampiezza di probabilità per l'evento è la somma delle ampiezze di probabilità per le varie alternative considerate separatamente. Si ha perciò interferenza:
	\begin{equation}
		\boxed{
			\begin{aligned}
			&\phi = \phi_1 + \phi _2,&  \\
			&P = \lvert \phi_1 + \phi _2 \rvert ^2;&
			\end{aligned}
			}
	\end{equation}
\item Se si effettua un'esperienza capace di determinare se una o l'altra delle possibili alternative è effettivamente realizzata, la probabilità dell'evento non è la somma delle probabilità per ciascuna delle alternative. Non si ha più interferenza:
	\begin{equation}
		\boxed{
			P= P_1 + P_2.
			}
	\end{equation}
\end{enumerate}

Sottolineiamo una differenza molto importante tra la meccanica classica e quella quantistica: \textbf{nella meccanica quantistica è impossibile prevedere esattamente ciò che accadrà in una data situazione. la sola cosa che è possibile prevedere é la probabilità di eventi differenti.}
\section{Il principio di indeterminazione}
La presenza di interferenza nell'esperimento delle due fenditure, con gli elettroni, mette in risalto come, nel caso di particelle microscopiche, il \textbf{concetto di traiettoria}, che sta a fondamento della meccanica classica, \textbf{viene a perdere di significato nella meccanica quantistica}. Tale circostanza trova la sua espressione nel cosiddetto \textbf{principio di indeterminazione}, uno dei principi basilari della meccanica quantistica, scoperto da \textbf{Heisenberg} nel \textbf{1927}.\\
Se, in seguito ad una misura, ad un elettrone vengono assegnate coordinate determinate, esso non ha, in generale, nessuna velocità determinata. Viceversa se è dotato di una velocità determinata, l'elettrone non potrà avere una posizione determinata nello spazio. Infatti \textbf{l'esistenza simultanea ad ogni istante delle coordinate e delle velocità significherebbe l'esistenza di una traiettoria determinata, che l'elettrone non ha}. Di conseguenza nella meccanica quantistica \textbf{le coordinate e le velocità dell'elettrone sono grandezze che non possono essere misurate con precisione allo stesso istante, cioè non possono avere simultaneamente valori determinati. Si può dire che le coordinate e le velocità dell'elettrone sono grandezze non esistenti simultaneamente.}\\
Una formulazione matematica del principio di indeterminazione è data dalla relazione:
	\begin{equation}
		\boxed{\boxed{
			\Delta p_x \Delta x \geq \frac{\hbar}{2}.
			}}
	\end{equation}
\subsection{Il principio di indeterminazione e l'esperimento delle due fenditure}
Mostriamo, in un caso particolare, come il principio di indeterminazione di Heisenberg deve essere valido al fine di evitare situazioni inconsistenti.\\
Immaginiamo di modificare l'esperimento di interferenza degli elettroni sostituendo la parete fissa, con le due fenditure, con una lamina montata su due cuscinetti che si può muovere liberamente in direzione dell'asse x:
\begin{figure}[!htbp]
\begin{center}
\includegraphics[width=.5\textwidth]{immagini/cap_2/fig_2_5.png}
\end{center}
\end{figure}\\
Osservando il moto della lamina possiamo provare a determinare attraverso quale foro passa un elettrone. Consideriamo infatti il caso in cui il rivelatore è posto in $x=0$. Ci aspettiamo che un elettrone che passi per il foro 1 debba essere deflesso verso il basso dalla lamina per poter arrivare al rivelatore. Poiché la componente verticale dell'impulso dell'elettrone è variata la lamina deve muoversi in direzione opposta con lo stesso impulso. La lamina riceverà quindi una spinta verso l'alto. Se invece l'elettrone passa dal foro inferiore la lamina dovrebbe subire una spinta verso il basso. È chiaro che per ogni posizione del rivelatore l'impulso ricevuto dalla lamina avrà un valore differente a seconda che l'elettrone attraversi il foro 1 o il foro 2. Quindi, \textbf{senza per nulla perturbare gli elettroni, ma solo osservando la lamina, possiamo determinare il percorso scelto dall'elettrone.\\
Tuttavia, per determinare di quanto è variato l'impulso della lamina dopo il passaggio dell'elettrone occorre conoscere l'impulso di questa prima che l'elettrone la attraversi.} Calcoliamo l'impulso che l'elettrone trasmette alla lamina attraversando un foro:\\
\begin{minipage}{.5\textwidth}
\includegraphics[width=.8\textwidth]{immagini/cap_2/fig_2_6.png}	
\end{minipage}
\hspace{.5cm}
\begin{minipage}{.4\textwidth}
\begin{equation} \tan \theta \simeq \theta \simeq \frac{a}{2D}
\end{equation}
\begin{equation} p_x = p \sin \theta = \frac{pa}{2D}
\end{equation}
\end{minipage}\\
\vspace{.5cm}

L'impulso trasmesso è dell'ordine di
	\begin{equation}
		\boxed{
		\Delta p \simeq 2p_x = 2p \sin \theta \simeq \frac{pa}{D},
		}
	\end{equation}
e questa quantità rappresenta anche l'incertezza massima con la quale è necessario conoscere l'impulso della lamina prima che l'elettrone l'attraversi per poter distinguere se l'elettrone è passato attraverso il foro 1 o il foro 2.\\
In base al \textbf{principio di indeterminazione}, se l'impulso è noto con una precisione maggiore di $\Delta  p$, allora \textbf{la posizione della lamina stessa non può essere conosciuta con una precisione maggiore di}:
	\begin{equation}
		\boxed{
		\Delta x \simeq \frac{\hbar}{\Delta p} \simeq \frac{\hbar D}{pa}= \frac{\lambda D}{a},
		}
	\end{equation}
dove $\lambda = \hbar / p$ è la lunghezza d'onda di De Broglie associata all'elettrone che si muove con impulso $p$.\\

L'incertezza $\Delta x$ è allora anche l'incertezza con cui è definita la posizione delle due fenditure, che saranno quindi in diverse posizioni per ogni elettrone che l'attraversi. Questo significa che \textbf{il centro delle frange di interferenza avrà una posizione differente per i vari elettroni.}\\

Dimostreremo ora che la lunghezza $\Delta x$, di cui oscillano lungo l'asse $x$ le frange di interferenza, è circa uguale alla distanza tra due massimi vicini. \textbf{Un tale movimento, che avviene a caso, è giusto sufficiente a distruggere le oscillazioni del grafico e quindi a far sì che non si osservi più interferenza.}\\
\vspace{1cm}
\begin{minipage}{.5\textwidth}
\includegraphics[width=.8\textwidth]{immagini/cap_2/fig_2_7.png}	
\end{minipage}
\begin{minipage}{.5\textwidth}
\begin{equation}
\Delta = a \sin \theta \simeq a \theta = \frac{ax}{D}
\end{equation}
\end{minipage}\\
\vspace{.5cm}
La differenza di fase tra le onde che giungono nel punto $x$ dalle due fenditure è:
	\begin{equation}
		\delta = k\Delta =\frac{2 \pi }{\lambda}\Delta = \frac{2 \pi a x}{D}.
	\end{equation}
I massimi di interferenza si hanno quando la differenza di fase $\delta $ è pari ad un multiplo intero di $2 \pi$, ossia nei punti d coordinate
	\begin{equation}
		x_n =n\frac{\lambda D}{a}, \qquad n=0,\pm 1, \pm 2, \dots 
	\end{equation}
Due massimi consecutivi si trovano dunque a distanza 
	\begin{equation}
		\boxed{
			\Delta x = \frac{\lambda D}{a},
			}
	\end{equation}
che coincide con lo spostamento tipico del centro delle frange di interferenza per ciascun elettrone.\\

\textbf{il principio di indeterminazione garantisce quindi che l'aver osservato la fenditura attraverso la quale è passato l'elettrone porta alla scomparsa dell'interferenza.}
%DEFINITIVO
\input{Capitolo_3_(Paolo)/capitolo3.tex}%DEFINITIVO
\chapter[Misure e osservabili]{Misure, osservabili e principio di indeterminazione}
Discutiamo qui la \textbf{teoria quantistica della misura}. In meccanica quantistica, per un sistema che si trovi in un determinato stato iniziale $\vert \varphi \rangle$, una misura di una quantità fisica non produce, in generale, sempre lo stesso risultato. Piuttosto si possono ottenere differenti risultati, ciascuno con una determinata probabilità.\\

Se indichiamo con $a_i$ i possibili risultati di una misura dell'osservabile $A$ e con $P_i$ le relative probabilità (riferite ad un sistema che si trovi nello stato $\vert \varphi \rangle$) possiamo scrivere il \textbf{valore medio} dei risultati di una misura $A$ come
	\begin{equation}
		\tcboxmath[sharp corners=downhill ,colback=white,colframe=red!75!black]{
			\langle A \rangle = \sum _i a_i P_i.
			}
	\end{equation}
\\

\textbf{Nella meccanica quantistica si associa ad ogni grandezza fisica $A$ un operatore lineare che la rappresenta:}
	\begin{equation}
		\tcboxmath[enhanced,sharp corners=downhill ,colback=yellow!50!white,colframe=red!75!black, borderline={2pt}{-3pt}{red!50!black}]{
			\begin{array}{c}
			\textrm{Grandezza fisica}\\
			A
			\end{array}
			\longleftrightarrow
			\begin{array}{c}
			\textrm{Operatore}\\
			A
			\end{array}
			}
	\end{equation}
\\

\textbf{L'operatore $A$ viene definito in maniera tale che, per un sistema che si trovi in uno stato $\vert \varphi \rangle $, valga la relazione}
	\begin{equation}
		\tcboxmath[enhanced,sharp corners=downhill ,colback=yellow!50!white,colframe=red!75!black, borderline={2pt}{-3pt}{red!50!black}]{
			\langle A \rangle = \langle \varphi \vert A \vert \varphi \rangle,
			}
	\end{equation}
ossia il valor medio dei possibili risultati di una misura di $A$ è dato dal \textbf{valore di aspettazione} dell'operatore $A$ sullo stato $\vert \varphi \rangle$.\\

Naturalmente \textbf{i valori medi di qualsiasi grandezza fisica reale, in qualunque stato, sono reali}. Questa circostanza pone determinate limitazioni alle proprietà degli operatori che corrispondono, nella meccanica quantistica, alle grandezze fisiche. Assumiamo infatti che $\langle \varphi \vert A \vert \varphi \rangle$ sia reale per qualunque scelta del vettore di stato $\vert \varphi \rangle$. Consideriamo poi un vettore $\vert \varphi \rangle$ della forma
	\begin{equation}
		\vert \varphi \rangle = \alpha \vert u \rangle + \beta \vert v \rangle,
	\end{equation}
dove $\alpha$ e $\beta$ sono numeri complessi. Il valore di aspettazione di $A$ su questo stato è dato da:
	\begin{eqnarray}
		& &\left( \alpha ^* \langle u \vert + \beta ^* \langle v \vert \right) A \left(\alpha \vert u \rangle + \beta \vert v \rangle \right) = \nonumber \\
		& &=\vert \alpha \vert ^2 \langle u \vert A \vert u \rangle + \vert \beta \vert ^2 \langle v \vert A \vert v \rangle + \alpha ^* \beta \langle u \vert A \vert v \rangle \alpha  \beta ^* \langle v \vert A \vert u \rangle.
	\end{eqnarray}
Per ipotesi, i primi due termini che entrano in questa espressione sono reali. Deve dunque essere reale anche la somma dei due secondi termini. Uguagliando la parte immaginaria di questa somma a zero otteniamo:
	\begin{equation}
		\label{eq:cap4_1}
		\alpha ^* \beta \left(\langle u \vert A \vert v \rangle -\langle u \vert A^{+} \vert v \rangle \right)-  \alpha  \beta ^* \left( \langle v \vert A^{+} \vert u \rangle -\langle v \vert A \vert u \rangle \right) =0, 
	\end{equation}
da cui segue:
	\begin{equation}
		\langle u \vert A \vert v \rangle = \langle u \vert A^{+} \vert v \rangle ,
	\end{equation}
Ma poiché i vettori $\vert u \rangle$ e $\vert v \rangle$ sono arbitrari concludiamo che
	\begin{equation}
		\tcboxmath[enhanced,sharp corners=downhill ,colback=yellow!50!white,colframe=red!75!black, borderline={2pt}{-3pt}{red!50!black}]{
			A= A^{+},
			}
	\end{equation}
ossia \textbf{gli operatori che rappresentano le osservabili in meccanica quantistica sono operatori hermitiani} (viceversa se $A= A^{+}$ allora $\langle \varphi \vert A \vert \varphi \rangle$ è reale per definizione di $A^{+}$).\\ \\

La condizione (\ref{eq:cap4_1}) può essere scritta anche nella forma
	\begin{equation}
		\alpha ^* \beta m - \left( \alpha ^*  \beta \right) ^* m^* =0, 
	\end{equation}
dove $m= \left( \langle u \vert A \vert v \rangle - \langle u \vert A^{+} \vert v \rangle \right)$. Scegliendo allora $\alpha ^* \beta$ reale oppure $\alpha ^* \beta$ immaginario puro si ottiene rispettivamente
	\begin{equation}
		\begin{cases}
		\alpha ^* \beta \left( m- m^* \right) =0 \\
		\alpha ^* \beta \left( m + m^* \right) =0
		\end{cases}
		\Rightarrow
		\begin{cases}
		m= m^*\\
		m= - m^*
		\end{cases}
	\end{equation}
che implicano necessariamente
	\begin{equation}
		m=0.
	\end{equation}
\section{Autovalori ed autovettori di osservabili}
Cerchiamo di determinare i possibili valori della grandezza $A$ e gli stati $\vert a' \rangle$ nei quali tale grandezza non può avere che un solo valore determinato $a'$. Per tali stati lo \textbf{scarto quadratico medio} 	
	\begin{equation}
		\tcboxmath[sharp corners=downhill ,colback=white,colframe=red!75!black]{
			\langle \left( \Delta A \right) ^2 \rangle \equiv \langle \left( A- \langle A\rangle \right) ^2 \rangle
			}
	\end{equation}
deve essere nullo.\\

Calcoliamo allora esplicitamente il valore di aspettazione di $(\Delta A )^2$ sullo stato $\vert a' \rangle$, ponendo dunque $\langle A \rangle = a'$. Utilizzando la relazione di completezza per un insieme arbitrario di stati di base si ottiene:
	\begin{eqnarray}
		0 & = & \langle a' \vert (\Delta A )^2 \vert a'\rangle = \langle a' \vert (A-a' )(A-a') \vert a'\rangle= \nonumber \\
		& = & \sum _i \langle a' \vert (A-a' ) \vert i \rangle \langle i \vert (A-a' ) \vert a' \rangle = \nonumber \\
		& = & \sum _i \langle i \vert (A^{+}-a' ) \vert a' \rangle ^* \langle i \vert (A-a' ) \vert a' \rangle = \nonumber \\
		& = & \sum _i \vert \langle i \vert (A-a' ) \vert a' \rangle \vert ^2.
	\end{eqnarray}
Poiché ciascun termine della sommatoria è positivo o nullo, questa condizione può essere soddisfatta solo se
	\begin{equation}
		\langle i \vert A-a' \vert a' \rangle =0 \qquad \textrm{ per ogni } \vert i \rangle .
	\end{equation}
Otteniamo quindi la relazione
	\begin{equation}
		\tcboxmath[enhanced,sharp corners=downhill ,colback=yellow!50!white,colframe=red!75!black, borderline={2pt}{-3pt}{red!50!black}]{
			A \vert a' \rangle =a' \vert a' \rangle .
			}
	\end{equation}
Questa equazione è detta \textbf{equazione agli autovalori}. I numeri $a'$ sono detti \textbf{autovalori} dell'operatore $A$ ed i corrispondenti stati $\vert a' \rangle $ prendono il nome di \textbf{autostati od autovettori} dell'operatore.\\

Abbiamo così mostrato che \textbf{se un sistema si trova in uno stato corrispondente ad un autostato dell'operatore $A$ con autovalore $a'$, allora una misura dell'osservabile $A$ produce con certezza il valore $a'$. Viceversa, se una misura dell'osservabile $A$ in un determinato stato produce con certezza il valore $a'$, allora lo stato in questione è un autostato di $A$ corrispondente all'autovalore $a'$.}\\

\textbf{In meccanica quantistica si postula inoltre che la totalità degli autovalori dell'operatore $A$ è identica alla totalità di tutti i possibili risultati di una misura della grandezza $A$ corrispondente.}\\

È utile sottolineare la \textbf{distinzione tra autovalori e valori di aspettazione.} Per esempio, per una particella di spin $1/2$, i risultai di una misura della componente $z$ dello spin possono assumere solo i valori $\pm \hbar/2$ (corrispondenti agli autovalori dell'operatore $S_z$) mentre il valore di aspettazione di $S_z$ in un determinato stato può assumere in generale qualunque valore compreso tra $-\hbar /2$ e $+\hbar /2$.
\section{Autovettori di osservabili come vettori di base}
Consideriamo due \textbf{proprietà degli operatori hermitiani:}
\begin{enumerate}
\item \textbf{Gli autovalori di un operatore hermitiano sono reali.} Indichiamo con $a'$ un autovalore di $A$ e con $\vert a' \rangle $ il corrispondente autovettore convenientemente normalizzato$\langle a' \vert a' \rangle =1 $. Possiamo allora scrivere
	\begin{equation}
		\langle a' \vert A \vert a' \rangle = a' \langle a' \vert a' \rangle = a'.
	\end{equation}
D'altra parte, per l'hermitianità dell'operatore $A$ si ha pure
	\begin{equation}
		\langle a' \vert A \vert a' \rangle = \langle a' \vert A \vert a' \rangle  ^* = a^{\prime \, *},
	\end{equation}
da cui segue
	\begin{equation}
		\tcboxmath[sharp corners=downhill ,colback=white,colframe=red!75!black]{
			a'=a^{\prime \, *} .
			}
	\end{equation}
Questo risultato è consistente con l'assunzione che gli autovalori di un operatore hermitiano $A$ rappresentano i possibili risultati di una misura della grandezza fisica reale $A$.
\item \textbf{Gli autovettori di un operatore hermitiano corrispondenti ad autovalori distinti sono ortogonali.}\\
indichiamo con $a' $ e $a''$ due autovalori distinti di $A$ e con $\vert a' \rangle$ ed $\vert a'' \rangle$ i corrispondenti autovettori. Si ha:
	\begin{equation}
		\langle a' \vert A \vert a'' \rangle = a'' \langle a' \vert a'' \rangle ,
	\end{equation}
	\begin{equation}
		\langle a' \vert A \vert a'' \rangle = \langle a'' \vert A \vert a' \rangle ^* = a' \langle a'' \vert a' \rangle ^* = a' \langle a' \vert a'' \rangle ,
	\end{equation}
dove nella seconda equazione si è utilizzato il risultato secondo cui gli autovalori di un operatore hermitiano sono reali. Sottraendo allora membro a membro le due equazioni si ottiene
	\begin{equation}
		(a'-a'') \langle a' \vert a'' \rangle =0,
	\end{equation}
ossia
	\begin{equation}
		\tcboxmath[sharp corners=downhill ,colback=white,colframe=red!75!black]{
		\langle a' \vert a'' \rangle =0 \qquad \textrm{ per } a' \neq a'' .
		}
	\end{equation}
\end{enumerate}

Osserviamo che \textbf{in generale gli autostati associati ad uno stesso autovalore non sono ortogonali.} Poiché però una qualunque combinazione lineare di autostati degeneri è ancora un autostato associato allo stesso autovalore, \textbf{risulta sempre possibile scegliere tali autostati in modo che siano a due a due ortogonali.}\\

Per ogni operatore hermitiano $A$ è possibile dunque definire un insieme ortonormale di autovettori che soddisfa cioè la relazione 
	\begin{equation}
		\tcboxmath[enhanced,sharp corners=downhill ,colback=yellow!50!white,colframe=red!75!black, borderline={2pt}{-3pt}{red!50!black}]{
			\langle a' \vert a'' \rangle = \delta _{a' a''},
			}
	\end{equation}
e che rappresenta una \textbf{base} nello spazio dei vettori di stato.\\

\textbf{Risulta allora possibile sviluppare un vettore di stato arbitrario $\vert \varphi \rangle $ come combinazione lineare di autostati dell'operatore $A$:}
	\begin{equation}
		\tcboxmath[enhanced,sharp corners=downhill ,colback=yellow!50!white,colframe=red!75!black, borderline={2pt}{-3pt}{red!50!black}]{
			\vert \varphi \rangle = \sum _{a'} c_{a'} \vert a' \rangle .
			}
	\end{equation}
\\

I coefficienti dello sviluppo si ottengono moltiplicando  a sinistra per $\langle a'' \vert $ ed utilizzando l'ortonormalità degli autostati di $A$:
	\begin{equation}
		\tcboxmath[enhanced,sharp corners=downhill ,colback=yellow!50!white,colframe=red!75!black, borderline={2pt}{-3pt}{red!50!black}]{
			c_{a'} = \langle a' \vert \varphi \rangle .
			}
	\end{equation}
	\\
	
Cerchiamo il significato fisico delle ampiezze $c_{a'}$. in termini di queste ampiezze, il valore medio di $A$ sullo stato $\vert \varphi \rangle$ si scrive:
	\begin{equation}
		\langle A \rangle =  \langle \varphi \vert A \vert \varphi \rangle= \sum _{a',a''} c_{a'}^* c_{a''} \langle a' \vert A \vert a'' \rangle =   \sum _{a',a''} c_{a'}^* c_{a''}\  a'' \langle a' \vert a'' \rangle = \sum _{a'} \vert c_{a'} \vert ^2\ a'. 
	\end{equation}
D'altra parte, la condizione di normalizzazione del vettore di stato $\vert \varphi \rangle$ comporta:
	\begin{equation}
	\langle \varphi \vert \varphi \rangle = 1 = \sum _{a',a''} c_{a'} ^* c_{a''}  \langle a' \vert a'' \rangle = \sum _{a'} \vert c_{a'} \vert ^2 .
	\end{equation}
\textbf{Dalle due uguaglianze:}
	\begin{equation}
		\tcboxmath[enhanced,sharp corners=downhill ,colback=yellow!50!white,colframe=red!75!black, borderline={2pt}{-3pt}{red!50!black}]{
		\langle A \rangle = \sum _{a'} \vert c_{a'} \vert ^2 a',
		}
		\qquad
		\tcboxmath[enhanced,sharp corners=downhill ,colback=yellow!50!white,colframe=red!75!black, borderline={2pt}{-3pt}{red!50!black}]{
		\sum _{a'} \vert c_{a'} \vert ^2 =1 ,
		}
	\end{equation}
\textbf{si deduce che il modulo quadro delle ampiezze $c_{a'}$ rappresenta la probabilità di trovare, in seguito ad una misura della grandezza fisica $A$, il valore $a'$:}
	\begin{equation}
		\tcboxmath[enhanced,sharp corners=downhill ,colback=yellow!50!white,colframe=red!75!black, borderline={2pt}{-3pt}{red!50!black}]{
		\textrm{Probablità per }a'= \vert c_{a'} \vert ^2 = \vert \langle a' \vert \varphi \rangle \vert ^2
		}
	\end{equation}
(purché lo stato $\vert \varphi \rangle$ sia normalizzato).\\

Questa interpretazione è del tutto naturale, nel formalismo che stiamo sviluppando: la quantità $ \langle a' \vert \varphi \rangle$ rappresenta infatti l'ampiezza di probabilità che lo stato $\vert \varphi \rangle $ ``si porta''  nello stato $\vert a' \rangle $, stato in cui una misura di $A$ produce con certezza il valore $a'$.\\
 
\textbf{Se di un sistema nello stato $\vert \varphi \rangle $ si effettua una misura dell'osservabile $A$ e si ottiene come risultato il valore $a'$, allora, per effetto della misura, il sistema \textit{precipita} nello stato $ \vert a' \rangle $ }
	\begin{equation}
		\tcboxmath[enhanced,sharp corners=downhill ,colback=yellow!50!white,colframe=red!75!black, borderline={2pt}{-3pt}{red!50!black}]{
			\vert \varphi \rangle \xrightarrow[\textrm{Misura di }A]{ } \vert a' \rangle
			}
	\end{equation}
\textbf{In questo senso il processo di misura, in meccanica quantistica, influisce sempre sullo stato del sistema.} La sola eccezion è quando lo stato iniziale è già autostato dell'osservabile che viene misurata.
\subsection{Misura selettiva}
Nello studio dell'esperienza di Stern-Gerlach ideale Abbiamo considerato apparecchi di Stern-Gerlach con filtri, del tipo
	\begin{equation}
		\begin{matrix}
		\begin{Bmatrix}
 			\makebox[\widthof{$-$}]{$+$}\makebox[\widthof{$|$}]{$\ $} \\ \makebox[\widthof{$-$}]{$0$}|\\ - | 
		\end{Bmatrix} \\[0.7cm]
			S
		\end{matrix}
	\end{equation}
Siamo ora in grado di dare un'espressione esplicita dell'operatore corrispondente ad un apparecchio di questo tipo. In generale, un processo di misura che seleziona solo uno degli autoket di un'osservabile $A$, diciamo $\vert a' \rangle $, ed elimina tutti gli altri, è detto \textbf{misura selettiva}. È evidente che un tale processo è descritto matematicamente dall'\textbf{operatore di proiezione}
	\begin{equation}
		\tcboxmath[sharp corners=downhill ,colback=white,colframe=red!75!black]{
		\Lambda _{a'} = \vert a' \rangle \langle a'\vert.
		}
	\end{equation}
\section{Osservabili compatibili ed operatori commutanti}
\textbf{Poiché in uno stato due osservabili $A$ e $B$ abbiano simultaneamente valori ben determinati} (ossia $\langle (\Delta A ) ^2 \rangle = \langle (\Delta B ) ^2 \rangle =0$) \textbf{è necessario che tale stato sia autovettore comune degli operatori $A$ e $B$.} \textbf{È possibile mostrare che due operatori hanno una base di autostati in comune se e solo se i due operatori, diciamo $A$ e $B$, commutano tra loro:}
	\begin{equation}
		\tcboxmath[sharp corners=downhill ,colback=white,colframe=red!75!black]{
			\left[ A, B \right] =0.
			}
	\end{equation}
In questo caso le due \textbf{osservabili} si dicono \textbf{compatibili}. Se $\left[ A, B\right] \neq 0$, le \textbf{osservabili} si dicono \textbf{incompatibili.}\\

Dimostriamo in primo luogo che due osservabili che ammettono una base di autostati in comune commutano tra loro. Indichiamo con $\vert a', b' \rangle$ ali autostati, per i quali
	\begin{equation}
		A\vert a', b' \rangle= a'\vert a', b' \rangle \qquad B\vert a', b' \rangle= b'\vert a', b' \rangle.
	\end{equation}
Si ha allora
	\begin{equation}
		\left[ A, B \right]\vert a', b' \rangle = \left(AB-BA\right)\vert a', b' \rangle=\left( a'b'-b'a'\right)\vert a', b' \rangle=0,
	\end{equation}
ossia
	\begin{equation}
		\left[A,B\right]\vert a', b' \rangle=0 \qquad \forall \vert a', b' \rangle.
	\end{equation}
Poiché questa identità vale per qualunque autostato di base, allora vale anche per qualunque vettore di stato $\vert \varphi \rangle$. ma allora l'operatore $\left[ A, B \right]$ deve essere identicamente nullo:
	\begin{equation}
		\left[ A, B \right] =0 \quad \textrm{(C.V.D.)}
	\end{equation}
	\\
	
Supponiamo ora che i due operatori $A$ e $B$ commutino tra loro. Dimostriamo che in questo caso ammettono una base di autostati in comune. Consideriamo l'elemento di matrice di $[A,B]$ tra due autostati dell'operatore $A$. Si ha:
	\begin{equation}
		0 = \langle a' \vert \left[A, B\right] \vert a'' \rangle = \langle a' \vert \left( AB-BA \right) \vert a'' \rangle = \left( a'-a''\right) \langle a' \vert B \vert a'' \rangle . 
	\end{equation}
Ora, se tutti gli autovalori dell'operatore $A$ sono diversi tra loro, si avrà $(a'-a'') \neq 0$ e dunque
	\begin{equation}
		\langle a' \vert B \vert a'' \rangle =0,
	\end{equation}
per $a' \neq a''$.
in altri termini, nella rappresentazione degli autostati di $A$ anche la matrice $B$ risulta diagonale. Se invece la matrice $A$ è degenere, allora in generale qualcuno degli elementi non diagonali di $B$ può risultare diverso da zero. Tuttavia in questo caso è sempre possibile scegliere come stati di base una combinazione lineare lineare di autostati degeneri di $A$ in modo tale che anche con tale scelta la matrice $B$ risulti diagonale.\\

In conclusione, abbiamo dimostrato che \textbf{la commutatività degli operatori è condizione necessaria e sufficiente perché due grandezze fisiche possano avere simultaneamente valori determinati, ossia siano simultaneamente misurabili.}\\

Consideriamo le misure di A e B quando le due osservabili sono compatibili. Supponiamo di misurare $A$ per primo e ottenere il risultato $a'$. Successivamente possiamo misurare $B$ ed ottenere il risultato $b'$. Una terza misura di $A$ darà allora come risultato $a'$ con certezza, cioè la seconda misura ($B$) non distrugge la precedente informazione contenuta nella prima misura ($A$). Questo è ovvio quando gli operatori di $A$ sono non degeneri:
	\begin{equation}
		\vert \varphi \rangle \xrightarrow[\textrm{Misura di }A]{ } \vert a',b' \rangle \xrightarrow[\textrm{Misura di }B]{ } \vert a',b' \rangle \xrightarrow[\textrm{Misura di }A]{ } \vert a',b' \rangle  
	\end{equation}
Quando c'è degenerazione l'argomento  è il seguente: dopo la prima misura di $A$, che dà $a'$, il sistema precipita in qualche combinazione lineare
	\begin{equation}
		\sum _{i=1} ^n c_{a'} ^{(i)} \vert a', b^{(i)} \rangle ,
	\end{equation}
dove $n$ è il grado di degenerazione ed i ket $\vert a', b^{(i)} \rangle $ sono autoket simultanei di $A$ e $B$ corrispondenti tutti allo stesso autovalore di $A$. La seconda misura ($B$) può selezionare proprio uno dei termini di questa combinazione lineare, diciamo $\vert a', b^{(j)} \rangle $, ma la terza misura applicata a questo stato fornisce ancora $a'$. Pertanto indipendentemente dalla presenza o meno di degenerazione, \textbf{le misure di $A$ e $B$ non interferiscono. Per questa ragione le osservabili vengono dette compatibili.}\\

In generale si possono avere diverse osservabili mutuamente compatibili (ossia più di due), cioè:
	\begin{equation}
		\tcboxmath[sharp corners=downhill ,colback=white,colframe=red!75!black]{
			\left[A,B\right]=\left[B,C\right]=\left[A,C\right]=\dots =0.
			}
	\end{equation}
\textbf{Supponiamo allora di aver trovato un insieme massimale di osservabili che commutano. In questo caso gli autovalori dei singoli operatori $A$, $B$, $C$,$\dots$ possono avere degenerazione, ma se specifichiamo una combinazione ($a'$, $b'$, $c'$, $\dots$) allora il corrispondente autoket simultaneo di $A$, $B$, $C$ risulta univocamente determinato. In altri termini, uno stato di un sistema risulta completamente determinato dall'assegnazione di un insieme di numeri quantici di numero pari al numero massimo di osservabili mutuamente compatibili esistente per il sistema.}
\section[Osservabili incompatibili e relazione di indeterminazione]{Osservabili incompatibili e relazione di\\ indeterminazione}
Le osservabili incompatibili non ammettono un insieme completo di autostati in comune. È bene osservare, tuttavia, che possono esistere autostati simultanei di osservabili non compatibili. Tali autostati però non costituiscono un insieme completo. (Ad esempio l'autostato di $L^2$ ed $L_z$ corrispondente ad $l=n=0$ è anche autostato di $L_x$ ed $L_y$, sebbene $L_x$, $L_y$ d $L_z$ non commutino tra loro).\\

Per quanto detto, in generale, \textbf{grandezze fisiche associate a due operatori non commutanti non possono essere determinate simultaneamente.} Se si effettuano misure successive di due osservabili $A$ e $B$ incompatibili, allora la seconda misura ($B$) comporta una perdita di informazione circa lo stato del sistema a seguito della prima misura $A$.\\

Le affermazioni precedenti trovano una loro espressione quantitativa nella cosiddetta \textbf{relazione di indeterminazione}, secondo cui, per ogni stato, vale la seguente disuguaglianza:
	\begin{equation}
		\label{eq:cap4_2}
		\tcboxmath[enhanced,sharp corners=downhill ,colback=yellow!50!white,colframe=red!75!black, borderline={2pt}{-3pt}{red!50!black}]{
			\langle \left( \Delta A \right) ^2 \rangle \langle \left( \Delta B \right) ^2 \rangle \geq\frac{1}{4}\vert \langle \left[A,B \right] \rangle \vert ^2.
			}
	\end{equation}
In altri termini: \textbf{le dispersioni} (o scarto quadratico medio) \textbf{di due osservabili non commutanti non possono risultare (in generale) simultaneamente nulle.}\\

La relazione di indeterminazione non è altro che la forma generale della famosa relazione di indeterminazione di Heisenberg:
	\begin{equation}
		\tcboxmath[sharp corners=downhill ,colback=white,colframe=red!75!black]{
		\Delta x \cdot \Delta p \geq \frac{\hbar}{2},
		}
	\end{equation}
che costituisce l'espressione matematica del principio di indeterminazione.\\
Per dimostrare la relazione di indeterminazione, consideriamo uno stato $\vert \varphi \rangle$ della forma:
	\begin{equation}
		\vert \varphi \rangle = R\vert \alpha \rangle +i \lambda S \vert \alpha \rangle ,
	\end{equation}
dove $R$ ed $S$ sono due operatori hermitiani e $\lambda $ una costante reale. L'ampiezza $\langle \varphi \vert \varphi \rangle$ è, per definizione, una quantità reale positiva o nulla. Questo implica:
	\begin{eqnarray}
		\langle \varphi \vert \varphi \rangle &=&\left( \langle \alpha \vert R -i \lambda \langle \alpha \vert S \right) \left( R\vert \alpha \rangle +i \lambda S \vert \alpha \rangle  \right) = \nonumber \\
		&=&\langle \alpha \vert R^2 \vert \alpha \rangle + i\lambda \langle \alpha \vert RS \vert \alpha \rangle - i\lambda \langle \alpha \vert SR \vert \alpha \rangle + \lambda ^2 \langle \alpha \vert S^2 \vert \alpha \rangle \nonumber \\
		&=& \langle \alpha \vert R^2 \vert \alpha \rangle + \lambda \langle \alpha \vert i\left[ R,S\right] \vert \alpha \rangle + \lambda ^2 \langle \alpha \vert S^2 \vert \alpha \rangle \geq 0 
\end{eqnarray}
ossia
	\begin{equation}
		\label{eq:cap4_3}
		\langle  R^2  \rangle + \lambda \langle i\left[ R,S \right] \rangle + \lambda ^2 \langle S^2 \rangle \geq 0,
	\end{equation}
dove i valori medi possono essere calcolati su uno stato arbitrario. 
Per inciso: l'eq.~\eqref{eq:cap4_3} indica che il valore medio $\langle i\left[ R,S \right] \rangle$ deve essere un numero reale. In altri termini l'operatore $i\left[R,S	\right]$ è un operatore hermitiano, se $R$ ed $S$ sono operatori hermitiani, o equivalentemente, il commutatore è un operatore antihermitiano:
	\begin{equation}
		\left( \left[R,S \right] \right) ^{+} = - \left[R,S \right] \quad \textrm{se: } \begin{array}{c}
		R=R^{+},\\
		S=S^{+}.
		\end{array}
	\end{equation}
Questo risultato può essere anche dimostrato per verifica diretta.\\ 

La condizione \eqref{eq:cap4_3} deve risultare soddisfatta per qualunque valore della variabile reale $\lambda$. A tale scopo è necessario richiedere che il discriminante dell'equazione sia negativo o nullo, di modo che l'equazione con il segno di uguaglianza non ammetta soluzioni reali  al più ne ammetta una sola.

\begin{figure}[!htbp]
\begin{center}
\includegraphics[width=.9\textwidth]{immagini/cap_4/fig_4_1.png}
\end{center}
\end{figure}
\noindent Pertanto
\begin{equation}
\langle i \left[R,S \right] \rangle ^2 - 4\langle R^2 \rangle \langle S^2 \rangle \leq 0,
\end{equation}
ossia
\begin{equation}
\langle R^2 \rangle \langle S^2 \rangle \geq \frac{1}{4}\langle i \left[R,S \right] \rangle ^2.
\end{equation}
Scegliendo infine $R=\Delta A$ ed $S= \Delta B$ si ottiene la relazione di indeterminazione \eqref{eq:cap4_2}.
\section[Esempio: autovalori ed autovettori dello spin per particelle di spin 1/2 e relazioni di indeterminazione]{Esempio: autovalori ed autovettori dello spin per\\ particelle di spin 1/2 e relazioni di indeterminazione}
Consideriamo nuovamente un sistema costituito da particelle di spin 1/2. Abbiamo già visto che i corrispondenti stati di singola particella possono essere espressi come combinazione lineare di sue stati di base, che qui indichiamo con $\vert +z\rangle $ e $\vert - z \rangle$, che rappresentano gli stati in cui la proiezione dello spin della particella lungo l'asse $z$ vale rispettivamente $S_z = \pm \hbar /2$. È evidente che in questa base l'operatore $S_z$ è rappresentato dalla seguente matrice:
	\begin{equation}
		\tcboxmath[sharp corners=downhill ,colback=white,colframe=red!75!black]{
			S_z = \left( 
			\begin{array}{cc}
			\langle +z \vert S_z \vert +z \rangle &  \langle +z \vert S_z \vert -z \rangle \\
			\langle -z \vert S_z \vert +z \rangle & \langle -z \vert 	S_z \vert -z \rangle  
			\end{array}
			\right)= \frac{\hbar}{2}\begin{pmatrix}
			1 & 0 \\
			0 & -1
			\end{pmatrix}
			}
	\end{equation}
o, equivalentemente
	\begin{equation}
		\tcboxmath[sharp corners=downhill ,colback=white,colframe=red!75!black]{
			S_Z =\frac{\hbar}{2}\sigma _z,
			}
	\end{equation}
dove $\sigma _z$ è la matrice di Pauli. Utilizzando le proprietà generali dell'operatore di spin è possibile anche dimostrare che per gli analoghi operatori di $S_x$ ed $S_y$ valgano le corrispondenti rappresentazioni:
	\begin{equation}
		\tcboxmath[sharp corners=downhill ,colback=white,colframe=red!75!black]{
			S_x = \frac{\hbar}{2}\sigma _x= \frac{\hbar}{2}\begin{pmatrix}
			0 & 1 \\
			1 & 0
			\end{pmatrix},
			}
	\end{equation}
	\begin{equation}
		\tcboxmath[sharp corners=downhill ,colback=white,colframe=red!75!black]{
			S_y= \frac{\hbar}{2}\sigma _y = \frac{\hbar}{2} \begin{pmatrix}
			0 & -i \\
			i & 0
			\end{pmatrix}.
			}
	\end{equation}
Poiché le tre matrici di Pauli hanno determinante uguale a $-1$ e traccia nulla, i loro autovalori sono $+1$ o $-1$. Corrispondentemente, gli autovalori della proiezione dello spin lungo un qualunque asse sono $\pm \hbar/2$. È immediato calcolare i corrispondenti autovettori:
	\begin{equation}
		\label{eq:cap4_4}
		\tcboxmath[sharp corners=downhill ,colback=white,colframe=red!75!black]{
			\begin{aligned}
			&\displaystyle{S_x = \frac{\hbar}{2}, \quad \vert x \rangle = \frac{1}{\sqrt{2}}\vert +z \rangle + \frac{1}{\sqrt{2}}\vert -z \rangle}= \frac{1}{\sqrt{2}}
			\begin{pmatrix}
			1\\
			1
			\end{pmatrix}, \\
			&\displaystyle{S_x = -\frac{\hbar}{2}, \quad \vert -x \rangle = \frac{1}{\sqrt{2}}\vert +z \rangle - \frac{1}{\sqrt{2}}\vert -z \rangle}= \frac{1}{\sqrt{2}} \begin{pmatrix}
			1\\
			-1
			\end{pmatrix},
			\end{aligned}
			}
	\end{equation}
	\begin{equation}
		\label{eq:cap4_5}
		\tcboxmath[sharp corners=downhill ,colback=white,colframe=red!75!black]{
			\begin{aligned}
			&\displaystyle{S_y = \frac{\hbar}{2}, \quad \vert y \rangle = \frac{1}{\sqrt{2}}\vert +z \rangle + \frac{i}{\sqrt{2}}\vert -z \rangle}= \frac{1}{\sqrt{2}} 
			\begin{pmatrix}
			1\\
			i
			\end{pmatrix} ,\\
			&\displaystyle{S_y = -\frac{\hbar}{2}, \quad \vert -y \rangle = \frac{1}{\sqrt{2}}\vert +z \rangle - \frac{i}{\sqrt{2}}\vert -z \rangle}= \frac{1}{\sqrt{2}} 
			\begin{pmatrix}
			1\\
			-i
			\end{pmatrix},
			\end{aligned}
			}	
	\end{equation}
\begin{center}
\begin{tcolorbox}[toprule=3mm, width=.9\textwidth]
Calcoliamo esplicitamente l'autostato di $\vert +x\rangle$ di $S_x$ con autovalore $+\hbar /2$. Scriviamo questo stato come combinazione lineare generica di autostati di $S_z$:
	\begin{equation}
		\vert +x \rangle = c_1\vert +z \rangle+ c_2\vert -z \rangle \doteq 
		\begin{pmatrix}
		c_1 \\
		c_2
		\end{pmatrix}
	\end{equation}
Applicando l'equazione agli autovalori $S_x \vert+x\rangle =\hbar /2 \vert +x\rangle $ si ottiene:
	\begin{equation}
		\frac{\hbar}{2}
		\begin{pmatrix}
		0 & 1 \\
		1 & 0
		\end{pmatrix}
		\begin{pmatrix}
		c_1 \\
		c_2
		\end{pmatrix} =
		\frac{h}{2}
		\begin{pmatrix}
		c_1 \\
		c_2
		\end{pmatrix}
		\quad \Rightarrow \quad
		\begin{cases}
		c_2=c_1\\
		c_1=c_2
		\end{cases}
	\end{equation}
La condizione di normalizzazione dello stato implica:
	\begin{equation}
		\langle +x \vert +x \rangle = \left( c_1 ^*\ c_2 ^*\right)\begin{pmatrix}
		c_1\\c_2
		\end{pmatrix} = \vert c_1 \vert ^2 +\vert c_2 \vert ^2 =1, 
	\end{equation}
che, unitamente alla condizione $c_1=c_2$ implica $\vert c_1 \vert =1/\sqrt{2}$. Possiamo allora scrivere:
	\begin{equation}
		\tcboxmath[sharp corners=downhill ,colback=white,colframe=red!75!black]{c_1 =c_2 = \frac{e^{i\delta}}{\sqrt{2}}}\ \Rightarrow \ \tcboxmath[sharp corners=downhill ,colback=white,colframe=red!75!black]{\vert +x \rangle \frac{e^{i\delta}}{\sqrt{2}} \left(\vert +z \rangle + \vert -z \rangle \right).}
	\end{equation}
Il fattore di fase moltiplicativo, nel vettore di stato, è completamente arbitrario, ossia non interviene mai nelle quantità fisiche che sono la probabilità ($\vert \langle \alpha \vert +x \rangle \vert ^2$) ed i valori di aspettazione ($\langle +x \vert A \vert +x \rangle $). Una scelta possibile è 	quella effettuata effettuata in eq. \eqref{eq:cap4_4}, ossia $e^{i\delta}=1$.
\end{tcolorbox}

\end{center}
È semplice verificare i risultati (\ref{eq:cap4_4}) e (\ref{eq:cap4_5}). Ad esempio si ha:
	\begin{equation}
		S_y \vert - y \rangle = \frac{\hbar}{2\sqrt{2}}\begin{pmatrix}
		0 & -i\\
		i & 0
		\end{pmatrix}
		\begin{pmatrix}
		1\\-i
		\end{pmatrix} = \frac{\hbar}{2\sqrt{2}} 
		\begin{pmatrix}
		-1 \\ i
		\end{pmatrix} = -\frac{\hbar}{2} \vert -y \rangle .
	\end{equation}
Gli operatori corrispondenti alle proiezioni dello spin lungo i tre assi non commutano tra loro. Per esempio, un calcolo esplicito del commutatore di $S_x$ ed $S_y$ conduce a:
	\begin{eqnarray}
		\left[ S_x , S_y \right] & = & S_x S_y-S_y S_x= 
		\frac{\hbar ^2}{4} \left[\begin{pmatrix}
		0 & 1\\
		1 & 0
		\end{pmatrix} \begin{pmatrix}
		0 & -i\\
		i & 0
		\end{pmatrix}- \begin{pmatrix}
		0 & -i\\
		i & 0
		\end{pmatrix}\begin{pmatrix}
		0 & 1\\
		1 & 0
		\end{pmatrix}
		\right] = \nonumber \\
		& = &\frac{\hbar ^2}{4} \left[\begin{pmatrix}
		i & 0\\
		0 & -i
		\end{pmatrix} - \begin{pmatrix}
		-i & 0\\
		0 & i
		\end{pmatrix}\right] = 	
		\frac{\hbar ^2}{2}\begin{pmatrix}
		i & 0\\
		0 & -i
		\end{pmatrix},
	\end{eqnarray}
ossia
	\begin{equation}
		\tcboxmath[sharp corners=downhill ,colback=white,colframe=red!75!black]{
			\left[ S_x, S_y \right] = i \hbar S_z.
			}
	\end{equation}
In generale, è possibile verificare che le tre relazioni di commutazione indipendenti possono essere scritte nella forma compatta
	\begin{equation}
		\tcboxmath[sharp corners=downhill ,colback=white,colframe=red!75!black]{
			\left[ S_i , S_j\right] = i\ \varepsilon_{ijk}\ \hbar S_k,
			}
	\end{equation}
dove gli indici 1,2,3 sono associati rispettivamente alle componenti $x$, $y$ e $z$.\\
Le relazioni di commutazione qui derivate implicano che \textbf{le componenti dello spin lungo assi distinti non possono essere misurate simultaneamente, ossia che queste tre grandezze fisiche sono osservabili incompatibili}. Corrispondentemente, deve risultare soddisfatta una \textbf{relazione di indeterminazione}. Nel caso della misura di $S_x$ ed $S_y$, per esempio, questa relazione assume la forma
	\begin{equation}
		\label{eq:cap4_6}
		\tcboxmath[enhanced,sharp corners=downhill ,colback=yellow!50!white,colframe=red!75!black, borderline={2pt}{-3pt}{red!50!black}]{
			\langle (\Delta S_x) ^2 \rangle \langle (\Delta S_y) ^2 \rangle \geq \frac{1}{4} \vert \langle \left[ S_x , S_y \right] \rangle \vert ^2 = \frac{\hbar}{4} \langle S_z \rangle ^2 .
			}
	\end{equation}
Verifichiamo la validità di questa relazione per una particella che si trovi ad esempio nello stato $\vert +z\rangle$. In generale, lo scarto quadratico medio di una grandezza $A$ si esprime come
	\begin{equation}
		\langle (\Delta A ) ^2 \rangle = \langle (A- \langle A \rangle ) ^2 \rangle = \langle (A^2-2A\langle A \rangle + \langle A \rangle ^2) \rangle = \langle A^2\rangle - \langle A \rangle ^2,
	\end{equation}
e dipende dunque dai valori medi di $A$ e $A^2$. Nel caso di $(\Delta S_x) ^2$, allora, calcoliamo in primo luogo l'operatore $S_x ^2$:
	\begin{equation}
		S_x ^2 = \frac{\hbar ^2}{4} \sigma _x ^2=\frac{\hbar ^2}{4} \begin{pmatrix}
		0 & 1 \\
		1 & 0
		\end{pmatrix} \begin{pmatrix}
		0 & 1 \\
		1 & 0
		\end{pmatrix} = \frac{\hbar ^2}{4}\begin{pmatrix}
		1 & 0 \\
		0 & 1
		\end{pmatrix}.
	\end{equation}
Osserviamo che le matrici di Pauli soddisfano:
	\begin{equation}
		\tcboxmath[sharp corners=downhill ,colback=white,colframe=red!75!black]{
		\sigma _x ^2 = \sigma _y ^2 = \sigma _z ^2 =1.
		}
	\end{equation}
Poiché lo stato $\vert +z \rangle$, su cui vogliamo verifica la relazione di indeterminazione, è uno degli stati di base, i valori medi di $S_x$ ed $S_x ^2$ su questo stato si ottengono direttamente dai corrispondenti elementi di matrice (elementi 1,1):
	\begin{eqnarray}
		& &\displaystyle{\langle +z \vert S_x \vert +z \rangle = (S_x)_{11} = \frac{\hbar}{2}\cdot 0 =0}  \\
		& &\displaystyle{\langle +z \vert S_x ^2\vert +z \rangle = (S_x ^2)_{11} = \frac{\hbar ^2}{4}\cdot 1 =\frac{\hbar ^2}{4}} 
	\end{eqnarray}
Ne segue allora
	\begin{equation}
		\label{eq:cap4_7}
		\tcboxmath[sharp corners=downhill ,colback=white,colframe=red!75!black]{
			\langle +z \vert (\Delta S_x)^2\vert +z \rangle = \frac{\hbar ^2}{4}.
			}
	\end{equation}
Analogamente si calcola $\langle (\Delta S_y)^2 \rangle $ ($S_y ^2= \frac{\hbar ^2}{4}\cdot I$):
	\begin{eqnarray}
	& &\displaystyle{\langle +z \vert S_y \vert +z \rangle = (S_y)_{11} = \frac{\hbar}{2}\cdot 0 =0}  \\
	& &\displaystyle{\langle +z \vert S_y ^2\vert +z \rangle = (S_y ^2)_{11} = \frac{\hbar ^2}{4}\cdot 1 =\frac{\hbar ^2}{4}} \nonumber
	\end{eqnarray}
da cui
	\begin{equation}
		\label{eq:cap4_8}
		\tcboxmath[sharp corners=downhill ,colback=white,colframe=red!75!black]{
			\langle +z \vert (\Delta S_y)^2\vert +z \rangle = \frac{\hbar ^2}{4}.
			}
	\end{equation}
Il secondo membro della relazione di indeterminazione è in questo caso, a parte il fattore $\frac{\hbar ^2}{4}$, semplicemente il valore medio di $S_z$ al quadrato:
	\begin{equation}
		\label{eq:cap4_9}
		\tcboxmath[sharp corners=downhill ,colback=white,colframe=red!75!black]{
			\left( \langle +z \vert S_z \vert +z \rangle \right) ^2 = \frac{\hbar ^2}{4}.
			}
	\end{equation}
Mettendo insieme i risultati \eqref{eq:cap4_7}, \eqref{eq:cap4_8} e \eqref{eq:cap4_9} possiamo allora verificare esplicitamente la relazione di indeterminazione \eqref{eq:cap4_6}:
	\begin{equation}
		\langle  (\Delta S_x)^2 \rangle \langle(\Delta S_y)^2 \rangle = \left(\frac{\hbar ^2}{4} \right) ^2 \geq \frac{\hbar ^2}{4} \langle S_z  \rangle ^2 = \left(\frac{\hbar ^2}{4} \right) ^2.
	\end{equation}
La relazione di indeterminazione risulta quindi soddisfatta, in questo caso, con il segno di uguaglianza. In altri termini, sullo stato $\vert +z \rangle $ il prodotto delle indeterminazioni su $S_x$ e $S_y$ assume il valore minimo possibile compatibilmente con la relazione di indeterminazione.\\ \\

\textbf{Appendice:}\\
Vediamo come, considerando ad esempio lo stato $\vert + z \rangle $, i valori medi $\langle S_x \rangle $ e $\langle (\Delta S_x)^2 \rangle $ possano essere calcolati anche utilizzando la definizione di valore medio:
	\begin{equation}
		\langle A \rangle = \sum _i a_i p_i,
	\end{equation}
dove $A_I$ sono i possibili risultati della misura di $A$ e $p_i$ le corrispondenti probabilità.\\
Utilizzando le eq.\eqref{eq:cap4_7} è semplice trovare che lo stato $\vert +z \rangle$ si esprime come combinazione lineare degli autostati si $S_x$ nella forma
	\begin{equation}
		\tcboxmath[sharp corners=downhill ,colback=white,colframe=red!75!black]{
			\vert +z \rangle = \frac{1}{\sqrt{2}} \vert +x \rangle + \frac{1}{\sqrt{2}} \vert -x \rangle .
			}
	\end{equation}
I possibili risultati di una misura di $S_x$ e le corrispondenti probabilità su questo stato sono allora
	\begin{equation}
		\begin{cases}
		\displaystyle{S_x ^{(1)} = +\frac{\hbar}{2}, \ p_1 = \frac{1}{2}}, \\[.5cm]
		\displaystyle{S_x ^{(2)} = -\frac{\hbar}{2}, \ p_2 = \frac{1}{2}}.
		\end{cases}
	\end{equation}
Per i valori medi si $S_x$ si ottiene allora:
	\begin{equation}
		\langle S_x \rangle = \sum _i S_x ^{(i)} p_i= +\frac{\hbar}{2}\cdot\frac{1}{2}-\frac{\hbar}{2}\cdot\frac{1}{2}=0,
	\end{equation}
mentre per lo scarto quadratico medio
	\begin{equation}
		\langle (\Delta S_x)^2 \rangle = \sum _i ( S_x ^{(i)}- \underbrace{\langle S_x \rangle}_{=0}) ^2\ p_i= +\frac{\hbar ^2}{4}\cdot\frac{1}{2}+\frac{\hbar ^2}{4}\cdot\frac{1}{2}= \frac{\hbar ^2}{4},
	\end{equation}
in accordo con i precedenti risultati ottenuti.
%DEFINITIVO
\input{Capitolo_5_(Matteo)/capitolo5.tex}%DEFINITIVO
\input{Capitolo_6_(Ilaria-Matteo-Valerio)/capitolo6.tex}%DEFINITIVO 
\input{Capitolo_7_(Giulio)/capitolo7.tex} %DEFINITIVO
\input{Capitolo_8_(Valerio)/capitolo8.tex} %DEFINITIVO
\input{Capitolo_9_(Matteo)/capitolo9.tex} %DEFINITIVO
\chapter{Problemi unidimensionali}
\section{Proprietà generali dell'equazione di Schr\"{o}dinger}
\textbf{La funzione d'onda deve essere monotona e continua in tutto lo spazio}. \textbf{Le derivate della f.d.o. sono continue ovunque, anche su superfici di discontinuità del potenziale, eccetto il caso in cui $V$ diventa infinito al di fuori di tali superfici}.\\

\textbf{Dimostrazione}: utilizzando l'equazione di Schr\"{o}dinger, possiamo calcolare l'eventuale discontinuità della derivata prima in un generico punto $x_0$. Si ha:
	\begin{align}
		& \lim _{\varepsilon \rightarrow 0} \left[ \left(\frac{\partial \psi}{\partial x}\right) _{x_0+\varepsilon}-\left(\frac{\partial \psi}{\partial x}\right) _{x_0-\varepsilon}\right]= \lim _{\varepsilon \rightarrow 0} \int _{x_0+\varepsilon} ^{x_0+\varepsilon} dx\, \frac{d}{dx}\left(\frac{\partial\psi}{\partial x}\right)=\nonumber\\
		&= \lim _{\varepsilon \rightarrow 0} \left(-\frac{2m}{\hbar ^2}\right)\int _{x_0+\varepsilon} ^{x_0+\varepsilon} dx\, \left(E-V\left( x\right)\right)\psi \left( x \right) .
	\end{align}
È evidente che, se il potenziale $V(x)$ non diverge nel punto $x_0$, l'integrale al membro destro della precedente equazione non può che tendere a zero nel limite in cui la larghezza dell'intervallo di integrazione tende a zero. La derivata della funzione d'onda risulta allora continua nel punto $x_0$ (c.v.d.).\\

\textbf{Una particella non può penetrare un generale in una regione dello spazio dove $V =\infty$, cioè dappertutto in questa regione la f.d.o. è nulla}. La continuità delle f.d.o. esige che essa si annulli sul contorno di questa regione, quanto alle derivate della f.d.o. esse subiscono allora, in generale, un salto.\\

\textbf{Se il potenziale $V$ è ovunque finito, anche la f.d.o. deve essere finita in tutto lo spazio} (questa condizione deve essere ugualmente soddisfatta nei casi in cui $V$ diventa infinito in un punto ma non non troppo rapidamente: come $1/r^s$ con $s<2$).\\

Tutti gli autovalori dell'energia sono maggiori del minimo valore assoluto del potenziale $V$, cioè:
	\begin{equation}
		\tcboxmath[sharp corners=downhill, colback=white, colframe=black]{
			E_n >V_{min} 
			}\quad \forall n.
	\end{equation}
	
\textbf{Dimostrazione}:
	\begin{equation}
		E_n = \langle n \vert H \vert n \rangle =  \langle n \vert \frac{p ^2}{2m} \vert n \rangle + \langle n \vert V(x) \vert n \rangle .
	\end{equation}
I due  valori medi soddisfano poi:
	\begin{align}
		&\bullet \langle n \vert \frac{p ^2}{2m} \vert n \rangle =\frac{1}{2m} \langle n \vert p\cdot p\vert n \rangle \equiv \frac{1}{2m} \langle \phi \vert  \phi \rangle \ge 0 , \\[0.3cm]
		&\bullet \langle n \vert V(x) \vert n \rangle = \int _{-\infty} ^{\infty} dx' V(x') \vert \psi (x') \vert ^2 \geq V_{min} \int _{-\infty} ^{\infty} dx'  \vert \psi (x') \vert ^2 = V_{min} , 
	\end{align}
da cui segue $E_n \geq V_{min}$ (c.v.d.).\\

\textbf{Allo stesso tempo, una particella in Meccanica Quantistica può venire anche a trovarsi nelle regioni dello spazio in cui $E<V$}. Anche se la probabilità $|\psi|^2$ di presenza della particella tende rapidamente a zero all'interno di tale regione, essa è però differente da zero a tutte le distanze finite.
\section{Particella su un segmento (buca di potenziale infinita)}
Consideriamo una particella nel campo di potenziale:\\

\begin{minipage}{.55\textwidth}
	\begin{align}
		&\tcboxmath[enhanced, sharp corners=downhill, colframe=black, colback=white, borderline={2pt}{-3pt}{black}]{
			V(x)= 
			\begin{cases}
			\infty \quad x<0,\\
			0 \quad 0>x>a, \\
			\infty \quad x>a.
			\end{cases} }
			\\
			&\quad \textrm{(buca di potenziale infinita)} \nonumber
			\end{align}	
			\end{minipage}
\hspace{.5cm}
\begin{minipage}{.35\textwidth}
\includegraphics[width=\textwidth]{immagini/cap_10/fig_10_1.png}
\end{minipage}\\

Il moto della particella può avvenire soltanto nel segmento $\left( 0; a \right)$, giacché non è posisbile che questa penetri che questa penetri in una regione in cui $V=\infty$. Allora:
	\begin{equation}
		\tcboxmath[sharp corners=downhill, colback=white, colframe=black]{		
			\psi(x)= 
			\begin{cases}
			0\quad x<0,\\
			0 \quad x>a.
			\end{cases}
			}
	\end{equation}
All'interno della buca, dove $V(x)=0$, l'eq. di Schr\"{o}dinger ha la forma:
	\begin{equation}
		\tcboxmath[sharp corners=downhill, colback=white, colframe=black]{
			-\frac{\hbar ^2}{2m}\frac{d^2 \psi}{dx^2}= E\psi
			} \quad \Longrightarrow \quad 
		\tcboxmath[sharp corners=downhill, colback=white, colframe=black]{
			\frac{d^2 \psi}{dx^2}=-\frac{2mE}{\hbar^2}\psi,
			}
	\end{equation}
che deve essere risolta con le condizioni al contorno:
	\begin{equation}
		\tcboxmath[sharp corners=downhill, colback=white, colframe=black]{
			\psi(0)=\psi(a)=0,
			 }
	\end{equation}
richieste dalla continuità della f.d.o. in $x=0$ e $x=a$.\\

È facile verificare che \textbf{l'equazione non ha soluzioni corrispondenti ad $\textbf{E<0}$}, in accordo con la proprietà generale secondo cui gli autovalori dell'energia sono sempre maggiori del minimo del potenziale, in questo caso $V_{min}=0$. Infatti, per $E<0$ si può porre:
	\begin{equation}
		\lambda ^2 = -\frac{2mE}{\hbar ^2},
	\end{equation}
e l'equazione di Schr\"{o}dinger diventa:
	\begin{equation}
		\frac{d^2 \psi}{dx^2}=\lambda ^2\psi.
	\end{equation}
Le soluzioni di questa equazione sono della forma:
	\begin{equation}
		\psi(x)= Ae^{\lambda x}+ Be^{-\lambda x}.
	\end{equation}
la condizione in $\psi (0)=0$ implica:
	\begin{equation}
		\psi(0)= A+ B=0 \quad \Rightarrow \quad B=-A,
	\end{equation}
ossia
	\begin{equation}
		\psi(x)= 2A\ \sinh{\left( \lambda x \right)},
	\end{equation}
e la condizione $\psi(a)=0$ non risulta mai soddisfatta.\\

Considerando allora $\textbf{E>0}$ e ponendo
	\begin{equation}
		\tcboxmath[sharp corners=downhill, colback=white, colframe=black]{
			k ^2 = \frac{2mE}{\hbar ^2},
			}
	\end{equation}
otteniamo per l'equazione di Schr\"{o}dinger l'espressione:
	\begin{equation}
		\tcboxmath[sharp corners=downhill, colback=white, colframe=black]{
			\frac{d^2 \psi}{dx^2}=-k ^2\psi.
			}
	\end{equation}
La soluzione generale di questa equazione è
	\begin{equation}
		\psi(x)= Ae^{ik x}+ Be^{-ik x}.
	\end{equation}
la condizione $\psi (0)=0$ richiede:
	\begin{equation}
		\psi(0)= A+ B=0 \quad \Rightarrow \quad B=-A,
	\end{equation}
ossia
	\begin{equation}
		\psi(x)= 2iA\ \sin{\left( k x \right)},
	\end{equation}
la condizione in $\psi (a)=0$ implica:
	\begin{equation}
		\tcboxmath[sharp corners=downhill, colback=white, colframe=black]{
			k_n a= n\pi
			} \qquad n=1,2,3\dots			
	\end{equation}
Questa è la condizione che determina gli \textbf{autovalori discreti dell'energia}:
	\begin{equation}
		\tcboxmath[enhanced, sharp corners=downhill, colframe=black, colback=white, borderline={2pt}{-3pt}{black}]{
			E_n= \frac{\hbar ^2 {k_n}^2}{2m}=\frac{\hbar ^2 \pi n^2}{2ma^2}
			} \quad n=1,2,3,\dots
	\end{equation}
Le corrispondenti autofunzioni sono:
	\begin{equation}
		\psi _n = 2iA_n\ \sin{\left( k_n x\right)} \Rightarrow 2A_n\ \sin{\left( k_n x \right)},
	\end{equation}
dove abbiamo tolto il fattore di fase irrilevante $i$.\\
La costante complessa $A_n$ può essere determinata dalla \textbf{condizione di normalizzazione}:
	\begin{equation}
		\tcboxmath[sharp corners=downhill, colback=white, colframe=black]{
			\int _{-\infty} ^{\infty} |\psi(x)| ^2 dx =1.
			}
	\end{equation}
Si trova:
	\begin{align}
		\int _{-\infty} ^{\infty} |\psi(x)| ^2 dx &= \int _{0} ^{a} 4|A_n| ^2 \sin ^2 {\left( k_n x \right)}\ dx= 4|A_n| ^2 \int _{0} ^{a} \left( \frac{e^{ik_n x}- e^{-ik_nx}}{2i}\right) ^2\ dx = \nonumber \\ 
		&= -|A_n| ^2 \int _{0} ^{a} \left( e^{2ik_n x}+ e^{-2ik_nx}-2\right)\  dx = \nonumber \\
		&= |A_n| ^2 \left[ 2a- 2 \int _{0} ^{a} \cos \left(\frac{2n \pi}{a} x\right)\   dx \right] = \nonumber \\
		&= |A_n| ^2 \left[ 2a- \frac{a}{n\pi} \int _{0} ^{2n\pi} \cos \alpha\   d\alpha \right] = 2a|A_n|^2= 1.
	\end{align}
Scegliendo la fase arbitraria della f.d.o.~in modo tale che $A_n$ risulti reale e positivo otteniamo:
	\begin{equation}
		A_n = \frac{1}{\sqrt{2a}},
	\end{equation}
ossia, in definitiva:
	\begin{equation}
		\tcboxmath[enhanced, sharp corners=downhill, colframe=black, colback=white, borderline={2pt}{-3pt}{black}]{
			\psi _n (x)= \sqrt{\frac{2}{a}} \sin{\left( k_n x\right)} = \sqrt{\frac{2}{a}} \sin{\left( \frac{n \pi x}{a}\right)};
			} \qquad 0<x<a.
\end{equation}\\

È semplice verificare che \textbf{le autofunzioni sono ortogonali tra loro}:
	\begin{equation}
		\tcboxmath[sharp corners=downhill, colback=white, colframe=black]{
			\int _{-\infty} ^{\infty} dx\ \psi_n ^* (x) \psi _m (x) = \delta _{mn},
			}
	\end{equation}
come previsto in generale per le autofunzioni di un operatore Hamiltoniano.\\ 

\textbf{Una qualunque f.d.o.} $\mathbf{\psi (x)}$\textbf{, soluzione dell'equazione di Schr\"{o}dinger può essere espressa come combinazione lineare delle autofunzioni dell' Hamiltoniano}:
	\begin{equation}
		\tcboxmath[sharp corners=downhill, colback=white, colframe=black]{
			\psi (x) = \sum _{n=1} ^{\infty} c_n \psi _n (x) = \sum _{n=1} ^{\infty} c_n \sqrt{\frac{2}{a}} \sin {\left( k_n x \right)}.
			}
	\end{equation}\\
	
L'ortonormalità delle autofunzioni può essere utilizzata per esprimere \textbf{i coefficienti} $\mathbf{c_n}$ \textbf{dello sviluppo come}:
	\begin{equation}
		\tcboxmath[sharp corners=downhill, colback=white, colframe=black]{
			c_n = \int _{-\infty} ^{+\infty} dx\ \psi _n ^* (x) \psi(x).
			}
	\end{equation}
Si ha infatti:
\begin{equation}
\int _{-\infty} ^{+\infty} dx\ \psi _n ^* (x) \psi(x)= \sum _m c_m \int _{-\infty} ^{+\infty} dx\ \psi _n ^* (x) \psi _m(x)= c_n.
\end{equation}
I moduli quadri $|c_n|^2$ rappresentano la probabilità che una misura dell'energia fornisca come risultato il valore $E=E_n$.\\

Il valore medio dell'energia nello stato descritto dalla f.d.o. $\psi$ è:
	\begin{equation}
		\tcboxmath[sharp corners=downhill, colback=white, colframe=black]{
		\begin{aligned}
			\langle H \rangle &= \int _{-\infty} ^{+\infty} dx\ \psi ^*(x) H \psi(x) =\sum _n c_n \int _{-\infty} ^{+\infty} dx\ \psi ^*(x) H \psi _n(x) = \\
			&= \sum _n c_n E_n \int _{-\infty} ^{+\infty} dx\ \psi ^*(x) \psi _n(x) =\sum _n |c_n|^2 E_n .
		\end{aligned}
		}
	\end{equation}\\
	
Se $\psi (x)$ rappresenta la f.d.o.~della particella al tempo $t=0$, la f.d.o.~vad un tempo $t$ successivo è data da:
	\begin{equation}
		\tcboxmath[sharp corners=downhill, colback=white, colframe=black]{
			\psi (x,t) =\sum _n c_n\ e^{-\frac{i}{\hbar}E_n t}\ \psi _n (x).
			}
	\end{equation}\\
	
I risultati ottenuti consentono di derivare semplicemente gli autovalori dell'energia e le autofunzioni per una \textbf{particella in una scatola}, ossia in una buca di potenziale tridimensionale descritta dal campo:
	\begin{equation}
		\tcboxmath[enhanced, sharp corners=downhill, colframe=black, colback=white, borderline={2pt}{-3pt}{black}]{
			V(x,y,z)= 
			\begin{cases}
			0 \quad \textrm{per}\ 0\leq x \leq a,\ 0\leq y \leq b,\ 0 \leq z \leq c;\\
			\infty \quad \textrm{fuori.}
			\end{cases}
			}
	\end{equation}
L'equazione di Schr\"{o}dinger per la particella all'interno della buca di potenziale si scrive:
	\begin{equation}
		\tcboxmath[sharp corners=downhill, colback=white, colframe=black]{
			-\frac{\hbar ^2}{2m} \left( \frac{\partial ^2 \psi}{\partial x^2}+\frac{\partial ^2 \psi}{\partial y^2}+\frac{\partial ^2 \psi}{\partial z^2}\right) = E \psi.
			}
	\end{equation}
L' \textbf{Hamiltoniano} si separa nella somma di tre termini, ciascuno dei quali è l' Hamiltoniano di una particella libera unidimensionale:
	\begin{equation}
		\tcboxmath[sharp corners=downhill, colback=white, colframe=black]{
			H= H_x+H_y+H_z
			} \qquad 
		\tcboxmath[sharp corners=downhill, colback=white, colframe=black]{
			H_x= -\frac{\hbar ^2}{2m}  \frac{\partial ^2 \psi}{\partial x^2}, \dots
			}
	\end{equation}
Le \textbf{autofunzioni} si scompongono nel prodotto di tre autofunzioni per la buca di potenziale unidimensionale:
	\begin{equation}
		\tcboxmath[sharp corners=downhill, colback=white, colframe=black]{
			\begin{aligned}
			\psi _{n_x n_y n_z} (x,y,z) &= \psi _{n_x}(x)\psi _{n_y}(y)\psi _{n_z}(z)= \\
			&=\sqrt{\frac{8}{abc}}\sin \frac{\pi n_x x}{a}\ \sin \frac{\pi n_y y}{b}\ \sin \frac{\pi n_z z}{c}.
			\end{aligned}
			}	
	\end{equation}
Gli \textbf{autovalori} dell'energia sono dati dalla somma dei corrispondenti autovalori del caso unidimensionale. Scrivendo esplicitamente l'equazione agli autovalori si trova infatti:
	\begin{eqnarray}
		H\psi  &=& \left(H_x+H_y+H_z\right) \psi _{n_x}(x)\psi _{n_y}(y)\psi _{n_z}(z)= \nonumber \\
		&=& \left( E_{n_x}+E_{n_y}+E_{n_z}\right) \psi _{n_x}(x)\psi _{n_y}(y)\psi _{n_z}(z)=  E\psi,
	\end{eqnarray}
ossia, esplicitamente:
	\begin{equation}
		\tcboxmath[sharp corners=downhill, colback=white, colframe=black]{
			E_{n_x n_y n_z}=\frac{\hbar ^2 \pi^2}{2m} \left(\frac{{n_x}^2}{a^2}+\frac{{n_y}^2}{b^2}+\frac{{n_z}^2}{c^2}\right)
			} \qquad
			n_x, n_y, n_z, = 1,2,3... 
	\end{equation}
\section{Buca di potenziale: stati legati}
Consideriamo il moto di una particella nel campo di potenziale:\\

\noindent
\begin{minipage}{.65\textwidth}
	\begin{align}
		&\tcboxmath[enhanced, sharp corners=downhill, colframe=black, colback=white, borderline={2pt}{-3pt}{black}]{
			\begin{cases}
			V(x)=-V_0 \quad -a< x < a;\\
			V(x)= 0 \quad x<-a,\ x> a.
			\end{cases}
			}\\
		&\qquad \ \textrm{(buca di potenziale finita)} \nonumber
	\end{align}	
\end{minipage}
\hspace{.2cm}
\begin{minipage}{.3\textwidth}
\includegraphics[width=5cm]{immagini/cap_10/fig_10_2.png}
\end{minipage}\\


Discutiamo in particolare le soluzioni dell'equazione di Schr\"{o}dinger corrispondenti agli \textbf{stati legati} della particella, ossia a valori negativi dell'energia, $\mathbf{E<0}$. Nelle regioni in cui il potenziale è rispettivamente nullo o pari a $-V_0$ l'equazione di Schr\"{o}dinger si scrive:
	\begin{equation}
		\begin{cases}
		\displaystyle{-\frac{\hbar ^2}{2m}\frac{d^2 \psi}{d x^2}=-|E| \psi \qquad  x<-a,\ x> a;}\\
		\\
		\displaystyle{-\frac{\hbar ^2}{2m}\frac{d^2 \psi}{d x^2}= \left( V_0-|E| \right)\psi \quad  -a< x < a.}
		\end{cases} 
	\end{equation}
Ponendo:
	\begin{equation}
		\tcboxmath[sharp corners=downhill, colback=white, colframe=black]{
			\lambda ^2 = \frac{2m}{\hbar ^2}|E|, 
			}\qquad
		\tcboxmath[sharp corners=downhill, colback=white, colframe=black]{
			 q ^2 = \frac{2m}{\hbar ^2}\left( V_0 -|E|\right),
			}
	\end{equation}
dove $q$ e $\lambda$ sono parametri reali, possiamo scrivere l'equazione di Schr\"{o}dinger nella forma:
	\begin{equation}
		\tcboxmath[sharp corners=downhill, colback=white, colframe=black]{
			\begin{cases}
			\displaystyle{\frac{d^2 \psi}{d x^2}=\lambda ^2 \psi \qquad  x<-a,\ x> a;}\\
			\\
			\displaystyle{\frac{d^2 \psi}{d x^2}= -q^2 \psi \quad  -a< x < a.}
			\end{cases} 
			}
	\end{equation}\\
	
Il potenziale cui è soggetta la particella è simmetrico rispetto allo scambio $x \rightarrow -x$:
	\begin{equation}
		\tcboxmath[sharp corners=downhill, colback=white, colframe=black]{
		V(x)= V(-x).
		}
	\end{equation}
L'operatore Hamiltoniano commuta allora con l'operatore di parità:
	\begin{equation}
		\tcboxmath[sharp corners=downhill, colback=white, colframe=black]{
		[H;P]=0,
		}
	\end{equation}
e \textbf{le autofunzioni di $H$ possono essere scelte simultaneamente come autofunzioni dell'operatore di parità}.\\

Nella regione al di fuori della buca di potenziale le soluzioni all'equazione di Schr\"{o}dinger sono della forma $e^{\pm \lambda x}$. La richiesta che queste autofunzioni siano limitate all'infinito implica allora:
	\begin{equation}
		\begin{cases}
		\psi (x) = Ae^{\lambda x} \quad x<-a;\\
		\psi (x) = A'e^{\lambda x} \quad x>a.\end{cases} 
	\end{equation}\\

All'interno della buca di potenziale le autofunzioni sono della forma $e^{\pm iq x}$, le cui combinazioni pari e dispari sono rispettivamente $\cos (qx)$ e $\sin (qx)$.\\

Consideriamo allora dapprima le \textbf{autofunzioni pari}, ossia le autofunzioni della forma:
	\begin{equation}
		\tcboxmath[sharp corners=downhill, colback=white, colframe=black]{
			\begin{cases}
			\psi (x) = Ae^{\lambda x} \quad x<-a;\\
			\psi (x) = B\cos(qx) \quad -a<x<a;\\
			\psi (x) = Ae^{-\lambda x} \quad x>a.
			\end{cases} 
			}
	\end{equation}\\
	
Per la simmetria della f.d.o.~è sufficiente imporre la condizione di continuità della funzione e della sua derivata prima nel solo punto $x=-a$. Automaticamente la stessa condizione risulterà soddisfatta nel punto $x=+a$. Si ha allora:
	\begin{equation}
		\begin{cases}
		Ae^{-\lambda a} =B\cos(qa) ;\\
		A \lambda e^{-\lambda a} = +Bq \sin(qa).\end{cases} 
	\end{equation}
Dividendo membro a membro le due equazioni si ottiene:
	\begin{equation}
		\tcboxmath[enhanced, sharp corners=downhill, colframe=black, colback=white, borderline={2pt}{-3pt}{black}]{
			q \tan (qa) =\lambda
			}
	\label{eq:cap10_1}
	\end{equation}
Questa equazione determina implicitamente i possibili \textbf{autovalori dell'energia}.\\

Studiamo graficamente le soluzioni dell'equazione (\ref{eq:cap10_1}). Poniamo per convenienza:
	\begin{equation}
		\tcboxmath[sharp corners=downhill, colback=white, colframe=black]{
			y=qa,
			}
	\end{equation}
e
	\begin{equation}
		\tcboxmath[sharp corners=downhill, colback=white, colframe=black]{
		k=\frac{2mV_0 a^2}{\hbar ^2}.
		}
	\end{equation}
Poiché
	\begin{equation}
		\lambda ^2 a^2 = \frac{2m }{\hbar ^2}|E|a^2=k-q^2 a^2 =k-y^2,
	\end{equation}
l'equazione (\ref{eq:cap10_1}) si scrive come:
	\begin{equation}
		\tcboxmath[sharp corners=downhill, colback=white, colframe=black]{
			\tan (y) = \frac{\sqrt{k-y^2}}{y}
			} \qquad (y^2 \leq k).
	\end{equation}
Le soluzioni dell'eq. (\ref{eq:cap10_1}) sono dunque le intersezioni della curve $\tan (y)$ e $\sqrt{k-y^2}/y$. Come appare dalla figura seguente, \textbf{i livelli di energia sono discreti}.
\begin{figure}[!htbp]
\includegraphics[width=\textwidth]{immagini/cap_10/fig_10_3.png}
\end{figure}
\newpage

Si noti come il numero di intersezioni cresce al crescere di $k$, ossia \textbf{esistono tanti più stati legati quanto più la buca di potenziale è profonda}. Inoltre esiste sempre almeno una intersezione (ed in particolare una sola per $k<\pi$), ossia \textbf{esiste sempre almeno uno stato legato per la particella}.\\

Consideriamo ora le \textbf{autofunzioni dispari}, della forma:
	\begin{equation}
		\tcboxmath[sharp corners=downhill, colback=white, colframe=black]{
			\begin{cases}
			\psi (x) = A'e^{\lambda x} \quad x<-a;\\
			\psi (x) = B'\sin(qx) \quad -a<x<a;\\
			\psi (x) = -A'e^{\lambda x} \quad x>a.\end{cases} 
			}
	\end{equation}
In questo caso le condizioni di continuità della f.d.o. e della sua derivata nel punto $x=-a$ si scrivono:
	\begin{equation}
		\begin{cases}
		A'e^{-\lambda a} =-B'\sin(qa) ;\\
		A' \lambda e^{-\lambda a} = B'q \cos (qa);\end{cases} 
	\end{equation}
e forniscono la condizione:
	\begin{equation}
		\tcboxmath[enhanced, sharp corners=downhill, colframe=black, colback=white, borderline={2pt}{-3pt}{black}]{
			q \cot (qa) =- \lambda.
			}
	\label{eq:cap10_2}
	\end{equation}
Espressa in termini di $y$ e $k$ questa equazione si scrive:
	\begin{equation}
		\tcboxmath[sharp corners=downhill, colback=white, colframe=black]{
		\frac{\sqrt{k-y^2}}{y}= -\cot (y)
		}  \qquad (y^2 \leq k),
	\end{equation}
e le soluzioni si ottengono graficamente dalle intersezioni di seguito rappresentate.
\newpage
\begin{figure}[!htbp]
\includegraphics[width=\textwidth]{immagini/cap_10/fig_10_4.png}
\[ \textrm{N.B. } \tan (y+\pi/2) = -\cot (y) \]
\end{figure}

Si noti come, anche in questo caso, il numero di intersezioni cresce al crescere di $k$, ossia \textbf{il numero di stati legati cresce al crescere della profondità della buca}.\\

Tuttavia l'eq. (\ref{eq:cap10_2}) \textbf{ammette soluzioni solo se} $\mathbf{k\geq \pi^2/4}$,\textbf{ ossia}:
	\begin{equation}
		\tcboxmath[sharp corners=downhill, colback=white, colframe=black]{
		\frac{2mV_0 a^2}{\hbar ^2} \geq \frac{\pi ^2}{4}.
		}
	\end{equation}
Se questa condizione non è soddisfatta, esiste solo uno stato legato corrispondente ad un'autofunzione pari.
\section{Gradino di potenziale}
Consideriamo un \textbf{campo di forze in cui il potenziale presenta una discontinuità}, della forma:\\
\begin{minipage}{.55\textwidth}
	\begin{equation}
		\tcboxmath[enhanced, sharp corners=downhill, colframe=black, colback=white, borderline={2pt}{-3pt}{black}]{
			V(x)=
			\begin{cases}
			0 \quad \textrm{ per } x<0;\\
			V_0 \quad \textrm{per } x>0.
			\end{cases}
			}
	\end{equation}
\end{minipage}
\hspace{.2cm}
\begin{minipage}{.4\textwidth}
\includegraphics[width=5cm]{immagini/cap_10/fig_10_5.png}
\end{minipage}\\

Secondo la \textbf{meccanica classica}, una particella con energia $E>V_0$, che si muove in un tale campo da sinistra verso destra, arrivata alla barriera di potenziale continua a muoversi nella stessa direzione ma con velocità minore. Se invece la sua energia è $E<V_0$, arrivata alla barriera la particella si riflette da essa e riprende il moto in direzione opposta.\\

Nella \textbf{meccanica quantistica} compare un fenomeno nuovo: anche per $E>V_0$ la particella può essere riflessa dalla barriera di potenziale. Ricaviamo tale risultato.\\

L'equazione di Schr\"{o}dinger in presenza del potenziale è:
	\begin{equation}
		-\frac{\hbar ^2}{2m}\frac{d^2 \psi}{dx^2}+ V(x)\psi= E\psi, 
	\end{equation}
ossia:
	\begin{equation}
		-\frac{\hbar ^2}{2m}\frac{d^2 \psi}{dx^2}= -\frac{2m}{\hbar ^2}\left( E-V(x)\right) \psi. 
	\end{equation}
Considerando il caso $\mathbf{E>V_0}$ e ponendo:
	\begin{equation}
		\tcboxmath[sharp corners=downhill, colback=white, colframe=black]{
			k^2=\frac{2mE}{\hbar ^2}
			}\qquad
		\tcboxmath[sharp corners=downhill, colback=white, colframe=black]{
			q^2=\frac{2m\left( E- V_0\right)}{\hbar ^2},
			}
	\end{equation}
otteniamo per l'equazione di Schr\"{o}dinger nelle due regioni $x<0$ ed $x>0$ la forma:
	\begin{equation}
		\tcboxmath[sharp corners=downhill, colback=white, colframe=black]{
			\begin{cases}
			\displaystyle{\frac{d^2 \psi}{dx^2}= -k^2 \psi \quad x<0,}\\
			\\
			\displaystyle{\frac{d^2 \psi}{dx^2}= -q^2 \psi \quad x>0.}\\
			\end{cases}
			}
	\end{equation}\\
	
Se assumiamo che la particella incidente giunga da sinistra verso destra possiamo scrivere la soluzione dell'equazione di Schr\"{o}dinger nella forma:
	\begin{equation}	
		\tcboxmath[sharp corners=downhill, colback=white, colframe=black]{
		\begin{aligned}
				&\psi(x) = Ae^{ikx}+Be^{-ikx} \quad x<0, \\
				&\psi(x) = Ce^{iqx} \qquad \ \quad \qquad x>0. 
		\end{aligned}
		}
		\end{equation}
Ovviamente nella regione $x>0$ non esiste un'onda di ritorno. Le componenti $\displaystyle{Be^{-ikx}}$ e $\displaystyle{Ce^{iqx}}$ rappresentano invece l'\textbf{onda riflessa} e l'\textbf{onda trasmessa} dal gradino di potenziale.\\

La f.d.o.~deve essere continua in tutto lo spazio, ed in particolare dunque nel punto $x=0$. Inoltre, malgrado la discontinuità del potenziale nel punto $x=0$, possiamo dimostrare che anche la derivata prima della f.d.o. deve essere continua in questo punto. (Vedi \emph{Proprietà generali dell'eq.~di Schr\"{o}dinger}).\\

Imponendo la continuità della f.d.o. e della sua derivata prima nel punto $x=0$ si ottiene:
	\begin{equation}
		\begin{cases}
		A+B=C \quad \qquad \quad \textrm{ continuità di }\psi\textrm{ in }x=0,\\
		k\left(A-B \right) =qC\qquad \textrm{ continuità di }\psi '\textrm{ in }x=0.
		\end{cases}
	\end{equation}
Da queste due condizioni risultano determinate due costanti, $B$ e $C$, mentre la terza, $A$, dipende dalla normalizzazione della f.d.o. ossia dal flusso incidente.\\

Moltiplicando la prima equazione per q e sottraendo da questa la seconda si ottiene:
	\begin{equation}
		\left( q-k \right) A + \left( q+k \right)B=0,
	\end{equation}
da cui:
	\begin{equation}
		\tcboxmath[sharp corners=downhill, colback=white, colframe=black]{
			B= \frac{k-q}{k+q}A.
			}
	\label{eq:cap10_3}
	\end{equation}\\
	
Similmente, moltiplicando la prima equazione per k e sommando a questa la seconda equazione si ottiene:
	\begin{equation}
		2kA=\left( k+q \right)C,
	\end{equation}
ossia:
	\begin{equation}
		\tcboxmath[sharp corners=downhill, colback=white, colframe=black]{
			C=\frac{2k}{k+q}A.
			}
	\label{eq:cap10_4}
	\end{equation}
Le equazioni (\ref{eq:cap10_3}) e (\ref{eq:cap10_4}) risolvono completamente il problema posto.\\

Definiamo il \textbf{coefficiente di riflessione $R$} come il rapporto tra la densità di corrente dell'onda riflessa ($\displaystyle{Be^{-ikx}}$) e la densità di corrente dell'onda incidente ($\displaystyle{Ae^{ikx}}$):
	\begin{equation}
		\tcboxmath[enhanced, sharp corners=downhill, colframe=black, colback=white, borderline={2pt}{-3pt}{black}]{
			R=\frac{|j_r|}{j_i}.
			}
	\end{equation}
Analogamente si definisce il \textbf{coefficiente di trasmissione $T$} come il rapporto tra la densità di corrente trasmessa ($\displaystyle{Ce^{iqx}}$) e la densità di corrente dell'onda incidente ($\displaystyle{Ae^{ikx}}$):
	\begin{equation}
		\tcboxmath[enhanced, sharp corners=downhill, colframe=black, colback=white, borderline={2pt}{-3pt}{black}]{
			T=\frac{j_t}{j_i}.
			}
	\end{equation}
\textbf{I coefficienti di riflessione e di trasmissione rappresentano allora le probabilità che la particella incidente sul gradino di potenziale venga rispettivamente riflessa oppure trasmessa}.\\
Per una generica onda piana della forma:
	\begin{equation}
		\psi= Ae^{ikx},
	\end{equation}
la densità di corrente è data da:
	\begin{equation}
		j=\frac{\hbar}{m}\, \Im \left(\psi ^* \frac{d\psi}{dx} \right)= |A|^2\frac{\hbar k}{m}.
	\end{equation}
I coefficienti di riflessione e trasmissione risultano allora:
	\begin{equation}
		\tcboxmath[enhanced, sharp corners=downhill, colframe=black, colback=white, borderline={2pt}{-3pt}{black}]{
			\begin{aligned}
			& R=\frac{|j_r|}{j_i}= \frac{|B|^2}{|A|^2}=\left(\frac{k-q}{k+q}\right)^2; \\[0.3cm]
			& T=\frac{j_t}{j_i}= \frac{|C|^2 q}{|A|^2 k}=\frac{4kq}{\left(k+q\right)^2}. 
			\end{aligned}
			}
	\end{equation}
Un valore di $R$ differente da zero indica una corrispondente probabilità non nulla che la particella venga riflessa dal gradino di potenziale in completo contrasto con le previsioni classiche. Tuttavia, in accordo con l'intuizione, nel limite in cui l'energia della particella è molto maggiore dell'altezza del gradino, $E\gg V_0$ ossia $k\simeq q$ la probabilità di riflessione tende a zero:
	\begin{equation}
		\tcboxmath[sharp corners=downhill, colback=white, colframe=black]{
			R\simeq 0\quad  \textrm{ per } E\gg V_0.
			}
	\end{equation}
Osserviamo che ovviamente è verificata la relazione:
	\begin{equation}
		\tcboxmath[enhanced, sharp corners=downhill, colframe=black, colback=white, borderline={2pt}{-3pt}{black}]{
			R+T=1,
			}
	\label{eq:cap10_5}
	\end{equation}
in accordo con l'interpretazione probabilistica dei coefficienti di riflessione e trasmissione.\\

L'eq. (\ref{eq:cap10_5}) può essere usata come conseguenza generale dell'equazione di continuità:
	\begin{equation}
		\frac{\partial \left(\psi^* \psi \right)}{\partial t}+ \frac{\partial j}{\partial x}=0,
	\end{equation}
che esprime appunto la conservazione della probabilità. Poiché infatti non vi è alcuna dipendenza dal tempo nel problema, questa equazione implica che $j(x)$ è indipendente da $x$. Quindi il flusso in $x<0$ deve essere uguale al flusso in $x>0$. Si ha:
	\begin{align}
		j\left( x<0 \right) &= \frac{\hbar}{m} \Im \left. \left( \psi ^* \frac{\partial \psi}{\partial x} \right) \right| _{x<0} = \nonumber \\
		&= \frac{\hbar}{m} \Im \left( \left( A^* e^{-ikx} + B^* e^{ikx}\right)ik \left( A e^{ikx} + B e^{-ikx}\right)\right)= \nonumber \\
		&=\frac{\hbar k}{m} \Im \left( i\left( |A|^2 - |B|^2 +AB^* e^{2ikx}- A^* B e^{-2ikx}\right)\right)= \nonumber \\
		&=\frac{\hbar k}{m} \Im \left( i\left( |A|^2 - |B|^2 +2i \Im \left( A B^* e^{2ikx}\right) \right)\right)= \nonumber \\
		&= \frac{\hbar k}{m}\left( |A|^2 - |B|^2\right)= j_i- |j_r|,
\end{align}
e
	\begin{equation}
		j\left( x>0 \right) = |C| \frac{\hbar q}{m}= j_t,
	\end{equation}
pertanto vale la condizione
	\begin{equation}
		\tcboxmath[sharp corners=downhill, colback=white, colframe=black]{
			j_i = |j_r|+ j_t,
			}
	\end{equation}
che equivale all'eq. (\ref{eq:cap10_5}) per i coefficienti di riflessione e trasmissione.\\

Un \textbf{fenomeno puramente quantistico} si osserva quando il potenziale $V_0$ è negativo ed in modulo molto grande:\\
\begin{minipage}{.7\textwidth}
\includegraphics[width=.9\textwidth]{immagini/cap_10/fig_10_6.png}
\end{minipage}
\hspace{.5cm}
\begin{minipage}{.1\textwidth}
\[V_0<0\]
\[|V_0| \gg E\]
\end{minipage}\\ \\

Classicamente la particella proveniente da sinistra verso destra, giunta al salto di potenziale, prosegue nella sua direzione con maggiore velocità. La previsione della meccanica quantistica è invece la seguente: poiché
	\begin{equation}
		q^2=\frac{2m}{\hbar ^2}\left( E+ |V_0| \right) \gg \frac{2mE}{\hbar ^2} = k^2,
	\end{equation}
Si trova:
	\begin{equation}
		\tcboxmath[sharp corners=downhill, colback=white, colframe=black]{
			R=\left( \frac{q-k}{q+k} \right) ^2 \simeq 1,
			} \qquad
		\tcboxmath[sharp corners=downhill, colback=white, colframe=black]{
			T=\frac{4qk}{\left( q+k \right) ^2} \simeq 0,
			}
	\end{equation}
ossia la \textbf{particella viene riflessa} con probabilità quasi uno.\\

Un altro caso che interessa considerare è quello in cui la particella incide sul gradino di potenziale con energia minore dell'altezza del gradino:\\
\begin{minipage}{.7\textwidth}
\includegraphics[width=.9\textwidth]{immagini/cap_10/fig_10_7.png}
\end{minipage}
\hspace{.5cm}
\begin{minipage}{.1\textwidth}
\[E<V_0\]
\end{minipage}\\ \\

In questo caso:
	\begin{equation}
		q^2=\frac{2m}{\hbar ^2}\left( E- V_0 \right)<0,
	\end{equation}
ossia $q$ è immaginario. La soluzione dell'equazione di Schr\"{o}dinger, nella regione $x<0$, è della forma:
	\begin{equation}
		\tcboxmath[sharp corners=downhill, colback=white, colframe=black]{	
			\psi (x) = Ce^{-|q|x}, \quad x>0
			}
	\end{equation}
(giacché la soluzione $\displaystyle{Ce^{+|q|x}}$ diverge all'infinito).\\
Tutti i risultati ottenuti precedentemente si modificano allora per la sostituzione:
	\begin{equation}
		q \rightarrow i|q|
	\end{equation}
e quindi, in particolare, per i coefficienti $B$ e $C$ si ha:
	\begin{equation}
		B=\frac{k-i|q|}{k+i|q|}A, \qquad C=\frac{2k}{k+i|q|}.
	\end{equation}
il coefficiente di riflessione $R$ risulta allora:
	\begin{equation}
		\tcboxmath[sharp corners=downhill, colback=white, colframe=black]{	
			R=\frac{|B|^2}{|A|^2}=\left(\frac{k-i|q|}{k+i|q|}\right)\left(\frac{k+i|q|}{k+i|q|}\right)=1,
			}
	\end{equation}
mentre il coefficiente di trasmissione è nullo,
	\begin{equation}
		\tcboxmath[sharp corners=downhill, colback=white, colframe=black]{			
			T=0,
			}
	\end{equation}
essendo sempre nulla la densità di corrente associata ad una f.d.o.~reale.\\

Pertanto, come in meccanica classica, si ha in questo caso \textbf{riflessione totale}. Ciononostante osserviamo che la \textbf{f.d.o.~non si annulla nella regione classicamente proibita} e vi è dunque una probabilità non nulla di osservare la particella in questa regione. Come vedremo in seguito, questo fenomeno puramente quantistico di penetrazione in una regione classicamente proibita consente il cosiddetto \textbf{effetto tunnel}, ossia l'attraversamento di una barriera di potenziale che bloccherebbe completamente la particella secondo la descrizione classica.
\section{Barriera di potenziale -  effetto tunnel}
Consideriamo un campo di forze il cui potenziale ha l'andamento\\
\begin{minipage}{.65\textwidth}
	\begin{equation}
		\tcboxmath[enhanced, sharp corners=downhill, colframe=black, colback=white, borderline={2pt}{-3pt}{black}]{	
			V(x)=
			\begin{cases}
			0 \quad x<-a,\\
			V_0 \quad -a<x<a,\\
			0 \quad x>-a.
			\end{cases}
			}
\end{equation}
\end{minipage}
\begin{minipage}{.3\textwidth}
\includegraphics[width=5cm]{immagini/cap_10/fig_10_8.png}
\end{minipage}\\

Studiamo il moto della particella che giunge da sinistra con energia $\mathbf{E<V_0}$. Classicamente la particella, giunta alla barriera, viene riflessa e torna indietro.\\
Consideriamo invece la descrizione quantistica. \textbf{L'equazione di Schr\"{o}dinger} è
	\begin{equation}
		\begin{cases}
		\displaystyle{-\frac{\hbar ^2}{2m}\frac{d^2 \psi}{d x^2}=E \psi 		\qquad \qquad x <-a \ \textrm{ e }\  x>a,}\\
		\\
		\displaystyle{-\frac{\hbar ^2}{2m}\frac{d^2 \psi}{d x^2}+V_0 \psi=E \psi \qquad -a  \leq x \leq a.}
		\end{cases}
	\end{equation}
Ponendo:
	\begin{equation}
		\tcboxmath[sharp corners=downhill, colback=white, colframe=black]{
			k^2= \frac{2mE}{\hbar ^2} ,
			} \qquad
		\tcboxmath[sharp corners=downhill, colback=white, colframe=black]{
			\lambda ^2= \frac{2m}{\hbar ^2} \left( V_0 - E \right),
			}
	\end{equation}
dove $k$ e $\lambda $ sono parametri reali, possiamo scrivere l'equazione di Schr\"{o}dinger nella forma:
	\begin{equation}
		\tcboxmath[sharp corners=downhill, colback=white, colframe=black]{	
			\begin{cases}
			\displaystyle{\frac{d^2 \psi}{d x^2}= -k ^2 \psi \qquad  x <-a \ \textrm{ e }\  x>a,}\\
			\\
			\displaystyle{\frac{d^2 \psi}{d x^2}= \lambda ^2 \psi \qquad 	\ -a  < x < a.}
			\end{cases}
			}
	\end{equation}\\
	
Le \textbf{soluzioni} di queste equazioni differenziali con la condizione al contorno che in assenza di barriera lo stato è rappresentato da un'onda piana che si propaga nella direzione delle $x$ positive sono:
	\begin{equation}
		\tcboxmath[sharp corners=downhill, colback=white, colframe=black]{
			\begin{cases}
			\displaystyle{\psi (x) = e^{ikx}+Ae^{-ikx} \qquad \ x <-a,}\\
			\\
			\displaystyle{\psi (x) = Be^{\lambda x}+Ce^{-\lambda x} \qquad  -a< x <a,}\\
			\\
			\displaystyle{\psi (x) = De^{ikx} \qquad \qquad \qquad \ x >a.}
			\end{cases}
			}
	\end{equation}	
La \textbf{normalizzazione} scelta corrisponde ad aver fissato a
	\begin{equation}
		\tcboxmath[sharp corners=downhill, colback=white, colframe=black]{
			j_i= \frac{\hbar k}{m}
			}
	\end{equation}
il flusso dell'onda incidente.\\

Imponendo la \textbf{continuità della f.d.o. e della sua derivata prima} nei punti $x=\pm a$ si ottiene il seguente sistema di equazioni:
	\begin{equation}
		\begin{cases}
		\displaystyle{e^{-ika}+ Ae^{ika}= Be^{-\lambda a} + C e^{\lambda a}}\\
		\\
		\displaystyle{ik \left(e^{-ika}- Ae^{ika}\right)= \lambda \left(Be^{-\lambda a} - C e^{\lambda a}\right)}\\
		\\
		\displaystyle{Be^{\lambda a} + Ce^{-\lambda a}= De^{ika}}\\
		\\
		\displaystyle{\lambda \left(Be^{\lambda a} -Ce^{-\lambda a} \right)= ikDe^{ika}}
		\end{cases}
	\end{equation}\\

Moltiplichiamo la prima equazione per $ik$ e sommiamo e sottraiamo la seconda equazione. Eseguiamo le stesse operazioni sulla terza e sulla quarta equazione. Si ottiene in tal modo:
	\begin{equation}
		\begin{cases}
		\displaystyle{2ike^{-ika}= B \left(ik+\lambda\right)e^{-\lambda a} +C \left(ik-\lambda\right)e^{\lambda a} }\\
		\\
		\displaystyle{2ikAe^{ika}= B \left(ik-\lambda\right)e^{-\lambda a} +C \left(ik+\lambda\right)e^{\lambda a} }\\
		\\
		\displaystyle{B\left(ik+\lambda\right)e^{\lambda a} + C\left(ik-\lambda\right)e^{-\lambda a} = 2ikD e^{ika}}\\
		\\
		\displaystyle{B\left(ik-\lambda\right)e^{\lambda a} + C\left(ik+\lambda\right)e^{-\lambda a} =0}
		\end{cases}
	\label{eq:cap10_10}
	\end{equation}\\

Frequentemente risulta soddisfatta la condizione
	\begin{equation}
		\tcboxmath[sharp corners=downhill, colback=white, colframe=black]{
			2\lambda a \gg 1.
			}
	\end{equation}
Poiché
	\begin{equation}
		\frac{1}{\lambda}=\frac{\hbar}{\sqrt{2m\left(V_0 - E \right)}}=\frac{\hbar}{\sqrt{2mE\left(V_0/E - 1 \right)}}=\frac{\hbar /p}{\sqrt{\left(V_0/E - 1 \right)}},
	\end{equation}
questa condizione equivale a
	\begin{equation}
		\tcboxmath[sharp corners=downhill, colback=white, colframe=black]{
			2a \gg \frac{\hbar /p}{\sqrt{\left(V_0/E - 1 \right)}},
			}
	\end{equation}
ossia la \textbf{larghezza della barriera} risulta \textbf{molto maggiore della lunghezza d'onda di De Broglie della particella} divisa per il fattore $\sqrt{\left(V_0/E - 1 \right)}$ (tipicamente di $O(1)$).\\

Risolviamo allora il sistema di equazioni (\ref{eq:cap10_10}) e risolviamolo nel limite ${\lambda a \ll 1}$ (si veda di seguito per la risoluzione esatta). Dalla quarta equazione del sistema (\ref{eq:cap10_10}) ricaviamo:
	\begin{equation}
		B=-\frac{(ik+\lambda)}{(ik-\lambda)}e^{-2 \lambda a } C,
	\end{equation}
ossia $B$ è esponenzialmente piccolo rispetto a $C$.  Nella prima equazione possiamo quindi trascurare il termine proporzionale a $Be^{-\lambda a}$ rispetto a quello proporzionale a $Ce^{\lambda a }$. Si ottiene in tal modo:
	\begin{equation}
		\tcboxmath[sharp corners=downhill, colback=white, colframe=black]{
			C= \frac{2ik}{(ik-\lambda)}e^{-ika}e^{-\lambda a},
			}
	\end{equation}
e dunque, combinando con la precedente equazione, anche
	\begin{equation}
		\tcboxmath[sharp corners=downhill, colback=white, colframe=black]{
			B=-\frac{2ik(ik+\lambda)}{(ik-\lambda)^2}e^{-ik a }e^{-3 \lambda a }.
			}
	\end{equation}
Possiamo ora ricavare il coefficiente $D$ utilizzando la terza delle equazioni del sistema. In questa equazione i termini $Be^{\lambda a }$ e $Ce^{-\lambda a }$ sono dello stesso ordine di grandezza (entrambi proporzionali a $e^{-2\lambda a}$). Si ottiene dunque
	\begin{align}
		2ikDe^{-ika}&= 2ike^{-ika}e^{-2\lambda a} \left[-\frac{(ik+\lambda)^2}{(ik-\lambda)^2}+1\right]= \nonumber \\
		&= 2ike^{-ika}e^{-2\lambda a}\frac{-(ik+\lambda)^2+(ik-\lambda)^2}{(ik-\lambda)^2}= \nonumber \\
		&= 2ike^{-ika}e^{-2\lambda a}\left( -\frac{4ik\lambda}{(ik-\lambda)^2}\right),
	\end{align}
da cui
	\begin{equation}
		\tcboxmath[sharp corners=downhill, colback=white, colframe=black]{
			D= -\frac{4ik\lambda}{(ik-\lambda)^2}e^{-2ika}e^{-2\lambda a}.
			}
	\end{equation}
Dalla seconda equazione del sistema, infine, trascurando il termine $Be^{-\lambda a}$ (di ordine $O(e^{-4\lambda a })$) rispetto al termine $Ce^{\lambda a}$ (di ordine $O(1)$) si ottiene
	\begin{equation}
		\tcboxmath[sharp corners=downhill, colback=white, colframe=black]{
			A=\frac{(ik+\lambda)}{(ik-\lambda)}e^{-2ika}.
			}
	\end{equation}\\
	
La quantità che ci interessa calcolare è il \textbf{coefficiente di trasmissione}
	\begin{equation}
		\tcboxmath[sharp corners=downhill, colback=white, colframe=black]{
			T=\frac{j_t}{j_i}=|D|^2,
			}
	\end{equation}
che risulta dunque valere
	\begin{align}
		T &= |D|^2= \left(\frac{4k\lambda}{k^2+\lambda ^2} \right) ^2 e^{-4\lambda a} =\left(\frac{4\lambda /k}{1+\lambda ^2/k^2} \right) ^2 e^{-4\lambda a}= \nonumber \\[0.3cm]
		&=16\frac{\left(V_0 /E-1\right)}{\left(V_0 /E \right) ^2}\ e^{-\left(4a\sqrt{\frac{2m}{\hbar ^2}\left(V_0-E\right)}\right)},
	\end{align}
ossia
	\begin{equation}
		\tcboxmath[enhanced, sharp corners=downhill, colframe=black, colback=white, borderline={2pt}{-3pt}{black}]{
			T \simeq 16 \left(\frac{E}{V_0} \right) \left(1-\frac{E}{V_0}\right) e^{-4a\sqrt{\frac{2m}{\hbar ^2}\left(V_0-E\right)}}.
			}
	\end{equation}
Questa espressione mostra come la probabilità di trasmissione della particella attraverso la barriera di potenziale decresce esponenzialmente con la lunghezza della barriera ($2a$) e quanto più l'altezza della barriera ($V_0$) eccede l'energia della particella ($E$)\footnote{Osserviamo inoltre che nel limite $2\lambda a \gg 1 $ per il coefficiente di riflessione si ottiene (ricordando che $R=1-T$): $
 R=|A|^2=1+O(e^{-4\lambda a})$.}.\\

%il coefficiente di trasmissione risulta allora
%\begin{equation}
%T=|D|^2= \left(\frac{4k\lambda}{k^2+\lambda^2}\right) ^2 e^{-4\lambda a},
%\end{equation}
%in accordo con quanto ottenuto precedentemente nel limite $\lambda a \ll 1$.\\
Le stesse conclusioni continuano a valere in condizioni più generali. Utilizzando la cosidetta \textbf{approssimazione WKB\footnote{WKB: Wentzel, Kramers, Brillouin}o approssimazione quasi-classica}, è possibile infatti derivare per il coefficiente di trasmissione l'espressione:
	\begin{equation}
		\tcboxmath[enhanced, sharp corners=downhill, colframe=black, colback=white, borderline={2pt}{-3pt}{black}]{
			T \approx e^{-2 \int_{barr.} dx\ \sqrt{\frac{2m}{\hbar ^2}\left[ V(x)-E \right]}},
			}
		 \end{equation}
valida per potenziali $V(x)$ non troppo rapidamente variabili nello spazio.\\

Gli esempi di effetto tunnel sono abbastanza comuni nella fisica atomica e nucleare. Discutiamo qui di seguito due esempi brevemente.
\subsection{Emissione fredda}
L'emissione fredda è un fenomeno che consiste nell' \textbf{emissione di elettroni da parte della superficie di un metallo quando all'esterno della superficie viene applicato un campo elettrico $E$}. Il fenomeno può essere spiegato nel modo seguente. Gli elettroni in un metallo sono confinati all'interno da un potenziale che in prima approssimazione può essere descritto da una buca di profondità finita:
\begin{center}
\includegraphics[width=0.5\textwidth]{immagini/cap_10/fig_10_9}
\end{center} 
Nello stato fondamentale del metallo tutti i livelli di energia di singolo elettrone sono riempiti fino all'energia di Fermi. La differenza tra l'energia di Fermi e la cima della buca rappresenta l'energia necessaria per estrarre un elettrone dal metallo, ossia la \textbf{funzione lavoro} $w$. 
Gli elettroni possono essere estratti dal metallo trasferendogli energia, o mediante fotoni o per riscaldamento.\\

Tuttavia \textbf{è} anche \textbf{possibile estrarre elettroni dal metallo senza fornire a questi alcuna energia} ma applicando un campo elettrico $\mathscr{E}$ all'esterno del metallo. In questo caso infatti, il potenziale visto da un elettrone che si trova al livello di fermi cambia, per effetto del campo esterno, da $w+\varepsilon _F$ a $w+\varepsilon _F-e\mathscr{E}x$:\\
\vspace{.5cm}
\begin{minipage}{.6\textwidth}
\includegraphics[width=\textwidth]{immagini/cap_10/fig_10_10.png}
\end{minipage}
\begin{minipage}{.4\textwidth}
\begin{equation}
w+\varepsilon _F-e\mathscr{E}a=\varepsilon _F ,
\end{equation}
\begin{equation}
\tcboxmath[sharp corners=downhill, colback=white, colframe=black]{
a= w/(e\mathscr{E}) .
}
\end{equation}
\end{minipage}
L'elettrone può allore essere emesso per \textbf{effetto tunnel}.\\
la probabilità di emissione è:
	\begin{equation}
		T \approx \exp \left[-2 \int_{0} ^{a} dx\ \sqrt{\frac{2m}{\hbar ^2} (\underbrace{w+\varepsilon _F-e\mathscr{E}x}_{V(x)}-\underbrace{\varepsilon _F}_{E}})\right].
	\end{equation}
Calcoliamo l'integrale:
	\begin{equation}
		\int _{0} ^{a} \left( w-e\mathscr{E}x \right) ^{1/2}= \left. -\frac{1}{e\mathscr{E}}\left( w-e\mathscr{E}x \right) ^{3/2}\frac{2}{3}\right| _0 ^a =\frac{2}{3}\frac{w^{3/2}}{e\mathscr{E}}.
	\end{equation}
Pertanto:
	\begin{equation}
		\tcboxmath[sharp corners=downhill, colback=white, colframe=black]{
		T \approx \exp \left[-\frac{4}{3}\left(\frac{2m}{\hbar ^2}\right)^{1/2}\frac{w^{3/2}}{e\mathscr{E}}\right].
		}
	\end{equation}
\subsection{Decadimento $\alpha$}
L'esperienza mostra che gli elementi di numero atomico $Z>81$ sono radioattivi, ossia decadono emettendo particelle $\alpha$ (nuclei di elio) o $\beta$ (elettroni) e trasformandosi così in altri elementi. Indichiamo qui un \textbf{semplice modello per descrivere i decadimenti} $\mathbf{\alpha}$ le cui previsioni risultano in buon accordo con le osservazioni sperimentali.\\

In questo modello il decadimento del nucleo in una particella $\alpha$ ed un nucleo ``prodotto'' è descritto come l'emissione, per effetto tunnel, della particella $\alpha$ attraverso una barriera di potenziale causata dall'interazione coulombiana tra il nucleo prodotto e la particella $\alpha$:
\begin{center}
\includegraphics[width=.65\textwidth]{immagini/cap_10/fig_10_11.png}
\end{center}
Il raggio $R$ definisce le dimensioni del nucleo. Per $r<R$ la particella risente  del potenziale all'interno del nucleo, schematizzato da una buca. Per $r>R$ il potenziale è coulombiano: $V(r)=Z_1 Z_2 e^2/r$ dove $Z_1$ è la carica del nucleo prodotto e $Z_2=2$ la carica della particella $\alpha$.\\

Affinché il decadimento $\alpha$ (per effetto tunnel) possa avere luogo, la particella $\alpha$ all'interno del nucleo in questo modello deve avere energia $E$ positiva, ossia non deve trovarsi in uno stato legato.\\

La \textbf{probabilità di emissione} della particella $\alpha$ per effetto tunnel è:
	\begin{equation}
		T \approx \exp \left[ -2 \int _R ^b dr\ \sqrt{\frac{2m}{\hbar ^2}\left(\frac{Z_1 Z_2 e^2}{r}-E\right)}\right].
	\end{equation}
Calcoliamo l'integrale:
	\begin{align}
		&\displaystyle{\int _R ^b dr\ \left(\frac{Z_1 Z_2 e^2}{r}-E\right)^{1/2}=\left(Z_1 Z_2 e^2\right)^{1/2}\int _R ^b \frac{dr}{\sqrt{r}}\left(1-\frac{Er}{Z_1 Z_2 e^2}\right)=}\nonumber \\
		&\displaystyle{=\left(Z_1 Z_2 e^2\right)^{1/2}\int _R ^b \frac{dr}{\sqrt{r}}\left(1-\frac{r}{b}\right)\overset{s=r/b}{=}}\displaystyle{\left(Z_1 Z_2 e^2 b\right)^{1/2}\int _{R/b} ^1 ds\ s^{-1/2} (1-s)^{1/2}}.
	\end{align}
La piccolezza del raggio nucleare fa sì che risulta sempre valida con buona approssimazione la condizione
	\begin{equation}
		R\ll b \quad \Rightarrow \quad E \ll \frac{Z_1 Z_2 e^2}{r}
	\end{equation}
Risulta pertanto possibile sostituire nel precedente integrale l'estremo inferiore di integrazione, $R/b$, con $0$. L'integrale si riduce quindi ad una costante adimensionale il cui valore,
	\begin{equation}
		\int _{0} ^1 ds\ s^{-1/2} (1-s)^{1/2}= \frac{\pi}{2},
	\end{equation}
si può ottenere effettuando il cambio di variabile $s= \sin^2 \alpha$ (ma non è comunque rilevante per quanto segue).\\
Si trova pertanto:
	\begin{equation}
		T\approx  \exp \left[ -\frac{\pi}{(\hbar ^2 /2m)^{1/2}}\left(Z_1 Z_2 e^2 b\right)^{1/2}\right]=  exp \left[ -\frac{\pi}{(\hbar ^2 /2m)^{1/2}}\frac{e^2}{\sqrt{E}}\right],
	\end{equation}
ossia
	\begin{equation}
		\tcboxmath[sharp corners=downhill, colback=white, colframe=black]{
			T \approx e^{-C\frac{Z_1}{\sqrt{E}}},
			}
	\end{equation}
che evidenzia la dipendenza della probabilità di decadimento dal numero atomico $Z_1$ del nucleo che decade e dall'energia $E=mv^2/2$ della particella $\alpha$ prodotta nel decadimento.\\

La probabilità di decadimento $\alpha$ del nucleo, T, è inversamente proporzionale alla \textbf{vita media} $\mathbf{\tau}$ del nucleo:
	\begin{equation}
		\tau=\frac{C'}{T}=e^{C'\frac{Z_1}{\sqrt{E}}},
	\end{equation}
o, equivalentemente,
	\begin{equation}
		\tcboxmath[sharp corners=downhill, colback=white, colframe=black]{
			\log \tau = \log C' + \frac{C Z_1}{\sqrt{E}}.
			}
	\end{equation}
Questo andamento risulta ben confermato dalle misure sperimentali.
\section{Risoluzione esatta dell'eq. di Schr\"{o}dinger per la barriera di potenziale (senza l'ipotesi $\mathbf{\lambda a \gg 1} $)}
Consideriamo nuovamente il sistema di equazioni (\ref{eq:cap10_10}) e risolviamolo esattamente, senza assumere la condizione $\lambda a \gg 1 $:
\begin{equation}
		\begin{cases}
		\displaystyle{2ike^{-ika}= B \left(ik+\lambda\right)e^{-\lambda a} +C \left(ik-\lambda\right)e^{\lambda a} }\\
		\\
		\displaystyle{2ikAe^{ika}= B \left(ik-\lambda\right)e^{-\lambda a} +C \left(ik+\lambda\right)e^{\lambda a} }\\
		\\
		\displaystyle{B\left(ik+\lambda\right)e^{\lambda a} + C\left(ik-\lambda\right)e^{-\lambda a} = 2ikD e^{ika}}\\
		\\
		\displaystyle{B\left(ik-\lambda\right)e^{\lambda a} + C\left(ik+\lambda\right)e^{-\lambda a} =0}
		\end{cases}
	\end{equation}\\

La prima e la quarta di queste equazioni costituiscono un sistema di due equazioni nelle due incognite $B$ e $C$. Moltiplicando la prima equazione per $(ik+\lambda) e^{-\lambda a }$ e la quarta per $(ik-\lambda) e^{\lambda a }$ e sottraendo poi la quarta dalla prima si ottiene
	\begin{equation}
		B\left[ (ik+\lambda)^2 e^{-2\lambda a }-(ik-\lambda)^2 e^{2\lambda a }\right]= 2ik(ik+\lambda) e^{-\left(ik+\lambda\right) a },
	\end{equation} 
da cui
	\begin{equation}
		B\left[ (-k^2+\lambda ^2+2ik\lambda) e^{-2\lambda a }-(-k^2+\lambda-2ik\lambda) e^{2\lambda a }\right]= 2ik(ik+\lambda) e^{-\left(ik+\lambda\right) a },
	\end{equation}
e infine
	\begin{equation}
		\tcboxmath[sharp corners=downhill, colback=white, colframe=black]{
			B=\frac{ik\left(ik+\lambda \right)e^{-\left(ik+\lambda\right) a }}{\left(k^2-\lambda^2\right)\sinh \left(2\lambda a\right)+2ik\lambda \cosh\left(2\lambda a\right)}.
			}
	\end{equation}\\
	
Il coefficiente $C$ si ottiene da $B$ scambiando $\lambda \rightarrow -\lambda$:
	\begin{equation}
		\tcboxmath[sharp corners=downhill, colback=white, colframe=black]{
			C=\frac{-ik\left(ik-\lambda \right)e^{-\left(ik-\lambda\right) a }}{\left(k^2-\lambda^2\right)\sinh \left(2\lambda a\right)+2ik\lambda \cosh\left(2\lambda a\right)}.
			}
	\end{equation}\\
	
I coefficienti $A$ e $D$ si ottengono infine sostituendo questi risultati nella seconda e terza equazione del sistema. Si trova
	\begin{align}
 2ikAe^{ika} &= B \left(ik-\lambda\right)e^{-\lambda a} +C \left(ik+\lambda\right)e^{\lambda a}=\nonumber \\[0.3cm]
	& = \displaystyle{\frac{ik\left(-k^2-\lambda ^2 \right)e^{-2\lambda a }e^{-ika }-ik\left(-k^2-\lambda ^2 \right)e^{2\lambda a }e^{-ika }}{\left(k^2-\lambda^2\right)\sinh \left(2\lambda a\right)+2ik\lambda \cosh\left(2\lambda a\right)}=} \nonumber \\[0.3cm]
	&\displaystyle{ =\frac{2ik\left(k^2+\lambda ^2 \right)\sinh\left(2\lambda a\right)e^{-ika}}{\left(k^2-\lambda^2\right)\sinh \left(2\lambda a\right)+2ik\lambda \cosh\left(2\lambda a\right)}},
	\end{align}
e
	\begin{align}
		2ikD e^{ika} &= B\left(ik+\lambda\right)e^{\lambda a} + C\left(ik-\lambda\right)e^{-\lambda a} = \nonumber \\[0.3cm]
		& = \displaystyle{\frac{ik\left(-k^2 + \lambda ^2 +2ik\lambda\right)e^{-ika}-ik\left(-k^2 + \lambda ^2 -2ik\lambda\right)e^{-ika}}{\left(k^2-\lambda^2\right)\sinh \left(2\lambda a\right)+2ik\lambda \cosh\left(2\lambda a\right)}=} \nonumber \\[0.3cm]
	&\displaystyle{ =\frac{-4k^2\lambda e^{-ika}}{\left(k^2-\lambda^2\right)\sinh \left(2\lambda a\right)+2ik\lambda \cosh\left(2\lambda a\right)}},
	\end{align}
ossia:
	\begin{equation}
		\tcboxmath[sharp corners=downhill, colback=white, colframe=black]{
			A=\frac{\left(k^2+\lambda ^2\right)\sinh\left(2\lambda a \right)e^{-2ika}}{\left(k^2-\lambda^2\right)\sinh \left(2\lambda a\right)+2ik\lambda \cosh\left(2\lambda a\right)},
			}
	\end{equation}
e
	\begin{equation}
		\tcboxmath[sharp corners=downhill, colback=white, colframe=black]{
			D=\frac{2ik\lambda e^{-2ika}}{\left(k^2-\lambda^2\right)\sinh \left(2\lambda a\right)+2ik\lambda \cosh\left(2\lambda a\right)}.
			}
	\end{equation}\\
	
La quantità che ci interessa calcolare è il \textbf{coefficiente di trasmissione}:
	\begin{equation}
		\tcboxmath[sharp corners=downhill, colback=white, colframe=black]{
			T=\frac{j_t}{j_i} =|D|^2,
			}
	\end{equation}
che risulta dunque valere
	\begin{equation}
		T=\frac{4k^2 \lambda ^2}{\left(k^2-\lambda^2\right) ^2\sinh ^2\left(2\lambda a\right)+4k^2\lambda ^2\cosh ^2\left(2\lambda a\right)},
	\end{equation}
ossia
	\begin{equation}
		\tcboxmath[enhanced, sharp corners=downhill, colframe=black, colback=white, borderline={2pt}{-3pt}{black}]{
			T=\frac{4k^2 \lambda ^2}{\left(k^2+\lambda^2\right) ^2\sinh ^2\left(2\lambda a\right)+4k^2\lambda ^2}.
			}
	\end{equation}
Il risultato per questo coefficiente (diverso da zero) implica che \textbf{vi è una probabilità non nulla che la particella attraversi la barriera di potenziale}. Questo fenomeno, puramente quantistico, è noto con il nome di \textbf{effetto tunnel}.\\

Possiamo ora verificare come, nel limite $2\lambda a \gg 1$, si riottenga il risultato derivato precedentemente. Si ha:
	\begin{equation}
		\sinh \left( 2 \lambda a \right) = \frac{e^{2\lambda a }+ e ^{-2\lambda a }}{2} \simeq \frac{1}{2} e^{2\lambda a} \left( 1+ O(e^{-4\lambda a}) \right).
\end{equation}
Sostituendo nell'equazione precedente per il coefficiente di trasmissione $T$ risulta
	\begin{equation}
		\tcboxmath[sharp corners=downhill, colback=white, colframe=black]{
		T \simeq \frac{16k^2 \lambda ^2}{\left( k^2+\lambda ^2\right) ^2}e^{-4\lambda a }}
	\end{equation}
che coincide con l'espressione derivata in precedenza.
\section{Risoluzione dell'eq. di Schr\"{o}dinger per la barriera di potenziale in termini di f.d.o. simmetriche e anti-simmetriche}
L'equazione di Schr\"{o}dinger per la barriera di potenziale può anche essere risolta, in modo conveniente, in termini di autofunzioni pari e dispari:
	\begin{equation}
		\psi _S (x) =
		\begin{cases}
		\displaystyle{e^{ikx} + A_S\ e^{-ikx} \qquad x<-a,}\\
		B_S \cosh (\lambda x) \quad -a<x<a,\\
		\displaystyle{e^{-ikx} + A_S\ e^{ikx} \qquad x>a.}
		\end{cases}
	\end{equation}
	\begin{equation}
		\psi _A (x) =
		\begin{cases}
		\displaystyle{e^{ikx} + A_A\ e^{-ikx} \qquad x<-a,}\\
		B_A \sinh (\lambda x) \quad -a<x<a,\\
		\displaystyle{-e^{-ikx} - A_A\ e^{ikx} \qquad x>a.}
		\end{cases}
	\end{equation}
Successivamente, una volta determinate le costanti $A$ e $B$ mediante le condizioni di continuità sulle f.d.o.~e sulle derivate prime, si costruisce l'opportuna combinazione lineare per il problema dato:
	\begin{equation}
		\psi (x) =\frac{1}{2}\left(\psi _S (x) + \psi _A (x) \right) =
		\begin{cases}
		\displaystyle{e^{ikx} + \frac{1}{2}\left(A_S+A_A\right) e^{-ikx}} \\
		\\
		\displaystyle{\frac{1}{2}B_S \cosh (\lambda x )+ \frac{1}{2} B_A \sinh (\lambda x)} \\
		\\
		\displaystyle{\frac{1}{2}\left(A_S - A_A \right) e^{ikx}}
		\end{cases}
	\end{equation}
Vediamo, in particolare, che $\displaystyle{D=\frac{1}{2}\left( A_s - A_A \right)}$. Calcoliamo il coefficiente $D$ utilizzando questo metodo: le condizioni di continuità per la f.d.o. simmetrica e per la sua derivata prima nel punto $x=a$ si scrivono
	\begin{equation}
		\begin{cases}
		\displaystyle{B_S \cosh (\lambda a) = e^{-ika}+ A_S\ e^{ika}}\\
		\displaystyle{\lambda B_S \sinh (\lambda a) = -ik e^{-ika}+ ikA_S\ e^{ika}}
		\end{cases}
	\label{eq:cap10_6}
	\end{equation}
Per la simmetria della f.d.o.~queste condizioni garantiscono anche la continuità della f.d.o. e della sua derivata nel punto $x=-a$. Il sistema di equazioni (\ref{eq:cap10_6}) può essere facilmente risolto per $A_S$ moltiplicando la prima equazione per $\lambda \sinh (\lambda a)$, la seconda per $\cosh (\lambda a)$ e sottraendo l'una dall'altra:
	\begin{equation}
		\left[\lambda \sinh (\lambda a)+ ik\cosh (\lambda a) \right]e^{-ika}+ A_S \left[\lambda \sinh (\lambda a)-ik \cosh (\lambda a) \right]e^{ika}=0,
	\end{equation}
da cui
	\begin{equation}
		\tcboxmath[sharp corners=downhill, colback=white, colframe=black]{	
			A_S=-\frac{\lambda \sinh (\lambda a)+ik \cosh (\lambda a)}{\lambda \sinh (\lambda a)-ik \cosh (\lambda a)}e^{-2ika}.
			}
	\label{eq:cap10_7}
	\end{equation}\\
	
La determinazione del coefficiente $A_A$ della f.d.o antisimmetrica procede in modo analogo. in questo caso le condizioni di continuità della f.d.o.~e della sua derivata conducono al sistema
	\begin{equation}
		\begin{cases}
		\displaystyle{B_A \sinh (\lambda a) = -e^{-ika}- A_A e^{ika}}\\
		\displaystyle{\lambda B_A \cosh (\lambda a) = ik e^{-ika}- ikA_A e^{ika}}
		\end{cases}
	\label{eq:cap10_8}
	\end{equation}
Questo sistema differisce dal sistema (\ref{eq:cap10_6}) per lo scambio\\ $\sinh (\lambda a) \leftrightarrow -\cosh (\lambda a)$. Possiamo quindi scrivere direttamente dall'eq. (\ref{eq:cap10_7}) la soluzione per il coefficiente $A_A$:
	\begin{equation}
		\tcboxmath[sharp corners=downhill, colback=white, colframe=black]{
			A_A=-\frac{\lambda \cosh (\lambda a)+ik \sinh (\lambda a)}{\lambda \cosh (\lambda a)-ik \sinh (\lambda a)}e^{-2ika}.
			}
	\label{eq:cap10_9}
	\end{equation}
dalle eq. (\ref{eq:cap10_7}) e (\ref{eq:cap10_9}) ricaviamo infine l'espressione cercata per il coefficiente $D$:
	\begin{eqnarray}
		D&=&\frac{1}{2}\left( A_S - A_A \right)= \nonumber \\
		&=& -\frac{1}{2}e^{-2ika}\left\lbrace \frac{\left( \lambda \sinh (\lambda a)+ ik \cosh (\lambda a )\right)\cdot\left( \lambda \cosh (\lambda a)- ik \sinh (\lambda a )\right)}{\left( \lambda\sinh ( \lambda a )-ik \cosh (\lambda a )\right)\cdot\left(\lambda \cosh ( \lambda a )-ik \sinh (\lambda a )\right)}\right. +\nonumber \\
		& &\quad \left. -\frac{\left( \lambda \cosh (\lambda a)+ ik \sinh (\lambda a )\right)\cdot \left( \lambda \sinh (\lambda a)- ik \cosh (\lambda a )\right)}{(\lambda \sinh (\lambda a)- ik\cosh (\lambda a )) \cdot (\lambda \cosh (\lambda a ) - ik \sinh (\lambda a ))}\right\rbrace= \nonumber \\
		&=& -\frac{1}{2}e^{-2ika} \frac{-2ik\lambda \sinh ^2 (\lambda a) + 2ik \lambda \cosh ^2 (\lambda a)}{ \left(\lambda ^2 - k^2 \right) \sinh (\lambda a ) \cosh (\lambda a) - ik\lambda \left(\sinh ^2 (\lambda a) + \cosh ^2 (\lambda a ) \right)}.\nonumber \\
	\end{eqnarray}
Utilizzando le relazioni:
	\begin{eqnarray}
		& &\cosh ^2 x - \sinh ^2 x=1 \nonumber \\
		& &\cosh ^2 x + \sinh ^2 x=\cosh 2x  \\
		& &2 \cosh  x \sinh  x=\sinh 2x \nonumber 
	\end{eqnarray}
possiamo riscrivere il coefficiente $D$ nella forma
	\begin{equation}
		\tcboxmath[sharp corners=downhill, colback=white, colframe=black]{	
			D=\frac{2ik\lambda e^{-2ika}}{\left(k^2-\lambda^2\right)\sinh \left(2\lambda a\right)+2ik\lambda \cosh\left(2\lambda a\right)},
			}
	\end{equation}
che coincide con il risultato precedentemente ottenuto.
 %DEFINITIVO
\chapter[Oscillatore Armonico]{Oscillatore Armonico}
Consideriamo una particella che compie piccole oscillazioni unidimensionale (il cosidetto \textbf{oscillatore armonico}). L'energia potenziale di tale particella è uguale a $\frac{1}{2}mw^2x^2$, dove $w$ rappresenta nella meccanica classica la frequenza propria delle oscillazioni. L'hamiltoniana dell'oscillatore è quindi:
	\begin{equation}
		\tcboxmath[enhanced, sharp corners=downhill, colback=yellow!50!white, colframe=red!75!black, borderline={2pt}{-3pt}{red!50!black}]{
			H=\frac{p^2}{2m}+\frac{1}{2}mw^2x^2.
			}
	\label{eq:cap11_1}
	\end{equation}
Poiché l'energia potenziale diventa infinita per $x=\pm \infty$, la particella può compiere soltanto un moto finito e, di conseguenza,  \textbf{tutto lo spettro} energetico dell'oscillatore \textbf{sarà discreto}.\\

 I livelli energetici dell'oscillatore armonico si possono determinare risolvendo \textbf{l'equazione di Schr\"{o}dinger indipendente dal tempo}:
	\begin{equation}
		\tcboxmath[sharp corners=downhill, colback=white, colframe=red!75!black]{	
			H\psi= -\frac{\hbar^2}{2m}\frac{d^2\psi}{dx^2}+\frac{1}{2}mw^2x^2\psi=E\psi,
			}
	\end{equation}
con le condizioni al contorno:
	\begin{equation}
		\tcboxmath[sharp corners=downhill, colback=white, colframe=red!75!black]{
			\lim _{x \rightarrow \pm \infty} \psi(x)=0.
			}
	\end{equation}\\
	
Noi invece risolviamo il problema della determinazione dei livelli energetici, e dei relativi autostati, seguendo un elegante \textbf{metodo operatoriale sviluppato da Dirac}. A tale scopo è conveniente in primo luogo introdurre degli operatori adimensionali, dividendo entrambe i membri dell'equazione (\ref{eq:cap11_1}) per $\hbar w$:
	\begin{equation} \label{eq:cap11_2}
		\frac{H}{\hbar w}=\frac{p^2}{2m\hbar w}+\frac{mw^2x^2}{2\hbar w}.
	\end{equation}
Definendo allora:
	\begin{gather}  
		\tcboxmath[enhanced, sharp corners=downhill, colback=yellow!50!white, colframe=red!75!black, borderline={2pt}{-3pt}{red!50!black}]{	
			\hat{H}=\frac{H}{\hbar w},
			}  \\[0.3cm]
		\tcboxmath[enhanced, sharp corners=downhill, colback=yellow!50!white, colframe=red!75!black, borderline={2pt}{-3pt}{red!50!black}]{
			 \hat{p}=\frac{p}{\sqrt{m \hbar w }},
			}  \\[0.3cm]
		\tcboxmath[enhanced, sharp corners=downhill, colback=yellow!50!white, colframe=red!75!black, borderline={2pt}{-3pt}{red!50!black}]{
			 \hat{x}= \sqrt{\frac{mw}{\hbar}}x,
			}
	\end{gather}
possiamo scrivere la precedente equazione nella forma:
	\begin{equation}
		\tcboxmath[sharp corners=downhill, colback=white, colframe=red!75!black]{
			\hat{H}=\frac{1}{2} (\hat{p}^2+\hat{x}^2).
			}
	\end{equation}\\
	 
Calcoliamo il commutatore tra $\hat{p}$ e $\hat{x}$:
	\begin{equation}
		[\hat{p},\hat{x}]= \frac{1}{\sqrt{\hbar w m}} \sqrt{\frac{mw}{\hbar}} [p,x]=\frac{1}{\hbar}(-i\hbar),
	\end{equation}
ossia:
	\begin{equation}
	\label{eq:cap11_3}
	 	\tcboxmath[sharp corners=downhill, colback=white, colframe=red!75!black]{
			[\hat{p},\hat{x}]=-i.
			}
	\end{equation}\\
	
In termini degli operatori $\hat{p}$ ed $\hat{x}$ risulta poi conveniente definire due operatori non hermitiani:
	\begin{equation}
	\label{eq:cap11_4}
		\tcboxmath[enhanced, sharp corners=downhill, colback=yellow!50!white, colframe=red!75!black, borderline={2pt}{-3pt}{red!50!black}]{
			\begin{aligned}
			&a=\frac{1}{\sqrt{2} } (\hat{x}+i\hat{p}),
			\\
			&a^+=\frac{1}{\sqrt{2} } (\hat{x}-i\hat{p}),
			\end{aligned}
	}
	\end{equation}
o, più semplicemente:
	\begin{equation} 
		\tcboxmath[sharp corners=downhill, colback=white, colframe=red!75!black]{	
		\begin{split}
			&a=\sqrt{\frac{mw}{2\hbar}}(x+i \frac{p}{mw}), \\
			&a^+=\sqrt{\frac{mw}{2\hbar}}(x-i \frac{p}{mw}).
		\end{split}
		}
	\end{equation}\\
	
Facendo uso della regola di commutazione canonica (\ref{eq:cap11_3}) possiamo calcolare il commutatore tra $a$ e $a^+$:
	\begin{equation}
		[a,a^+]=\frac{1}{2} [\hat{x}+i\hat{p},\hat{x}-i\hat{p}]=\frac{1}{2}  \left( +i[\hat{p},\hat{x}]-i[\hat{x},\hat{p}]   \right ) = i[\hat{p},\hat{x}],
	\end{equation}
e dunque:
	\begin{equation}
		\tcboxmath[sharp corners=downhill, colback=white, colframe=red!75!black]{
			[a,a^+]=1.
			}
	\end{equation}\\
	
Esprimiamo l'operatore $\hat{H}$ in termini degli operatori $a$ e $a^+$. A tale scopo invertiamo le equazioni (\ref{eq:cap11_4}) per ottenere:
	\begin{equation}
	\label{eq:cap11_5}
		\tcboxmath[sharp corners=downhill, colback=white, colframe=red!75!black]{ 
				\hat{x}=\frac{1}{\sqrt{2} } (a+a^+),
				} \quad
		\tcboxmath[sharp corners=downhill, colback=white, colframe=red!75!black]{
				\hat{p}=\frac{1}{\sqrt{2}i } (a-a^+).
				}
	\end{equation}
Si trova allora:
	\begin{align}
		\hat{H} &= \frac{1}{2} (\hat{p}^2+\hat{x}^2)=  \frac{1}{4} \left[ -(a-a^+)^2+(a+a^+)^2  \right]=  \frac{1}{4} (aa^++a^+a)\cdot 2= \nonumber \\
		&=\frac{1}{2}(aa^++a^+a)= \frac{1}{2}( [a,a^+]+2a^+a  )=\frac{1}{2}(1+2a^+a ),
	\end{align}
ossia:
	\begin{equation} 
		\tcboxmath[sharp corners=downhill, colback=white, colframe=red!75!black]{
			\hat{H} =a^+a +\frac{1}{2}.
			}
	\end{equation}
Equivalentemente:
	\begin{equation}
	\label{eq:cap11_6}
		\tcboxmath[enhanced, sharp corners=downhill, colback=yellow!50!white, colframe=red!75!black, borderline={2pt}{-3pt}{red!50!black}]{
			H =(a^+a+ \frac{1}{2}) \hbar w.
			}
	\end{equation}\\
	
Per comprendere il significato degli operatori $a$ e $a^+$ supponiamo di conoscere un autovalore $E_n$  dell'energia ed il corrispondente autostato $|n \rangle$:
	\begin{equation}
		\tcboxmath[sharp corners=downhill, colback=white, colframe=red!75!black]	{		
			H|n\rangle=E_n|n\rangle.
			}
	\end{equation}
Consideriamo quindi l'applicazione di H allo stato ottenuto applicando l'operatore $a$ ad $|n\rangle$:
	\begin{equation}
		Ha|n\rangle= [H,a]|n \rangle+aH|n\rangle=([H,a]+E_na)|n\rangle.
	\end{equation}
Calcoliamo il commutatore [H,$a$]:
	\begin{align}
		[H,a]&=\hbar w[a^+a+\frac{1}{2},a]=\hbar w [a^+a,a]= \nonumber\\
		&=\hbar w (a^+aa-aa^+a)=\hbar w[a^+,a]a=-\hbar wa.
	\end{align}
Allora:
	\begin{equation}
		\tcboxmath[sharp corners=downhill, colback=white, colframe=red!75!black]{
		Ha|n\rangle=(E_n-\hbar w)a|n\rangle.
		}
	\end{equation}
Pertanto, se \textbf{$\mathbf{|n\rangle}$ è un autostato dell'hamiltoniana con autovalore $E_n$ allora anche $a|n\rangle$ è  un autostato dell'hamiltoniana con autovalore $E_n$-$\hbar w$. Per questa ragione l'operatore $a$ è anche detto \textit{operatore di distruzione}}.  \\

 Similmente possiamo considerare l'applicazione di H sullo stato $a^+|n\rangle$:
	\begin{equation}
		Ha^+|n\rangle= [H,a^+]|n\rangle+a^+H|n\rangle=([H,a^+]+E_na^+)|n\rangle.
	\end{equation}
Il commutatore di H con $a^+$ risulta:
	\begin{align}
		[H,a^+]&= \hbar w[a^+a+\frac{1}{2},a^+]=\hbar w [a^+a,a^+]= \nonumber\\
		&= \hbar w (a^+aa^+-a^+a^+a)=\hbar w a^+ [a,a^+]=\hbar wa^+,
	\end{align}
o anche:
\begin{equation}
[H,a^+]=-[H,a]^+=\hbar w a^+.
\end{equation}
Allora:
\begin{equation}
Ha^+|n\rangle=(E_n+\hbar w)a^+|n\rangle,
\end{equation}
ossia se \textbf{$|n\rangle$ è un autostato dell'hamiltoniana con autovalore $E_n$ allora anche $a^+|n\rangle$ è  un autostato dell'hamiltoniana con autovalore $E_n$-$\hbar w$. Per questa ragione l'operatore $a^+$ è anche detto \textit{operatore di creazione}}. \\ 
 Questi risultati indicano che \textbf{i livelli di energia sono discreti e differiscono tra loro per un numero interno di unità $\hbar w$}.\\
 Un'altra importante osservazione è che gli autovalori dell'energia devono essere sempre positivi, ed anzi, più precisamente, maggiori od uguali di $\hbar w/2$. Si ha infatti:
\begin{eqnarray}
	E_n&=&\langle n|H|n \rangle= \hbar w \langle n|(a^+a+\frac{1}{2})|n\rangle= \nonumber \\
	&=&\hbar w (\langle n|(a^+a)|n\rangle+\frac{1}{2})=\hbar w (\langle n'|n'\rangle+\frac{1}{2}) \geq \frac{1}{2} \hbar w, 
\end{eqnarray}
giacché per qualunque ket $|n'\rangle$ si ha $\langle n'|n' \rangle\geq 0$ (e con $|n'\rangle= a|n\rangle$).
Deve dunque esistere uno \textbf{stato fondamentale}, il cui vettore di stato indicheremo con $|0\rangle$, la cui energia $E_0$ è maggiore o uguale di $\hbar w/2$. Poiché poi l'operatore $a$, se applicato ad un autostato, produce l'autostato di energia inferiore, deve valere la relazione:
\begin{equation}  \label{eq:cap11_7}
\mathbf{a|0\rangle=0}.
\end{equation}
 \textbf{L'energia del livello fondamentale} può essere ora facilmente calcolata:
\begin{equation}
H|0\rangle= \hbar w(a^+a+\frac{1}{2})|0\rangle= \frac{1}{2} \hbar w |0\rangle,
\end{equation}
ossia:
\begin{equation}
E_0=\frac{1}{2} \hbar w
\end{equation}
 I livelli di energia dell'oscillatore armonico risultano dunque dati da:
\begin{equation}
  \label{eq:cap11_8}
E_n=(n+\frac{1}{2}) \hbar w ,
\end{equation}
con   \textbf{n=0,1,2,...}

 L'equazione (\ref{eq:cap11_6}) implica che gli autostati $|n\rangle$ dell'hamiltoniana sono autostati simultanei dell'operatore $a^+a$. L'espressione (\ref{eq:cap11_8}) per gli autovalori dell'energia $E_n$ indica inoltre che i corrispondenti autovalori dell'operatore $a^+a$ sono i numeri interi n. Per tale ragione l'operatore hermitiano $a^+a$ è anche detto \textbf{operatore numero}:
\begin{equation}
\mathbf{N=a^+a},
\end{equation}
e si ha:
\begin{equation}
\mathbf{N|n\rangle=n|n\rangle}.
\end{equation}
\\\\
 Discutiamo ora come gli autostati $|n\rangle$ dell'hailtoniana possono essere costruiti a partire dall'autostato $|0\rangle$ corrispondente allo stato fondamentale dell'oscillatore.
 Il ruolo degli operatori $a$ e $a^+$ come operatori di distruzione e costruzione rispettivamente implica che gli stati $a|n\rangle$ ed $a^+|n\rangle$ coincidono, a meno di una costante di normalizzazione, con gli autostati $|n-1\rangle$ ed $|n+1\rangle$. Possiamo pertanto scrivere:
\begin{equation}
\begin{cases}
a|n\rangle=c_n|n-1\rangle,\\
a^+|n-1\rangle=d_n|n\rangle.
\end{cases}.
\end{equation}
 Per ricavare le costanti $c_n$ e $d_n$ osserviamo innanzitutto che:
\begin{equation}
c_n=\langle n-1|a|n\rangle=\langle n|a^+|n-1\rangle^*=d_n^*.
\end{equation}
 Inoltre, applicando l'operatore numero allo stato $|n\rangle$ troviamo:
\begin{equation}
N|n\rangle=n|n\rangle=a^+a|n\rangle=c_na^+|n-1\rangle=c_nd_n|n\rangle=|c_n|^2|n\rangle,
\end{equation}
ossia:
\begin{equation}
|c_n|^2=n.
\end{equation}
Scegliendo per convenzione $c_n$ reale e positivo (tale scelta essendo sempre possibile giacché i vettori di stato sono definiti a meno di un fattore di fase arbitrario) vediamo allora che $c_n=\sqrt{n}$. In definitiva abbiamo dimostrato le relazioni:
\begin{equation} \label{eq:cap11_9}
\begin{cases}
a|n\rangle= \sqrt{n} |n-1\rangle, \\
a^+|n-1\rangle=\sqrt{n}|n\rangle.
\end{cases}
\end{equation}
Queste relazioni consentono in particolare di costruire tutti gli autostati dell'hamiltoniana applicando in successione l'operatore $a^+$ allo stato fondamentale $|0\rangle$. Otteniamo:\\\\
\begin{eqnarray} \label{eq:cap11_10}  %non so come mandare tutto a sx
& &|1\rangle=a^+|0\rangle \nonumber \\
& &|2\rangle=\frac{1}{\sqrt{2}}a^+|1\rangle=\frac{1}{\sqrt{2}}(a^+)^2|0\rangle \nonumber \\
& &|3\rangle= \frac{1}{\sqrt{3}}a^+|2\rangle=\frac{1}{\sqrt{3!}}(a^+)^3|0\rangle   \\
& &... \nonumber\\
& &|n\rangle=\frac{1}{\sqrt{n!}}(a^+)^n|0\rangle \nonumber
\end{eqnarray}



 Il metodo operatoriale di Dirac consente anche di ricavare le \textbf{f.d.o nello spazio delle coordinate corrispondenti agli autostati dell'energia.}\\
 Consideriamo in primo luogo lo stato fondamentale definito dall'equazione (\ref{eq:cap11_7}). Moltiplicando a sinistra questa equazione per il $\langle x'|$ troviamo:
\begin{equation}
\langle x'|a|0 \rangle= \sqrt{\frac{mw}{2\hbar}}\langle x'|(x+\frac{ip}{mw})|0 \rangle=0.
\end{equation}
Ricordando le espressioni degli operatori x e p nella rappresentazione delle coordinate, otteniamo l'equazione:
\begin{equation} \label{eq:cap11_12}
\sqrt{\frac{mw}{2\hbar}}(x'+\frac{\hbar}{mw}\frac{d}{dx'})\psi_0(x')=0,
\end{equation}
dove si è indicata con:
\begin{equation}
\psi_0(x')=\langle x'|0 \rangle.
\end{equation}
l'autofunzione corrispondente allo stato fondamentale dell'oscillatore. \\
Ponendo:
\begin{equation}
x_0=\sqrt{\frac{\hbar}{mw}} \qquad \textrm{e} \qquad \xi=\frac{x'}{x_0},
\end{equation}
si può riscrivere l'equazione (\ref{eq:cap11_12}) nella forma:
\begin{equation}  \label{eq:cap11_13}
a\psi_o(\xi)=\frac{1}{\sqrt{2}}(\xi+\frac{d}{d\xi})\psi_0(\xi)=0.
\end{equation}
Per inciso vediamo che \textbf{nella rappresentazione delle coordinate, in termini della variabile adimensionale $\xi=x/x_0$, gli operatori di creazione e distruzione si esprimono come}:
\begin{equation}    \label{eq:cap11_14}
\begin{split}
	a=\frac{1}{\sqrt{2}} (\xi+\frac{d}{d\xi}) \\
	a^+=\frac{1}{\sqrt{2}} (\xi-\frac{d}{d\xi}) 
\end{split} \end{equation}
 L'equazione (\ref{eq:cap11_13}) si integra facilmente per separazione di variabili:
\begin{equation}
\frac{d\psi_0}{d\xi}=-\xi\psi_0 \rightarrow \frac{d\psi}{\psi_0}=-\xi d\xi \rightarrow ln\psi_0=-\frac{\xi^2}{2}+cost,
\end{equation}
ossia:
\begin{equation}
\psi_0(\xi)=Ce^{-\xi^2/2}.
\end{equation}
La costante C è determinata dalla \textbf{condizione di normalizzazione}:
\begin{equation}
\langle 0|0 \rangle=\int_{-\infty}^{+\infty} dx' \langle 0|x'\rangle\langle x'|0 \rangle=\int_{-\infty}^{+\infty} dx' |\psi_0(x')|^2=1.
\end{equation}
Troviamo in tal modo:
\begin{equation}
\int_{-\infty}^{+\infty} dx' |\psi_0(x')|^2=|C|^2x_0\int_{-\infty}^{+\infty} d\xi e^{-\xi^2}=|C|^2 x_0 \sqrt{\pi}=1,
\end{equation}
da cui, scegliendo C reale e positivo:
\begin{equation}
C=\frac{1}{\pi^{{1}/{4}}} \sqrt{x_0}.
\end{equation}
Pertanto \textbf{l'autofunzione normalizzata corrispondente allo stato fondamentale dell'oscillatore armonico è}:
\begin{equation}
\psi_0=\left( \frac{1}{\pi^{{1}/{4}}}  \right) e^{-\xi^2/2}, \qquad \qquad \xi=\frac{x}{x_0}.
\end{equation}
 Le equazioni (\ref{eq:cap11_10}) consentono poi di valutare le \textbf{autofunzioni dell'energia per gli stati eccitati.} Per la generica f.d.o dell'autostato $|n\rangle$ possiamo scrivere:
\begin{equation}
\psi_n(x')=\langle x'|n \rangle=\frac{1}{\sqrt{n!}}\langle x'|(a^+)^n|0 \rangle,
\end{equation}
da cui, utilizzando la rappresentazione espressa nell'equazione (\ref{eq:cap11_14}) per l'operatore $a^+$ ricaviamo:
\begin{equation} \label{eq:cap11_15}
\begin{split}
	\psi_n(\xi)=\frac{1}{\sqrt{2^nn!}}(\xi-\frac{d}{d\xi})^n \psi_0({\xi})=\\
	=\frac{1}{\pi^{1/4}\sqrt{2^nx_0n!}}(\xi-\frac{d}{d\xi})^n e^{-\xi^2/2}.
\end{split} \end{equation}

 Le funzioni H$_n({\xi})$ definite dall'equazione:
\begin{equation}
(\xi-\frac{d}{d\xi})^n e^{-\xi^2/2}=H_n(\xi)e^{-\xi^2/2},
\end{equation}
sono dei polinomi di grado $n$ in $\xi$ contenenti potenze della stessa parità del numero $n$, Queste funzioni dono dette \textbf{polinomi di Hermite.} L'equazione (\ref{eq:cap11_15}) si scrive allora, in termini di questi polinomi:
\begin{equation}
\psi_n(\xi)=\frac{1}{\pi^{1/4}\sqrt{2^nx_0n!}} H_n(\xi)e^{-\xi^2/2}.
\end{equation}








\newpage
\begin{figure}[htbp]
\begin{center}
\includegraphics[width=\textwidth]{immagini/cap_11/polHer1.png}
\end{center}
\end{figure}

\begin{figure}[htbp]
\begin{center}
\includegraphics[width=\textwidth]{immagini/cap_11/polHer2.png}
\end{center}
\end{figure}
 %DEFINITIVO
\chapter[Simmetrie e Leggi di Conservazione]{Simmetrie e Leggi di\\ Conservazione} 
\section{Derivata di un operatore rispetto al tempo} 

\textbf{Il concetto di derivata di una grandezza fisica rispetto al tempo non può essere definito in meccanica quantistica nel senso che esso ha in meccanica classica. Infatti, la definizione della derivata in meccanica classica è legata alla considerazione dei valori della grandezza in due istanti vicini ma differenti. Nella meccanica quantistica, invece, una grandezza avente un valore determinato in un certo istante non ha, in generale, valore determinato negli istanti successivi.} In altri termini, se il vettore di stato del sistema considerato è un autostato di una determinata osservabile ad un certo istante, negli istanti successivi il vettore di stato non sarà più, in generale, autostato della stessa osservabile.\\

Dunque, il concetto di derivata rispetto al tempo deve essere definito in meccanica quantistica in modo diverso.
\textbf{\'E naturale definire la derivata $\mathbf{{dA}/{dt}}$ di A come grandezza il cui valore medio è uguale alla derivata rispetto al tempo del valor medio $\mathbf{\langle A \rangle} $.
Si ha dunque, per definizione}:
	\begin{equation}
		\tcboxmath[enhanced, sharp corners=downhill, colframe=black, colback=white, borderline={2pt}{-3pt}{black}]{
		\left\langle \frac{dA}{dt} \right\rangle = \frac{d}{dt} \langle A \rangle .
		}
	\end{equation}\\

Partendo da questa definizione non è difficile ottenere l'espressione dell'operatore quantistico $dA/dt$
	\begin{align}
	\label{eq:cap12_1}
	\left\langle \frac{dA}{dt} \right\rangle &= \langle \alpha,t| \frac{dA}{dt}| \alpha,t \rangle = \frac{d}{dt} \langle A \rangle = \frac{d}{dt} \langle \alpha ,t |A| \alpha ,t\rangle= \nonumber\\
	&=\left(\frac{\partial }{\partial{t}}\langle \alpha,t|\right) A|\alpha,t \rangle + \langle \alpha,t| \frac{\partial A}{\partial{t}}|\alpha,t \rangle + \langle\alpha,t|A \left(\frac{\partial }{\partial{t}}|\alpha,t\rangle\right).
	\end{align} 
In questa espressione ${\partial A}/{\partial{t}}$ è un operatore dedotto per derivazione dell'operatore A rispetto al tempo, dal quale quest'ultimo può dipendere come da un parametro.\\

Utilizzando l'equazione di Schr\"{o}dinger per il ket $|\alpha,t\rangle$ e per il bra corrispondente,   $\langle\alpha,t|$ :
	\begin{equation}
		\begin{cases}
		\displaystyle{i\hbar\frac{\partial }{\partial{t}}|\alpha,t\rangle= H|\alpha,t\rangle }\\[0.3cm]
	\displaystyle{-i\hbar\frac{\partial }{\partial{t}}\langle\alpha,t|= \langle\alpha,t|H } 
		\end{cases}
	\end{equation}
otteniamo dall'eq. (\ref{eq:cap12_1})
	\begin{align}
		\langle\frac{dA}{dt} \rangle &= \frac{i}{\hbar} \langle\alpha,t|HA|\alpha,t\rangle + \langle \alpha,t|\frac{\partial A}{\partial{t}}|\alpha,t \rangle -\frac{i}{\hbar} \langle \alpha,t|AH|\alpha,t\rangle=  \nonumber\\
		&= \langle \alpha,t| (\frac{\partial A}{\partial{t}} + \frac{i}{\hbar}[H,A] ) |\alpha,t\rangle .
	\end{align}
Per definizione di valore medio, l'espressione tra parentesi rappresenta l'operatore cercato $dA/dt$ :
	\begin{equation}
	\label{eq:cap12_2}
		\tcboxmath[enhanced, sharp corners=downhill, colframe=black, colback=white, borderline={2pt}{-3pt}{black}]{
			\frac{dA}{dt}= \frac{\partial A}{\partial{t}}+ \frac{i}{\hbar}[H,A].
			}
	\end{equation}\\

È istruttivo confrontare questo risultato con l'equazione del moto classico nella forma di parentesi di Poisson. \textbf{In meccanica classica}, per la derivata totale rispetto al tempo di una grandezza $f$, che è funzione delle coordinate e dei momenti generalizzati $q_i$ e $p_i$ del sistema, si ha:
	\begin{equation}
		\tcboxmath[sharp corners=downhill, colback=white, colframe=black]{
			\frac{df}{dt}= \frac{\partial f}{\partial{t}} + \sum_{i}^{}{(\frac{\partial H}{\partial{p_i}}\frac{\partial f}{\partial{q_i}} - \frac{\partial H}{\partial{q_i}}\frac{\partial f}{\partial{p_i}}} )=
\frac{\partial f}{\partial{t}}+ \{H,f\}.
			}
	\end{equation}
Nuovamente riscontriamo che la regola di corrispondenza di Dirac 
	\begin{equation}
		\tcboxmath[sharp corners=downhill, colback=white, colframe=black]{
			\{ \quad,\quad   \}_{classica}  \leftrightarrow \frac{i}{\hbar} [ \quad, \quad]
			}
	\end{equation}
porta all'equazione corretta in meccanica quantistica. È bene sottolineare, tuttavia, come l'eq (\ref{eq:cap12_2}) risulti valida anche quando la grandezza $A$ non ha analogo classico.\\

Una classe molto importante di grandezze fisiche è costituita da \textbf{quelle grandezze i cui operatori non dipendono esplicitamente dal tempo e che, inoltre, commutano con l'hamiltoniano} in modo tale che 
	\begin{equation}
		\tcboxmath[sharp corners=downhill, colback=white, colframe=black]{
			\frac{dA}{dt}= \frac{\partial A}{\partial{t}} + \frac{i}{\hbar}[H,A]=0.
			}
	\end{equation}
tali grandezze si chiamano \textbf{conservative};\\

Per le grandezze conservative 
	\begin{equation} 
		\tcboxmath[sharp corners=downhill, colback=white, colframe=black]{
			\left\langle \frac{dA}{dt}\right\rangle= \frac{d}{dt}\langle A \rangle =0,
			}
	\end{equation}
cioè 
	\begin{equation}
		\tcboxmath[sharp corners=downhill, colback=white, colframe=black]{
			\langle A \rangle= costante.
			}
	\end{equation}
In altri termini \textbf{il valore medio della grandezza resta costante nel tempo}. Si può ugualmente affermare che \textbf{se, nello stato dato, la grandezza $A$ ha un valore determinato (cioè se lo stato è un autostato dell'operatore $A$) essa avrà anche negli istanti successivi un valore determinato ed esattamente lo stesso\footnote{$A| a, t_0\rangle= a |a,t_0\rangle \Rightarrow A|a, t \rangle = Ae^{-\frac{i}{\hbar}(t-t_0)}|a,t_0\rangle=e^{-\frac{i}{\hbar}(t-t_0)}a|a,t_0\rangle=a|a,t\rangle$}.}\\

\textbf{L'hamiltoniano di un sistema isolato (nonchè di un sistema che si trova in un campo esteso costante e non variabile) non può contenere il tempo esplicitamente.} Ciò risulta dal fatto che tutti gli istanti sono equivalenti relativamente a tale sistema fisico. Pertanto, per un tale sistema  
	\begin{equation}
		\tcboxmath[sharp corners=downhill, colback=white, colframe=black]{
			\frac{\partial H}{\partial{t}}=0;
			}
	\end{equation}
d'altra parte, poiché ogni operatore, come è ovvio, commuta con se stesso possiamo concludere che \textbf{l'energia di un sistema che non si trova in un campo esteso variabile si conserva}:
	\begin{equation}
		\tcboxmath[sharp corners=downhill, colback=white, colframe=black]{
			\frac{dH}{dt}=0.
			}
	\end{equation}
Pertanto \textbf{in meccanica quantistica la legge di conservazione dell'energia significa che, se nello stato dato l'energia ha un valore determinato, questo valore resterà costante nel tempo}.

\section[Simmetrie e leggi di conservazione  in meccanica quantistica]{Simmetrie e leggi di conservazione  in meccanica \\quantistica} 

La legge di conservazione dell'energia, discussa nel paragrafo precedente, rappresenta un esempio specifico di \textbf{connessione tra simmetrie e leggi di conservazione}. Questa legge può essere infatti enunciata dicendo che,se un sistema fisico è invariante rispetto a traslazioni temporali, allora il corrispondente generatore della trasformazione, ossia l'hamiltoniano $H$, è una quantità conservata.\\

\textbf{Così come in meccanica classica, anche in meccanica quantistica la connessione tra simmetria e leggi di conservazione può essere stabilita in forma del tutto generale}.\\

Per discutere questa connessione consideriamo una generica trasformazione che è rappresentata, in meccanica quantistica, da un operatore unitario agente sui vettori di stato:
	\begin{equation}
	\label{eq:cap12_3}
		\tcboxmath[enhanced, sharp corners=downhill, colframe=black, colback=white, borderline={2pt}{-3pt}{black}]{
			|\alpha \rangle \rightarrow {U}|\alpha \rangle.
			}
	\end{equation}
Possiamo pensare che ${U}$ rappresenti l'operatore di traslazione spaziale $T(d\vec{x})$ oppure l'operatore di evoluzione temporale, ${U}(t,t_0)$.\\

Consideriamo come cambia, per effetto della trasformazione, un generico elemento di matrice di un operatore ${X}$ tra due stati $|\alpha \rangle$ e $|\beta\rangle$ :
	\begin{equation}
		\tcboxmath[sharp corners=downhill, colback=white, colframe=black]{
			\langle \beta|{X}|\alpha \rangle   \rightarrow    \langle \beta |{U}^+{X}{U}|\alpha \rangle.
			}
	\end{equation}\\

Vediamo allora che la trasformazione può essere pensata o come una trasformazione (\ref{eq:cap12_3})  sui vettori di stato, che lascia gli operatori invariati, o come trasformazione sugli operatori:
	\begin{equation}
	\label{eq:cap12_4}
		\tcboxmath[enhanced, sharp corners=downhill, colframe=black, colback=white, borderline={2pt}{-3pt}{black}]{
		{X}\rightarrow {U}^+ {X}{U},}
	\end{equation}
lasciando invariati i vettori di stato. Formalmente i due approcci sono completamente equivalenti.\\

Come esempio discutiamo il caso di una traslazione spaziale infinitesima operata dall'operatore $T(d\vec{x}^{\,\prime})$.
Nell'approccio utilizzato in precedenza, la trasformazione era pensata come una trasformazione sui vettori di stato che lascia invariati gli operatori. Così ad esempio:
	\begin{align}
		  |\alpha \rangle &\rightarrow (1- \frac{i}{\hbar}\vec{p}\cdot d\vec{x}^{\, \prime}) |\alpha \rangle ,  \\
		  \vec{x}&\rightarrow \vec{x};  
	\end{align}
tuttavia possiamo considerare la trasformazione come una trasformazione degli operatori, che lascia invariati i vettori di stato:
	\begin{align}
			|\alpha \rangle &\rightarrow  |\alpha \rangle , \\
			\vec{x} &\rightarrow \left(1+ \frac{i}{\hbar} \vec{p}\cdot d\vec{x}^{\, \prime}\right) \vec{x} \left(1-\frac{i}{\hbar} \vec{p} \cdot d\vec{x}\right) \approx  \nonumber\\
			& \quad \approx\vec{x}+ \frac{i}{\hbar}[\vec{p}\cdot d\vec{x}^{\, \prime},\vec{x}]=  \vec{x}+ d\vec{x}^{\, \prime} .
	\end{align}
È immediato mostrare come entrambi gli approcci portano allo stesso risultato per il valore di aspettazione di $\vec{x}$:
	\begin{equation}
		\langle \vec{x} \rangle  \mapsto \langle \vec{x}\rangle + \langle d\vec{x}^{\, \prime} \rangle.
	\end{equation}\\

Nel discutere la connessione tra simmetrie e leggi di conservazione è conveniente considerare le trasformazioni come agenti sugli operatori. \textbf{Consideriamo allora una generica trasformazione operata dall'operatore unitario ${U}$ e supponiamo che il sistema fisico considerato sia invariate rispetto a tale trasformazione, ossia che la trasformazione rappresenti una simmetria del sistema. In particolare, allora, dovrà risultare invariate per tale trasformazione l'hamiltoniano H del sistema, che ne descrive la dinamica}. Per quanto discusso questo comporta
	\begin{equation}
		\tcboxmath[enhanced, sharp corners=downhill, colframe=black, colback=white, borderline={2pt}{-3pt}{black}]{
			{U}^+H{U}= H,
			}
	\end{equation}
o, equivalentemente 
	\begin{equation}
	\label{eq:cap12_5}
		\tcboxmath[enhanced, sharp corners=downhill, colframe=black, colback=white, borderline={2pt}{-3pt}{black}]{
			[H,{U}]=0.
			}
	\end{equation}\\

Se consideriamo una \textbf{trasformazione infinitesima}, sappiamo che l'operatore unitario ${U}$ può essere scritto nella forma 
	\begin{equation}
		\tcboxmath[enhanced, sharp corners=downhill, colframe=black, colback=white, borderline={2pt}{-3pt}{black}]{
			{U}= 1- \frac{i\varepsilon}{\hbar}G,
			}
	\end{equation}
dove G è il generatore hamiltoniano della trasformazione considerata. In questo caso l'equazione (\ref{eq:cap12_5}) equivale a 
	\begin{equation} 
		\tcboxmath[enhanced, sharp corners=downhill, colframe=black, colback=white, borderline={2pt}{-3pt}{black}]{
			[H,G]=0.
			}
	\end{equation}
Per l'eq. (\ref{eq:cap12_4}) si ha allora
	\begin{equation}
		\tcboxmath[enhanced, sharp corners=downhill, colframe=black, colback=white, borderline={2pt}{-3pt}{black}]{
			\frac{dG}{dt}=0,
			}
	\end{equation}
e quindi \textbf{G è una costante del moto}. Questo risultato stabilisce la connessione tra simmetrie e leggi di conservazione nella meccanica quantistica.\\

Consideriamo ad esempio un \textbf{sistema isolato soggetto a campi esterni}. Poiché tutte le posizioni di tale sistema in blocco sono equivalenti nello spazio, si può affermare che l'hamiltoniano del sistema non cambia in uno spostamento arbitrario del sistema. In altri termini l'hamiltoniano commuta con l'operazione di traslazione
	\begin{equation}
		\tcboxmath[sharp corners=downhill, colback=white, colframe=black]{
			[H,T (d\vec{x}^{\, \prime})]=0.
			}
\end{equation}
Ma questo equivale a dire che \textbf{l'hamiltoniano commuta con l'operatore impulso $\vec{p}$} del sistema è una quantità conservata:
	\begin{equation}
		\tcboxmath[sharp corners=downhill, colback=white, colframe=black]{
			\frac{d\vec{p}}{dt}=0.
			}
	\end{equation}
Possiamo quindi affermare che \textbf{in meccanica quantistica,come in meccanica classica, la legge di conservazione dell'impulso di un sistema isolato è una diretta conseguenza dell'omogeneità dello spazio.}\\

In seguito mostreremo che, \textbf{come conseguenza dell'invarianza rispetto a rotazioni spaziali di un sistema isolato non soggetto a campi esterni, si conserva in meccanica quantistica, come in meccanica classica, il momento angolare del sistema. La legge di conservazione del momento angolare è dunque una conseguenza dell'isotropia dello spazio.}\\

\textbf{L'invarianza di un sistema isolato, o di un sistema soggetto a campi esterni variabili nel tempo, rispetto a traslazioni temporali è conseguenza dell'omogeneità del tempo. La conservazione dell'energia è dunque una diretta conseguenza di tale simmetria}.\\

Sin qui abbiamo discusso in particolare il caso delle \textbf{simmetrie continue}, ossia delle simmetrie associate a trasformazioni che possano essere ottenute applicando successivamente trasformazioni infinitesime.
Esistono tuttavia trasformazioni che non godono di questa proprietà, e che sono associate a cosiddette \textbf{simmetrie discrete}. In questo caso non esiste alcun generatore hamiltoniano associato alla trasformazione e dunque potrebbe non esistere alcuna osservabile corrispondente ad una quantità conservata. Per alcune simmetrie discrete, tuttavia, lo stesso operatore unitario $U$ che opera la trasformazione può essere al tempo stesso, un operatore hermitiano. In vista dell'eq (\ref{eq:cap12_5}), allora l'operatore $U$ corrisponde ad una quantità osservabile conservata. Questo è il caso, ad esempio, della \textbf{trasformazione di inversione spaziale o parità}. Abbiamo già discusso come la parità sia infatti una quantità conservata per sistemi in cui il campo di forze estese è descritto da un potenziale simmetrico rispetto ad inversione degli assi.

\section[Teorema di Ehrenfest]{Teorema di Ehrenfest}
Consideriamo l'operatore di Hamilton per una particella:
	\begin{equation}
		\tcboxmath[sharp corners=downhill, colback=white, colframe=black]{
			H= \frac{\vec{p}^2}{2m} + {V}(\vec{x}).
			}
	\end{equation}
Calcoliamo l'operatore $d\vec{x}/dt$. Poichè l'operatore di posizione $\vec{x}$ non dipende esplicitamente dal tempo, in virtù dell'eq (\ref{eq:cap12_4}) si ha:
	\begin{equation}
		\frac{d\vec{x}}{dt}= \frac{i}{\hbar}[H,\vec{x}].
	\end{equation}\\

L'unico operatore tra i componenti di H che non commuta con l'operatore di posizione è $\vec{p}^2$. Si ha allora 
	\begin{align}
		\frac{dx_i}{dt}&=\frac{i}{\hbar} \left[ \frac{p_i^2}{2m},x_i \right]= \frac{i}{2m\hbar}\left[ p_i ^2 ,x_i \right] = \frac{i}{2m\hbar}\left(p_i\left[ p_i, x_i\right]+\left[p_i , x_i \right]p_i\right)= \nonumber \\
		&= \frac{i}{2m\hbar }\left( -i\hbar \right) \left( p_i +p_i \right)=\frac{p_i}{m},
	\end{align}
ossia
	\begin{equation}
   	\label{eq:cap12_6}
		\tcboxmath[enhanced, sharp corners=downhill, colframe=black, colback=white, borderline={2pt}{-3pt}{black}]{
			\frac{d\vec{x}}{dt}=\frac{\vec{p}}{m}.
			}
	\end{equation}\\
	
Calcoliamo ora l'operatore ${d\vec{p}}/{dt}$. Otteniamo facilmente il risultato cercato utilizzando per gli operatori la loro espressione nella rappresentazione delle coordinate:\\
	\begin{align}
		\frac{d\vec{p}}{dt}&=\frac{i}{\hbar}[\vec{H},\vec{p}]=\frac{i}{\hbar}[V(\vec{x}),\vec{p}]=  \frac{i}{\hbar}[V(\vec{x}),-i\hbar\vec{\nabla}]= \nonumber \\
		&= (V(\vec{x})\vec{\nabla}-\vec{\nabla}V(\vec{x}))= V(\vec{x})\vec{\nabla}-(\vec{\nabla}V(\vec{x}))-V	\vec{x}\vec{\nabla},
	\end{align}
ossia:
	\begin{equation}
	\label{eq:cap12_7}
		\tcboxmath[enhanced, sharp corners=downhill, colframe=black, colback=white, borderline={2pt}{-3pt}{black}]{
			\frac{d\vec{p}}{dt}=-\vec{\nabla}V(\vec{x}).
			}
	\end{equation}\\
	
Le due equazioni, (\ref{eq:cap12_6}) e (\ref{eq:cap12_7}), possono essere combinate insieme per ottenere:\\
	\begin{equation}
		\tcboxmath[enhanced, sharp corners=downhill, colframe=black, colback=white, borderline={2pt}{-3pt}{black}]{
			m\frac{d^2\vec{x}}{dt^2}=\frac{d\vec{p}}{dt}=-\vec{\nabla}V(\vec{x}),
			}
	\end{equation}
che esprimono, in forma operatoriale, la legge di Newton. Questo risultato è noto come \textbf{Teorema di Ehrenfest}. 

\section[Rappresentazioni di Schrödinger e di Heisenberg]{Rappresentazioni di Schrödinger e di Heisenberg; equazioni  del moto di Heisenberg}

Nel discutere la dinamica in meccanica quantistica, abbiamo considerato come i vettori di stato evolvono nel tempo secondo l'equazione di Schrödinger. Questo significa che abbiamo considerato la trasformazione di evoluzione temporale come una trasformazione applicata ai vettori di stato, che lascia invariati gli operatori. Questo approccio è noto come \textbf{rappresentazione~di~Schr\"{o}dinger}.\\

In accordo con la precedente discussione generale, possiamo considerare la trasformazione di evoluzione temporale nell'approccio alternativo, e completamente equivalente, secondo cui la trasformazione è applicata agli operatori, mentre i vettori di stato restano invariati nel tempo. Questo approccio è noto come $\textbf{rappresentazione~di~Heisenberg}$.\\

Consideriamo allora l'operatore di evoluzione temporale $U(t, t_0)$ e poniamo per semplicità $t_0 = 0$ :
	\begin{equation}
		\tcboxmath[sharp corners=downhill, colback=white, colframe=black]{
			U(t) \equiv U(t,t_0 = 0) = e^{-\frac{i}{\hbar}Ht}.
			}
	\end{equation}
	Nella rappresentazione di Schrödinger gli stati evolvono nel tempo e gli operatori restano invariati:
	\begin{equation}
		\tcboxmath[enhanced, sharp corners=downhill, colframe=black, colback=white, borderline={2pt}{-3pt}{black}]{
		\begin{aligned}
			&\ket{\alpha, t_0 = 0}_S \rightarrow \ket{\alpha, t}_S = U(t) \ket{\alpha, t=0}_S ,\\
			&A^{(S)} \rightarrow A^{(S)}.
		\end{aligned}
		}
	\end{equation}
Nella rappresentazione di Heisenberg, viceversa, gli stati restano invariati e gli operatori evolvono nel tempo:
	\begin{equation}
		\tcboxmath[enhanced, sharp corners=downhill, colframe=black, colback=white, borderline={2pt}{-3pt}{black}]{
		\begin{aligned}
			&\ket{\alpha}_H \rightarrow \ket{\alpha}_H, \\
&A^{(H)}(t_0 = 0) \rightarrow A^{(H)}(t) = U^\textbf{+}(t)A^{(H)}(t_0 = 0)U(t).
		\end{aligned}
		}
	\end{equation} \\

Per definizione i vettori di stato e gli operatori coincidono nelle rappresentazioni di Schrödinger e di Heisenberg al tempo $t_0 = 0$ :
	\begin{equation}
		\tcboxmath[sharp corners=downhill, colback=white, colframe=black]{	
		\begin{aligned}
			\ket{\alpha}_H = \ket{\alpha, t_0 = 0}_S, \\
			A^{(H)}(t_0 = 0) = A^{(S)}.
		\end{aligned}
		}
	\end{equation}\\

Il valore di aspettazione di un generico operatore $A$ su qualunque stato $\ket{\alpha}$ è ovviamente identico nelle due rappresentazioni:
	\begin{equation}
	{}_S\langle \alpha , t \vert A^{(S)} \vert \alpha , t \rangle _S = {}_S\langle \alpha, t_0=0\vert U^\textbf{+}(t) A^{(S)}U(t)\vert\alpha,t_0=0\rangle _S ={}_H\langle \alpha\vert A^{(H)}(t)\vert \alpha\rangle _H.
	\end{equation} \\

Così come nella rappresentazione di Schrödinger l'evoluzione temporale degli stati è definita dall'equazione di Schrödinger, in modo analogo è possibile derivare, nella rappresentazione di Heisenberg, un'equazione fondamentale che definisce l'evoluzione temporale degli operatori. Questa equazione può essere derivata derivando rispetto al tempo l'operatore $A^{(H)}(t)$ nella rappresentazione di Heisenberg:
	\begin{align}
		& \frac{dA^{(H)}(t)}{dt} = \frac{d}{dt}\left[U^\textbf{+}(t)\  A^{(H)}(t_0=0)\ U(t)\right] = \nonumber \\
		&= \frac{\partial U^\textbf{+}(t)}{dt} A^{(H)}(t_0=0)\ U(t) + U^\textbf{+}(t)\  A^{(H)}(t_0=0)\ \frac{\partial U(t)}{dt} = \nonumber \\
		&= \frac{i}{\hbar} H\ U^\textbf{+}(t)\ A^{(H)}(t_0=0)\ U(t) + U^\textbf{+}(t)\ A^{(H)}(t_0=0) \left(-\frac{i}{h}\right) H\ U(t) = \nonumber \\
		&= \frac{i}{\hbar} H\ A^{(H)}(t) - \frac{i}{h} A^{(H)}(t)\ H,
	\end{align}
dove si è supposto che l'operatore $A$, così come l'hamiltoniana $H$, non dipendano esplicitamente (ossia parametricamente) dal tempo. Dunque:
	\begin{equation}
	\label{eq:cap13_1}
		\tcboxmath[enhanced, sharp corners=downhill, colframe=black, colback=white, borderline={2pt}{-3pt}{black}]{
			\frac{d A^{(H)}(t)}{dt} = \frac{i}{h} \left[H, A^{(H)}(t) \right];
			}
	\end{equation}
questa equazione è nota come $\textbf{equazione del moto di Heisenberg}$.\\
 
Osserviamo come l'equazione del moto di Heisenberg sia formalmente analoga all'equazione (\ref{eq:cap12_2}), ma il suo significato è alquanto differente: l'eq.~(\ref{eq:cap12_2}) rappresenta la definizione dell'operatore $dA/dt$ della grandezza fisica corrispondente $dA/dt$, mentre il primo membro dell'equazione del moto di Heisenberg contiene la derivata rispetto al tempo dell'operatore della grandezza stessa $A$. Osserviamo anche che l'operatore hamiltoniano coincide nelle rappresentazioni di Schrödinger e di Heisenberg: $U^\textbf{+} H U = H$. Per questa ragione nell'eq.~(\ref{eq:cap13_1}) non abbiamo specificato la rappresentazione.
\section{Teorema del viriale in Meccanica Quantistica}
	\begin{equation}
	\label{eq:cap_13_2}
		\frac{d}{dt}\left( \sum _i x_i p_i \right) = \sum _i\dot{x}_i p_i + \sum _i	x_i \dot{p}_i = \sum _i \frac{p_i ^2}{m}- \sum _i \frac{\partial V}{\partial x_i} = 2T - \sum _i x_i \frac{\partial V}{\partial x_i}-
	\end{equation}
Se $V$ è una funzione omogenea delle coordinate allora:
	\begin{equation}
		\sum _ i x_i \frac{\partial V}{\partial x_i} = KV.
	\end{equation}
Prendendo dall'eq.~(\ref{eq:cap_13_2}) il valore medio su un autostato $\vert n \rangle$ di $H$ e considerando che per qualunque operatore $A$
	\begin{equation}
		\left\langle \frac{dA}{dt} \right\rangle _n = \langle n \vert \frac{dA}{dt}\vert n \rangle = \frac{d}{dt}\langle n \vert A \vert n \rangle = 0, 
	\end{equation}
si ottiene
	\begin{equation}
		\tcboxmath[sharp corners=downhill, colback=white, colframe=black]{
		\langle T \rangle _n = \frac{1}{2} K \langle V \rangle _n.
		}
	\end{equation}
Esempi:\begin{itemize}
\item Oscillatore armonico ($K=2$);
\item Potenziale coulombiano ($K=-1$).
\end{itemize}
 %DEFINITIVO
\chapter[Rappresentazioni di Schrödinger e di Heisenberg]{Rappresentazioni\\ di Schrödinger e di\\ Heisenberg ed equazioni\\ del moto di Heisenberg\footnote{S22, LL13}}

Nel discutere la dinamica in meccanica quantistica, abbiamo considerato come i vettori di stato evolvono nel tempo secondo l'equazione di Schrödinger. Questo significa che abbiamo considerato la trasformazione di evoluzione temporale come una trasformazione applicata ai vettori di stato, che lascia invariati gli operatori. Questo approccio è noto come \textbf{rappresentazione~di~Schr\"{o}dinger}.\\
In accordo con la precedente discussione generale, possiamo considerare la trasformazione di evoluzione temporale nell'approccio alternativo, e completamente equivalente, secondo cui la trasformazione è applicata agli operatori, mentre i vettori di stato restano invariati nel tempo. Questo approccio è noto come $\textbf{rappresentazione~di~Heisenberg}$.\\
\\
\noindent Consideriamo allora l'operatore di evoluzione temporale $U(t, t_0)$ e poniamo per semplicità $t_0 = 0$ :
 
\begin{equation}
U(t) \equiv U(t,t_0 = 0) = e^{-\frac{i}{\hbar}Ht}.
\end{equation}
\\
\noindent Nella rappresentazione di Schrödinger gli stati evolvono nel tempo e gli operatori restano invariati:

\begin{align}
\ket{\alpha, t_0 = 0}_S &\rightarrow \ket{\alpha, t}_S = U(t) \ket{\alpha, t=0}_S \nonumber \\
A^{(S)} &\rightarrow A^{(S)}.
\end{align}
\\
\noindent Nella rappresentazione di Heisenberg, viceversa, gli stati restano invariati e gli operatori evolvono nel tempo:

\begin{align}
&\ket{\alpha}_H \rightarrow \ket{\alpha}_H, \\ \nonumber
&A^{(H)}(t_0 = 0) \rightarrow A^{(H)}(t) = U^\textbf{+}(t)A^{(H)}(t_0 = 0)U(t).
\end{align}
\\
\noindent Per definizione i vettori di stato e gli operatori coincidono nelle rappresentazioni di Schrödinger e di Heisenberg al tempo $t_0 = 0$ :

\begin{align}
\ket{\alpha}_H = \ket{\alpha, t_0 = 0}_S, \\ \nonumber
A^{(H)}(t_0 = 0) = A^{(S)}.
\end{align}
\\
\noindent Il valore di aspettazione di un generico operatore $A$ su qualunque stato $\ket{\alpha}$ è ovviamente identico nelle due rappresentazioni:

\begin{eqnarray}
~_S\braket{\alpha,t|A^{(S)}|\alpha,t}_S &=& _S\langle \alpha, t_0=0\vert U^\textbf{+}(t) A^{(S)}U(t)\vert\alpha,t_0=0\rangle _S \nonumber \\
&=&~_H\braket{\alpha|A^{(H)}(t)|\alpha}_H.
\end{eqnarray}
\\
\noindent Così come nella rappresentazione di Schrödinger l'evoluzione temporale degli stati è definita dall'equazione di Schrödinger, in modo analogo è possibile derivare, nella rappresentazione di Heisenberg, un'equazione fondamentale che definisce l'evoluzione temporale degli operatori. Questa equazione può essere derivata derivando rispetto al tempo l'operatore $A^{(H)}(t)$ nella rappresentazione di Heisenberg:

\begin{align}
& \frac{dA^{(H)}(t)}{dt} = \frac{d}{dt}\left[U^\textbf{+}(t) A^{(H)}(t_0=0)U(t)\right] = \nonumber \\
&= \frac{\partial U^\textbf{+}(t)}{dt} A^{(H)}(t_0=0) U(t) + U^\textbf{+}(t) A^{(H)}(t_0=0) \frac{\partial U(t)}{dt} = \nonumber \\
&= \frac{i}{\hbar} H U^\textbf{+}(t) A^{(H)}(t_0=0) U(t) + U^\textbf{+}(t) A^{(H)}(t_0=0) \left(-\frac{i}{h}\right) H U(t) = \nonumber \\
&= \frac{i}{\hbar} H A^{(H)}(t) - \frac{i}{h} A^{(H)}(t) H,
\end{align}

\noindent dove si è supposto che l'operatore $A$, così come l'hamiltoniana $H$, non dipendano esplicitamente (ossia parametricamente) dal tempo. Dunque:

\begin{equation} \label{eq:cap13_1}
\frac{d A^{(H)}(t)}{dt} = \frac{i}{h} \left[H, A^{(H)}(t) \right].
\end{equation}

\noindent Questa equazione è nota come $\textbf{equazione del moto di Heisenberg}$.\\
\\
Osserviamo come l'equazione del moto di Heisenberg sia formalmente analoga all'equazione (\ref{eq:cap12_2}), ma il suo significato è alquanto differente: l'eq. (\ref{eq:cap12_2}) rappresenta la definizione dell'operatore $dA/dt$ della grandezza fisica corrispondente $dA/dt$, mentre il primo membro dell'equazione del moto di Heisenberg contiene la derivata rispetto al tempo dell'operatore della grandezza stessa $A$.\\
Osserviamo anche che l'operatore hamiltoniano coincide nelle rappresentazioni di Schrödinger e di Heisenberg: $U^\textbf{+} H U = H$. Per questa ragione nell'eq. (\ref{eq:cap13_1}) non abbiamo specificato la rappresentazione.
 %DEFINITIVO
\pagestyle{VS}
\chapter[T.d.P. indipendenti dal tempo]{Teoria delle perturbazioni indipendenti dal tempo}
La soluzione esatta dell'equazione di Schr\"{o}dinger può essere trovata solamente per un numero relativamente piccolo di casi molto semplici.\\

Tuttavia, nelle condizioni del problema figurano spesso \textbf{grandezze piccole} trascurando le quali il problema si semplifica in modo tale da rendere possibile una soluzione esatta. Allora il primo passo nella risoluzione del problema fisico posto consiste nel trovare la soluzione esatta del problema ma semplificato e il secondo nel calcolare, in modo approssimato, le correzioni dovute ai termini piccoli trascurati nel problema semplificato.\\

Il metodo generale che permette di calcolare queste correzioni prende il nome di \textbf{teoria delle perturbazioni}.\\

Supponiamo che l'hamiltoniano del sistema fisico considerato abbia la forma
	\begin{equation}
		\tcboxmath[enhanced, sharp corners=downhill, colback=yellow!50!white, colframe=red!75!black, borderline={2pt}{-3pt}{red!50!black}]{
			H= H_0+ V,
			}
	\end{equation}
dove V è una piccola correzione (\textbf{perturbazione}) dell'operatore ``\textbf{imperturbato}'' $H_0$. Le condizioni necessarie perché l'operatore $V$ possa essere considerato come ``piccolo'' rispetto all'operatore $H_0$ saranno dedotte più avanti.\\

La risoluzione del problema mediante la teoria delle perturbazioni dipende in maniera essenziale dalla degenerazione o meno dei livelli di energia del sistema imperturbato, descritto dall'hamiltoniano $H_0$. I due casi devono dunque essere trattati separatamente.
\section{Caso non degenere}
Supponiamo che siano noti gli autostati $\vert n^{(0)} \rangle$ e gli autovalori $E_n ^{(0)}$ dell'operatore imperturbato $H_0$, cioè che siano note le soluzioni esatte dell'equazione
	\begin{equation}
		\tcboxmath[enhanced, sharp corners=downhill, colback=yellow!50!white, colframe=red!75!black, borderline={2pt}{-3pt}{red!50!black}]{
			H_0\vert n^{(0)} \rangle=E_n ^{(0)}\vert n^{(0)} \rangle.
			}
	\label{eq:cap14_1}
	\end{equation}
Assumiamo qui che \textbf{gli autovalori} $E_n ^{(0)}$ appartengano allo \textbf{spettro discreto} e siano \textbf{non degeneri}\footnote{per semplicità assumeremo dapprima che esiste uno spettro discreto di livelli energetici.}. Il problema posto consiste nel trovare le soluzioni approssimate dell'equazione:
	\begin{equation}
		\tcboxmath[enhanced, sharp corners=downhill, colback=yellow!50!white, colframe=red!75!black, borderline={2pt}{-3pt}{red!50!black}]{
			H\vert n \rangle=\left( H_0+ V \right) \vert n \rangle= E_n \vert n \rangle,
			}
	\label{eq:cap14_2}
	\end{equation}
cioè le espressioni approssimate degli autostati $\vert n \rangle$ e degli autovalori $E_n$ dell'operatore perturbato $H$.\\

È comodo condurre i calcoli sin dall'inizio in forma matriciale. A tale scopo sviluppiamo gli autostati cercati $\vert n \rangle$ in serie di autostati $\vert n^{(0)} \rangle$:
	\begin{equation}
		\tcboxmath[enhanced, sharp corners=downhill, colback=yellow!50!white, colframe=red!75!black, borderline={2pt}{-3pt}{red!50!black}]{
			\vert n \rangle= \sum _m c_m \vert m^{(0)} \rangle.
			}
	\label{eq:cap14_3}
	\end{equation}
Sostituendo questo sviluppo nella (\ref{eq:cap14_2}) si ottiene:
	\begin{equation}
		\sum _m c_m \left( H_0+V\right)\vert m^{(0)} \rangle =\sum _m c_m \left( E_m ^{(0)}+V\right)\vert m^{(0)} \rangle= E_n\sum _m c_m \vert m^{(0)} \rangle,
	\end{equation}
ossia
	\begin{equation}
		\sum _m c_m \left( E_n-E_m ^{(0)}\right)\vert m^{(0)}\rangle=\sum _m c_m\ V\vert m^{(0)}\rangle.
	\end{equation}
Moltiplicando quindi entrambi i membri di questa uguaglianza per il bra $\langle k^{(0)}\vert$ si trova:
	\begin{equation}
		\tcboxmath[sharp corners=downhill, colback=white, colframe=red!75!black]{
			\left( E_n-E_k ^{(0)}\right)c_k =\sum _m \langle k^{(0)}\vert V\vert m^{(0)}\rangle c_m.
			}
	\label{eq:cap14_4}
	\end{equation}\\
	
Introduciamo, per comodità di notazione, gli elementi di matrice $V_{km}$ della perturbazione $V$ nella base degli autostati imperturbati:
	\begin{equation}
		\tcboxmath[sharp corners=downhill, colback=white, colframe=red!75!black]{
			V_{km} = \langle k^{(0)}\vert V\vert m^{(0)}\rangle.
			}
	\end{equation}\\
	
L'eq. (\ref{eq:cap14_4}) si scrive allora nella forma:
	\begin{equation}
		\tcboxmath[enhanced, sharp corners=downhill, colback=yellow!50!white, colframe=red!75!black, borderline={2pt}{-3pt}{red!50!black}]{	(E_n - E_k ^{(0)}) c_k = \sum _m V_{km}\ c_m .
		}
	\label{eq:cap14_5}
	\end{equation}
Osserviamo che questa equazione, le cui incognite sono rappresentate dai coefficienti $c_m$ dello sviluppo (\ref{eq:cap14_3}) e dagli autovalori $E_n$ dell'hamiltoniano imperturbato, è un'equazione esatta.\\

\textbf{Cerchiamo ora i valori dei coefficienti} $c_m$ \textbf{e dell'energia} $E_n$ \textbf{sotto forma di serie}:
	\begin{equation}
		\tcboxmath[enhanced, sharp corners=downhill, colback=yellow!50!white, colframe=red!75!black, borderline={2pt}{-3pt}{red!50!black}]{
		\begin{aligned}
			& E_n = E_n ^{(0)}+E_n ^{(1)}+E_n ^{(2)}+\dots  \\
			& c_m = c_m ^{(0)}+c_m ^{(1)}+c_m ^{(2)}+\dots
		\end{aligned}
		}
	\end{equation}
dove le quantità $E_n ^{(1)}$, $c_m ^{(1)}$ sono dello stesso ordine della perturbazione $V$, le quantità $E_n ^{(2)}$, $c_m ^{(2)}$ sono del secondo ordine, eccetera. Allora, evidentemente, $E_n ^{(0)}$ coincide con l'autovalore di energia imperturbato.\\

Per determinare le quantità $E_n ^{(2)}$ e $c_m ^{(2)}$ risolviamo l'equazione (\ref{eq:cap14_3}) ordine per ordine. \textbf{All'ordine zero}, si ha:\\

$\bullet$ \textbf{[Ordine 0]}\\
	\begin{equation}
		\left(E_n ^{(0)}- E_k ^{(0)}\right)c_k ^{(0)}=0,
	\end{equation}
che fornisce evidentemente
	\begin{equation}
		\tcboxmath[sharp corners=downhill, colback=white, colframe=red!75!black]{	
			c_k ^{(0)}=0, per k\neq n.
			}
	\label{eq:cap14_6}
	\end{equation}
Quanto al coefficiente $c_n$ all'ordine zero, questo è determinato dalla condizione di normalizzazione
	\begin{equation}
		\langle n \vert n \rangle = \sum _m \vert c_m \vert ^2 =1.
	\end{equation}
Scegliendo \textbf{$c_n$ reale e positivo}, questa condizione all'ordine zero fornisce
	\begin{equation}
		\tcboxmath[sharp corners=downhill, colback=white, colframe=red!75!black]{
			c_n ^{(0)}=1.
			}
	\label{eq:cap14_7}
	\end{equation}\\
	
Consideriamo ora l'eq.~(\ref{eq:cap14_5}) \textbf{al primo ordine} dello sviluppo perturbativo:\\

$\bullet$ \textbf{[Ordine 1]}\\
	\begin{equation}
		E_n ^{(1)}c_k ^{(0)}+\left(E_n ^{(0)}- E_k ^{(0)}\right)c_k ^{(1)}=\sum _m V_{km}\ c_m ^{(0)} = V_{kn},
	\label{eq:cap14_8}
	\end{equation}
dove, a secondo membro, si sono sostituiti i risultati (\ref{eq:cap14_6}) e (\ref{eq:cap14_7}) per i coefficienti di ordine zero. L'eq.~(\ref{eq:cap14_8}) con $k=n$ dà:
	\begin{equation}
		\tcboxmath[enhanced, sharp corners=downhill, colback=yellow!50!white, colframe=red!75!black, borderline={2pt}{-3pt}{red!50!black}]{
			E_n  ^{(1)} = V_{nn} = \langle n^{(0)}\vert V \vert n^{(0)} \rangle ;
			}
	\end{equation}
pertanto \textbf{in prima approssimazione la correzione all'autovalore} $E_n ^{(0)}$ \textbf{è uguale al valore medio della perturbazione nello stato} $\vert n^{(0)} \rangle$.\\

L'eq.~(\ref{eq:cap14_8})  con $k\neq n$ fornisce:
	\begin{equation}
		\tcboxmath[enhanced, sharp corners=downhill, colback=yellow!50!white, colframe=red!75!black, borderline={2pt}{-3pt}{red!50!black}]{
			c_k ^{(1)} = \frac{V_{kn}}{E_n ^{(0)}-E_k ^{(0)}}=\frac{\langle k^{(0)}\vert V \vert n^{(0)} \rangle}{E_n ^{(0)}-E_k ^{(0)}}, }\qquad \qquad (k\neq n).			
	\label{eq:cap14_9}
	\end{equation}\\
	  
Quanto al coefficiente $c_n ^{(1)}$ che ricordiamo per convenzione abbiamo scelto essere reale, questo è fissato nuovamente dalla condizione di normalizzazione che a meno di termini del secondo ordine fornisce:
	\begin{equation}
		1= \sum _m \vert c_m \vert ^2 = \left( 1+ c_n ^{(1)}\right) ^2+ \sum _{m\neq n } \vert c_m ^{(1)} \vert ^2 \simeq \left( 1+ 2c_n ^{(1)}\right),
	\end{equation}
ossia
	\begin{equation}
		\tcboxmath[enhanced, sharp corners=downhill, colback=yellow!50!white, colframe=red!75!black, borderline={2pt}{-3pt}{red!50!black}]{
			c_n ^{(1)} =0.
			}
	\label{eq:cap14_10}
	\end{equation}\\
	
La formula (\ref{eq:cap14_9}) dà la correzione in prima approssimazione agli autostati dell'hamiltoniano. Da essa, tra l'altro, si vede quali sono le \textbf{condizioni di applicabilità del metodo considerato}. Precisamente, dovendo risultare i coefficienti al primo ordine molto minori del coefficiente di ordine zero ($c_n ^{(0)} =1$) deve valere la diseguaglianza
	\begin{equation}
		\tcboxmath[enhanced, sharp corners=downhill, colback=yellow!50!white, colframe=red!75!black, borderline={2pt}{-3pt}{red!50!black}]{
			\vert V_{kn} \vert \ll E_ n ^{(0)}-E_ k ^{(0)},
		}
	\end{equation}
cioè \textbf{gli elementi di matrice della perturbazione devono essere piccoli rispetto alle differenze corrispondenti dei livelli energetici imperturbati}.\\

Determiniamo ancora la correzione in seconda approssimazione all'autovalore $E_n ^{(0)}$. A tale scopo consideriamo l'equazione (\ref{eq:cap14_5}) per i termini del secondo ordine:\\

$\bullet$ \textbf{[Ordine 2]}\\
	\begin{equation}
			E_n^{(2)}c_k^{(0)}+E_n^{(1)}c_k^{(1)}+ \left( E_n^{(0)}-E_k ^{(0)}\right) c_k^{(2)}  = \sum _m V_{km} c_m ^{(1)} = \sum _{m\neq n} \frac{V_{km} V_{mn}}{E_n^{(0)}-E_m ^{(0)}},
	\label{eq:cap14_11}
	\end{equation}
dove abbiamo sostituito a secondo membro le espressioni (\ref{eq:cap14_9}) e (\ref{eq:cap14_10}) per i coefficienti di ordine uno. Scegliendo nell'eq.~(\ref{eq:cap14_11}) $k=n$ si ottiene:
	\begin{equation}
		\tcboxmath[enhanced, sharp corners=downhill, colback=yellow!50!white, colframe=red!75!black, borderline={2pt}{-3pt}{red!50!black}]{
			E_n ^{(2)} = \sum _{m \neq n } \frac{\vert V_{mn} \vert ^2}{E_n ^{(0)}-E_m ^{(0)}}.
			}
	\label{eq:cap14_12}
	\end{equation}
Possiamo allora riassumere i risultati ottenuti mediante le formule
	\begin{align}
		& \tcboxmath[enhanced, sharp corners=downhill, colback=yellow!50!white, colframe=red!75!black, borderline={2pt}{-3pt}{red!50!black}]{
			E_n = E_n ^{(0)}+ V_{nn} +\sum _{m \neq n } \frac{\vert V_{mn} \vert ^2}{E_n ^{(0)}-E_m ^{(0)}}+ \dots
			} \\[0.5cm]
		& \tcboxmath[enhanced, sharp corners=downhill, colback=yellow!50!white, colframe=red!75!black, borderline={2pt}{-3pt}{red!50!black}]{
 			\vert n \rangle = \vert n^{(0)} \rangle +\sum _{m \neq n } \frac{V_{mn} }{E_n ^{(0)}-E_m ^{(0)}}\, \vert m^{(0)}\rangle+ \dots 
			 }
	\end{align}
che esprimono gli \textbf{autovalori ed autovettori dell'hamiltoniano completo} $H$ \textbf{rispettivamente al secondo ed al primo ordine nella perturbazione}. le approssimazioni successive si possono calcolare in modo analogo.\\

I risultati ottenuti si generalizzano direttamente al \textbf{caso in cui l'operatore} $H_0$ \textbf{ha anche uno spettro continuo (si tratta però sempre di una perturbazione dello spettro discreto)}. A tale scopo occorre solamente aggiungere alle somme sullo spettro discreto gli integrali corrispondenti allo spettro continuo. Così, ad esempio l'eq. (\ref{eq:cap14_12}) si scrive:
	\begin{equation}
		E_n ^{(2)} = \sum _{m \neq n} \frac{\vert V_{mn} \vert ^2}{E_n ^{(0)}-E_m ^{(0)}}+ \int d\nu \frac{\vert V_{\nu n} \vert ^2}{E_n ^{(0)}-E_{\nu} ^{(0)}} 
	\end{equation}
\section{Caso degenere}
Vediamo ora il caso in cui l'operatore imperturbato $H_0$ ha \textbf{autovalori degeneri}. Indichiamo con
	\begin{equation}
		\vert n^{(0)} \rangle, \vert n^{\prime \,(0)} \rangle, \vert n^{\prime \prime \,(0)} \rangle, \dots
	\end{equation}
gli autostati relativi ad uno stesso autovalore $E_n ^{(0)}$. Come è noto, la scelta di questi autostati non è univoca: in luogo di essi si possono scegliere $s$ combinazioni lineari indipendenti di questi stati, dove $s$ è l'ordine di degenerazione del livello $E_n ^{(0)}$.\\

\textbf{Il metodo perturbativo sviluppato in precedenza non è più valido quando gli autostati dell'energia sono degeneri}. Nello sviluppo di questo metodo abbiamo infatti assunto l'esistenza di un unico e ben definito vettore di stato imperturbato $\vert n ^{(0)}\rangle$ a cui tende il vettore di stato perturbato quando la perturbazione $V$ tende a zero. In presenza di degenerazione, tuttavia, non è ovvio a priori a quale vettore di stato, combinazione lineare degli $\vert n^{\prime \,(0)}\rangle$, tende il ket perturbato in questo limite.\\

Per determinare il vettore di stato imperturbato cui tende il ket perturbato nel limite in cui la perturbazione tende a zero, e simultaneamente le correzioni al primo ordine dell'energia, consideriamo nuovamente l'eq.~(\ref{eq:cap14_5}):
	\begin{equation}
		\tcboxmath[enhanced, sharp corners=downhill, colback=yellow!50!white, colframe=red!75!black, borderline={2pt}{-3pt}{red!50!black}]{
			(E_n - E_k ^{(0)}) c_k = \sum _m V_{km}\ c_m.
			}
	\end{equation}
Studiamo in primo luogo l'equazione all'ordine zero:
	\begin{equation}
		\left( E_n ^{(0)} - E_k ^{(0)} \right) c_ k ^{(0)} =0,
	\end{equation}
che fornisce, evidentemente,
	\begin{equation}
		\tcboxmath[sharp corners=downhill, colback=white, colframe=red!75!black]{
			c_k ^{(0)} =0\quad \textrm{per} \quad k \neq n, n',\dots
			}
	\end{equation}
si ha infatti $E_n ^{(0)} = E_{n'} ^{(0)} = \dots $ per tutti i livelli degeneri.\\

Poniamo quindi nel'eq.~(\ref{eq:cap14_5}) $k=n$, e consideriamo i termini del primo ordine. Per le grandezze $c_k$ è allora sufficiente limitarsi all'approssimazione di ordine zero:
	\begin{equation}
		\tcboxmath[sharp corners=downhill, colback=white, colframe=red!75!black]{
		\begin{aligned}
			&c_n = c_n ^{(0)}, \quad c_{n'} = c_{n'} ^{(0)},\dots \\
			&c_m =0 \quad \textrm{per}\quad m\neq n, n', \dots 
		\end{aligned}
		}
	\end{equation}
Si ottiene allora
	\begin{equation}
		E_n ^{(1)}c_n ^{(0)}= \sum _{n'} V_{n n'}\ c_{n'} ^{(0)},
	\end{equation}
ossia
	\begin{equation}
		\tcboxmath[enhanced, sharp corners=downhill, colback=yellow!50!white, colframe=red!75!black, borderline={2pt}{-3pt}{red!50!black}]{
			\sum _{n'} \left( V_{n n'}- E_n ^{(1)}\delta _{nn'} \right) c_{n'} ^{(0)}=0,
			}
	\label{eq:cap14_13}
	\end{equation}
dove $n, n'$ assumono tutti i valori che numerano gli stati relativi all'operatore imperturbato $E_n ^{(0)}$.\\

Il sistema (\ref{eq:cap14_13}) rappresenta un sistema di equazioni lineari omogenee che ammette, rispetto alle grandezze $c_{n'} ^{(0)}$, soluzioni non nulle a condizione che il determinante formato con i coefficienti delle incognite si annulli. Si ottiene quindi l'equazione
	\begin{equation}
		\tcboxmath[enhanced, sharp corners=downhill, colback=yellow!50!white, colframe=red!75!black, borderline={2pt}{-3pt}{red!50!black}]{
				\det \left( V - E_n ^{(1)} I \right) =0,
			}
	\end{equation}
detta \textbf{equazione secolare}.\\

L'\textbf{equazione secolare} è un'equazione di grado $s$ in $E^{(1)}$ ed \textbf{ammette, in generale,} $s$ \textbf{radici reali distinte}. Sono precisamente queste radici che costituiscono le \textbf{correzioni agli autovalori in prima approssimazione}.\\
Sostituendo successivamente le radici dell'equazione secolare nel sistema (\ref{eq:cap14_13}) e risolvendo quest'ultimo, troviamo i coefficienti $c_n ^{(0)}$ ed otteniamo così \textbf{gli autostati nell'approssimazione zero}. Questi autostati rappresentano le particolari \textbf{combinazioni lineari di autostati degeneri} $\vert n^{\prime \,(0)}\rangle $ \textbf{di} $H_0$ \textbf{cui si riducono gli autostati perturbati di} $H$ \textbf{ nel limite in cui la perturbazione} $V$ \textbf{tende a zero}.\\

Per effetto della perturbazione, il livello energetico inizialmente degenere cessa di essere tale; le radici dell'equazione secolare sono infatti generalmente distinte. Si dice che \textbf{la perturbazione ``rimuove'' la degenerazione}. Questa rimozione della degenerazione puà essere completa o parziale.\\

Per il calcolo delle \textbf{correzioni di ordine superiore} agli autovalori ed agli autostati dell'hamiltoniano $H$ si procede in modo analogo al caso della teoria perturbativa non degenere. Si ottengono allora per queste correzioni, le stesse espressioni ottenute nel caso non degenere, con la sola differenza che, nella sommatoria, vengono esclusi tutti gli stati dell'hamiltoniano imperturbato che appartengono al sottospazio degenere.
 %DEFINITIVO
\pagestyle{VS}
\chapter[T.d.P. dipendenti dal tempo]{Teoria delle perturbazioni dipendenti dal tempo}
Consideriamo qui le \textbf{perturbazioni dipendenti esplicitamente dal tempo}, in presenza delle quali, cioè, l'hamiltoniano completo ha la forma
	\begin{equation}
		\tcboxmath[enhanced, sharp corners=downhill, colback=yellow!50!white, colframe=red!75!black, borderline={2pt}{-3pt}{red!50!black}]{					H=H_0+V(t),
		}
	\end{equation}
dove $H_0$ non contiene il tempo esplicitamente. Si assume inoltre risolta l'equazione di Schr\"{o}dinger per $V(t)=0$, nel senso che gli autostati dell'hamiltoniano $H_0$ ed i suoi autovalori, definiti dall'equazione
	\begin{equation}
		\tcboxmath[enhanced, sharp corners=downhill, colback=yellow!50!white, colframe=red!75!black, borderline={2pt}{-3pt}{red!50!black}]{
			H_0 \vert n ^{(0)} \rangle = E_n ^{(0)} \vert n ^{(0)} \rangle 
			}
	\end{equation}
sono noti completamente.\\

Notiamo che, essendo l'hamiltoniano $H$ dipendente dal tempo, non si può più parlare di correzioni agli autovalori dell'energia. \textbf{L'energia non si conserva}, e gli stati stazionari non esistono. Il problema consiste qui nel calcolo approssimato degli stati del sistema nella base degli stati stazionari del sistema imperturbato.\\

Scriviamo lo stato del sistema fisico al tempo $t=0$ come combinaizone lineare di autostati di $H_0$:
	\begin{equation}
		\tcboxmath[enhanced, sharp corners=downhill, colback=yellow!50!white, colframe=red!75!black, borderline={2pt}{-3pt}{red!50!black}]{
			\vert \alpha ,t=0  \rangle = \sum _k c_k (0) \vert k ^{(0)} \rangle .
			}
	\label{eq:cap15_1}
	\end{equation}
Lo \textbf{stato del sistema al tempo} $t>0$ potrà allora essere espresso come
	\begin{equation}
		\tcboxmath[enhanced, sharp corners=downhill, colback=yellow!50!white, colframe=red!75!black, borderline={2pt}{-3pt}{red!50!black}]{
			\vert \alpha ,t  \rangle = \sum _k c_k (t)\ e^{-\frac{i}{\hbar}E_k ^{(0)} t } \vert k ^{(0)} \rangle ,
			}
	\label{eq:cap15_2}
	\end{equation}
dove la disposizione temporale degli stati stazionari $\vert k ^{(0)} \rangle$, dovuta all'hamiltoniano imperturbato $H_0$, è stata esplicitamente separata dai coefficienti $c_k {(t)}$. In questo modo, l'evoluzione temporale dei coefficienti è dovuta esplicitamente alla presenza della perturbazione $V(t)$  (si dice che i coefficienti $c_k (t)$ definiscono lo sviluppo del vettore di stato $\vert \alpha , t \rangle $ nella ``rappresentazione di interazione'').\\

I coefficienti $c_k (t)$ soddisfano un insieme di equazioni che può essere ottenuto applicando al vettore di stato $\vert \alpha , t \rangle $ l'equazione di Schr\"{o}dinger dipendente dal tempo:
	\begin{align}
		i\hbar \frac{\partial}{\partial t} \vert \alpha , t \rangle &= \sum _ k \left( i\hbar \frac{dc_k}{dt}+ E_k ^{(0)} c_k\right)e^{-\frac{i}{\hbar}E_k ^{(0)} t } \vert k ^{(0)} \rangle =\nonumber \\
		&=  H\vert \alpha , t \rangle  = \left( H_0 + V\right)\sum _k c_ke^{-\frac{i}{\hbar}E_k ^{(0)} t } \vert k ^{(0)} \rangle  = \nonumber \\
		&=   \sum _k \left( E_k ^{(0)}+ V\right)c_k\ e^{-\frac{i}{\hbar}E_k ^{(0)} t } \vert k ^{(0)} \rangle ,
	\end{align}
da cui si ottiene
\begin{equation}
\sum _ k i\hbar \frac{dc_k}{dt} \ e^{-\frac{i}{\hbar}E_k ^{(0)} t } \vert k ^{(0)} \rangle = \sum _ k V c_k\ e^{-\frac{i}{\hbar}E_k ^{(0)} t } \vert k ^{(0)} \rangle .
\end{equation}\\

Moltiplicando entrambi i membri di questa equazione per il bra $\langle n^{(0)} \vert$ otteniamo infine il \textbf{sistema di equazioni esatte}
	\begin{equation}
		\tcboxmath[enhanced, sharp corners=downhill, colback=yellow!50!white, colframe=red!75!black, borderline={2pt}{-3pt}{red!50!black}]{
			i\hbar \frac{dc_n (t)}{dt} = \sum _ k V_{nk} (t) c_k\ e^{i\omega _{nk}t } ,
			}
	\label{eq:cap15_3}
	\end{equation}
dove si è posto
	\begin{equation}
		\tcboxmath[enhanced, sharp corners=downhill, colback=yellow!50!white, colframe=red!75!black, borderline={2pt}{-3pt}{red!50!black}]{
			\omega _{nk} = \frac{E_n ^{(0})-E_k ^{(0})}{\hbar}
			}
	\end{equation}
e $V_{kn} (t)$ sono gli elementi di matrice, dipendenti esplicitamente dal tempo,
	\begin{equation}
		\tcboxmath[enhanced, sharp corners=downhill, colback=yellow!50!white, colframe=red!75!black, borderline={2pt}{-3pt}{red!50!black}]{
			V_{nk} (t) = \langle n ^{(0)} \vert V(t) \vert k ^{(0)} \rangle .
			}
	\end{equation}\\
	
Ci proponiamo ora di risolvere le eq.~(\ref{eq:cap15_3}) utilizzando la \textbf{teoria delle perturbazioni}. A tale scopo sviluppiamo i coefficienti $c_n (t)$ nella serie
	\begin{equation}
		\tcboxmath[enhanced, sharp corners=downhill, colback=yellow!50!white, colframe=red!75!black, borderline={2pt}{-3pt}{red!50!black}]{
			c_n (t) = c_n ^{(0)}(t)+c_n ^{(1)}(t)+c_n ^{(2)}(t)+\dots
			}
	\end{equation}
dove il generico $c_n ^{(k)}(t)$ è di ordine $k$ nella perturbazione.\\

Sostituendo il precedente sviluppo nell'eq.~(\ref{eq:cap15_3}) e considerando i termini di \textbf{ordine zero} si ottiene
	\begin{equation}
		\tcboxmath[enhanced, sharp corners=downhill, colback=yellow!50!white, colframe=red!75!black, borderline={2pt}{-3pt}{red!50!black}]{
			i\hbar \frac{d c_n ^{(0)}(t)}{dt}=0 ,
			}
	\end{equation}
da cui
	\begin{equation}
		\tcboxmath[enhanced, sharp corners=downhill, colback=yellow!50!white, colframe=red!75!black, borderline={2pt}{-3pt}{red!50!black}]{
			c_n ^{(0)}(t) = \textrm{costante}= c_n ^{(0)} .
			}
	\label{eq:cap15_4}
	\end{equation}
Questo risultato è consistente con l'aver definito i coefficienti $c_n (t)$ in modo tale che la loro dipendenza esplicita dal tempo sia determinata esplicitamente dalla presenza della perturbazione $V(t)$.\\

Considerando nell'eq.~(\ref{eq:cap15_3}) i termini del \textbf{primo ordine} in $V$ si ottiene poi:
	\begin{equation}
		\tcboxmath[enhanced, sharp corners=downhill, colback=yellow!50!white, colframe=red!75!black, borderline={2pt}{-3pt}{red!50!black}]{
			i\hbar \frac{d c_n ^{(1)}(t)}{dt}=\sum _k V_{nk} (t) c_k (0)\ e^{i\omega _{nk} t}  ,
			}
	\end{equation}
avendo sostituito $c_k ^{(0)} (t)$ con $c_k (0)$, in accordo con l'eq.~(\ref{eq:cap15_4}). Questa espressione può essere esplicitamente integrata con la condizione al contorno $ c_n ^{(1)} (0) =0$, che segue direttamente dalla (\ref{eq:cap15_4}): $c_n (0) = c_n ^{(0)} (0)$. Si ottiene in tal modo:
	\begin{equation}
		\tcboxmath[enhanced, sharp corners=downhill, colback=yellow!50!white, colframe=red!75!black, borderline={2pt}{-3pt}{red!50!black}]{
			c_n ^{(1)} (t) = -\frac{i}{\hbar} \sum _k c_k (0) \int _0 ^t dt'\ V_{nk} (t') e ^{i \omega _{nk} t'} ,
			}
	\label{eq:cap15_5}
	\end{equation}
che determina lo stato del sistema al primo ordine dello sviluppo perturbativo. In molti casi pratici questa approssimazione risulta sufficiente e non ci soffermeremo qui a descrivere i termini di ordine più elevato.\\

Nel seguito considereremo sempre il caso in cui all'istante $t=0$ il sistema si trova in un autostato dell'hamiltoniana imperturbata $H_0$, ossia
	\begin{equation}
		\tcboxmath[enhanced, sharp corners=downhill, colback=yellow!50!white, colframe=red!75!black, borderline={2pt}{-3pt}{red!50!black}]{
			\vert \alpha , t=0\rangle = \vert i ^{(0)} \rangle
			}\quad \textrm{e dunque}\quad
		\tcboxmath[enhanced, sharp corners=downhill, colback=yellow!50!white, colframe=red!75!black, borderline={2pt}{-3pt}{red!50!black}]{
			\begin{cases}
			   c_i (0)=1, \\c_k (0)=0, \textrm{ per }k\neq i.
   			\end{cases}
   			}
	\end{equation}
In questo caso il modulo quadro del coefficiente $c_n (t)$ fornisce la probabilità che il sistema abbia effettuato, dopo un tempo $t$, una transizione nell'autostato $\vert n^{(0)} \rangle $ di $H_0$:
	\begin{equation}
		\tcboxmath[enhanced, sharp corners=downhill, colback=yellow!50!white, colframe=red!75!black, borderline={2pt}{-3pt}{red!50!black}]{
			P_{i\rightarrow n} (t) = \vert c_n (t) \vert ^2 = \vert c_n ^{(1)}(t)+c_n ^{(2)}(t)+\dots \vert ^2 ,
			}
	\end{equation}
per $n \neq i$. Al primo ordine $P_{i\rightarrow n} (t) \simeq \vert c_n  ^{(1)}(t) \vert ^2$ e l'eq.~(\ref{eq:cap15_5}) fornisce semplicemente
	\begin{equation}
		\tcboxmath[enhanced, sharp corners=downhill, colback=yellow!50!white, colframe=red!75!black, borderline={2pt}{-3pt}{red!50!black}]{
			\vert c_n  ^{(1)}(t) \vert ^2 = \frac{1}{\hbar ^2}\left\vert \int _0 ^t dt'\ V_{ni} (t') e ^{i \omega _{ni} t'} \right\vert ^2 .
			}
	\end{equation}
\section{Transizione per effetto di una perturbazione costante e relazione di indeterminazione tempo-energia}
Il metodo sviluppato è ancora valido quando l'energia di perturbazione $V$ non dipende esplicitamente dal tempo $t$. In questo caso potremmo, se lo volessimo, trattare il sistema  mediante la teoria delle perturbazioni indipendenti dal tempo e trovare i suoi stati stazionari. Tuttavia se ciò che vogliamo calcolare si riferisce esplicitamente al tempo, ad esempio dobbiamo calcolare le probabilità che il sistema si trovi in un certo stato ad un istante determinato, quando sia noto che esso si trovava in un certo stato ad un altro istante, allora il metodo della teoria delle perturbazioni dipendente dal tempo qui considerato risulta più conveniente. Assumiamo dunque
	\begin{equation}
		\tcboxmath[enhanced, sharp corners=downhill, colback=yellow!50!white, colframe=red!75!black, borderline={2pt}{-3pt}{red!50!black}]{
			V(t) = V \quad \textrm{(indipendente da }t\textrm{),}
			}
	\end{equation}
dove l'operatore $V$ è in generale funzione degli operatori impulso, posizione e spin.\\

Assumendo anche, come discusso in precedenza, che al tempo $t=0$ il sistema si trovi nello stato stazionario $\vert i^{(0)}\rangle $ dell'hamiltoniano imperturbato $H_0$, troviamo dall'eq.~(\ref{eq:cap15_5}) l'espressione al primo ordine per il coefficiente $c_n (t)$:
	\begin{align}
		c_n ^{(1)} (t) & =  -\frac{i}{\hbar} \int _0 ^t dt'\ V_{ni}\, e^{i \omega _{ni} t'} = \nonumber \\
		&= -\frac{i}{\hbar} V_{ni} \frac{e^{i \omega _{ni} t}-1}{i \omega _{ni}}= -\frac{2i}{\hbar} V_{ni}\ e^{i \omega _{ni} \frac{t}{2}}\ \frac{\sin{\left(\omega _{ni} \frac{t}{2}\right)}}{\omega _{ni}} .
	\end{align}\\
	
La probabilità di transizione $i\rightarrow n $, determinata al primo ordine dal modulo quadro $\vert c_n ^{(1)} (t) \vert ^2$, risulta dunque\footnote{Allo stesso risultato si giunge naturalmente utilizzando la teoria delle perturbazioni indipendente dal tempo evolvendo $ \vert \alpha , t =0 \rangle \rightarrow \vert \alpha , t \rangle$} (ricordando che $\omega _{ni} = ( E_n ^{(0)}- E_i ^{(0)})/ \hbar$)
	\begin{equation}
		\tcboxmath[enhanced, sharp corners=downhill, colback=yellow!50!white, colframe=red!75!black, borderline={2pt}{-3pt}{red!50!black}]{
			\vert c_n ^{(1)} (t) \vert ^2 = \frac{4\vert V_{ni} \vert ^2}{\left( E_n ^{(0)} - E_i ^{(0)} \right) ^2}\ \sin ^2\left(\frac{\left( E_n ^{(0)} - E_i ^{(0)} \right) t}{2 \hbar}\right).
			}
	\label{eq:cap15_6}
	\end{equation}\\
	
La probabilità di transizione allo stato $\vert n ^{(0)} \rangle$, dunque, oltre ad essere proporzionale all'elemento di matrice modulo quadro $\vert V_{ni}\vert ^2$, dipende dalla differenza delle energie imperturbate $(E_n ^{(0)} - E_i ^{(0)}) $. Per tempi sufficientemente grandi la funzione ha la forma
\begin{figure}[!htbp]
\begin{center}
\includegraphics[width=8cm]{immagini/cap_15/fig_15_1.png}
\end{center}
\end{figure}\\
ossia la probabilità di transizione differisce apprezzabilmente da zero solo per quegli stati finali che soddisfano
\begin{equation}
\vert \omega \vert = \frac{\vert E_n ^{(0)} - E_i ^{(0)}\vert}{\hbar} \leq \frac{2\pi}{t}.
\end{equation}
In altri termini, se indichiamo con $\Delta t$ l'intervallo di tempo durante il quale la perturbazione ha agito sul sistema e con $\Delta E = \vert E_n ^{(0)} - E_i ^{(0)}\vert$ la differenza delle energie imperturbate relative agli stati finale ed iniziale, si può avere con probabilità apprezzabile una transizione sole se
	\begin{equation}
		\tcboxmath[enhanced, sharp corners=downhill, colback=yellow!50!white, colframe=red!75!black, borderline={2pt}{-3pt}{red!50!black}]{
			\Delta E \Delta t \sim \hbar .
			}
	\label{eq:cap15_7}
	\end{equation}
Questa relazione è anche detta \textbf{relazione di indeterminazione tempo-energia}.\\

L'espressione (\ref{eq:cap15_6}) per la probabilità di transizione $i\rightarrow n$ e la condizione (\ref{eq:cap15_7}) che ne segue esprimono il seguente risultato: se $E_n ^{(0)}$ differisce apprezzabilmente da   $E_i ^{(0)}$, la probabilità di transizione $i \rightarrow n$ è piccola e tale rimane per tutti i valori di $t$. Questo risultato è richiesto dalla  \textbf{legge di conservazione dell'energia}. L'energia totale $E$ è costante, perché l'hamiltoniana $H$ non dipende esplicitamente dal tempo, pertanto l'energia propria $E^{(0)}$ (cioè l'energia che si ottiene trascurando la parte $V$ dovuta alla perturbazione), essendo approssimativamente uguale ad $E$, deve essere approssimativamente costante. Ciò significa che se $E^{(0)}$ ha inizialmente il valore $E _i ^{(0)}$, in qualsiasi istante successivo ci deve essere solo una probabilità piccola che esso abbia un valore considerevolmente diverso da $E _i ^{(0)}$.\\

La relazione (\ref{eq:cap15_7}) ci dice anche che un'interazione, sia pure arbitrariamente debole, che agisce per un tempo $\Delta t$, può variare l'energia propria del sistema di una quantità $\Delta E \sim \hbar /\Delta t$. Questo risultato, puramente quantistico, ha un significato fisico profondo. Esso prova che \textbf{nella meccanica quantistica la legge di conservazione dell'energia può essere verificata mediante una misura soltanto a meno di una grandezza dell'ordine di} $\hbar /\Delta t$, \textbf{dove } $\Delta t$ \textbf{è la durata del processo di misura}. Infatti, una sia pur debole interazione tra lo strumento di misura ed il sistema in esame, che agisce per un tempo $\Delta t $, può sempre variare l'energia del sistema di una quantità $\Delta E \sim \hbar /\Delta t$.\\

È utile sottolineare che il \textbf{significato} della relazione (\ref{eq:cap15_7}) è \textbf{essenzialmente diverso da quello della relazione di indeterminazione} $\Delta p \Delta x\sim \hbar$, per la coordinata e la quantità di moto. In quest'ultima $\Delta p$ e $\Delta x$ sono le indeterminazioni nei valori della quantità di moto e della coordinata in uno stesso istante;  essa mostra che queste due grandezze non possono avere contemporaneamente valori esattamente determinati. L'energia $E ^{(0)}$, invece, può essere misurata in ogni istante con la precisione voluta. La grandezza $\Delta E = E_n ^{(0)} - E_i ^{(0)}$ è la differenza di sue valori esattamente misurati dell'energia $E^{(0)}$ in due stati, e non l'indeterminazione nei valori dell'energia in un istante determinato.\\

È utile mostrate che per \textbf{grandi tempi} $t$, \textbf{la probabilità di transizione} $P_{i\rightarrow n} (t) \simeq \vert c_n ^{(1)} (t) \vert ^2$ \textbf{espressa dalla relazione (\ref{eq:cap15_6}), può essere considerata proporzionale a } $t$. A questo scopo osserviamo che vale la formula seguente:
	\begin{equation}
		\tcboxmath[sharp corners=downhill, colback=white, colframe=red!75!black]{
			\lim _{t \rightarrow \infty} \frac{\sin ^2 (\alpha t)}{\pi t \alpha ^2} = \delta (\alpha),
			}
	\label{eq:cap15_8}
	\end{equation}
infatti per $\alpha \neq 0$ questo limite è nullo, mentre per  $\alpha = 0$. Si ha $\sin ^2 (\alpha t)/\alpha ^2 t = t$, cosicché il limite è infinito. Integrando poi in $d \alpha$ da $-\infty$ a $+\infty$ si ottiene (sostituendo $\alpha t = \xi$)
	\begin{equation}
		\frac{1}{\pi t}\int _{-\infty} ^{+\infty} d\alpha \ \frac{\sin ^2 (\alpha t)}{ \alpha ^2} = \frac{1}{\pi}\int _{-\infty} ^{+\infty} d\xi \ \frac{\sin ^2 \xi}{ \xi ^2}=1. 
	\end{equation}
In tal modo la funzione a primo membro della (\ref{eq:cap15_8}) soddisfa effettivamente tutte le condizioni che definiscono una funzione $\delta$. Il risultato espresso nell'eq. (\ref{eq:cap15_8}) può essere anche visualizzato dal grafico della funzione $f(\omega)$ presentato in precedenza. Al crescere del tempo $t$ l'altezza del picco centrale della funzione cresce (proporzionalmente a $t^2$) e la sua larghezza decresce (proporzionalmente ad $1/t$). L'integrale della funzione, ossia l'area sottesa dalla curva è proporzionale a $t^2\cdot 1/t =t$, e dunque la funzione $f(\omega)/t$ ha, per grandi $t$, area costante, come richiesto dalla funzione $\delta$.\\

Conformemente a quanto detto, troviamo dall'eq. (\ref{eq:cap15_6}) che \textbf{per grandi tempi}
	\begin{equation}
		\vert c_n ^{(1)} (t) \vert ^2 \simeq \frac{1}{\hbar ^2} \vert V_{ni} \vert ^2 \pi t\ \delta\left(\frac{E_n ^{(0)}-E_i ^{(0)}}{2\hbar} \right),
	\end{equation}
ossia, osservando che $\delta (ax) = \frac{1}{a} \delta (x)$,
	\begin{equation}
		\tcboxmath[sharp corners=downhill, colback=white, colframe=red!75!black]{
			\vert c_n ^{(1)} (t) \vert ^2 \simeq \frac{2\pi}{\hbar } \vert V_{ni} \vert ^2  t\ \delta (E_n ^{(0)}-E_i ^{(0)} ),
			}
	\label{eq:cap15_9}
	\end{equation}
che mostra, come anticipato, che la probabilità di transizione cresce linearmente con il tempo $t$. È consuetudine considerare la \textbf{probabilità di transizione per unità di tempo}, definita come
	\begin{equation}
		\tcboxmath[sharp corners=downhill, colback=white, colframe=red!75!black]{
			W_{i\rightarrow n} (t) =\frac{d}{dt} P_{i\rightarrow n} (t),
			}
	\end{equation}
che è dunque costante per grandi t. Dall'eq. (\ref{eq:cap15_9}) otteniamo allora:
	\begin{equation}
		\tcboxmath[enhanced, sharp corners=downhill, colback=yellow!50!white, colframe=red!75!black, borderline={2pt}{-3pt}{red!50!black}]{
			W_{i\rightarrow n} (t) = \frac{2\pi}{\hbar } \vert V_{ni} \vert ^2\ \delta (E_n ^{(0)}-E_i ^{(0)} ).
			}
	\label{eq:cap15_10}
	\end{equation}
Questo risultato,di grande importanza pratica, è chiamato \textbf{regola d'oro di Fermi} (sebbene la teoria perturbativa dipendete dal tempo sia stata di fatto sviluppata da Dirac).\\
		
L'eq. (\ref{eq:cap15_10}) ha particolare rilevanza per le transizioni nello spettro continuo che, praticamente, sono sempre degeneri. In questo caso le probabilità di transizione a tutti i possibili stati finali con energia $E_n ^{(0)}= E_i ^{(0)}$ si ottiene integrando l'eq. (\ref{eq:cap15_10}) con
	\begin{equation}
		\tcboxmath[sharp corners=downhill, colback=white, colframe=red!75!black]{
			\int dE_n ^{(0)} \ \rho (E_n ^{(0)}),
			}
	\end{equation}
dove $\rho (E)$ rappresenta la \textbf{densità degli stati}, ossia il numero di stati nell'intervallo di energia $(E, E+dE)$.
\subsection{Determinazione dell'integrale $\int _{-\infty} ^{-\infty} \myfont{\textbf{dx}}\ \myfont{\textbf{sin}} ^{\myfont{\textbf{2}}} \myfont{\textbf{x/}}\myfont{\textbf{x}}^{\myfont{\textbf{2}}}$}
Determiniamo l'integrale
	\begin{equation}
		\tcboxmath[sharp corners=downhill, colback=white, colframe=red!75!black]{
			I= \int _{-\infty} ^{-\infty} dx\ \frac{\sin ^2 x}{x^2}.
			}
	\end{equation}
in primo luogo trasformiamo l'integrale con una integrazione per parti:
	\begin{align}
		I &= \int _{-\infty} ^{-\infty} dx\ \frac{d}{dx}\left(\frac{1}{x}\right)\sin ^2 x =  \left. -\frac{1}{x}\sin ^2 x\right\vert _{-\infty} ^{-\infty} +\int _{-\infty} ^{-\infty} dx\ \frac{1}{x} 2 \sin x \cos x  =\nonumber \\
		&= \int _{-\infty} ^{-\infty} dx\ \frac{\sin 2 x}{x}, 
	\end{align}
ossia
	\begin{equation}
		\tcboxmath[sharp corners=downhill, colback=white, colframe=red!75!black]{
		I= \int _{-\infty} ^{-\infty} dx\ \frac{\sin  x}{x}.
		} \quad \textrm{[Integrale di Dirichelet]}
	\end{equation}
Possiamo poi trasformare l'integrale in un integrale nel piano complesso:
	\begin{equation}
		I= \int _{-\infty} ^{-\infty} dx\ \frac{\sin  x}{x}= \textrm{Im}\left\{\textrm{PV} \int _{-\infty} ^{-\infty} dz\ \frac{e^{iz}}{z}\right\},
	\label{eq:cap15_14}
	\end{equation}
dove ``PV'' indica il valor principale (la parte reale dell'integrale non è altrimenti definita per il polo in $z=0$). \newpage

Per calcolare l'integrale scegliamo il seguente contorno:
\begin{figure}[!htbp]
\begin{center}
\includegraphics[width=8cm]{immagini/cap_15/fig_15_4.png}
\end{center}
\end{figure}
poiché la funzione integranda non ha poli all'interno del contorno possiamo scrivere:
	\begin{align}
		0 & =  \oint  dz\ \frac{e^{iz}}{z} = \nonumber \\
		&= \lim _{R\rightarrow \infty} \lim _{\varepsilon\rightarrow 0} \left[ \int _{-R} ^{-\varepsilon}dz\ \frac{e^{iz}}{z}+\int _{\varepsilon} ^{R}dz\ \frac{e^{iz}}{z}+\int _{\Gamma _{\varepsilon}} dz\ \frac{e^{iz}}{z} +\int _{\Gamma _{R}} dz\ \frac{e^{iz}}{z}\right] = \nonumber\\
		&= \textrm{PV}\int _{-\infty} ^{+\infty}dz\ \frac{e^{iz}}{z}+ \lim _{\varepsilon\rightarrow 0}  \int _{\Gamma _{\varepsilon}} dz\ \frac{e^{iz}}{z}+\lim _{R\rightarrow \infty}\int _{\Gamma _{R}} dz\ \frac{e^{iz}}{z}.
	\label{eq:cap15_15}
	\end{align}
L'integrale sul semicerchio esterno tende evidentemente a zero nel limite $R\rightarrow \infty $:
	\begin{equation}
		\int _{\Gamma _{R}} dz\ \frac{e^{iz}}{z} \overset{z= Re^{i\varphi}}{=} \int _0 ^{\pi} d\varphi \ i R e^{i\varphi}\ \frac{e^{iRe^{i\varphi}}}{Re^{i\varphi}} = i\int _0 ^{\pi} d\varphi \ e^{iRe^{i\varphi}} \underset{R\rightarrow \infty}{\longrightarrow}0.
	\end{equation}
L'integrale su $\Gamma _{\varepsilon}$ dà invece contributo non nullo:
	\begin{equation}
		\int _{\Gamma _{\varepsilon}} dz\ \frac{e^{iz}}{z} \overset{z= \varepsilon e^{i\varphi}}{=} \int _0 ^{\pi} d\varphi \ i \varepsilon\, e^{i\varphi}\ \frac{e^{i\varepsilon  e^{i\varphi}}}{\varepsilon\, e^{i\varphi}} =  i\int _0 ^{\pi} d\varphi \ e^{i\varepsilon e^{i\varphi}} \underset{\varepsilon\rightarrow 0}{\longrightarrow}i\int _0 ^{\pi} d\varphi = -i\pi.
	\end{equation}
Sostituendo questo risultato nella (\ref{eq:cap15_15}) si ottiene
	\begin{equation}
		\textrm{PV} \int _{-\infty} ^{-\infty} dz\ \frac{e^{iz}}{z} =-\lim _{\varepsilon \rightarrow 0} \int _{\Gamma _{\varepsilon}} dz\ \frac{e^{iz}}{z}= +i\pi.
	\end{equation}
L'eq. (\ref{eq:cap15_14}) implica allora:
	\begin{equation}
		\tcboxmath[sharp corners=downhill, colback=white, colframe=red!75!black]{
			I= \int _{-\infty} ^{-\infty} dx\ \frac{\sin ^2  x}{x^2}= \int _{-\infty} ^{-\infty} dx\ \frac{\sin  x}{x}= \pi
			}
	\end{equation}
\newpage
\subsection{Coefficienti c$_\textrm{n}$(t) al secondo ordine}
Determiniamo l'espressione per i coefficienti $c_n(t)$ al secondo ordine della teoria delle perturbazioni. L'eq. (\ref{eq:cap15_3}) comporta:
	\begin{equation}
		i\hbar \frac{dc_n ^{(2)} (t)}{dt} = \sum _k V_{nk} (t)\ c_k ^{(1)}(t) e^{i\omega _{nk}t},
	\end{equation}
che può essere direttamente integrata, con la condizione iniziale $c_n ^{(2)} (0)=0 $, per dare:
	\begin{equation}
		c_n ^{(2)} (t)= -\frac{i}{\hbar}\sum _k \int _0 ^t dt'\ V_{nk} (t')\ c_k ^{(1)} (t') e ^{i\omega _{nk} t'}.
	\end{equation}
Sostituendo quindi in questa espressione il risultato (\ref{eq:cap15_5}) per i coefficienti al primo ordine si ottiene dunque
	\begin{equation}
		\tcboxmath[sharp corners=downhill, colback=white, colframe=red!75!black]{
		\begin{split}
			c_n ^{(2)} (t)= \left(-\frac{i}{\hbar}\right)^2 \sum _{k,k'} \int _0 ^t dt'\ V_{nk} (t')\ c_k ^{(1)} (t') e ^{i\omega _{nk} t'}\cdot \qquad\\
			\cdot\int _0 ^t dt''\ V_{kk'} (t'')\  e ^{i\omega _{kk'} t''}\ c_{k'} ^{(1)} (0).
	\end{split}
	}
	\end{equation}
\subsection{Perturbazione costante: trattazione con la teoria delle perturbazioni dipendente dal tempo}
I risultati ottenuti sulla transizione per effetto di una perturbazione costante con il formalismo della teoria delle perturbazioni dipendente dal tempo possono essere riottenuti, seppure con una procedura più laboriosa, utilizzando il formalismo della teoria delle perturbazioni dipendente dal tempo.\\

In questo contesto, l'evoluzione temporale del vettore di stato del sistema al tempo $t=0$,
	\begin{equation}
		\vert \alpha , t=0 \rangle = \vert i^{(0)} \rangle ,
	\end{equation}
si determina esprimendo il vettore di stato come combinazione lineare di autostati $\vert k \rangle $ dell'Hamiltoniano completo $H=H_0 +V$,
	\begin{equation}
	\vert \alpha , t=0\rangle = \sum _k c_k \vert k \rangle , \quad c_k = \langle k \vert i^{(0)} \rangle ,
	\end{equation}
ed applicando l'operatore di evoluzione temporale:
	\begin{equation}
		\vert \alpha , t \rangle = e^{-\frac{i}{\hbar} H t}\, \vert \alpha , t=0\rangle = \sum _k c_k e^{-\frac{i}{\hbar} E_k t }\, \vert k \rangle .
	\end{equation}
L'ampiezza di probabilità $\langle n^{(0)} \vert \alpha , t \rangle $ che il sistema venga a trovarsi nell'autostato $\vert n^{(0)} \rangle $ di $H_0$ al tempo $t$ è allora:
	\begin{equation}
		\langle n^{(0)} \vert \alpha , t \rangle = \sum _k  c_k\, e^{-\frac{i}{\hbar} E_k t}\, \langle n^{(0)} \vert k \rangle = \sum _k \langle n^{(0)} \vert k \rangle \langle k \vert i^{(0)} \rangle e^{-\frac{i}{\hbar}E_k t}.
	\label{eq:cap15_16}
	\end{equation}
Questa espressione è esatta. Determiniamo ora il valore approssimato al primo ordine della teoria delle perturbazioni sostituendo al posto degli autostati $\vert k \rangle $ e degli autovalori $E_k$ di $H$ le loro approssimazioni al primo ordine:
	\begin{align}
		&\vert k \rangle \simeq \vert k^{(0)} \rangle + \sum _{m\neq k} \frac{V_{mk}}{E_k ^{(0)} - E_m ^{(0)}}\vert m ^{(0)} \rangle ,\\
	 	&E_k \simeq E_l ^{(0)} +V_{kk} .
	\end{align}
È già evidente che per le transizioni tra stati iniziali e finali diversi, $n\neq i$, il prodotto $\langle n^{(0)} \vert k \rangle \langle k \vert i^{(0)} \rangle$ è almeno di ordine 1 nello sviluppo. Nella (\ref{eq:cap15_16}) possiamo allora sostituire l'autovalore $E_k$ nell'esponenziale con la sua espressione all'ordine zero. Quanto alle ampiezze si ha:
	\begin{equation}
		\langle n^{(0)} \vert k \rangle \simeq \langle n^{(0)} \vert k^{(0)} \rangle + \sum _{m\neq k} \frac{V_{mk}}{E_k ^{(0)}-E_m ^{(0)}} \langle n^{(0)} \vert m^{(0)} \rangle = \delta _{nk} + \frac{V_{nk}}{E_k ^{(0)} - E_n ^{(0)}}\left( 1-\delta _{nk} \right),
	\label{eq:cap15_17}	
	\end{equation}
si noti che per $n=k$ il secondo termine della precedente espressione, contenente la somma $\displaystyle{\sum _{m\neq k }}$, dà contribuito nullo.\\

Allo stesso modo, prendendo il complesso coniugato della precedente espressione e sostituendo $n\rightarrow i $, si trova:
	\begin{equation}
		\langle k\vert i^{(0)}  \rangle  = \delta _{ik} + \frac{V_{ki}}{E_k ^{(0)} - E_i ^{(0)}}\left( 1-\delta _{ik} \right);
	\label{eq:cap15_18}	
	\end{equation}
sostituendo le (\ref{eq:cap15_17}) e (\ref{eq:cap15_18}) nella (\ref{eq:cap15_16}) si trova allora:
	\begin{align}
		\langle n ^{(0)}\vert \alpha , t \rangle &= \sum _ k \left[ \delta _{nk} + \frac{V_{nk}}{E_k ^{(0)}- E_n ^{(0)}} \left(1-\delta _{nk} \right)\right]\cdot \nonumber \\
		&\qquad\qquad \cdot \left[ \delta _{ik} + \frac{V_{ki}}{E_k ^{(0)}- E_i ^{(0)}} \left(1-\delta _{ik} \right)\right] e^{-\frac{i}{\hbar} E_k ^{(0)} t} = \nonumber \\
		&= \delta_{ni} +\frac{V_{ni}}{E_n ^{(0)} -E_i ^{(0)}}\left(1-\delta_{ni} \right) e^{-\frac{i}{\hbar} E_n ^{(0)} t}+ \nonumber \\
		&\qquad \qquad	+\frac{V_{ni}}{E_i ^{(0)} -E_n ^{(0)}}\left(1-\delta_{ni} \right) e^{-\frac{i}{\hbar} E_i ^{(0)} t}+ O(V^2) = \nonumber \\
		& \overset{ n\neq i}{=}\frac{V_{ni}}{E_n ^{(0)} -E_i ^{(0)}} \left(e^{-\frac{i}{\hbar} E_n ^{(0)} t} -e^{-\frac{i}{\hbar} E_i ^{(0)} t}\right) = \nonumber \\
		&= \frac{V_{ni}}{E_n ^{(0)} -E_i ^{(0)}}e^{-\frac{i}{\hbar} E_n ^{(0)} t}\left( 1- e^{-\frac{i}{\hbar} \left(E_n ^{(0)}- E_i ^{(0)}\right) t}\right) = \nonumber \\
&= \frac{-2iV_{ni}}{E_n ^{(0)} -E_i ^{(0)}}e^{-\frac{i}{\hbar} E_n ^{(0)} t}  e^{\frac{i}{2\hbar} \left( E_n ^{(0)} -E_i ^{(0)}\right)t} \, \sin \left( \frac{ E_n ^{(0)} -E_i ^{(0)}}{2\hbar}t \right).
	\end{align}\\
	
Calcolando infine il modulo quadro di questa espressione otteniamo la probabilità
	\begin{equation}
		\tcboxmath[enhanced, sharp corners=downhill, colback=yellow!50!white, colframe=red!75!black, borderline={2pt}{-3pt}{red!50!black}]{
			P_{i\rightarrow n} (t) = 	\vert\langle n ^{(0)}\vert \alpha , t \rangle \vert ^2 = \frac{4\abs{V_{ni}}^2 }{\left(E_n ^{(0)} -E_i ^{(0)}\right) ^2}\,\sin ^2\left( \frac{ E_n ^{(0)} -E_i ^{(0)}}{2\hbar}t \right),
			}
	\end{equation}
in accordo con il risultato ottenuto con la teoria delle perturbazioni dipendente dal tempo.
\section{Transizioni per effetto di una perturbazione periodica}
Consideriamo ora le transizioni indotte da una \textbf{perturbazione periodica}, della forma cioè
	\begin{equation}
		\tcboxmath[enhanced, sharp corners=downhill, colback=yellow!50!white, colframe=red!75!black, borderline={2pt}{-3pt}{red!50!black}]{
			V(t) = Fe^{-i\omega t} + F^+ e ^{i\omega t},
			}
	\label{eq:cap15_11}
	\end{equation}
dove $F$ è un operatore non dipendente dal tempo (funzione in generale degli operatori impulso, posizione e spin). Osserviamo che $V(t)$ è un operatore hermitiano, come deve essere.	\\

Assumendo ancora che al tempo $t=0$ il sistema si trovi nell'autostato $\vert i ^{(0)}\rangle$ dell'hamiltoniano imperturbato $H_0$ (per cui $c_k (0) = \delta _{ik}$) e sostituendo il potenziale (\ref{eq:cap15_11}) nella (\ref{eq:cap15_5}), si ricava:
	\begin{align}
		c_n ^{(1)}(t) & =  -\frac{i}{\hbar}\int _0 ^t dt'\ V_{ni} (t') \ e^{i\omega _{ni}t'} = \nonumber \\
		&= -\frac{i}{\hbar}\int _0 ^t dt'\left[ F_{ni}\ e^{i(\omega _{ni}-\omega) t'} + F_{ni} ^+\ e ^{i(\omega _{ni} +\omega) t}\right] = \nonumber \\
		&= -\frac{i}{\hbar}\left[ F_{ni}\ \frac{e^{i(\omega _{ni}-\omega) t}-1}{i(\omega _{ni}-\omega)} + F_{ni} ^+\ \frac{e ^{i(\omega _{ni} +\omega) t}-1}{i(\omega _{ni}+\omega)}\right], 
	\end{align}
ossia
	\begin{equation}
		\tcboxmath[sharp corners=downhill, colback=white, colframe=red!75!black]{
			c_n ^{(1)}(t)= -\frac{2i}{\hbar}\left[ F_{ni}\ e^{i\frac{(\omega _{ni}-\omega) t}{2}}\frac{\sin(\frac{(\omega _{ni}-\omega) t}{2})}{(\omega _{ni}-\omega)}  + F_{ni} ^+\ e^{i\frac{(\omega _{ni}+\omega) t}{2}}\frac{\sin(\frac{(\omega _{ni}+\omega) t}{2})}{(\omega _{ni}+\omega)}\right].
			}
	\label{eq:cap15_12}
	\end{equation}\\
	
Dai risultati del paragrafo precedente è evidente che, nel calcolo delle probabilità di transizione $\vert c_n ^{(1)}(t)\vert ^2$, il primo termine della (\ref{eq:cap15_12}) fornisce, per tempi grandi, un contributo significativo solo alle transizioni verso quegli stati con energia $E_n ^{(0)} \simeq E_i ^{(0)}+ \hbar \omega $, tali cioè che la differenza  $\omega _{ni} - \omega$ sia piccola. Analogamente il secondo termine della (\ref{eq:cap15_12}) fornisce un contributo significativo solo per le transizioni verso quegli stati con energia  $E_n ^{(0)} \simeq E_i ^{(0)}- \hbar \omega $, per i quali cioè la somma $\omega _{ni} + \omega$ è piccola. Infine  il termine di ``doppio prodotto'' nel calcolo di  $\vert c_n ^{(1)}(t)\vert ^2$ fornisce un contributo che è sempre trascurabile nel limite di tempi grandi, giacché le due condizioni $\omega _{ni} - \omega\simeq 0$ e $\omega _{ni} + \omega\simeq 0$ non sono mai simultaneamente verificate.\\

Dai risultati del paragrafo precedente possiamo ottenere direttamente l'espressione delle \textbf{probabilità di transizione per unità di tempo} valida nel limite di grandi $t$:
	\begin{align}
		\tcboxmath[enhanced, sharp corners=downhill, colback=yellow!50!white, colframe=red!75!black, borderline={2pt}{-3pt}{red!50!black}]{
			W_{i\rightarrow n} = \frac{2\pi}{\hbar}\left[ \vert F_{ni}\vert ^2 \ \delta (E_n^{(0)}-E_i^{(0)}-\hbar \omega) + \vert F_{ni} ^+\vert ^2 \ \delta (E_n^{(0)}-E_i^{(0)}+\hbar \omega)\right],
			} \nonumber \\[-0.5cm]
			\,
	\label{eq:cap15_13}
	\end{align}
in accordo nuovamente con la \textbf{regolo d'oro di Fermi}.\\

Il significato fisico dei due termini nell'eq. (\ref{eq:cap15_13}) è evidente. Il primo termine corrisponde alle transizioni verso quegli stati, tipicamente nello spettro del continuo, la cui energia $E_n ^{(0)}$ è accresciuta rispetto all'energia dello stato iniziale di una quantità $\hbar \omega$. Questo termine descrive dunque \textbf{l'assorbimento da parte del sistema di una quantità di energia $\hbar \omega$ dal potenziale periodico $V$}. Il secondo termine della (\ref{eq:cap15_13}) corrisponde invece alle transizioni verso quegli stati la cui energia $E_n ^{(0)}$ è minore di una quantità $\hbar \omega$ rispetto all'energia dello stato iniziale $E_i ^{(0)}$. Questo termine descrive il cosiddetto processo di \textbf{emissione stimolata, in cui il sistema cede una quantità $\hbar \omega$ della sua energia al campo esterno $V$. Così una perturbazione dipendente dal tempo può essere vista come una inesauribile sorgente o pozzo di energia}.\\
I due processi di assorbimento ed emissione stimolato posso essere schematicamente rappresentati nel modo seguente:
\begin{figure}[!htbp]
\begin{center}
\includegraphics[width=6cm]{immagini/cap_15/fig_15_2.png}\hspace{1cm}
\includegraphics[width=6cm]{immagini/cap_15/fig_15_3.png}
\end{center}
\end{figure}
 %DEFINITIVO
\chapter{Momento angolare}
\section[Rotazioni, momento angolare e regole di commutazione]{Rotazioni, momento angolare e regole di commutazione per gli operatori momento angolare}
Nella meccanica quantistica, così come nella meccanica classica, \textbf{il momento angolare è il generatore delle rotazioni infinitesime}.\\

Se indichiamo con $D_{\widehat{n}} (d\varphi)$ l'operatore unitario che induce una rotazione di un angolo infinitesimo $d\varphi$ attorno all'asse caratterizzato dal vettore $\widehat{n}$ abbiamo allora
	\begin{equation}
		\tcboxmath[enhanced, sharp corners=downhill, colback=yellow!50!white, colframe=red!75!black, borderline={2pt}{-3pt}{red!50!black}]{
			D_{\widehat{n}} (d\varphi)=1-\frac{i}{\hbar}\vec{J}\cdot \widehat{n}\ d\varphi ,
			}
	\end{equation}
dove $\vec{J}$ è \textbf{l'operatore momento angolare}. Questa equazione può essere considerata la \textbf{definizione} nella meccanica quantistica dell'operatore momento angolare.\\

Una rotazione finita si può ottenere associando successivamente rotazioni infinitesime attorno allo stesso asse. Così ade esempio, per una rotazione finita di un angolo $\phi$ attorno all'asse $z$, otteniamo
	\begin{equation}
		D_z (\phi) = \lim _{N\rightarrow \infty} \left(1-\frac{i}{\hbar} J_z \frac{\phi}{N}\right) ^N= \lim _{N\rightarrow \infty} e^{N\log\left(1-\frac{i}{\hbar} J_z \frac{\phi}{N}\right)}= \lim _{N\rightarrow \infty} e^{N\left(-\frac{i}{\hbar} J_z \ \frac{\phi}{N}\right)},
	\end{equation}
ossia
	\begin{equation}
		\tcboxmath[enhanced, sharp corners=downhill, colback=yellow!50!white, colframe=red!75!black, borderline={2pt}{-3pt}{red!50!black}]{
			D_z (\phi)=e^{-\frac{i}{\hbar} J_z \phi}.
			}
	\end{equation}
\textbf{L'avere assunto che il momento angolare è il generatore delle rotazioni spaziali implica che, per un sistema invariante rispetto a rotazioni attorno a un determinato asse, si conserva la componete del momento angolare lungo quell'asse.}\\

In particolare poi, le proprietà di isotropia dello spazio (ossia l'equivalenza di tutte le direzioni nello spazio) implica che l'hamiltoniano di un sistema isolato deve essere invariante rispetto a rotazioni di un angolo arbitrario attorno a un asse qualsiasi. \textbf{La legge di conservazione del momento angolare di un sistema isolato è dunque conseguenza della proprietà di isotropia dello spazio}.\\

È una proprietà ben nota delle rotazioni il fatto che \textbf{rotazioni attorno ad uno stesso asse commutano, mentre rotazioni attorno ad assi diversi non commutano}. Così ad esempio una rotazione di $\pi /2$ attorno all'asse $z$ seguita da una rotazione di $\pi /2$ attorno all'asse $x$ produce un risultato diverso di quello ottenuto con una rotazione di $\pi /2$ attorno all'asse $x$ seguita da una rotazione di $\pi /2$ attorno all'asse $z$:\\
\begin{figure}[!htbp]
\begin{center}
\includegraphics[width=10cm]{immagini/cap_16/fig_16_1.png}
\end{center}
\end{figure}


In termini dell'azione degli operatori di rotazione du un generico vettore di stato $\vert \alpha \rangle$ questo implica
	\begin{equation}
		D_x (\pi/2) D_z (\pi/2) \vert \alpha \rangle \neq D_x (\pi/2) D_x (\pi/2) \vert \alpha \rangle ,
	\end{equation}
o, equivalentemente, per l'arbitrarietà dello stato $\vert \alpha \rangle$,
\begin{equation}
[D_x (\pi/2); D_z (\pi/2)] \neq 0 .
\end{equation}\\

Per stabilire quantitativamente le regole di commutazione degli operatori di rotazione attorno ad assi diversi, dobbiamo considerare con maggior dettaglio le proprietà di commutazione delle rotazioni.\\

A tale scopo ricordiamo che a ciascuna rotazione, diciamo di un angolo $\varphi$ attorno ad un asse definito dal versore $\widehat{n}$, può essere associata una \textbf{matrice ortogonale} $R_{\widehat{n}} (\varphi)$. Il significato di questa matrice è che un vettore $\vec{v}$ di componenti $(v_x, v_y v_z) $ (che può rappresentare ad esempio la posizione di una particella nello spazio), viene trasformato, a seguito della rotazione nel vettore $\vec{v}^{\, \prime}$ legato a $\vec{v}$ dalla relazione
	\begin{equation}
		\tcboxmath[sharp corners=downhill, colback=white, colframe=red!75!black]{
			\vec{v}^{\, \prime}= R_{\widehat{n}} (\varphi)\vec{v}.
			}
	\end{equation}
L'ortogonalità della matrice $R$ segue dal fatto che le rotazioni lasciano invariati i moduli dei vettori:
	\begin{equation}
		\vec{v}^{\, \prime}\cdot\vec{v} = \vec{v}R^T R \vec{v}= \vec{v}\cdot\vec{v} \qquad \Rightarrow \quad
		\tcboxmath[sharp corners=downhill, colback=white, colframe=red!75!black]{
 		R^T R =1.
 		}
	\end{equation}\\
	
È semplice derivare l'espressione della matrice di rotazione associata ad esempio ad una rotazione di un angolo $d\varphi $ attorno all'asse $z$. Esprimendo le componenti del vettore $\vec{v}$ in coordinate polari,
	\begin{equation}
		\begin{cases}
			v_x = v \sin\theta \cos \varphi ,\\
			v_y = v \sin\theta \sin \varphi ,\\
			v_z = v \cos\theta ,
		\end{cases}
	\end{equation}
si ha che il vettore $\vec{v}$ si trasforma, per effetto della rotazione, nel vettore $\vec{v}^{\, \prime}$ di componenti:
	\begin{equation}
		\begin{cases}
			v'_x = v \sin\theta \cos (\varphi +d\varphi )= v_x \cos d\varphi - v_y \sin d\varphi ,\\
			v'_y = v \sin\theta \sin (\varphi +d\varphi ) = v_x \sin d\varphi + v_y \cos d\varphi ,\\
			v'_z = v \cos\theta = v_z .
	\end{cases}
	\end{equation}
La matrice $R_z (d\varphi)$ ha dunque la forma
	\begin{equation}
		R_z (d\varphi)=
		\begin{pmatrix}
			\cos d\varphi & -\sin d\varphi & 0\\
			\sin d\varphi & \cos d\varphi & 0 \\
			0 & 0 & 1 \\
		\end{pmatrix}
	\end{equation}\\
	
Per studiare le proprietà di commutazione delle rotazioni è conveniente considerare rotazioni infinitesime. Ponendo $\varepsilon =d\varphi$ e sviluppando la precedente matrice fino al secondo ordine in $\varepsilon$ troviamo:
	\begin{equation}
		R_z (\varepsilon)=
		\begin{pmatrix}
			1-\varepsilon ^2 /2 & -\varepsilon & 0\\
			\varepsilon & 1-\varepsilon ^2/2 & 0 \\
			0 & 0 & 1 \\
		\end{pmatrix}
		+ O(\varepsilon ^3).
	\end{equation}\\
	
Le corrispondenti matrici si rotazione attorno agli assi $x$ e $y$ si possono ottenere con permutazioni cicliche di $x$, $y$ e $z$:
	\begin{equation}
		x\rightarrow y,\ y\rightarrow z,\ z\rightarrow x,
	\end{equation}
si ottiene in tal modo:
	\begin{equation}
		R_x (\varepsilon)=
		\begin{pmatrix}
			1 & 0 & 0\\
			0 & 1-\varepsilon ^2/2 & -\varepsilon \\
			0 & \varepsilon & 1-\varepsilon ^2/2 \\
		\end{pmatrix}
		+ O(\varepsilon ^3),
	\end{equation}
	\begin{equation}
		R_y (\varepsilon)=
		\begin{pmatrix}
			1-\varepsilon ^2/2 & 0 & \varepsilon\\
			0 & 1 &0 \\
			- \varepsilon & 0 & 1-\varepsilon ^2/2 \\
		\end{pmatrix}
		+ O(\varepsilon ^3).
	\end{equation}\\
	
Consideriamo ora una rotazione di angolo $\varepsilon$ attorno all'asse $y$ seguita da una rotazione di angolo $\varepsilon$ attorno all'asse $x$. La corrispondente matrice di rotazione è:
	\begin{equation}
		R_x (\varepsilon)R_y (\varepsilon)=
		\begin{pmatrix}
			1-\varepsilon ^2/2 & 0 & \varepsilon \\
			 \varepsilon ^2 & 1-\varepsilon ^2/2 & -\varepsilon \\
			-  \varepsilon & \varepsilon & 1-\varepsilon ^2 \\
		\end{pmatrix}
		+ O(\varepsilon ^3)
	\end{equation}\\
	
Se consideriamo invece rotazione di angolo $\varepsilon$ attorno all'asse $x$ seguita da una rotazione di angolo $\varepsilon$ attorno all'asse $y$ otteniamo la matrice di rotazione\footnote{$\displaystyle{\left[\left( R_y R_x\right) ^T = R_x ^T R_y ^T=R_xR_y+(\varepsilon\rightarrow -\varepsilon)\right]}$.}.
	\begin{equation}
		R_y (\varepsilon)R_x (\varepsilon)=
		\begin{pmatrix}
			1-\varepsilon ^2/2 & \varepsilon ^2 & \varepsilon \\
			 0 & 1-\varepsilon ^2/2 & -\varepsilon \\
			-  \varepsilon & \varepsilon & 1-\varepsilon ^2 \\
		\end{pmatrix}
		+ O(\varepsilon ^3)
	\end{equation}\\
	
Dal confronto di questi risultati vediamo che
	\begin{equation}
		R_x (\varepsilon)R_y (\varepsilon)-R_y (\varepsilon)R_x (\varepsilon)=
		\begin{pmatrix}
			0 & - \varepsilon ^2 & 0 \\
			 \varepsilon ^2 & 0 & 0 \\
			0 & 0& 0\\
		\end{pmatrix}
		= R_z(\varepsilon ^2)-1+ O(\varepsilon ^3),
	\end{equation}
che definisce la proprietà di commutazione di due rotazioni successive infinitesime attorno agli assi $x$ ed $y$.\\

La stessa proprietà deve essere soddisfatta dagli operatori che inducono le corrispondenti rotazioni dei vettori di stato in meccanica quantistica, ossia:
	\begin{equation}
		D_x (\varepsilon)D_y (\varepsilon)-D_y (\varepsilon)D_x (\varepsilon)=D_z(\varepsilon ^2)-1+ O(\varepsilon ^3).
	\end{equation}\\
	
In termini degli operatori momento angolare, questa relazione implica:
	\begin{align}
		& D_x (\varepsilon)D_y (\varepsilon)-D_y (\varepsilon)D_x (\varepsilon)= \nonumber\\
		&  = \left(1-\frac{i\varepsilon}{\hbar} J_x -\frac{\varepsilon ^2}{2\hbar ^2}J_x ^2\right)\left(1-\frac{i\varepsilon}{\hbar} J_y -\frac{\varepsilon ^2}{2\hbar ^2}J_y ^2\right)- \left( x \leftrightarrow y\right) =\nonumber \\
		& = \left(1-\frac{i\varepsilon}{\hbar} J_x-\frac{i\varepsilon}{\hbar} J_y-\frac{\varepsilon ^2}{2\hbar ^2}J_x-\frac{\varepsilon ^2}{2\hbar ^2}J_y-\frac{\varepsilon ^2}{\hbar ^2}J_xJ_y \right) - \left( x \leftrightarrow y\right) = \nonumber \\
		&  = -\frac{\varepsilon ^2}{\hbar ^2}\left(J_xJ_y-J_y J_x\right) = D_z (\varepsilon ^2)- 1= \left(1-\frac{i\varepsilon ^2}{\hbar} J_z-1\right)=-\frac{i\varepsilon ^2}{\hbar} J_z, 
	\end{align}
ossia
	\begin{equation}
		\tcboxmath[enhanced, sharp corners=downhill, colback=yellow!50!white, colframe=red!75!black, borderline={2pt}{-3pt}{red!50!black}]{
			[ J_x, J_y ] = i\hbar J_z.
			 }
	\end{equation}\\
	
Utilizzando le permutazioni cicliche di $x$, $y$ e $z$ possiamo derivare le regole di commutazione per le altre componenti del momento angolare:
	\begin{equation}
		\tcboxmath[enhanced, sharp corners=downhill, colback=yellow!50!white, colframe=red!75!black, borderline={2pt}{-3pt}{red!50!black}]{
			[ J_y, J_z ] = i\hbar J_x,
			}
	\end{equation}
	\begin{equation}
		\tcboxmath[enhanced, sharp corners=downhill, colback=yellow!50!white, colframe=red!75!black, borderline={2pt}{-3pt}{red!50!black}]{
			[ J_z, J_x ] = i\hbar J_y,
			}
	\end{equation}
o in forma compatta per due componenti arbitrarie
	\begin{equation}
		\tcboxmath[enhanced, sharp corners=downhill, colback=yellow!50!white, colframe=red!75!black, borderline={2pt}{-3pt}{red!50!black}]{
			[ J_i, J_j ] = i\hbar\, \varepsilon _{ikj}\,J_k.
			}
	\label{eq:cap16_1}
	\end{equation}
Queste equazioni sono note come le \textbf{relazioni fondamentali di commutazione del momento angolare}.\\

Sottolineiamo come \textbf{queste relazioni di commutazione seguono solamente dall'assunzione che $J_k$ è il generatore della rotazione attorno all'asse k-esimo e dalla proprietà di non commutatività delle rotazioni}.\\

\textbf{Le relazioni di commutazione (\ref{eq:cap16_1}) implicano che, in generale, le tre componenti del momento angolare non possono avere simultaneamente valori determinati}. A questo proposito il momento angolare differisce essenzialmente dall'impulso, le cui tre componenti sono misurabili simultaneamente.\\
		
Con gli operatori $J_x$, $J_y$ e $J_z$ formiamo l'\textbf{operatore} del \textbf{quadrato del momento angolare}:
	\begin{equation}
		\tcboxmath[enhanced, sharp corners=downhill, colback=yellow!50!white, colframe=red!75!black, borderline={2pt}{-3pt}{red!50!black}]{
			J^2 = J_x ^2 + J_y ^2 + J_z ^2.
			}
	\end{equation}
Questo operatore commuta con ciascuno degli operatori $J_x$, $J_y$ e $J_z$:
	\begin{equation}
		\tcboxmath[enhanced, sharp corners=downhill, colback=yellow!50!white, colframe=red!75!black, borderline={2pt}{-3pt}{red!50!black}]{
			[J^2, J_k]=0.
			} \quad (k=1,2,3)
	\label{eq:cap16_2}
	\end{equation}
Infatti, considerando ad esempio il caso $k=3$ ed utilizzando le relazioni di commutazione (\ref{eq:cap16_1}) otteniamo:
	\begin{align}
		[J^2, J_z] &= [J_x ^2+J_y^2+J_z ^2, J_z] = [J_x ^2, J_z]+[J_y ^2, J_z]= \nonumber \\
	& = J_x[J_x, J_z]+[J_x, J_z]J_x+J_y[J_y, J_z]+[J_y, J_z]J_y =  \nonumber \\
	& =  J_x (-i\hbar J_y)+ (-i\hbar J_y)J_x+ J_y (i\hbar J_x)+(i\hbar J_x)J_y =  0.  
	\end{align}
(Nota bene: abbiamo usato la proprietà generale dei commutatori $[AB,C]=A[B,C]+[A,C]B$).\\

Dal punto di vista fisico, le relazioni (\ref{eq:cap16_2}) significano che \textbf{il quadrato del momento angolare (cioè il suo valore assoluto) può avere valori determinati contemporaneamente con una delle sue componenti}.
\section[Autovalori ed elementi di matrice degli operatori di momento angolare]{Autovalori ed elementi di matrice degli operatori di momento angolare}
Consideriamo il problema di determinare gli autovalori e gli autovettori simultanei del quadrato del momento angolare $J^2$ e di una sua componente lungo un determinato asse, diciamo $J_z$. Indichiamo rispettivamente con $a$ e $b$ questi autovalori;  cerchiamo cioè le soluzioni delle equazioni
	\begin{align}
		&\tcboxmath[sharp corners=downhill, colback=white, colframe=red!75!black]{ J^2 \vert a, b \rangle = a \vert a, b \rangle ,} \\
		& \tcboxmath[sharp corners=downhill, colback=white, colframe=red!75!black]{J_z \vert a, b \rangle = b \vert a, b \rangle .}
	\end{align}\\
	
A tale scopo risulta conveniente, in luogo degli operatori $J_x$ e $J_y$, introdurre le loro combinazioni complesse
	\begin{equation}
		\tcboxmath[enhanced, sharp corners=downhill, colback=yellow!50!white, colframe=red!75!black, borderline={2pt}{-3pt}{red!50!black}]{
			J_{\pm} = J_x \pm i J_y .
			}
	\end{equation}\\
Utilizzando le relazioni di commutazione (\ref{eq:cap16_1}) per le componenti del momento angolare possiamo calcolare
	\begin{align}
		[J_z , J_{\pm}] & =  [J_z , J_x \pm i J_y] = [J_z , J_x] \pm i [J_z ,J_y] = \nonumber \\
		& = i\hbar J_y \pm i (-i\hbar J_x) = \pm \hbar (J_x \pm i J_y ) = \pm \hbar J_{\pm} .
	\end{align}
Dunque
	\begin{equation}
		\tcboxmath[enhanced, sharp corners=downhill, colback=yellow!50!white, colframe=red!75!black, borderline={2pt}{-3pt}{red!50!black}]{
			[J_z , J_{\pm}] = \pm \hbar J_{\pm} .
			}
	\label{eq:cap16_3}
	\end{equation}
Dalle relazioni di commutazione (\ref{eq:cap16_2}) segue anche immediatamente
	\begin{equation}
		\tcboxmath[enhanced, sharp corners=downhill, colback=yellow!50!white, colframe=red!75!black, borderline={2pt}{-3pt}{red!50!black}]{
			[J^2 , J_{\pm}]=0 .
			}
	\label{eq:cap16_4}
	\end{equation}\\
	
Per determinare il significato fisico degli operatori $J_{\pm}$ esaminiamo l'azione di $J_z$ sugli stati $J_{\pm} \vert a, b \rangle$:
\begin{equation}
J_zJ_{\pm} \vert a, b \rangle =\left( [J_z,J_{\pm}] + J_{\pm}J_{z}\right)\vert a, b \rangle = (b \pm \hbar )J_{\pm} \vert a, b \rangle , 
\end{equation}
dove abbiamo fatto uso delle relazioni (\ref{eq:cap16_3}). Ciò prova che \textbf{gli stati $ J_{\pm} \vert a, b \rangle$ sono ancora (a meno di una costante di normalizzazione) autostati dell'operatore $J_z$ corrispondenti agli autovalori $b\pm \hbar$}. Per questa ragione gli operatori $J_{\pm}$ sono anche noti con il nome di \textbf{operatori ``a scala''}.\\

Applichiamo ora agli stati $ J_{\pm} \vert a, b \rangle$ l'operatore $J^2$. Utilizzando le regole di commutazione (\ref{eq:cap16_4}) troviamo:
	\begin{equation}
		J^2J_{\pm} \vert a, b \rangle = \left( [J^2,J_{\pm}] + J_{\pm}J^2\right)\vert a, b \rangle = a J_{\pm} \vert a, b \rangle .
	\end{equation}
In altri termini, gli stati $J_{\pm} \vert a, b \rangle$ sono ancora autostati dell'operatore $J^2$ corrispondenti allo stesso autovalore $a$.\\

In conclusione possiamo scrivere:
	\begin{equation}
		\tcboxmath[sharp corners=downhill, colback=white, colframe=red!75!black]{
			J_{\pm} \vert a, b \rangle = c_{\pm}\vert a, b\pm \hbar \rangle,
			}
	\label{eq:cap16_5}
	\end{equation}
dove le costanti $c_{\pm}$ sono determinate imponendo la corretta normalizzazione dei vettori di stato.\\

Supponiamo ora di applicare più volte, in successione, l'operatore $J_+$ ad un autostato $\vert a,b \rangle $. Ad ogni applicazione l'autovalore di $J_z$ aumenta di $\hbar$, mentre l'autovalore di $J^2$ è invariato. Questo processo, tuttavia, non può continuare in modo indefinito giacché, per un fissato valore $a$ di $J^2$, deve esistere un valore massimo, $b_{MAX}$ per $J_z$. Questo segue dal fatto che la differenza $J^2-J_z ^2=J_x ^2+J_y ^2$ è l'operatore di una grandezza fisica essenzialmente positiva e i suoi autovalori non possono essere negativi:
	\begin{align}
		\langle a,b \vert J^2-J_z ^2 \vert a, b \rangle &= (a-b^2) =  \langle a,b \vert J_x ^2+J_y ^2 \vert a, b \rangle = \nonumber \\
		&=\left(\langle a,b \vert J_x ^+\right) \left(J_x \vert a, b \rangle\right)+\left(\langle a,b \vert J_y ^+\right) \left(J_y \vert a, b \rangle\right) \geq 0.
	\end{align}
Dunque
	\begin{equation}
		\tcboxmath[sharp corners=downhill, colback=white, colframe=red!75!black]{
			b^2 \leq a .
			}
	\label{eq:cap16_6}
	\end{equation}
Deve allora esistere un $b_{MAX}$ tale che:
	\begin{equation}
		\tcboxmath[sharp corners=downhill, colback=white, colframe=red!75!black]{
			J_{+} \vert a, b_{MAX} \rangle =0.
			}
	\end{equation}\\
	
Per calcolare il valore di $b_{MAX}$ possiamo applicare l'operatore $J_-$ alla precedente equazione ed osservare che 
	\begin{equation}
		J_-J_+ = (J_x-iJ_y)(J_x+iJ_y)= J_x^2 +J_y ^2+i[J_x, J_y] ,
	\end{equation}
ossia 
	\begin{equation}
		\tcboxmath[sharp corners=downhill, colback=white, colframe=red!75!black]{
			J_-J_+ = J^2 -J_z ^2 -\hbar J_z .
			}
	\label{eq:cap16_7}
	\end{equation}
Si ha allora:
	\begin{align}
		0&=J_{-}J_{+} \vert a, b_{MAX} \rangle = (J^2 -J_z ^2 -\hbar J_z)\vert a, b_{MAX} \rangle = \nonumber \\
		&=(a- b_{MAX} ^2 - \hbar b_{MAX} )\vert a, b_{MAX} \rangle,
	\end{align}
ossia
	\begin{equation}
		\tcboxmath[sharp corners=downhill, colback=white, colframe=red!75!black]{
			b_{MAX}(b_{MAX}+\hbar) =a .
			}
	\label{eq:cap16_8}
	\end{equation}\\
	
Similmente, l'eq. (\ref{eq:cap16_6}) implica anche l'esistenza di un valore minimo di $b$,  $b_{MIN}$, definito dall'equazione
	\begin{equation}
		\tcboxmath[sharp corners=downhill, colback=white, colframe=red!75!black]{
			J_{-} \vert a, b_{MIN} \rangle =0.
			}
	\end{equation}
Tale valore si può calcolare applicando l'operatore $J_+$ alla precedente equazione ed osservando che
	\begin{equation}
		J_+J_- = (J_x+iJ_y)(J_x-iJ_y)= J_x^2 +J_y ^2-i[J_x, J_y] ,
	\end{equation}
ossia
	\begin{equation}
		\tcboxmath[sharp corners=downhill, colback=white, colframe=red!75!black]{
			J_-J_+ = J^2 -J_z ^2 +\hbar J_z .
			}
	\label{eq:cap16_9}
	\end{equation}
Si trova allora
	\begin{align}
		0&=J_{+}J_{-} \vert a, b_{MIN} \rangle = (J^2 -J_z ^2 +\hbar J_z)\vert a, b_{MIN} \rangle = \nonumber \\
		&=(a- b_{MIN} ^2 - \hbar b_{MIN} )\vert a, b_{MIN} \rangle ,
	\end{align}
da cui
	\begin{equation}
		\tcboxmath[sharp corners=downhill, colback=white, colframe=red!75!black]{
			b_{MIN}(b_{MIN}-\hbar) =a .
			}
	\label{eq:cap16_10}
	\end{equation}\\
	
Dal confronto delle eq. (\ref{eq:cap16_8}) e (\ref{eq:cap16_10}) (con l'ipotesi $b_{MAX} > b_{MIN}$) vediamo che
	\begin{equation}
		\tcboxmath[sharp corners=downhill, colback=white, colframe=red!75!black]{
			b_{MAX} = - b_{MIN} ,
			}
	\end{equation}
e dunque i valori possibili per $b$ sono compresi nell'intervallo
	\begin{equation}
		\tcboxmath[sharp corners=downhill, colback=white, colframe=red!75!black]{
			-b_{MAX} \leq b \leq b_{MAX} .
			}
	\end{equation}\\
	
Osserviamo che l'autostato corrispondente all'autovalore massimo di $J_z$, $\vert a, b_{MAX}\rangle$, può essere ottenuto applicando un numero $n$ (intero) di volte l'operatore $J_+$ all'autostato corrispondente all'autovalore minimo, $\vert a, - b_{MAX}\rangle $. Questo implica
	\begin{equation}
		b_{MAX}=-b_{MAX}+n\hbar, 
	\end{equation}
cioè
	\begin{equation}
		\tcboxmath[sharp corners=downhill, colback=white, colframe=red!75!black]{
			b_{MAX}=\frac{n\hbar}{2},
			} \qquad n\textrm{ intero.}
	\end{equation}\\
	
Solitamente il valore di $b_{MAX}$ in unità $\hbar$ è indicato con $j$, così che
	\begin{equation}
		\tcboxmath[sharp corners=downhill, colback=white, colframe=red!75!black]{
			j= \frac{b_{MAX}}{\hbar} \frac{n}{2},
			} \qquad n \textrm{ intero.}
	\end{equation}
L'eq. (\ref{eq:cap16_8}) indica poi che
	\begin{equation}
		\tcboxmath[sharp corners=downhill, colback=white, colframe=red!75!black]{
			a= \hbar ^2 j(j+1).
			}
	\end{equation}\\

Definiamo anche $m$ come il generico autovalore di $J_z$ in unità $\hbar$:
	\begin{equation}
		\tcboxmath[sharp corners=downhill, colback=white, colframe=red!75!black]{
			b=m\hbar .
			}
	\end{equation}
Il numero $m$ può assumere allora tutti i valori compresi tra $-j$ e $j$ e distanziati tra loro di 1.\\

Possiamo quindi riassumete i \textbf{risultati} derivati per gli autovalori di $J^2$ e $J_z$ nella forma seguente:
	\begin{equation}
		 \tcboxmath[enhanced, sharp corners=downhill, colback=yellow!50!white, colframe=red!75!black, borderline={2pt}{-3pt}{red!50!black}]{
		 	J^2\vert j, m \rangle =\hbar ^2 j (j+1)\vert j, m \rangle , 
		 	}
	\end{equation}
	\begin{equation}
		 \tcboxmath[enhanced, sharp corners=downhill, colback=yellow!50!white, colframe=red!75!black, borderline={2pt}{-3pt}{red!50!black}]{
		 	J_z\vert j, m \rangle =\hbar  m\vert j, m \rangle ,
		 	}
	\end{equation}
\textbf{dove $j$ può assumere tutti i valori, interi o seminteri positivi, incluso  lo zero,}
	\begin{equation}
		\tcboxmath[enhanced, sharp corners=downhill, colback=yellow!50!white, colframe=red!75!black, borderline={2pt}{-3pt}{red!50!black}]{
			j= \frac{n}{2},
			}
	\end{equation}
\textbf{ed $m$ può assumere i valori}:
	\begin{equation}
		\tcboxmath[enhanced, sharp corners=downhill, colback=yellow!50!white, colframe=red!75!black, borderline={2pt}{-3pt}{red!50!black}]{
			m=-j, -j+1,\dots , j-1, j.
			}
	\end{equation}\\
	
È importante osservare come \textbf{la quantizzazione del momento angolare è una diretta conseguenza della sole regole di commutazione del momento  angolare che a loro volta discendono dalle proprietà delle rotazioni e della definizione di $\vec{j}$ come generatore delle rotazioni}.\\

Per concludere questo studio \textbf{deduciamo le espressioni per gli elementi di matrice delle componenti $J_x$ e $J_y$ del momento angolare nella base degli autostati di $J^2$ e $J_z$}.\\

È conveniente a tale proposito considerare dapprima gli elementi di matrice degli operatori a scala $J_+$ e $J_-$. Scriviamo allora le equazioni (\ref{eq:cap16_5}) nella forma:
	\begin{equation}
		\begin{cases}
			J_+\vert j, m \rangle = c_{jm} ^+\vert j, m+1 \rangle , \\
			J_-\vert j, m \rangle = c_{jm} ^-\vert j, m-1 \rangle ,
		\end{cases}
	\end{equation}
Utilizzando le equazioni (\ref{eq:cap16_7}) e (\ref{eq:cap16_9})otteniamo:
	\begin{align}
		\vert c_{jm} ^+\vert ^2 &= \langle j, m \vert J_- J_+\vert j, m \rangle = \langle j, m \vert J^2- J_z ^2 -\hbar J_z\vert j, m \rangle =\nonumber \\
		&= \hbar ^2 \left( j(j+1) - m (m+1)\right) ,
	\end{align}
e
	\begin{align}
		\vert c_{jm} ^-\vert ^2 &= \langle j, m \vert J_+ J_-\vert j, m \rangle = \langle j, m \vert J^2- J_z ^2 +\hbar J_z\vert j, m \rangle =\nonumber \\
		&= \hbar ^2 \left( j(j+1) - m (m-1)\right) .
	\end{align}\\
	
Le precedenti equazioni determinano i coefficienti $c^+$ e $c^-$ a meno di un fattore di fase. In generale si usa scegliere $c^+$ e $c^-$ reali e positivi, definendo in tal modo la fase (arbitraria) degli autostati $\vert j,m \rangle$ di $J^2$ e $J_z$. Si trova quindi:
	\begin{equation}
		\tcboxmath[enhanced, sharp corners=downhill, colback=yellow!50!white, colframe=red!75!black, borderline={2pt}{-3pt}{red!50!black}]{
 			J_+\vert j, m \rangle =\hbar \sqrt{j(j+1)-m(m+1)}\vert j, m+1 \rangle ,
 			}
	\label{eq:cap16_11}
	\end{equation}
	\begin{equation}
		\tcboxmath[enhanced, sharp corners=downhill, colback=yellow!50!white, colframe=red!75!black, borderline={2pt}{-3pt}{red!50!black}]{
			J_-\vert j, m \rangle =\hbar \sqrt{j(j+1)-m(m-1)}\vert j, m-1 \rangle  .
			}
	\label{eq:cap16_12}
	\end{equation}
Poiché le componenti $J_x$ e $J_y$ del momento angolare sono legate agli operatori a scala dalle semplici relazioni:
\begin{equation}
J_x= \frac{1}{2}\left(J_+ + J_-\right),\qquad J_y= \frac{1}{2i}\left(J_+ - J_-\right)
\end{equation}
le equazioni (\ref{eq:cap16_11}) e (\ref{eq:cap16_12}) ci consentono di ricavare immediatamente le espressioni cercate per gli elementi di matrice di $J_x$ e $J_y$:
	\begin{equation}
		\tcboxmath[enhanced, sharp corners=downhill, colback=yellow!50!white, colframe=red!75!black, borderline={2pt}{-3pt}{red!50!black}]{
		\begin{aligned}
			\langle j, m-1\vert J_x\vert j, m \rangle &=\langle j, \vert J_x\vert j, m-1 \rangle =   \\
			&= \frac{\hbar}{2}\sqrt{j(j+1)-m(m-1)} , 
		\end{aligned}
		}
	\end{equation}
	\begin{equation}
		\tcboxmath[enhanced, sharp corners=downhill, colback=yellow!50!white, colframe=red!75!black, borderline={2pt}{-3pt}{red!50!black}]{
		\begin{aligned}
			\langle j, m-1\vert J_y\vert j, m \rangle &= -\langle j, \vert J_y\vert j, m-1 \rangle =  \\
			&= \frac{i\hbar}{2}\sqrt{j(j+1)-m(m-1)} , 
		\end{aligned}
		}
	\end{equation}
tutti gli altri elementi di matrice sono invece nulli.\\

Osserviamo in particolare che nelle matrici $J_x$ e $J_y$ gli elementi diagonali sono tutti nulli. Poiché l'elemento di matrice diagonale dà il valore medio delle grandezze nello stato corrispondente, ciò significa che \textbf{negli stati con valori determinati di $J_z$, i valori medi di $J_x$ e $J_y$ sono nulli}:
	\begin{equation}
		\tcboxmath[sharp corners=downhill, colback=white, colframe=red!75!black]{
			\langle J_x \rangle = \langle J_y \rangle =0.
		}
	\end{equation}
 %DEFINITIVO
\pagestyle{VS}
\chapter[Momento angolare orbitale]{Momento angolare orbitale}
Abbiamo introdotto il momento angolare definendolo come il generatore delle rotazioni infinitesime. Ma quando il momento di spin è nullo (o comunque può essere ignorato) il momento angolare per una particella coincide con il \textbf{momento angolare orbitale}, definito da:
	\begin{equation}
		\tcboxmath[enhanced, sharp corners=downhill, colback=yellow!50!white, colframe=red!75!black, borderline={2pt}{-3pt}{red!50!black}]{
			\vec{L}=\vec{r}\wedge\vec{p}.
			}
	\end{equation}\\
	
Per il momento angolare orbitale, le \textbf{regole di commutazione} fondamentali,
	\begin{equation}
		\tcboxmath[enhanced, sharp corners=downhill, colback=yellow!50!white, colframe=red!75!black, borderline={2pt}{-3pt}{red!50!black}]{
			[L_i, L_j]= i \varepsilon _{ijk}L_k ,
			}
	\label{eq:cap17_1}
	\end{equation}
seguono direttamente dalle regole di commutazione degli operatori di posizione ed impulso. Si ha, ad esempio,
	\begin{align}
		[L_x , L_y] &= [y p_z - zp_y, zp_x -xp_z] =  [y p_z , zp_x] + [zp_y, xp_z] = \nonumber \\
		&= yp_x [p_z , z] + p_y x [z , p_z] =   -i\hbar (yp_x -xp_y) = i\hbar L_z ,
	\end{align}
e analoghe per le altre componenti.\\

Possiamo mostrare esplicitamente come l'operatore momento angolare, definito dall'eq. (\ref{eq:cap17_1}), coincida, per particelle di spin nullo, con il \textbf{generatore delle rotazioni infinitesime}. Applichiamo infatti l'operatore:
	\begin{equation}
		\left(1- \frac{i}{\hbar}\delta \varphi L_z\right)
	\end{equation}
su un autostato arbitrario della posizione, e mostriamo come lo stato risultante sia ancora un autostato della posizione ma ruotato, rispetto allo stato di partenza, di un angolo $\delta \varphi$ attorno all'asse $z$. Questo risultato segue dalla considerazione che l'impulso è il generatore delle traslazioni infinitesime. Si ha infatti:
	\begin{equation}
		\tcboxmath[sharp corners=downhill, colback=white, colframe=red!75!black]{
		\begin{aligned}
			\left(1- \frac{i}{\hbar}\delta \varphi L_z\right)\vert x', y', z'\rangle &= \underbrace{\left[1- \frac{i}{\hbar}\delta \varphi\left(x'p_y-y'p_x\right)\right]}_{\begin{array}{cc}
			\scriptstyle{(1- \frac{i}{\hbar}\vec{p}\cdot d\vec{x})}\\
			\scriptstyle{d\vec{x}= (-y\delta \varphi ,\ x \delta \varphi ,\ 0)}
			\end{array}}\vert x', y', z'\rangle =  \\
			&=\vert x'-y'\delta \varphi , y'+x'\delta \varphi , z'\rangle ,
		\end{aligned}
		}
	\end{equation}
Ricordando l'espressione della matrice di rotazione $R_z (\delta \varphi) \vec{x}^{\, \prime}$, vediamo che lo stato ottenuto è effettivamente ruotato, rispetto all'autostato iniziale della posizione, di un angolo $\delta \varphi$ attorno all'asse $z$:			\begin{equation}
		R_z (\delta \varphi)\vec{x}^{\, \prime}=
		\begin{pmatrix}
		1 & -\delta \varphi & 0\\
		\delta \varphi & 1 & 0 \\
		0 & 0 & 1 \\
		\end{pmatrix}
		\begin{pmatrix}
		x' \\ y' \\ z'
		\end{pmatrix} =
		\begin{pmatrix}
		x'-y'\delta\varphi \\y'+ x'\delta \varphi \\z'
		\end{pmatrix}
	\end{equation}
Lo stesso risultato può essere convenientemente espresso utilizzando, per definire l'autostato della posizione, un sistema di coordinate polari:
	\begin{equation}
		\tcboxmath[sharp corners=downhill, colback=white, colframe=red!75!black]{
			\left(1- \frac{i}{\hbar}\delta \varphi L_z\right)\vert r, \theta, \varphi\rangle = \vert r, \theta, \varphi + \delta\varphi\rangle .
			}
	\label{eq:cap17_2}
	\end{equation}\\
	
Proponiamoci ora di derivare l'\textbf{espressione dell'operatore $L_z$ nella rappresentazione delle coordinate}. Utilizzando sempre un sistema di coordinate polari, e tenendo conto dell'eq. (\ref{eq:cap17_2}), troviamo:
	\begin{align}
		\langle r, \theta, \varphi \vert \left(1- \frac{i}{\hbar}\delta \varphi L_z\right) \vert \alpha\rangle &= \langle r, \theta, \varphi \vert \left(1+ \frac{i}{\hbar}\delta \varphi L_z\right) ^+ \vert \alpha\rangle =  \langle r, \theta, \varphi - \delta \varphi \vert\alpha\rangle  = \nonumber \\
		&= \langle r, \theta, \varphi \vert \alpha\rangle - \delta \varphi \frac{\partial}{\partial \varphi}\langle r, \theta, \varphi \vert\alpha\rangle .
	\end{align}
per l'arbitrarietà dello stato $\vert \alpha \rangle$, questo risultato implica:
	\begin{equation}
		\tcboxmath[enhanced, sharp corners=downhill, colback=yellow!50!white, colframe=red!75!black, borderline={2pt}{-3pt}{red!50!black}]{
			\langle r, \theta, \varphi \vert  L_z \vert \alpha\rangle =- i\hbar\ \frac{\partial}{\partial \varphi}\langle r, \theta, \varphi \vert\alpha\rangle ,
			}
	\end{equation}
ossia nella rappresentazione delle coordinate
	\begin{equation}
		\tcboxmath[enhanced, sharp corners=downhill, colback=yellow!50!white, colframe=red!75!black, borderline={2pt}{-3pt}{red!50!black}]{
			L_z = -i\hbar \frac{\partial}{\partial \varphi}.
			}
	\label{eq:cap17_4}
	\end{equation}\\
	
Allo stesso stato si può giungere, altrettanto facilmente, utilizzando l'espressione nella rappresentazione delle coordinate  dell'operatore impulso. Si ha infatti:
	\begin{equation}
		\tcboxmath[sharp corners=downhill, colback=white, colframe=red!75!black]{
			L_z= xp_y-yp_x = -i\hbar \left( x\frac{\partial}{\partial y} -y\frac{\partial}{\partial x}\right) .
			}
	\label{eq:cap17_3}
	\end{equation}
Effettuando il cambio di variabile da coordinate cartesiane a coordinate polari si riottiene l'espressione (\ref{eq:cap17_4}). I calcoli espliciti sono riportati in appendice. Nella stessa appendice sono anche riportate le analoghe derivazioni delle \textbf{espressioni nella rappresentazione delle coordinate degli operatori $L_x$, $L_y$ e del quadrato $L^2$}. In coordinate cartesiane si ha
	\begin{equation}
		\tcboxmath[sharp corners=downhill, colback=white, colframe=red!75!black]{
		\begin{aligned}
			L_x &= yp_z - zp_y = -i\hbar \left( y\frac{\partial}{\partial z}- z\frac{\partial}{\partial y}\right), \\
			L_y &= zp_x - xp_z = -i\hbar \left( z\frac{\partial}{\partial x}- x\frac{\partial}{\partial z}\right),
		\end{aligned}
	}
	\end{equation}
e, in coordinate polari,
	\begin{equation}
		\tcboxmath[enhanced, sharp corners=downhill, colback=yellow!50!white, colframe=red!75!black, borderline={2pt}{-3pt}{red!50!black}]{
		\begin{aligned}
			L_x &= i\hbar \left(  \sin \varphi \frac{\partial}{\partial \theta} + \cot \theta \cos \varphi \frac{\partial}{\partial \varphi}\right), \\
			L_y &= i\hbar \left( - \cos \varphi \frac{\partial}{\partial \theta} + \cot \theta \sin \varphi \frac{\partial}{\partial \varphi}\right).
		\end{aligned}
		}
	\end{equation}
Quanto al quadrato del momento angolare orbitale questo risulta espresso da
	\begin{equation}
		\tcboxmath[enhanced, sharp corners=downhill, colback=yellow!50!white, colframe=red!75!black, borderline={2pt}{-3pt}{red!50!black}]{
			L^2 = -\hbar ^2 \left[\frac{1}{\sin ^2 \theta}\frac{\partial ^2}{\partial \varphi ^2}+\frac{1}{\sin \theta}\frac{\partial}{\partial \theta} \left(\sin \theta \frac{\partial}{\partial \theta}\right)\right] .
		}
	\end{equation}
Osserviamo che, a meno di un fattore, $L^2$ è la parte angolare dell'operatore di Laplace $\nabla ^2$.
\section[Autovalori del momento angolare orbitale e armoniche sferiche]{Autovalori del momento angolare orbitale e armoniche sferiche}
Consideriamo gli autostati simultanei degli operatori $L^2$ ed $L_z$, definiti dalle equazioni:
	\begin{equation}
		 \tcboxmath[enhanced, sharp corners=downhill, colback=yellow!50!white, colframe=red!75!black, borderline={2pt}{-3pt}{red!50!black}]{
			L^2 \vert l, m \rangle = \hbar ^2 l(l+1) \vert l, m \rangle ,\label{eq:cap17_6}
		}
	\end{equation}
	\begin{equation}
		\tcboxmath[enhanced, sharp corners=downhill, colback=yellow!50!white, colframe=red!75!black, borderline={2pt}{-3pt}{red!50!black}]{
L_z \vert l, m \rangle = \hbar m \vert l, m \rangle
\label{eq:cap17_7}.
		}
	\end{equation}
Seguendo una pratica usuale abbiamo qui indicato con $l$ il numero quantico $j$ riferito al momento angolare orbitale. La componente $z$ del momento angolare può quindi assumere i valori definiti da
	\begin{equation}
		\tcboxmath[enhanced, sharp corners=downhill, colback=yellow!50!white, colframe=red!75!black, borderline={2pt}{-3pt}{red!50!black}]{
			m = -l, -l+1, \dots,l-1, l .
			}
	\end{equation}\\

\textbf{L'assegnazione dei valori $l$ ed $m$ non definisce completamente lo stato della particella}. Ciò è evidente già dal fatto che le espressioni degli operatori $L^2$ e $L_z$, in coordinate sferiche, contengono solamente gli angoli $\theta$ e $\varphi$, così che le loro autofunzioni possono contenere un fattore arbitrario dipendente da $r$- In questo contesto ci limitiamo allora solo a considerare la parte dipendente dagli angoli $\theta$ e $\varphi$ degli autostati di posizione; in altri termini, indicheremo con $\vert \theta ,\varphi\rangle$ il vettore di stato di una particella che si trova in un punto dello spazio individuato dagli angoli $\theta $ e $\varphi$, ma a distanza $r$ arbitraria dall'origine delle coordinate.\\

Moltiplichiamo scalarmente entrambi i membri delle equazioni (\ref{eq:cap17_6}) e (\ref{eq:cap17_7}) per il bra $\langle \theta , \varphi \vert$:
	\begin{equation}
		\tcboxmath[sharp corners=downhill, colback=white, colframe=red!75!black]{
			\langle \theta , \varphi \vert L^2 \vert l, m \rangle = \hbar ^2 l(l+1) \langle \theta , \varphi \vert l, m \rangle ,
			}
	\end{equation}
	\begin{equation}
		\tcboxmath[sharp corners=downhill, colback=white, colframe=red!75!black]{
			\langle \theta , \varphi \vert L_z \vert l, m \rangle = \hbar m\langle \theta , \varphi \vert l, m \rangle .
			}
	\end{equation}
Le funzioni
	\begin{equation}
		\tcboxmath[enhanced, sharp corners=downhill, colback=yellow!50!white, colframe=red!75!black, borderline={2pt}{-3pt}{red!50!black}]{
			Y_{l,m} (\theta, \varphi )= \langle \theta , \varphi \vert l, m \rangle 
			}
	\end{equation}
sono dunque le \textbf{autofunzioni simultanee degli operatori $L^2$ ed $L_z$} e sono note con il nome di \textbf{armoniche sferiche}. Fisicamente, queste funzioni rappresentano l'\textbf{ampiezza di probabilità che un sistema, caratterizzato dai valori $l$ ed $m$ dei numeri quantici del momento angolare, si trovi in una posizione la cui direzione è definita dal valori $\theta$ e $\varphi$ degli angoli delle coordinate polari}.\\

Le espressioni derivate in precedenza per gli operatori del momento angolare nella rappresentazione delle coordinate, ci consentono di scrivere esplicitamente le \textbf{equazioni agli autovalori che definiscono le armoniche sferiche}:
	\begin{align}
		\tcboxmath[enhanced, sharp corners=downhill, colback=yellow!50!white, colframe=red!75!black, borderline={2pt}{-3pt}{red!50!black}]{
		\begin{aligned}
			L^2 Y_{l,m} (\theta, \varphi) &=-\hbar ^2 \left[\frac{1}{\sin ^2 \theta}\frac{\partial ^2}{\partial \varphi ^2}+\frac{1}{\sin \theta}\frac{\partial}{\partial \theta} \left(\sin \theta \frac{\partial}{\partial \theta}\right)\right]Y_{l,m}(\theta, \varphi) =   \\
			&= \hbar ^2 l(l+1)Y_{l,m}(\theta, \varphi) ,
		\end{aligned}
		}\nonumber \\
		\quad
	\label{eq:cap17_8}
	\end{align}
	\begin{equation}
		\tcboxmath[enhanced, sharp corners=downhill, colback=yellow!50!white, colframe=red!75!black, borderline={2pt}{-3pt}{red!50!black}]{
			L_zY_{l,m}(\theta, \varphi)= -i\hbar \frac{\partial}{\partial \varphi}Y_{l,m}(\theta, \varphi)= \hbar m Y_{l,m}(\theta, \varphi).
			}
	\label{eq:cap17_9}.
	\end{equation}\\

Dalla condizione di normalizzazione degli autostati $\vert l,m \rangle$ e dalla completezza degli autostati $\vert \theta , \varphi\rangle $ della posizione, segue la condizione di normalizzazione delle armoniche sferiche:
	\begin{equation}
		\langle l' , m' \vert l,m \rangle = \int d\Omega \ \langle l' ,m'\vert \theta, \varphi \rangle \langle \theta , \varphi \vert l, m \rangle = \delta _{ll'} \delta _{mm'} ,
	\end{equation}
ossia
	\begin{equation}
		\tcboxmath[enhanced, sharp corners=downhill, colback=yellow!50!white, colframe=red!75!black, borderline={2pt}{-3pt}{red!50!black}]{
		\int _0 ^{2\pi} d\varphi \int _{-1} ^1 d(\cos \theta )\ Y_{l',m'}^* (\theta , \varphi ) Y_{l,m} (\theta, \varphi )=\delta _{ll'} \delta _{mm'}.
		}
	\end{equation}\\
	
Le equazioni differenziali (\ref{eq:cap17_8}) e (\ref{eq:cap17_9}), che definiscono le armoniche sferiche, ammettono una soluzione per separazione delle variabili $\theta$ e $\varphi)$, della forma:
	\begin{equation}
		\tcboxmath[sharp corners=downhill, colback=white, colframe=red!75!black]{
			Y_{l,m} (\theta , \varphi ) = \Theta _{l,m}(\theta)\Phi _m (\varphi) ,
			}
	\end{equation} 
dove le funzioni $\Theta _{l,m}(\theta)$ e $\Phi _m (\varphi)$ sono separatamente normalizzate:
	\begin{equation}
		\tcboxmath[sharp corners=downhill, colback=white, colframe=red!75!black]{
		\begin{aligned}
			&  \int _0 ^{2\pi} d\varphi\ \vert \Phi _m (\varphi)\vert ^2 =1, \\
			& \int  _{-1} ^1 d\varphi\ \vert \Theta _{l,m}(\theta) \vert ^2 =1. 
			\end{aligned} 
	}
	\end{equation}\\
	
L'eq. (\ref{eq:cap17_9}) indica che le funzioni $\Phi _m$ sono in particolare le autofunzioni della componente $z$ del momento angolare, mentre le funzioni $\Theta _{l,m}$ non sono di per sé autofunzioni di qualche operatore del momento angolare.\\

L'eq. (\ref{eq:cap17_9}), che possiamo scrivere nella forma:
	\begin{equation}
		\tcboxmath[sharp corners=downhill, colback=white, colframe=red!75!black]{
		-i\frac{\partial}{\partial \varphi}\Phi _m (\varphi )= m \Phi _m (\varphi ) ,
		}
	\end{equation}
ha come soluzioni normalizzate le funzioni:
	\begin{equation}
		\tcboxmath[enhanced, sharp corners=downhill, colback=yellow!50!white, colframe=red!75!black, borderline={2pt}{-3pt}{red!50!black}]{
			\Phi _m (\varphi ) = \frac{1}{\sqrt{2\pi}}e^{im\varphi} .
			}
	\end{equation}\\
	
La condizione che la funzione d'onda sia monotona (ossia ad un solo valore) implica in particolare
	\begin{equation}
		\Phi _m (0) =\Phi _m (2\pi ) ,
	\label{eq:cap17_10}
	\end{equation}
e dunque \textbf{l'autovalore $m$ può assumere solo valori interi} (positivi o negativi):
	\begin{equation}
		\tcboxmath[enhanced, sharp corners=downhill, colback=yellow!50!white, colframe=red!75!black, borderline={2pt}{-3pt}{red!50!black}]{
			m= 0, \pm 1,\pm 2, \dots
			}
	\end{equation}
Poiché i valori di $m$ sono anche definiti dalla relazione
	\begin{equation}
		\tcboxmath[enhanced, sharp corners=downhill, colback=yellow!50!white, colframe=red!75!black, borderline={2pt}{-3pt}{red!50!black}]{
			m = -l, -l+1,\dots, l-1,l
			}
	\end{equation}
ne segue che anche \textbf{il numero $l$ può assumere solo valori interi} (positivi, incluso lo zero):
	\begin{equation}
		\tcboxmath[enhanced, sharp corners=downhill, colback=yellow!50!white, colframe=red!75!black, borderline={2pt}{-3pt}{red!50!black}]{
			l= 0,1,2,\dots
			}
	\end{equation}\\
	
\textbf{Sottolineiamo come le regole di commutazione del momento angolare implichino soltanto la condizione che $j$ (o $l$) e, dunque, $m$  siano numeri intero o semi-interi. La condizione che $l$ ed $m$ siano invece numeri interi è una condizione aggiunta valida specificatamente per il momento angolare orbitale e che non si applica pertanto al momento angolare di spin}.\\

La determinazione delle funzioni $\Theta _{lm} (\theta)$ può essere effettuata risolvendo l'equazione agli autovalori per l'operatore $L^2$ e sostituendo per $\Phi _m (\varphi)$ la sua espressione (\ref{eq:cap17_10}). Risulta tuttavia conveniente seguire un'altra strada\footnote{Il procedimento qui esposto è il cosiddetto metodo matriciale, analogo a quello discusso per le autofunzioni dell'oscillatore atmonico.}. Consideriamo in primo luogo l'autostato corrispondente ad $m=l$. Questo soddisfa l'equazione
	\begin{equation}
		\tcboxmath[sharp corners=downhill, colback=white, colframe=red!75!black]{
			L_+ \vert l,l\rangle =0,
			}
	\end{equation}
la cui espressione, nella rappresentazione delle coordinate si ottiene moltiplicando l'equazione per il bra $\vert \theta, \varphi \rangle$ ed utilizzando per  l'operatore $L_+$ la sua rappresentazione(\ref{eq:cap17_5}). Si ottiene in tal modo:
	\begin{equation}
		\tcboxmath[sharp corners=downhill, colback=white, colframe=red!75!black]{
			L_+Y_{ll}(\theta , \varphi ) = \hbar e^{i\varphi}\left[\frac{\partial}{\partial\theta}+i \cot\theta \frac{\partial}{\partial \varphi}\right]Y_{ll}(\theta , \varphi ) =0 .
			}
	\end{equation}
Sostituendo nella precedente equazione 
	\begin{equation}
		Y_{ll}(\theta , \varphi ) = \frac{1}{\sqrt{2\pi}}e^{il\varphi}\ \Theta _{ll} (\theta ) ,
	\end{equation}
Otteniamo per $\Theta _{ll} (\theta )$ l'equazione
	\begin{equation}
		\tcboxmath[sharp corners=downhill, colback=white, colframe=red!75!black]{
			\frac{d\Theta _{ll}}{d \theta}-l\cot\theta \ \Theta _{ll} (\theta )=0 ,
			}
	\end{equation}
la cui soluzione si ricava facilmente
	\begin{align}
	& \frac{d\Theta _{ll}}{d \theta}=l\cot\theta \ \Theta _{ll} (\theta ) \Rightarrow \nonumber \\
	& \Rightarrow \ln \Theta _{ll} = l \int d\theta \ \frac{\cos \theta}{\sin \theta}+ \textrm{cost} = l \int  \frac{d\sin \theta}{\sin \theta}+ \textrm{cost} = l \log \sin \theta + \textrm{cost} ,
	\end{align}
ossia
	\begin{equation}
		\tcboxmath[sharp corners=downhill, colback=white, colframe=red!75!black]{
			\Theta _{ll} (\theta )= c_l \sin ^l \theta .
			}
	\end{equation}
la costante $c_l$ si ricava (a meno di una fase arbitraria) dalla condizione di normalizzazione:
	\begin{equation}
		\int _{-1} ^1 d(\cos \theta )\ \vert \Theta _{ll} (\theta ) \vert ^2 = \vert c_l \vert ^2 \int _{-1} ^1 d(\cos \theta )\ \left( \sin \theta \right) ^{2l} =1.
	\end{equation}
Per determinare il valore dell'integrale effettuiamo in primo luogo un'integrazione per parti:
	\begin{eqnarray}
\int _{-1} ^1 d(\cos \theta )\ (\sin \theta ) ^{2l} 			&=& \int _{0} ^{\pi} d\theta \sin \theta\ (\sin \theta ) ^{2l} = \nonumber \\
		&=& \left. -\cos \theta (\sin \theta ) ^{2l} \right\vert _0 ^{\pi} +2l\int _{0} ^{\pi} d\theta \cos ^2 \theta \ (\sin \theta ) ^{2l-1} = \nonumber \\
		&=&  2l\int _{0} ^{\pi} d\theta \ (\sin \theta ) ^{2l-1} - 2l\int _{0} ^{\pi} d\theta  \ (\sin \theta ) ^{2l+1} .
	\end{eqnarray}
Vediamo allora che 
	\begin{equation}
		\int _{0} ^{\pi} d\theta \ (\sin \theta ) ^{2l+1}=\frac{2l}{2l+1} \int _{0} ^{\pi} d\theta \ (\sin \theta ) ^{2l-1}.
	\end{equation}
Applicando ricorsivamente questa formula otteniamo:
	\begin{eqnarray}
		\int _{0} ^{\pi} d\theta \ (\sin \theta ) ^{2l+1} & = & \frac{2l(2l-2)}{(2l+1)(2l-1)} \int _{0} ^{\pi} d\theta \ (\sin \theta ) ^{2l-3} = \nonumber \\
		&=& \frac{2l(2l-2)\dots 2}{(2l+1)(2l-1)\dots 3} \int _{0} ^{\pi} d\theta \ \sin \theta =\nonumber \\
		&=& \frac{[2l(2l-2)\dots 2]^2}{(2l+1)(2l-1)\dots 2\cdot 1}\cdot 2 \nonumber \\
		& = &  \frac{[2^l\ l(l-1)(l-2)\dots 1]^2}{(2l+1)!}\cdot 2 \nonumber \\
		&=& \frac{2[2^l\ l!]^2}{(2l+1)!}.
	\end{eqnarray}
L'inverso di questo integrale è pari a $\vert c_l \vert ^2$. La fase di $c_l$ è scelta per convenzione uguale a $(-1)^l$, così che in definitiva si ottiene
	\begin{equation}
		\tcboxmath[sharp corners=downhill, colback=white, colframe=red!75!black]{
			\Theta _{ll} (\theta ) = (-1)^l \sqrt{\frac{(2l+1)!}{2}}\frac{1}{2^l\ l!}\sin^l \theta ,
			}
	\end{equation}
e per l'autofunzione completa l'espressione
	\begin{equation}
		\tcboxmath[enhanced, sharp corners=downhill, colback=yellow!50!white, colframe=red!75!black, borderline={2pt}{-3pt}{red!50!black}]{
			Y_{ll} (\theta , \varphi) = (-1)^l \sqrt{\frac{(2l+1)!}{4\pi}}\frac{1}{2^l\ l!}e^{il\varphi} \sin ^l \theta  .
			}
	\end{equation}\\
	
Le autofunzioni $Y_{lm}(\theta , \varphi)$ con $m<l$ possono essere determinate mediante successive applicazione dell'operatore a scala $L_-$. Dalla relazione
	\begin{equation}
		L_-\vert l, m+1 \rangle = \hbar \sqrt{l(l+1) - m(m+1)} \vert l,m \rangle = \hbar \sqrt{(l-m)(l+m+1)} \vert l,m \rangle ,
	\end{equation}
vediamo che 
	\begin{equation}
	(L_-)^2 \vert l, m+2 \rangle = \hbar ^2 \sqrt{(l-m)(l-m-1)(l+m+1)(l+m+2)}\vert l,m \rangle ,
	\end{equation}
e dunque
	\begin{equation}
	(L_-)^{l-m} \vert l, l \rangle = \hbar ^{l-m} \sqrt{\frac{(l-m)!(2l)!}{(l+m)!}}\vert l,m \rangle .
	\end{equation}
Moltiplicando scalarmente questa relazione per il bra $\langle \theta , \varphi \vert $ otteniamo per le autofunzioni del momento angolare
	\begin{equation}
		\tcboxmath[enhanced, sharp corners=downhill, colback=yellow!50!white, colframe=red!75!black, borderline={2pt}{-3pt}{red!50!black}]{	Y_{lm} (\theta ,  \varphi ) = \frac{(L_-)^{l-m}}{(\hbar )^{l-m}}\sqrt{\frac{(l+m)!}{(l-m)!(2l)!}} Y_{ll} (\theta , \varphi) . 
		}
	\end{equation}
Questa formula, insieme all'espressione (\ref{eq:cap17_5}) dell'operatore $L_-$ nella rappresentazione delle coordinate, risolve completamente il problema posto.\\

È anche possibile derivare una formula esplicita per l'applicazione successiva dell'operatore $L_-$. Secondo questa formula:
	\begin{equation}
		\frac{1}{(\hbar) ^{l-m}}(L_-)^{l-m} \left( e^{il\varphi} f(\theta) \right) = \frac{e^{im\varphi}}{(\sin \theta ) ^m} \frac{d^{l-m}}{(d\cos \theta )^{l-m}}\left( \sin ^l \theta \ f(\theta ) \right).
	\end{equation}
Utilizzando questo risultato si ottiene allora:
	\begin{align}
		\tcboxmath[enhanced, sharp corners=downhill, colback=yellow!50!white, colframe=red!75!black, borderline={2pt}{-3pt}{red!50!black}]{
			Y_{lm} (\theta ,  \varphi ) = \frac{(-1)^{l}}{2^l\ l!}\sqrt{\frac{(2l+1)}{4\pi}\frac{(l+m)!}{(l-m)!}}  \frac{e^{im\varphi}}{(\sin \theta ) ^m} \frac{d^{l-m}}{(d\cos \theta )^{l-m}}\left( \sin  \theta \right) ^{2l}.
			}\nonumber \\[0.1cm] 
	\end{align}
La dipendenza dall'angolo $\theta$ delle funzioni armoniche sferiche è rappresentata da una classe speciale di polinomi in $\cos \theta$, detti \textbf{polinomi associati di Legendre}, indicati solitamente con il simbolo $P_l ^m (\cos \theta)$. In particolare, per i valori di $m$ positivi si ha 
	\begin{equation}
		\tcboxmath[enhanced, sharp corners=downhill, colback=yellow!50!white, colframe=red!75!black, borderline={2pt}{-3pt}{red!50!black}]{
			Y_{lm} (\theta ,  \varphi ) = \sqrt{\frac{(2l+1)}{4\pi}\frac{(l+m)!}{(l-m)!}}\ P_l ^m (\cos \theta)\ e^{im\varphi}
			}\qquad m>0.
	\end{equation}
Le armoniche sferiche con valori negativi di $m$ si possono poi scrivere, in termini delle armoniche sferiche con $m$ positivo, nella forma
	\begin{equation}
		\tcboxmath[enhanced, sharp corners=downhill, colback=yellow!50!white, colframe=red!75!black, borderline={2pt}{-3pt}{red!50!black}]{
			Y_{lm} (\theta ,  \varphi ) = (-1)^{|m|}\left(Y_{l|m|} (\theta ,  \varphi )\right) ^*
			} \qquad m<0.
\end{equation}
\subsection{Appendice: derivazione degli operatori di momento angolare nella rappresentazione delle coordinate in coordinate polari}
In questa appendice deriviamo le espressioni in coordinate polari degli operatori di momento angolare orbitale $L_x$, $L_y$, $L_z$ e del quadrato $L^2$ a partire dalle analoghe espressioni in coordinate cartesiane. Considerando dapprima la componente $L_z$ si ha
\begin{equation}
L_z= xp_y - yp_x =-i\hbar \left(x\frac{\partial}{\partial y}- y\frac{\partial}{\partial x}\right)
\label{eq:cap17_11}
\end{equation}
Esprimiamo questo risultato utilizzando un sistema di coordinate polari. Dalle relazioni
\begin{equation}
\begin{cases}
x=r\sin \theta \cos \varphi \\
y=r\sin \theta \sin \varphi \\
z= r\cos\theta
\end{cases}
\qquad
\begin{cases}
r=(x^2+y^2+z^2)^{1/2} \\
\theta=\arctan\left(\sqrt{(x^2+y^2)/z^2}\right) \\
\varphi= \arctan \left(y/x\right)
\end{cases}
\end{equation}
si derivano, con una semplice algebra, le relazioni\footnote{$\partial/\partial x \arctan x = 1/(1+x^2)$}
\begin{equation}
\begin{cases}
\displaystyle{\frac{\partial}{\partial x} = \frac{\partial r}{\partial x}\frac{\partial}{\partial r}+\frac{\partial \theta}{\partial x}\frac{\partial}{\partial \theta}+ \frac{\partial \varphi}{\partial x}\frac{\partial}{\partial \varphi} = }\\
\qquad = \displaystyle{\sin\theta \cos \varphi \frac{\partial}{\partial r}+\cos\theta \cos \varphi\frac{1}{r} \frac{\partial}{\partial \theta}-\frac{\sin \varphi}{r\ \sin \theta} \frac{\partial}{\partial \varphi}}; \\
\\
\displaystyle{\frac{\partial}{\partial y} = \frac{\partial r}{\partial y}\frac{\partial}{\partial r}+\frac{\partial \theta}{\partial y}\frac{\partial}{\partial \theta}+ \frac{\partial \varphi}{\partial y}\frac{\partial}{\partial \varphi} = }\\
\qquad = \displaystyle{\sin\theta \sin \varphi \frac{\partial}{\partial r}+\cos\theta \sin \varphi\frac{1}{r} \frac{\partial}{\partial \theta}+\frac{\cos \varphi}{r\ \sin \theta} \frac{\partial}{\partial \varphi}}; \\
\\
\displaystyle{\frac{\partial}{\partial z} = \frac{\partial r}{\partial z}\frac{\partial}{\partial r}+\frac{\partial \theta}{\partial z}\frac{\partial}{\partial \theta}+ \frac{\partial \varphi}{\partial z}\frac{\partial}{\partial \varphi} = }\\
\qquad = \displaystyle{\cos\theta\frac{\partial}{\partial r}-\frac{\sin \theta}{r} \frac{\partial}{\partial \theta}}.
\end{cases}
\end{equation}
Dall'espressione (\ref{eq:cap17_11}) dell'operatore $L_z$ si ottiene pertanto
\begin{eqnarray}
L_z &=&-i\hbar \left[ r\sin \theta \cos \varphi \left(\sin \theta \sin \varphi \frac{\partial}{\partial r}+\cos \theta \sin \varphi\frac{1}{r} \frac{\partial}{\partial \theta}+ \right.\right. \nonumber \\
& &\left. + \frac{\cos \varphi}{r\ \sin \theta} \frac{\partial}{\partial \varphi}\right) - r \sin \theta \sin \varphi \left( \sin \theta \cos \varphi \frac{\partial}{\partial r}+\right. \nonumber \\
& & \left. \left. \cos \theta \cos \varphi\frac{1}{r} \frac{\partial}{\partial \theta}-\frac{\sin \varphi}{r\ \sin \theta} \frac{\partial}{\partial \varphi}\right)\right] ,
\end{eqnarray}
ossia
	\begin{equation}
		\tcboxmath[sharp corners=downhill, colback=white, colframe=red!75!black]{
		L_z = -i\hbar \frac{\partial}{\partial \varphi},
		}
	\end{equation}
in accordo con quanto ricavato precedentemente.\\

In modo analogo si possono derivare le \textbf{espressioni degli operatori nella rappresentazione delle coordinate}. Risulta conveniente, a tale scopo, derivare prima queste espressioni per gli operatori a scala $L_{\pm}$:
	\begin{align}
		L_{\pm} &= L_x \pm iL_y = (yp_z-zp_y) \pm i(zp_x-xp_z)=  \mp i (x\pm iy)p_z \pm iz(p_x\pm ip_y)= \nonumber \\
		&=\mp \hbar \left[ \left(x\pm i y\right) \frac{\partial}{\partial z}- z \left( \frac{\partial}{\partial x}\pm i\frac{\partial}{\partial y}\right) \right] ,
	\end{align}
da cui, sostituendo le coordinate polari:
	\begin{eqnarray}
		L_{\pm} &=&\mp \hbar \left[ r\sin \theta\ e^{\pm i \varphi}\ \left( \cos \theta\frac{\partial}{\partial r} -\frac{\sin \theta}{r}\frac{\partial}{\partial \theta} \right)- \right. \nonumber \\
		& &\left. - r\cos \theta \left( \sin \theta\ e^{\pm i \varphi}\frac{\partial}{\partial r}+\frac{\cos \theta}{r} e^{\pm i \varphi}\frac{\partial}{\partial \theta}\pm\frac{i e^{\pm i \varphi}}{r\sin \theta}\frac{\partial}{\partial \varphi} \right)\right]= \nonumber \\
		&=& \mp \hbar e^{\pm i \varphi}\left(-\frac{\partial}{\partial \theta}\mp i\ \cot \theta\frac{\partial}{\partial \varphi}\right) ,
	\end{eqnarray}
ossia, infine
	\begin{equation}
		\tcboxmath[enhanced, sharp corners=downhill, colback=yellow!50!white, colframe=red!75!black, borderline={2pt}{-3pt}{red!50!black}]{
			L_{\pm}=\hbar e^{\pm i \varphi}\left[\pm \frac{\partial}{\partial \theta}+ i\ \cot \theta\frac{\partial}{\partial \varphi}\right] .
			}
	\label{eq:cap17_5}
	\end{equation}
Da questo risultato è poi immediato ricavare le espressioni degli operatori $L_x$ ed $L_y$ nella rappresentazione delle coordinate. Ricordando le relazioni:
	\begin{equation}
		\tcboxmath[sharp corners=downhill, colback=white, colframe=red!75!black]{
			L_x = \frac{1}{2} (L_+ + L_-)
			}\qquad
		\tcboxmath[sharp corners=downhill, colback=white, colframe=red!75!black]{
			L_y = \frac{1}{2i} (L_+ + L_-),
			}
	\end{equation}
si ottiene
	\begin{align}
		& \tcboxmath[enhanced, sharp corners=downhill, colback=yellow!50!white, colframe=red!75!black, borderline={2pt}{-3pt}{red!50!black}]{L_x =i\hbar \left( \sin \varphi \frac{\partial}{\partial \theta}+\cot \theta \cos \varphi \frac{\partial}{\partial \varphi}\right),}\\[0.3cm]
		& \tcboxmath[enhanced, sharp corners=downhill, colback=yellow!50!white, colframe=red!75!black, borderline={2pt}{-3pt}{red!50!black}]{L_y =i\hbar \left(- \cos \varphi \frac{\partial}{\partial \theta}+\cot \theta \sin \varphi \frac{\partial}{\partial \varphi}\right) .}
	\end{align}\\
	
Risulta infine utile determinare l'\textbf{espressione, nella rappresentazione delle coordinate, del quadrato del momento angolare orbitale}. È conveniente partire dalla relazione (refeq:cap$16_7$) che si scrive, nel caso del momento angolare orbitale,
	\begin{equation}
		\tcboxmath[sharp corners=downhill, colback=white, colframe=red!75!black]{
			L_- L_+ = L^2 -L_z ^2 -\hbar L_z .
			}
	\end{equation}
Ricavando $L^2$ da questa relazione e sostituendovi le espressioni (\ref{eq:cap17_4}) e (\ref{eq:cap17_5}) per $L_z$ ed $L_{\pm}$, otteniamo:
\begin{eqnarray}
L^2 &=& L_-L_+ L_z^2 +\hbar L_z = \nonumber \\
&=& \hbar ^2 \left[e^{-i\varphi}\ \left(-\frac{\partial}{\partial \theta}+i \cot \theta\ \frac{\partial}{\partial \varphi}\right)e^{i\varphi}\ \left(\frac{\partial}{\partial \theta}+i \cot \theta\ \frac{\partial}{\partial \varphi}\right)-\right. \nonumber \\
& &\qquad \left. -\frac{\partial ^2}{\partial \varphi ^2}- i\frac{\partial}{\partial \varphi}\right] = \nonumber \\
&=&\hbar ^ 2 \left[ \left( -\frac{\partial}{\partial \theta}-\cot\theta +i \cot\theta\ \frac{\partial}{\partial \varphi}\right)\left(\frac{\partial}{\partial \theta}+i \cot \theta \ \frac{\partial}{\partial \varphi}\right)-\right. \nonumber \\
& &\qquad \left. -\frac{\partial ^2}{\partial \varphi ^2}- i\frac{\partial}{\partial \varphi}\right] = \nonumber \\
&=& \hbar ^2 \left[ -\frac{\partial ^2}{\partial \theta ^2}-i\left( \frac{\partial \cot \theta}{\partial \theta}\right) \frac{\partial }{\partial \varphi }-i\cot \theta \frac{\partial ^2}{\partial \theta \partial \varphi}-\cot \theta\ \frac{\partial }{\partial \theta}-\right. \nonumber \\
& & \left. -i \cot ^2 \theta \frac{\partial }{\partial \varphi}+i \cot\theta \frac{\partial ^2}{\partial \theta \partial \varphi}-\cot ^2 \theta \frac{\partial ^2}{\partial \varphi ^2}-\frac{\partial ^2}{\partial \varphi ^2}-i\frac{\partial }{\partial \varphi}\right] = \nonumber \\
&=& \hbar^2 \left[ -\frac{\partial ^2}{\partial \theta ^2}-\cot \theta \frac{\partial }{\partial \theta}-\left(1+ \cot ^2 \theta\right) \frac{\partial ^2}{\partial \varphi ^2}+\right. \nonumber \\
& &\qquad \left. -i\left( 1+\cot ^2 \theta +\frac{\partial \cot \theta}{\partial \theta}\right)\frac{\partial}{\partial \varphi}\right] .
\end{eqnarray}
Considerando le relazioni
	\begin{equation}
		1+ \cot ^2 \theta = 1+\frac{\cos ^2 \theta}{\sin ^2 \theta} = \frac{1}{\sin ^2 \theta},
	\end{equation}
	\begin{equation}
		\frac{\partial \cot \theta}{\partial \theta}= \frac{\partial}{\partial \theta}\left( \frac{\cos  \theta}{\sin \theta} \right) = \frac{-\sin ^ 2 \theta - \cos ^2 \theta}{\sin ^2 \theta}=-\frac{1}{\sin ^2 \theta},
	\end{equation}
possiamo scrivere
	\begin{equation}
		L^2 = \hbar ^2 \left[- \frac{\partial ^2}{\partial \theta ^2}-\cot \theta \frac{\partial}{\partial \theta}- \frac{1}{\sin ^2 \theta}\frac{\partial ^2}{\partial \varphi ^2}\right] ,
	\end{equation}
o, equivalentemente:
	\begin{equation}
		\tcboxmath[enhanced, sharp corners=downhill, colback=yellow!50!white, colframe=red!75!black, borderline={2pt}{-3pt}{red!50!black}]{
			L^2 = -\hbar ^2 \left[\frac{1}{\sin ^2 \theta}\frac{\partial ^2}{\partial \varphi ^2}+\frac{1}{\sin \theta}\frac{\partial}{\partial \theta} \left(\sin \theta \frac{\partial}{\partial \theta}\right)\right] .
			}
	\end{equation}
\newpage
\begin{figure}[!htbp]
\begin{center}
\includegraphics[width=\textwidth]{immagini/cap_17/fig_17_1.png}\\
\end{center}
\end{figure}
\begin{figure}[!htbp]
\begin{center}
\includegraphics[width=\textwidth]{immagini/cap_17/fig_17_2.png}\\
\end{center}
\end{figure}
 %DEFINITIVO
\chapter{Spin}
Nella meccanica quantistica si deve assegnare ad ogni particella un \textbf{momento angolare intrinseco} non legato con il suo moto nello spazio. Questa \textbf{proprietà} della materia è \textbf{squisitamente quantistica} (essa scompare nel passaggio al limite $\hbar\rightarrow0$) e, di conseguenza, non ammette un'interpretazione classica.\\ Il momento angolare intrinseco di una particella è chiamato \textbf{spin} della particella, a differenza del momento angolare legato al moto della particella nello spazio, che è detto momento angolare orbitale.\\ Lo spin di una particella, misurato in unità $\hbar$, sarà indicato con s. Gli operatori delle componenti del momento angolare di spin, trattandosi di operatori di momento angolare, soddisfano le regole di commutazione
\begin{equation}
[s_{i},s_{j}]=i\varepsilon_{ijk}\hbar s_{k} ,
\label{18.1}
\end{equation}
con tutte le conseguenze fisiche che ne derivano.\\
In particolare gli \textbf{autovalori del quadrato dello spin}, $s^{2}$. sono uguali a:
\begin{equation}
\hbar^{2}s(s+1) ,
\end{equation}
dove \textbf{s può essere o un numero intero(compreso lo zero) o semintero}.\\ \textbf{Per un dato s la componente $s_{z}$ dello spin può prendere i valori}
\begin{equation}
-s, -s+1,.....,s-1,s ,
\end{equation}
\textbf{in unità $\hbar$}, in totale 2s+1 valori.\\ \textbf{Poichè s è per ogni tipo di particella un numero dato, nel passaggio al limite classico ($\hbar\rightarrow0$) il momento angolare di spin $\hbar s$ si annulla}. Per il momento angolare orbitale questo ragionamento non ha senso, poichè \textit{l} può avere valori arbitrari. Il passaggio alla meccanica classica significa che, contemporaneamente $\hbar$ tende a zero ed \textit{l} all'infinito, cosicchè il prodotto $\hbar l$ resta finito.\\ \textbf{Per le particelle aventi uno spin, la descrizione dello stato mediante la funzione d'onda deve determinare non soltanto la probabilità delle sue diverse posizioni nello spazio, ma anche la probabilità delle diverse orientazioni possibili del suo spin}.
In altre parole, la funzione d'onda deve dipendere non soltanto da tre variabili continue, cioè dalle coordinate della particella,ma anche da una variabile di spin discreta che determina il valore della proiezione dello spin su una direzione scelta nello spazio(asse z) e suscettibile di un valore limitato di valori discreti(che indicheremo con la lettera $\sigma$). Tale funzione d'onda si scrive
\begin{equation}
\psi_{\sigma}(\vec{x'})=\langle\vec{x'}, \sigma|\alpha\rangle .
\end{equation} 
Essa rappresenta sostanzialmente un insieme di funzioni delle coordinate, che corrispondono a diversi valori di $\sigma$. Queste funzioni sono dette \textbf{componenti di spin della funzione d'onda}.\\ Frequentemente, le diverse componenti di spin della funzione d'onda sono arrangiate in un vettore colonna:
\[
\begin{pmatrix}
\psi_{\sigma_1}(\vec{x'})\\ \psi_{\sigma_2}(\vec{x'})\\ \vdots \\\psi_{2s+1}(\vec{x'})
\end{pmatrix}
\] 
Il modulo quadro 
\begin{equation}
\vert\psi_{\sigma}(\vec{x'})\vert^{2}  , 
\end{equation}
rappresenta la \textbf{densità di probabilità di trovare la particella nella posizione $\vec{x'}$ con un determinato valore di $\sigma$}. Pertanto la probabilità che la particella abbia un determinato valore di $\sigma$ è determinata dall'integrale
\begin{equation}
\int_{-\infty}^{+\infty} \vert\psi_{\sigma}(\vec{x'})\vert^{2}\,\vec{dx'} .
\end{equation}
Quanto alla probabilità della particella di trovarsi nell'elemento di volume $\vec{dx'}$, ma con un valore arbitrario di $\sigma$, essa è:
\begin{equation}
\sum_{\sigma} \vert\psi_{\sigma}(\vec{x'})\vert^{2}\vec{dx'}  .
\end{equation}
È evidente come, per una particella dotata di spin, la condizione di normalizzazione della funzione d'onda si scrive nella forma:
\begin{equation}
\sum_{\sigma} \int_{-\infty}^{+\infty}\vec{dx'}\vert\psi_{\sigma}(\vec{x'})\vert^{2}=\sum_{\sigma} \int_{-\infty}^{+\infty}\vec{dx'}\langle\alpha|\vec{x'}, \sigma\rangle\langle\vec{x'}, \sigma|\alpha\rangle=1 .
\end{equation}
Questo risultato segue anche dalla relazione di completezza
\begin{equation}
\sum_{\sigma} \int \vec{dx'}|\vec{x'}, \sigma\rangle\langle\vec{x'}, \sigma|=1 ,
\end{equation}
valida per i vettori di stato rappresentativi di particelle con spin.
\section{Operatori di spin e formalismo di Pauli per spin 1/2}
In questo capitolo non ci interesseremo, e non indicheremo pertanto esplicitamente, la dipendenza dalle coordinate spaziali dei vettori di stato.\\
Gli operatori di spin agiscono appunto sulla varietà di spin $\sigma$ e si possono rappresentare in forma di \textbf{matrici di dimensione (2s+1)}.\\
Consideriamo gli elementi di matrice degli operatori di spin nella base in cui il quadrato dello spin $s^{2}$ e la sua componente lungo l'asse z, $s_{z}$, sono diagonali. Questa base è definita dagli autovettori $|s,\sigma\rangle$ che soddisfano:
\begin{eqnarray}
& &s^{2} |s, \sigma\rangle =\hbar^{2}s(s+1)|s, \sigma\rangle ,\\
& &s_{z} |s, \sigma\rangle =\hbar \sigma |s, \sigma\rangle . 
\end{eqnarray}
Gli elementi di matrice degli operatori $s_{x}$ e $s_{y}$ in questa base sono espressi dalle relazioni derivate  nella discussione generale della teoria del momento angolare:
\begin{eqnarray}
& &\langle s, \sigma -1\vert s_{x}\vert s, \sigma\rangle = \langle s, \sigma\vert s_{x}\vert s, \sigma -1\rangle =\frac{\hbar}{2}\sqrt{s(s+1)-\sigma(\sigma-1)} , \nonumber \\
\\
& &\langle s, \sigma -1\vert s_{y}\vert s, \sigma\rangle =- \langle s, \sigma\vert s_{y}\vert s, \sigma -1\rangle =\frac{i\hbar}{2}\sqrt{s(s+1)-\sigma(\sigma-1)} . \nonumber \\
\end{eqnarray}
Nel caso più importante in cui lo \textbf{spin} è uguale a \textbf{1/2} le matrici rappresentative degli operatori di spin hanno dimensione 2 e sono della forma
\begin{equation}
\vec{s}\doteq\frac{\hbar}{2}\vec{\sigma} ,
\end{equation}
dove
\begin{equation}
\sigma_{x}=\begin{pmatrix}
0 & 1 \\
1 & 0
\end{pmatrix}, \qquad \sigma_{y}=\begin{pmatrix}
0 & -i \\
i & 0
\end{pmatrix},\qquad \sigma_{z}=\begin{pmatrix}
1 & 0 \\
0 & -1
\end{pmatrix},
\end{equation}
le matrici $\vec{\sigma}$ si chiamano \textbf{matrici di Pauli}.\\
In accordo con le regole (\ref{18.1}), le matrici di Pauli soddisfano le \textbf{relazioni di commutazione}
\begin{equation}
[\sigma_{i}, \sigma_{j}]=2i\varepsilon_{ijk}\sigma_{k} .
\end{equation}
Possiamo inoltre derivare con un calcolo diretto, le seguenti proprietà specifiche delle matrici di Pauli
\begin{equation}
\sigma_{i}^{2}=1, \quad \sigma_{i}\sigma_{j}+\sigma_{j}\sigma_{i}=0 \quad per \; i \neq j .
\end{equation}
Queste relazioni sono ovviamente equivalenti alle \textbf{relazioni di anticommutazione}\footnote{Tutte queste proprietà sono riassunte dall'unica relazione $\sigma_{i}\sigma_{j}=\delta_{ij}+i\varepsilon_{ijk}$}:
\begin{equation}
\lbrace\sigma_{i}, \sigma_{j}\rbrace=2\delta_{ij}
\end{equation}
Quanto all'operatore $s^{2}$, la sua rappresentazione in questa base risulta:
\begin{equation}
s^{2}=s_{x}^{2}+s_{y}^{2}+s_{z}^{2}=\frac{\hbar^{2}}{4}(\sigma_{x}^{2}+\sigma_{y}^{2}+\sigma_{z}^{2})=
\frac{3}{4}\hbar^{2}\doteq\frac{3}{4}\begin{pmatrix}
1 & 0 \\
0 & 1
\end{pmatrix} ,
\end{equation} 
in accordo con il fatto che per una particella di spin 1/2 gli autostati $\vert s, \sigma\rangle$ sono autostati  dell'operatore $s^{2}$ con autovalore $\hbar^{2}s(s+1)=\frac{3}{4}\hbar^{2}$.\\
Gli autostati $\vert s, \sigma\rangle$ per una particella di spin 1/2 corrispondenti agli autovalori $\sigma=\pm\hbar/2$ vengono frequentemente indicati con i simboli $\vert +\rangle$ e $\vert -\rangle$ (o anche $\vert \uparrow \rangle$ e $\vert \downarrow \rangle$). Quanto alla loro rappresentazione matriciale nella base da essi stessi costituita questa è ovviamente
\begin{equation}
\vert +\rangle\doteq\begin{pmatrix}
1 \\
0
\end{pmatrix}\equiv \chi_{+}\;, \qquad \vert - \rangle\doteq\begin{pmatrix}
0 \\
1
\end{pmatrix}\equiv \chi_{-} ,
\end{equation}
e per i corrispondenti vettori bra:
\begin{equation}
\langle + \vert \doteq \begin{pmatrix}
1 & 0
\end{pmatrix} \equiv \chi_{+}^{+} \; , \qquad \langle - \vert \doteq \begin{pmatrix}
0 & 1
\end{pmatrix} \equiv \chi_{-}^{+} .
\end{equation}
Un generico vettore di stato $\vert \alpha \rangle$ è esprimibile come combinazione lineare dei due stati di base $\vert + \rangle$ e $\vert - \rangle$ nella forma:
\begin{equation}
\alpha = c_{+}\vert + \rangle + c_{-}\vert - \rangle\doteq \begin{pmatrix}
c_{+}\\
c_{-}
\end{pmatrix}= c_{+}\chi_{+}+c_{-}\chi_{-} ,
\end{equation}
dove i coefficienti complessi corrispondono alle ampiezze di probabilità
\[
c_{+}=\langle + \vert \alpha \rangle \; , \qquad c_{-}=\langle - \vert \alpha \rangle .
\]
Il vettore colonna $\begin{pmatrix}
c_{+}\\
c_{-}
\end{pmatrix}$ è chiamato \textbf{spinore} a due componenti.%DEFINITIVO
\chapter[Composizione di momenti angolari]{Composizione di momenti angolari\footnote{S3.7, LL31}}
Consideriamo un sistema composto da due parti. Il momento angolare totale di questo sistema, $\vec{J}$, può essere scritto come la somma dei momenti angolari $\vec{J_1}$ e $\vec{J_2}$ delle sue parti:
\begin{align} \label{eq:cap19_01}
\vec{J} = \vec{J_1} + \vec{J_2}.
\end{align}
Se le due parti che costituiscono il sistema interagiscono tra loro, la legge di conservazione del momento angolare si applica soltanto al momento angolare totale $\vec{J}$, e non ai momenti angolari $\vec{J_1}$ e $\vec{J_2}$ presi separatamente.\\
Considerazioni analoghe possono essere effettuate per un sistema con momento angolare di spin $\vec{S}$ diverso da zero. Per un tale sistema la legge di conservazione del momento angolare si applica (in generale) soltanto al momento angolare totale del sistema, composto dalla componente orbitale e dalla componente di spin:
\begin{equation}
\vec{J} = \vec{L} + \vec{S}.
\end{equation}
Nello studio di tali sistemi si pone il problema della legge di composizione dei momenti angolari. Quali sono i valori possibili di $j$ dati i valori di $j_1$ e $j_2$? \\
Quanto alla legge di composizione delle proiezioni del momento angolare, essa è evidente: dal fatto che $J_z = J_{1z} + J_{2z}$ segue che 
\begin{equation} \label{eq:cap19_02}
m= m_1 + m_2.
\end{equation}
Per dedurre la legge di composizione dei quadrati dei momenti angolari ragioniamo nel modo seguente: prendiamo come sistema completo di grandezze fisiche le grandezze
\begin{equation}
J_1^2 ,~ J_2^2 ,~ J_{1z} ,~ J_{2z} ,
\end{equation}
e una serie di altre grandezze che, con le quattro indicate, costituiscono un sistema completo. Poichè queste altre grandezze non intervengono nei ragionamenti successivi, per abbreviare le espressioni non le considereremo affatto.\\
Ogni stato sarà determinato allora dai numeri $j_1 , ~ j_2 ,~ m_1 , ~ m_2$ e lo indicheremo pertanto con
\begin{equation}
| j_1 , ~ j_2 ,~ m_1 ,~ m_2 \rangle .
\end{equation}
Per $j_1$ e $j_2$, i numeri $m_1$ e $m_2$ assumono rispettivamente $(2 j_1 + 1)$ e $(2 j_2 + 1)$ 
valori, cosicché si hanno in tutto
\begin{equation} \label{eq:cap19_03}
N= (2j_1 + 1) (2j_2 + 1) 
\end{equation}
stati diversi con gli stessi $j_1$ e $j_2$. \\
In luogo delle quattro grandezze indicate si possono prendere anche. come sistema completo, le quattro grandezze:
\begin{equation}
J^2 ,~ J_z ,~ J_1^2 ,~ J_2^2 . 
\end{equation}
Si trova infatti che, ad esempio, l'operatore $J_1^2$  commuta con $J_2^2$, come risulta evidente dall'espressione:
\begin{align}
J^2 = (\vec{J_1} + \vec{J_2})^2 = J_1^2 + J_2^2 + 2 (J_{1x} J_{2x} + J_{1y} J_{2y} + J_{1z} J_{2z}),
\end{align}
(sottolineiamo che gli operatori $\vec{J_1}$ e $\vec{J_2}$, agendo sui diversi sottospazi, commutano tra loro).\\
 In questo caso ogni stato sarà caratterizzato da valori dei numeri $j, ~  m, ~ j_1,~  j_2$ e lo indicheremo con
\begin{center}
$| j ,~ m,~j_1, ~ j_2 \rangle $ .
\end{center}
Dati $j_1$ e $j_2$ si debbono avere, ovviamente, come prima $N = (2j_1+1)(2j_2+1)$ stati diversi, cioè dati $j_1$ e $j_2$ la coppia di numeri $j$ ed $m$ può prendere $(2j_1+1)(2j_2+1)$ coppie di valori. Il ragionamento seguente permette di determinare questi valori.\\
Il valore massimo possibile di $m$, in accordo con l'eq. \eqref{eq:cap19_02}, è $m = j_1 + j_2$ e a questo corrisponde un solo stato $| j_1 ,~ j_2,~m_1, ~ m_2 \rangle $ (coppia di valori $m_1$, $m_2$). Pertanto, il valore massimo possibile di m negli stati $\mid j ,~ m,~j_1, ~ j_2 > $  e, di conseguenza, il valore massimo di $j$ è $j_1 + j_2$. 

\begin{center}
\begin{tabular}{c||c|c||c|c||c}

	& 	$m_1$&	$m_2$&	$m$&	$j$\\
\cline{2-5}
\cline{2-5}
	1)&	$j_1$&	$j_2$&	$j_1+j_2$&	$j_1+j_2$&\\
	
\cdashline{2-5}
\end{tabular}
\end{center}
Esistono inoltre due stati i $| j_1 ,~ j_2,~m_1, ~ m_2 \rangle $ con $m= j_1 + j_2 -1$. Di conseguenza, ci debbomo essere ugualmente due stati $| j ,~ m,~j_1, ~ j_2 \rangle $ con questo valore di m. Uno di essi è lo stato con $j= j_1 + j_2$ (con $m=j-1$) e l'altro con $j= j_1+ j_2 -1$ (per $m=j$)
\begin{center}
\begin{tabular}{c||c|c||c|c||c}

	& 	$m_1$&	$m_2$&	$m$&	$j$\\
\cline{2-5}
\cline{2-5}
	1)&	$j_1$&	$j_2-1$&	\multirow{2}{*}{$j_1+j_2-1$}&	$j_1+j_2$	&1)\\ \hhline{~--~-~}
	2)&	$j_1-1$&	$j_2$&	&$j_1+j_2-1$	&2)\\	
\cdashline{2-5}
\end{tabular}
\captionof*{table}{\textit {~~Stati~$| j_1 ,~ j_2,~m_1, ~ m_2 \rangle $~~~~~~~~Stati~$| j ,~ m,~j_1, ~ j_2 \rangle $}}
\end{center}
Per il valore $m= j_1 + j_2 -2$ esistono tre diversi stati $| j_1 ,~ j_2,~m_1, ~ m_2 \rangle $. Ciò significa che oltre ai valori $j= j_1 + j_2, j= j_1 + j_2 -1$ è possibile anche il valore $j= j_1 +j_2 -2$. \\

\begin{center}
\begin{tabular}{c||c|c||c|c||c}

	& 	$m_1$&	$m_2$&	$m$&	$j$\\
\cline{2-5}
\cline{2-5}
	1)&	$j_1$&	$j_2-2$&	\multirow{3}{*}{$j_1+j_2-2$}&	$j_1+j_2$	&1)\\ \hhline{~--~-~}
	2)&	$j_1-1$&	$j_2-1$&	&$j_1+j_2-1$	&2)\\	\hhline{~--~-~}
	3)&	$j_1-2$&	$j_2$&	&$j_1+j_2-2$	&3)\\
\cdashline{2-5}
\end{tabular}
\captionof*{table}{\textit {~~Stati~$| j_1 ,~ j_2,~m_1, ~ m_2 \rangle $~~~~~~~~Stati~$| j ,~ m,~j_1, ~ j_2 \rangle $~}}
\end{center}
Questo ragionamento si può continuare nello stesso modo finché diminuendo m di 1 aumenta di 1 il numero di stati con il dato valore di m. E' facile capire che ciò si verifica finché, assumendo ad esempio $j_1 \geq j_2$,m non raggiunge il valore $j_1-j_2$. Diminuendo ulteriormente m, il numero di stati cessa di crescere, restando uguale a $2j_2+1$; il valore di $m_2$ infatti, non può essere minore di $-j_2$. Ciò significa che $j_1-j_2$, o in generale $|j_1-j_2|$, è il valore minimo possibile di $j$. \\
Così siamo giunti al risultato che, dati $j_1$ e $j_2$, il numero $j$ può prendere i valori:  
\begin{equation} \label{eq:cap19_04}
j= |j_1-j_2|, |j_1-j_2|+1, ......, j_1+j_2-1, j_1+j_2,
\end{equation}
 cioè in totale $2j_2 + 1$ valori diversi (supponendo che $j_2 \leq j_1$).  \\
Tenendo conto per ogni valore di $j$ esistono $(2j+1)$ stati $| j ,~ m,~j_1, ~ j_2 \rangle $  corrispondenti ai diversi possibili valori di $m$ possiamo verificare che il numero totale di stati è (per $j_2 \leq j_1$):
\begin{align}
N~ &= \sum_{j=j_1-j_2}^{j_1+j_2}{(2_j + 1)} = \nonumber \\
&= \sum_{k=0}^{2j_2}{[2(k+j_1-j_2)+1]}= \nonumber \\
&= 2 \sum_{k=0}^{2j_2}{k} + [2(j_1-j_2)+1](2j_2+1) =\nonumber \\
&= 2j_2(2j_2+1)+[2(j_1-j_2)+1](2j_2+1) = \nonumber \\ \nonumber \\
&= (2j_1+1)(2j_2+1)
\end{align}
in accordo con il risultato (\ref{eq:cap19_03}).\\
Il risultato ottenuto per i possibili valori di $j$ (eq.(\ref{eq:cap19_04})) può essere illustrato mediante il cosiddetto \emph{modello vettoriale}. Se introduciamo due vettori $\vec{J_1}$ e $\vec{J_2}$ con moduli $j_1$ e $j_2$, i valori di $j$ saranno rappresentati come moduli interi dei vettori $\vec{J}$, ottenuti componendo vettorialmente i $\vec{J_1}$ e $\vec{J_2}$. Il valore massimo $(j_1+j_2)$ di $j$ si ottiene allorché $\vec{J_1}$ e  $\vec{J_2}$ sono paralleli, e il valore minimo $(|j_1-j_2|)$ allorché essi sono antiparalleli.
%%%%%%%%%%%%%%%%%%%%%%%%%%%%%%%%%%%%%%%%%%%%%%%%%%%%%%5
\section[Coefficienti di Clebsh-Gordan]{Coefficienti di Clebsh-Gordan\footnote{S3.7}}
Consideriamo la trasformazione unitaria che connette la base degli autostati di $J_1^2 ,~ J_2^2 ,~ J_{1z} ,~ J_{2z}$ alla base degli autostati di  $J^2 ,~ J_z ,~ J_1^2 ,~ J_2^2$.
\begin{align} \label{eq:cap19_05}
| j ,~ m,~j_1, ~ j_2 \rangle  = \sum_{m_1, m_2} {| j_1 ,~ j_2,~m_1, ~ m_2 \rangle \langle j_1 ,~ j_2,~m_1, ~ m_2 |  j ,~ m,~j_1, ~ j_2 \rangle }.
\end{align}
Per scrivere questa relazione abbiamo utilizzato la relazione di completezza
\begin{align}
\sum_{m_1, m_2} {| j_1 ,~ j_2,~m_1, ~ m_2 \rangle \langle j_1 ,~ j_2,~m_1, ~ m_2 | = 1 },
\end{align}
dove il secondo membro è l'operatore di identità nello spazio dei vettori di stato con $j_1$ e $j_2$ assegnati. \\
Gli elementi di matrice unitaria che effettua il cambiamento di base, ossia la grandezza
\begin{equation}
\langle j_1 ,~ j_2,~m_1, ~ m_2 ~|~ j ,~ m,~j_1, ~ j_2 \rangle ,
\end{equation}
sono detti \textbf{coefficienti di Clebsh-Gordan}.\\
Dalle leggi di composizione del momento angolare totale e della sua componente $z$ segue ovviamente che i coefficienti di C.G. sono zero a meno che
\begin{equation}
m=m_1+m_2\quad \textrm{e}\quad |j_1-j_2| \leq j \leq j_1+j_2.
\end{equation}
Per convenzione, i coefficienti di C.G. sono definiti reali:
\begin{align}
\langle j_1 ,~ j_2,~m_1, ~ m_2~ |~ j ,~ m,~j_1, ~ j_2 \rangle  = \langle j ,~ m,~j_1, ~ j_2~ |~ j_1 ,~ j_2,~m_1, ~ m_2 \rangle .
\end{align}
La condizione di ortogonalità degli autostati $\mid j ,~ m,~j_1, ~ j_2 >$ applicata all'eq. (\ref{eq:cap19_05}), unitamente alla condizione di realtà dei coefficienti di C.G., fornisce:
\begin{align}
 \sum_{m_1, m_2} {\langle j_1 , j_2, m_1, m_2 | j' , m', j_1, j_2 \rangle  \langle j_1 , j_2, m_1, m_2  | j , m, j_1, j_2 \rangle } = \delta_{j j'} \delta{m m'},
\end{align}
che coincide con la condizione di unitarietà della matrice dei coefficienti di C.G.. In particolare, per $j=j'$ ed $m=m'$ si ottiene:
\begin{align}
 \sum_{m_1, m_2} {|\langle j_1 ,~ j_2,~m_1, ~ m_2 ~|~ j ,~ m,~j_1, ~ j_2 \rangle|^2} =1 , 
\end{align}
che non è altro che la condizione di normalizzazione degli stati $| j ,~ m,~j_1, ~ j_2 \rangle $. \\
Un metodo conveniente per determinare i coefficienti di C.G. consiste nell'utilizzare gli operatori a scala secondo la seguente procedura. \\
Nella composizione di due momenti angolari $\vec{J_1}$ e  $\vec{J_2}$ lo stato corrispondente al valore massimo della componente $z$ del momento angolare, ossia $m= j_1+j_2$ e dunque $j=j_1+j_2$ coincide necessariamente, a mo' di un fattore di fase, con lo stato avente $m_1=j_1$ ed $m_2=j_2$. Il fattore di fase è posto per convenzione uguale ad 1. Allora
\begin{align}
| j= j_1 + j_2 ,~m= m_1 +m_2  \rangle ~= ~| j_1 ,~j_2,  m_1=j_1, ~m_2= j_2 \rangle .
\end{align}
Applicando ad entrambi i membri di questa equazione l'operatore $J=J_1+J_2$ si può determinare lo stato con $j=j_1+j_2$ ed $m=j_1+j_2-1$ in termini degli stati con $(m_1=j_1-1 , m_2=j_2)$ e $(m_1=j_1, m_2=j_2-1)$. Lo stato con lo stesso valore di $m$ ma $j=j_1+j_2-1$ è poi costruito imponendo l'ortogonalità con il precedente. In questo modo si possono determinare tutti i coefficienti di C.G. del sistema considerato. \\
Consideriamo questa procedura discutendo un esempio specifico importante.
%%%%%%%%%%%%%%%%%%%%%%%%%%%%%%%%%%%%%%%%%%%%%%%%
\section[Composizione di due momenti angolari di spin 1/2. Stati di tripletto e di singoletto]{Composizione di due momenti angolari di spin 1/2. Stati di tripletto e di singoletto\footnote{S3.7}}
Consideriamo la composizione angolare di sin per due particelle di spin 1/2. Il momento angolare di spin totale delle due particelle,
\begin{equation}
\vec{S} = \vec{S_1} + \vec{S_2} ,
\end{equation}
può assumere in questo caso solo due valori, corrispondenti ad:
\begin{equation}
S~= ~0, ~1 ,
\end{equation}
in accordo con la legge di composizione (\ref{eq:cap19_04}).\\
Nella base degli autostati comuni degli operatori $S_1^2 ,~ S_2^2, ~ S_{1z}, ~ S_{2z}$ esistono quattro vettori di stato che possiamo indicare con \\ 
\begin{equation}
|++\rangle ,~ |+-\rangle,~ |-+\rangle , ~|--\rangle  ,
\end{equation}
dove, ad esempio, \\
\begin{equation}
|++\rangle~ \equiv ~|s_1=1/2, ~s_2=1/2,~\sigma_1 = 1/2, ~\sigma_2 = 1/2\rangle ,
\end{equation}
e analoghe. \\
Corrispondentemente esistono quattro stati di base corrispondenti agli operatori $S^2 ,~ S_z, ~ S_1^2, ~ S_2^2$. Possiamo indicare questi stati con \\
\begin{equation}
|1,1\rangle, ~|1,0\rangle, ~|1,-1\rangle, ~|0,0\rangle ,
\end{equation}dove \\
\begin{equation}
|1,1\rangle ~\equiv~ |s=1, ~ \sigma=1, ~s_1=1/2, ~s_2=1/2\rangle ,
\end{equation}
ecc... \\
Gli stati corrispondenti a momento angolare di spin $S=1$ sono detti \emph{stati di tripletto}. Lo stato corrispondente a momento angolare di spin $S=0$ \emph{stato di singoletto}. \\
Calcoliamo i coefficienti di C.G. che consentono di esprimere gli autostati di $S^2$ ed $S_z$ in termini degli autostati di $S_1^2 ,~ S_{1z}, ~ S_2^2, ~ S_{2z}$. \\
Lo stato $\sigma=1$ si può ottenere solo per $\sigma_1=\sigma_2=1/2$. \\
Pertanto:\\
\begin{equation}
|1,1\rangle = |++\rangle  .\\
\end{equation}
Applichiamo ora ad entrambi i membri di questa equazione l'operatore a scala $S_-$:
\begin{align}
 S_1 |1,1\rangle & = \sqrt{s(s+1) - \sigma(\sigma -1)}  |1,0\rangle =\sqrt{2}  |1,0\rangle= \nonumber \\
&= (S_1+S_2)  |+,+\rangle = \nonumber \\
&= \sqrt{1/2 (1/2+1) -1/2 (1/2-1)} ( |+,-\rangle + |-,+\rangle)= \nonumber \\
&= |+,-\rangle   +   |-,+\rangle ,
\end{align}
ossia
\begin{equation}
|1,0\rangle = \frac{1}{\sqrt{2}} (|+-\rangle + |-+\rangle) .
\end{equation}
Lo stato $|1,-1\rangle$ si può ottenere mediamente una seconda applicazione dell'operatore a scala. Ma in questo caso si ottiene evidentemente:
\begin{equation}
|1,-1\rangle = |--\rangle .
\end{equation}
Infine, lo stato $|0,0\rangle$ si può ottenere imponendo l'ortogonalità con lo stato $|1,0\rangle$. S trova allora:
\begin{equation}
|0,0\rangle = \frac{1}{\sqrt{2}} (|+-\rangle - |-+\rangle). \\
\end{equation}
In definitiva, abbiamo trovato:

\begin{align}
\begin{cases} 
|1,1\rangle = |++\rangle \nonumber \\
|1,0\rangle = \frac{1}{\sqrt{2}} (~|+-\rangle + |-+\rangle ~)  \nonumber  & \mbox{\textrm{\textbf{Tripletto}}} \\
|1,-1\rangle = |--\rangle  \nonumber \\
\end{cases}
\end{align}

\begin{align}
\begin{cases} 
|0,0\rangle = \frac{1}{\sqrt{2}} (~|+-\rangle - |-+\rangle~) \nonumber  & \mbox{\textrm{\textbf{Singoletto}}}\\
\end{cases}
\end{align}

%%%%%%%%%%%%%%%%%%%%%%%%%%%%%%%%%%%%%%%%%%%%%%%%%%%%%

\section[Composizione dei momenti angolari orbitale e di spin per una particella di spin 1/2]{Composizione dei momenti angolari orbitale e di spin per una particella di spin 1/2\footnote{S3.7}}

Consideriamo la composizione del momento angolare orbitale e del momento angolare di spin per una particella di spin 1/2.\\
Se  la particella si trova in un autostato del quadrato del momento angolare orbitale $L$ corrispondente all'autovalore $\hbar^2l(l+1)$, allora i valori possibili per il numero quantico $j$, corrispondente al momento angolare totale $\vec{J}=\vec{L}+\vec{S}$ sono dati da:
\begin{equation}
j =  l \pm 1/2 , ~~~~~~~~~(l>0)
\end{equation}
(il caso $l=0$, ossia assenza di momento angolare orbitale, corrisponde a $\vec{J}=\vec{S}$ e non verrà qui considerato).\\
\\
Determiniamo i coefficienti di Clebsch-Gordan che definiscono lo sviluppo delle autofunzioni $\mathcal{Y}_{j,j_z}$ di $J^2$ e $J_z$ in termini delle autofunzioni $Y_{l,m}$ di $L^2$, $L_z$ e $\chi_\pm$ di $S^2$ ed $S_z$.\\
La proiezione:
\begin{gather}
j_z = m + m_s ,\\
(m_s=\pm1/2)~~,~~(m=-l,-l+1,...,l),
\end{gather}
del momento angolare totale può assumere solo valori seminteri. In particolare, un determinato valore $j_z=m+1/2$ può essere ottenuto solo dalle combinazioni $l_z=m$, $s_z=1/2$ o $l_z=m+1$, $s_z=-1/2$. Devono allora valere gli sviluppi ortogonali:
\begin{align} \label{eq:cap19_06}
\begin{cases} 
\mathcal{Y}_{j=l+1/2,~ j_z=m+1/2} = c_+Y_{l,m}\chi_+ + c_-Y_{l, m+1}\chi_- \\
\mathcal{Y}_{j=l-1/2,~ j_z=m+1/2} = -c_-Y_{l,m}\chi_+ + c_+Y_{l, m+1}\chi_- 
\end{cases}
\end{align}
$c_\pm$ rappresentano i coefficienti di Clebsch-Gordan cercati.\\
Per determinare questi coefficienti, consideriamo in partenza lo stato in cui la proiezione $J_z$ del momento angolare totale assume il suo valore massimo: $j_z=l+1/2$. Questo stato corrisponde evidentemente alla combinazione:
\begin{equation} \label{eq:cap19_07}
 \mathcal{Y}_{j=l+1/2,~j_z=l+1/2} = Y_{l,l} \chi_+ .
\end{equation}
Applichiamo ad entrambi i membri di questa equazione l'operatore a scala $J_- = L_-+S_-$.\\
Ricordando il risultato:
\begin{align} \label{eq:cap19_08}
J_-~ \mathcal{Y}_{j,j_z} &= \sqrt{j(j+1)-j_z(j_z-1)}~\mathcal{Y}_{j,j_z-1} = \\ \nonumber
&= \sqrt{(j+j_z)(j-j_z+1)}~\mathcal{Y}_{j,j_z-1} ,
\end{align}
(e le espressioni analoghe per $L_-$ ed $S_-$), otteniamo dal primo membro dell'eq. (\ref{eq:cap19_07}) :
\begin{equation} \label{eq:cap19_09}
J_-~ \mathcal{Y}_{j=l+1/2,~j_z=l+1/2} = \sqrt{2l+1}~\mathcal{Y}_{j=l+1/2,~j_z=l-1/2} .
\end{equation}
Per ottenere l'autofunzione corrispondente a $j_z=m+1/2$ dobbiamo applicare l'operatore $J_-$ $(l-m)$ volte. Una seconda applicazione di $J_-$ all'equazione (\ref{eq:cap19_09}) fornisce:
\begin{align}
(J_-)^2~&\mathcal{Y}_{j=l+1/2,~j_z=l+1/2} = \sqrt{(2l+1)}~J_-~\mathcal{Y}_{j=l+1/2,~j_z=l-1/2} = \nonumber \\
=~ &\sqrt{(2l+1)\cdot 2 \cdot 2l}~\mathcal{Y}_{j=l+1/2,~j_z=l-3/2},
\end{align}
ed una terza applicazione di $J_-$ conduce a:
\begin{align}
(J_-)^3&~\mathcal{Y}_{j=l+1/2,~j_z=l+1/2} = \sqrt{2\cdot(2l+1)\cdot 2l}~J_-~\mathcal{Y}_{j=l+1/2,~j_z=l-3/2} = \nonumber\\
=~&\sqrt{2\cdot 3\cdot(2l+1) \cdot (2l) \cdot (2l-1)}~\mathcal{Y}_{j=l+1/2,~j_z=l-5/2}.
\end{align}
Risulta allora chiaro come il coefficiente che si ottiene applicando $J_-$ un numero $k$ di volte sulla autofunzione iniziale $\mathcal{Y}_{j=l+1/2,~j_z=l+1/2}$ sia:
\begin{equation}
\sqrt{2\cdot 3\cdot~...\cdot k~(2l+1) \cdot (2l) \cdot ~...\cdot(2l+2-k)} = \sqrt{\frac{k! (2l+1)!}{(2l+1-k)!}},
\end{equation}
ossia
\begin{equation} \label{eq:cap19_10}
(J_-)^k~\mathcal{Y}_{j=l+1/2,~j_z=l+1/2} = \sqrt{\frac{k! (2l+1)!}{(2l+1-k)!}}~\mathcal{Y}_{j=l+1/2,~j_z=l+1/2-k}.
\end{equation}
In particolare, scegliendo $k=l-m$, si trova:
\begin{equation} \label{eq:cap19_11}
(J_-)^{l-m}~\mathcal{Y}_{j=l+1/2,~j_z=l+1/2} = \sqrt{\frac{(l-m)! (2l+1)!}{(l+m+1)!}}~\mathcal{Y}_{j=l+1/2,~j_z=m+1/2}.
\end{equation}
Dobbiamo ora considerare l'applicazione dell'operatore $(J_-)^{l-m} = (L_-+S_-)^{l-m}$ sul secondo membro dell'eq. (\ref{eq:cap19_07}).\\
A tale scopo osserviamo che l'applicazione successiva dell'operatore $(S_-)$ un numero $\ge 2$ di volte sullo stato $\chi_+$ produce un risultato nullo:
\begin{equation}
(S_-)^2~\chi_+ = 0
\end{equation}
Si ha pertanto:
\begin{align} \label{eq:cap19_12}
&(L_-+S_-)^{l-m}~Y_{l,l} \chi_+ = [(L_-)^{l-m} + (l-m)(L_-)^{l-m-1}(S_-)]~ Y_{l,l} \chi_+ = \nonumber \\
=~& (L_-)^{l-m}~Y_{l,l} \chi_+ + (l-m)(L_-)^{l-m-1}~Y_{l,l} \chi_- .
\end{align}
L'eq. (\ref{eq:cap19_10}), con la sostituzione $l\rightarrow l-1/2$, ci fornisce direttamente il risultato dell'applicazione dell'operatore $(L_-)^k$ sull'autofunzione $Y_{l,l}$. Si ha:
\begin{equation} \label{eq:cap19_13}
(L_-)^k ~Y_{l,l} = \sqrt{\frac{k!~(2l)!}{(2l-k)!}}~Y_{l,l-k}.
\end{equation}
Utilizzando questo risultato, possiamo allora riscrivere l'eq. (\ref{eq:cap19_12}) nella forma:
\begin{align} \label{eq:cap19_14}
&(L_-+S_-)^{l-m}~Y_{l,l} \chi_+ = \\ \nonumber
&= \sqrt{\frac{(l-m)!~(2l)!}{(l+m)!}}~Y_{l,m} \chi_+ + (l-m)~\sqrt{\frac{(l-m-1)!~(2l)!}{(l+m+1)!}}~Y_{l,m+1} \chi_- .
\end{align}
I coefficienti di Clebsch-Gordan definiti dall'eq. (\ref{eq:cap19_06}) si ottengono in definitiva applicando ad entrambi i membri dell'eq. (\ref{eq:cap19_07}) l'operatore $(J_-)^{l-m}=(L_-+S_-)^{l-m}$ ed utilizzando i risultati (\ref{eq:cap19_11}) e (\ref{eq:cap19_14}). In tal modo si trova:
\begin{align}
&c_+ = \sqrt{\frac{(l-m)!~(2l)!}{(l+m)!}} \cdot \sqrt{\frac{(l+m+1)!}{(l-m)!~(2l+1)!}} = \sqrt{\frac{l+m+1}{2l+1}}, \\
&c_- = \sqrt{\frac{(l-m-1)!~(2l)!}{(l+m+1)!}} \cdot \sqrt{\frac{(l+m+1)!}{(l-m)!~(2l+1)!}} = \sqrt{\frac{l-m}{2l+1}}.
\end{align}
da cui risulta:
\begin{align}
\mathcal{Y}_{j=l+1/2,~ j_z=m+1/2} = \sqrt{\frac{l+m+1}{2l+1}}~Y_{l,m}~\chi_+ + \sqrt{\frac{l-m}{2l+1}}~Y_{l, m+1}~\chi_-, \\
\mathcal{Y}_{j=l-1/2,~ j_z=m+1/2} = -\sqrt{\frac{l-m}{2l+1}}~Y_{l,m}~\chi_+ + \sqrt{\frac{l+m+1}{2l+1}}~Y_{l, m+1}~\chi_-. 
\end{align}
\`E immediato verificare come queste autofunzioni siano ortogonali tra loro e correttamente normalizzate.%DEFINITIVO
\chapter[Particelle identiche]{Particelle identiche\footnote{S6.1,6.2,6.3; LL61,62, G8}}
Nella meccanica classica le particelle identiche (per esempio, elettroni), malgrado l'identità delle loro proprietà fisiche, non perdono però una loro "individualità": si può immaginare di numerare in un certo istante le particelle di un sistema fisico dato e seguire poi il moto di ciascuna di esse lungo la sua traiettoria; sarà allora possibile identificare la particella in qualsiasi istante.\\ 
Nella meccanica quantistica, invece, la situazione \`e completamente diversa. In virtù del principio di indeterminazione, il concetto di traiettoria della particella perde completamente significato. Di conseguenza, localizzate e numerate le particelle ad un certo istante, questo non ci dà la possibilità di identificarle negli istanti successivi.\\
Cos\`i nella meccanica quantistica non esiste, in linea di principio, alcuna possibilità di seguire separatamente ciascuna delle particelle identiche, e quindi di distinguerle. L'identità delle particelle relativa alle loro proprietà fisiche ha quindi un significato molto profondo: essa porta all'indistinguibilità totale delle particelle.\\
Questo principio di indistinguibilità delle particelle identiche ha un ruolo fondamentale nella teoria quantistica dei sistemi formati da particelle identiche.\\
Consideriamo, per iniziare, un sistema formato da due sole particelle identiche. Siano $|a\rangle$, $|b\rangle$,... i vettori di stato di ciascuna particella considerato solo come un sistema dinamico. Possiamo ottenere un vettore di stato per il sistema costituito dalle due particelle prendendo il prodotto di ket per ciascuna particella considerata da sola. Per esempio:
\begin{equation}
|a\rangle|b\rangle,
\label{eq:cap20_1}
\end{equation}
rappresenta lo stato in cui la prima particella si trova nello stato a e la seconda particella nello stato b.\\
Nella \ref{eq:cap20_1} possiamo scambiare il ruolo delle due particelle e ottenere un altro vettore di stato per il sistema costituito dalle due particelle, ossia il vettore di stato:\\
\begin{equation}
|b\rangle|a\rangle.
\label{eq:cap20_2}
\end{equation}
Questo rappresenta lo stato in cui la prima particella si trova nello stato $|b\rangle$,e la seconda nello stato $|a\rangle$.\\
Il processo di scambiare tra loro le due particelle è un operatore lineare che può essere applicato ai vettori di stato del sistema costituito dalle due particelle. Indicando con $P_{12}$ questo operatore si ha ad esempio:
\begin{equation}
P_{12}|a\rangle|b\rangle = |b\rangle|a\rangle.
\end{equation}
Supponiamo di effettuare una misura sul sistema costituito dalle due particelle, e di trovare che una particella si trova nello stato a e l'altra nello stato b, Tuttavia non sappiamo a priori se lo stato sia $|a\rangle|b\rangle$ o $|b\rangle|a\rangle$ oppure una qualsiasi combinazione lineare dei due, della forma:
\begin{equation}
|\psi\rangle = C_1|a\rangle|b\rangle + C_2 |b\rangle|a\rangle.
\label{eq:cap20_3}
\end{equation}
In altri termini, tutti i vettori di stato della forma \ref{eq:cap20_3} portano allo stesso insieme di autovalori quando si esegue la misura. Ciò è noto come \textbf{degenerazione di scambio}.\\
La degenerazione di scambio sembra rappresentare una difficoltà, poiché, contrariamente al caso di una particella singola, l'assegnazione degli autovalori di un insieme completo di osservabili non determina completamente il vettore di stato.\\
Tuttavia, \textbf{il principio di indistinguibilità delle particelle identiche implica che gli stati del sistema che si ottengono l'uno dall'altro semplicemente scambiando le due particelle, devono essere fisicamente del tutto equivalenti. Questo significa che, come risultato dello scambio, il vettore di stato del sistema può variare soltanto di un fattore di fase inessenziale.} Ossia:
\begin{equation}
P_{12}|\psi\rangle =  e^{i\alpha}|\psi\rangle,
\end{equation}
dove $\alpha$ è una costante reale.\\
Scambiando ancora una volta le due particelle si deve riottenere, evidentemente, lo stato iniziale. L'operatore $P_{12}$ soddisfa cioè:\\
\begin{equation}
P_{12}^2 = 1.
\end{equation}
L'applicazione dell'operatore $P_{12}^2$ allo stato $|\psi\rangle$ equivale a moltiplicare il fattore di stato per $e^{2i}$. Ne segue che $e^{2i\alpha} = 1 $, ossia $e^{i\alpha}=\pm1$. Di conseguenza:
\begin{equation}
P_{12} |\psi\rangle= \pm |\psi\rangle .
\end{equation}
\textbf{Siamo quindi giunti al risultato fondamentale che esistono in tutto due possibilità: o il vettore di stato di un sistema costituito da due particelle identiche è simmetrico, cioè non cambia nello scambio delle due particelle, o esso è antisimmetrico, cioè nello scambio cambia di segno}. Le due combinazioni corrispondono rispettivamente agli stati:
\begin{eqnarray}
\label{eq:cap20_4}
|\psi\rangle= \frac{1}{\sqrt{2}}\left(|a\rangle |b\rangle + |b\rangle|a\rangle \right) ,\nonumber \\
\\
|\psi\rangle= \frac{1}{\sqrt{2}}\left(|a\rangle |b\rangle - |b\rangle|a\rangle \right). \nonumber
\end{eqnarray}
\textbf{È evidente, inoltre, che i vettori di stato rappresentativi di tutti gli stati dello stesso sistema devono godere della stessa simmetria. Se così non fosse, infatti, il vettore di stato che rappresenta la sovrapposizione di stati con diverse simmetrie, non sarebbe né simmetrico né antisimmetrico}.\\
Questo risultato si generalizza immediatamente ai sistemi formati da un \textbf{numero qualsiasi di particelle identiche}. Infatti, a causa dell'identità delle particelle, è chiaro che se una coppia di queste particelle goda della proprietà di poter essere scritta, per esempio, da vettori di stato simmetrici, tutte le altre coppie di particelle avranno la stessa proprietà. \textbf{Quindi il vettore di stato delle particelle identiche deve o restare o assolutamente immutato per lo scambio di qualsiasi coppia di particelle, o cambiare di segno per lo scambio di ogni coppia.}\\
\textbf{Le proprietà del sistema di poter essere descritto da vettori di stato simmetrici o antisimmetrici dipende dalla natura delle particelle che lo compongono.} Delle particelle descritte da vettori di stato simmetrici, si dice che ubbidiscano alla \textbf{statistica di Bose-Einstein}, o che sono \textbf{bosoni}, delle particelle descritte da vettori di stato antisimmetrici, si dice che ubbidiscano alla \textbf{statistica di Fermi-Dirac}, ovvero che sono \textbf{fermioni}. Così, indicando con $P_{ij}$ l'operatore che scambia la i-esima e la j-esima particella si ha:
\begin{eqnarray}
&P_{ij} |N_{\textrm{bosoni identici}}\rangle= +|N_{\textrm{bosoni identici}}\rangle & \nonumber \\
\\
&P_{ij} |N_{\textrm{fermioni identici}}\rangle= -|N_{\textrm{fermioni identici}}\rangle & \nonumber
\end{eqnarray}
Utilizzando le leggi della meccanica quantistica relativistica è possibile mostrare che l\textbf{a statistica cui obbediscono le particelle è univocamente legata al loro spin: le particelle con spin intero sono bosoni, quelle con spin semintero sono fermioni.}\\
È semplice generalizzare le espressioni \ref{eq:cap20_4} al caso di sistemi con un numero arbitrario di particelle identiche. Nel caso generale di un sistema con un numero arbitrario di \textbf{N bosoni identici}, il vettore di stato normalizzato è:\\
\begin{equation}
|\psi\rangle= \left(\frac{N_{1}! N_{2}!\dots}{N!}\right)^{1/2} \begin{matrix} \sum_{k=1}^N |p^k\rangle_{k} \end{matrix}, \\
\end{equation}
dove $|p^{(k)}\rangle_i$ rappresenta il ket della particella i-esima che si trova nello stato $p^k$, la somma \`e estesa a tutte le permutazioni distinte degli indici $p^{(1)}...p^{(N)}$, ed il numero $N_{k}$ indica quante volte il valore $p^{(k)}$ compare nella combinazione sotto il segno di somma.\\
Per un sistema di \textbf{N fermioni identici} il vettore di stato \`e la combinazione antisimmetrica dei prodotti $|p^1\rangle_1...|p^N\rangle_N$. Questa combinazione pu\`o essere scritta nella forma di determinante:\\
\begin{equation}
|\psi\rangle= \frac{1}{\sqrt{N}} \begin{vmatrix} |p_{1}\rangle_1 & |p_{1}\rangle_2 & ... & |p_{1}\rangle_N \\|p_{2}\rangle_1 & |p_{2}\rangle_2 & ... & |p_{2}\rangle_N \\ ... & ... & ... & ... \\ ... & ... & ... & ... \\ ... & ... & ... & ... \\|p_{N}\rangle_1 & |p_{N}\rangle_2 & ... & |p_{N}\rangle_N\end{vmatrix}.
\label{eq:cap20_5}
\end{equation}
Allo scambio di due particelle corrisponde in questo caso lo scambio di due colonne dal determinante, ciò che causa il cambiamento di segno di quest'ultimo.\\
Dall'espressione \ref{eq:cap20_5} segue un risultato importante: se fra gli indici $p_1, p_2...$ ve ne sono due identici, due righe del determinante risulteranno identiche e quindi esso si annulla identicamente \textbf{Di conseguenza, in un sistema di fermioni identici due (o pi\`u) particelle non possono trovarsi in uno stesso stato.} Questo è il cosiddetto \textbf{principio di Pauli.} Nel caso particolare di un sistema costituito da due soli fermioni identici, ad esempio, l'espressione \ref{eq:cap20_4} mostra come il vettore di stato antisimmetrico si annulla identicamente quando i due fermioni si trovano nello stesso stato: $|a\rangle= |b\rangle$.\\
La proprietà di simmetria di un vettore di stato per un sistema di particelle identiche deve conservarsi invariata nel tempo. Così, ad esempio, un vettore di stato simmetrico all'istante iniziale t=0, deve risultare simmetrico a qualunque istante di tempo successivo. \textbf{Questa circostanza è garantita dal fatto che l'operatore hamiltoniano per un sistema composto da N particelle identiche commuta con l'operatore di scambio $P_{i}$ di una coppia qualunque di particelle i e j del sistema:}\\
\begin{equation}
\left [ H, P_{ij}\right ]=0
\end{equation}
Per dimostrare questa relazione limitiamoci a considerare, per semplicit\`a, un sistema costituito da una sola coppia di particelle identiche. Indichiamo poi con $\xi_{1}$ e $\xi_{2}$ gli insiemi delle tre coordinate e della proiezione dello spin di ciascuna delle particelle. Possiamo allora introdurre la funzione d'onda del sistema:\\
\begin{equation}
|\psi_{\alpha}(\xi_1 \xi_2)\rangle =\langle(\xi_1, \xi_2)|\alpha\rangle ,
\end{equation}
e l'espressione dell'hamiltoniano del sistema del sistema nella rappresentazione delle coordinate e dello spin:
\begin{equation}
\langle \xi_1 \xi_2 |H|\alpha \rangle = H(\xi_1 \xi_2)\langle \xi_1 \xi_2 |\alpha \rangle =  H(\xi_1 \xi_2)\psi_{\alpha}(\xi_1 \xi_2) .
\end{equation}
L'espressione dell'operatore H($\xi_1, \xi_2$) sarà della forma:
\begin{equation}
H(\xi_1, \xi_2) = \frac{-\hbar^2}{2m}\bigtriangledown_1^2-\frac{-\hbar^2}{2m}\bigtriangledown_2^2+V(x_1, S_1)+V(x_2,S_2)+U(x_1,x_2,S_1,S_2) .
\end{equation}
È evidente che, \textbf{in virtù dell'identità della particella, l'operatore H($\xi_1, \xi_2$) dovrà essere simmetrico rispetto allo scambio simultaneo delle coordinate spaziali e delle variabili di spin delle due particelle:}
\begin{equation}
H(\xi_1, \xi_2)=H(\xi_2, \xi_1)
\end{equation}
Questa proprietà comporta allora:
\begin{eqnarray}
\langle \xi_1 \xi_2|P_{12}H|\alpha\rangle &=& \langle \xi_2 \xi_1|H|\alpha\rangle=H(\xi_2 \xi_1)\langle \xi_2 \xi_1|\alpha\rangle= \nonumber \\
&=& H(\xi_1 \xi_2)\langle \xi_1 \xi_2|P_{12}|\alpha\rangle=\langle \xi_1 \xi_2|HP_{12}|\alpha\rangle .
\end{eqnarray}
Poiché questa relazione risulta valida per un vettore di stato arbitrario $|\alpha\rangle$ essa deve corrispondere ad un'identità operatoriale, che possiamo scrivere quindi nella forma:
\begin{equation}
\left [ H, P_{12}\right ]=0 .
\end{equation}
In virtù di questa relazione, per un vettore di stato al tempo generico t, ottenuto evolvendo un vettore di stato rispettivamente simmetrico o antisimmetrico al tempo iniziale t=0, troviamo
\begin{equation}
P_{12}|\psi_{S,A}(t)\rangle= P_{12}e^{-\frac{iHt}{\hbar}}|\psi_{S,A}(0)\rangle=e^{
{-\frac{iHt}{\hbar}}
}P_{12}|\psi_{S,A}(0)\rangle=\\\pm e^{-\frac{iHt}{\hbar}}|\psi_{S,A}(0)\rangle ,
\end{equation}
ossia:
\begin{equation}
P_{12}|\psi_{S,A}(t)\rangle= \pm |\psi_{S,A}(t)\rangle .
\end{equation}
Le proprietà di simmetria dei vettori di stato restano dunque costanti nel tempo.
\section{Funzioni d'onda per un sistema\\ composto da due particelle identiche\\ ed interazione di scambio}
Consideriamo la f.d.o. $\psi_{\xi_1 \xi_2}$ per un sistema composto da due particelle identiche rispettivamente bosoni o fermioni. Possiamo osservare che valgono le seguenti relazioni:
\begin{eqnarray}
& &\psi_S (\xi_2, \xi_1) =\langle \xi_2, \xi_1|\psi_S\rangle= \langle \xi_1, \xi_2|P_{12}|\psi_S\rangle=\langle \xi_1, \xi_2|\psi_S|P_{12}\rangle  , \\ 
\nonumber \\
& & \psi_A (\xi_2, \xi_1) =\langle \xi_2, \xi_1|\psi_S\rangle= \langle \xi_1, \xi_2|P_{12}|\psi_S\rangle=-\langle \xi_1, \xi_2|\psi_S|P_{12}\rangle ,
\end{eqnarray}
da cui
\begin{eqnarray}
& &\psi_S (\xi_2, \xi_1) = +|\psi_S(\xi_1, \xi_2)\rangle ,\\
 \nonumber \\
& &\psi_A (\xi_2, \xi_1) = -|\psi_A(\xi_1, \xi_2)\rangle .
\end{eqnarray}
\textbf{Pertanto le f.d.o. corrispondenti agli stati simmetrici ed antisimmetrici dei sistemi costituiti da una coppia di particelle identiche risultano rispettivamente simmetriche ed antisimmetriche rispetto allo scambio simultaneo delle coordinate spaziali e delle variabili di spin delle due particelle.}\\
\textbf{Osserviamo che per un sistema di particelle interagenti elettricamente, in assenza di campo magnetico, la hamiltoniana non dipende dagli operatori di spin.} Quindi l'equazione di Schr\"{o}dinger è soddisfatta da ogni componente di spin della funzione d'onda. In altre parole, la funzione d'onda di un sistema di particelle identiche può essere scritta in forma di prodotto di una funzione $\phi(x_1, x_2...)$, dipendente soltanto dalle coordinate delle particelle, e di una funzione $\chi(\sigma_1,\sigma_2,...)$ dipendente soltanto dallo spin. Per cui un sistema di due particelle si ha ad esempio:
\begin{equation}
|\psi(\xi_1, \xi_2)\rangle= \phi(x_1, x_2)\chi(\sigma_1, \sigma_2) .
\end{equation}
In questi casi l'equazione di Schr\"{o}dinger determina soltanto la funzione delle coordinate $\phi$. Sebbene l'interazione elettrica delle particelle sia indipendente dal loro spin, \textbf{esiste} tuttavia u\textbf{na dipendenza peculiare dell'energia del sistema dal suo spin totale che è originata}, in ultima analisi, dal \textbf{principio di indistinguibilità delle particelle identiche.}\\
Consideriamo un sistema composto da due particelle identiche per il quale la hamiltoniana sia indipendente dagli operatori di spin della particella. Risolvendo l'equazione di Schr\"{o}dinger troviamo una serie di livelli energetici a ciascuno dei quali corrisponde una determinata f.d.o. delle coordinate simmetrica o antisimmetrica:
\begin{equation}
\phi_{S,A}(x_2,x_1)= \pm \phi_{S,A}(x_1,x_2) .
\end{equation}
Infatti, a causa dell'identità delle particelle, la hamiltoniana è invariante rispetto allo scambio di esse.\\
Supponiamo dapprima che le particelle abbiano \textbf{spin nullo}. Il fattore di spin per tali particelle non esiste affatto, e la f.d.o. si riferisce alla sola funzione delle coordinate $\varphi(x_1, x_2)$, che deve essere simmetrica poiché le particelle ubbidiscono alla \textbf{statistica di Bose}. Così \textbf{non tutti i livelli energetici si ottengono dalla soluzione formale della equazione di Schr\"{o}dinger possono realmente esistere. Quelli di essi, a cui corrispondono funzioni $\phi$ antisimmetriche, sono impossibili per il sistema considerato.}\\
Supponiamo ora che il sistema sia costituito da due particelle con \textbf{spin $1/2$} (per esempio, da elettroni). Allora la f.d.o. totale del sistema (cioè il prodotto della funzione $\varphi(x_1,x_2)$ con la funzione di spin $\chi(\sigma_1, \sigma_2)$ deve essere necessariamente antisimmetrica rispetto allo scambio delle due particelle. Quindi \textbf{se la funzione delle coordinate è simmetrica la funzione di spin deve essere antisimmetrica e viceversa.}\\
Sappiamo che per un sistema costituito da due particelle di spin $1/2$ la funzione d'onda di spin simmetrica descrive un sistema con spin totale uguale ad 1 (\textbf{tripletto}) e la funzione antisimmetrica corrisponde allo spin 0 (\textbf{singoletto}). Si hanno quindi le seguenti possibilità:
\begin{eqnarray}
& &\varphi_S(x_1,x_2)\chi_A(\sigma_1,\sigma_2)\qquad \qquad \chi_A=\frac{1}{\sqrt{2}}(\chi_+\chi_--\chi_-\chi_+) , \\
\nonumber \\
& &\varphi_S(x_1,x_2)\chi_S(\sigma_1,\sigma_2)\qquad \qquad \chi_S=
\begin{cases}
\chi_+\chi_- , \\
\frac{1}{\sqrt{2}}(\chi_+\chi_-+\chi_-\chi_+), \\
\chi_-\chi_-.
\end{cases} 
\end{eqnarray}
In altri termini, i livelli energetici cui corrispondono le soluzioni simmetriche dell'equazione di Schr\"{o}dinger possono di fatto esistere se lo spin totale del sistema è uguale a zero, mentre i livelli energetici corrispondenti a funzioni antisimmetriche richiedono che lo spin totale sia uguale ad 1.\\
La circostanza per cui i valori energetici possibili di un sistema di elettroni risultano dipendenti dallo spin totale ci permette di parlare di una interazione peculiare della particella che porta questa dipendenza. Tale interazione si chiama \textbf{interazione di scambio}.%DEFINITIVO
\chapter{Atomo di idrogeno}
\section{Il problema dei due corpi e il moto in un campo centrale }
Il problema del moto di due particelle interagenti nella meccanica quantistica può essere ridotto al problema di una sola particella, in modo analogo a come può essere fatto in meccanica classica.\\

L'hamiltoniano di due particelle (con massa $m_1$ ed $m_2$) che interagiscono secondo la legge $V\left(r\right)$, dove $r$ è la distanza tra le particelle, ha la forma
	\begin{equation}
		\tcboxmath[sharp corners=downhill, colback=white, colframe=black]{
			H=\frac{\vec{p}_1^{\ 2}}{2m_1}+\frac{\vec{p}_2^{\ 2}}{2m_2}+V\left(|\vec{r}_1-\vec{r}_2|\right),
			}
	\end{equation} 
dove
	\begin{equation}
		\tcboxmath[enhanced, sharp corners=downhill, colframe=black, colback=white, borderline={2pt}{-3pt}{black}]{
			V(r) =-\frac{Ze^2}{r},
			}
	\end{equation}
per l'atomo idrogenoide.\\

Introduciamo in luogo dei raggi vettori delle particelle,$\vec{r_1}$ ed $\vec{r_2}$, le nuove variabili
	\begin{equation}
		\tcboxmath[sharp corners=downhill, colback=white, colframe=black]{
			\vec{R}=\frac{m_1\vec{r}_1+m_2\vec{r}_2}{m_1+m_2},
			} \qquad
		\tcboxmath[sharp corners=downhill, colback=white, colframe=black]{
			\vec{r}=\vec{r}_1-\vec{r}_2 ,
			}
	\end{equation}
dove \textbf{$\vec{R}$ è il raggio vettore del centro di massa della particella ed $\vec{r}$ è il vettore della distanza mutua}.\\

Con un semplice calcolo possiamo ottenere le espressioni dell'operatore di energia cinetica in termini degli impulsi coniugati alle variabili $\vec{R}$ ed $\vec{r}$. Si ha:
	\begin{align} 
		\frac{\partial}{\partial x_1} & = \frac{\partial R_x}{\partial x_1} \frac{\partial}{\partial R_x}+\frac{\partial R_y}{\partial x_1} \frac{\partial}{\partial R_y}+\frac{\partial R_z}{\partial x_1} \frac{\partial}{\partial R_z}+\frac{\partial r_x}{\partial x_1} \frac{\partial}{\partial r_x}+\frac{\partial r_y}{\partial x_1} \frac{\partial}{\partial r_y}+\frac{\partial r_z}{\partial x_1} \frac{\partial}{\partial r_z}= \nonumber\\
 		& = \frac{m_1}{m_1+m_2}\frac{\partial}{\partial R_1}+\frac{\partial}{\partial r_1} ,
	\end{align}
e dunque
	\begin{equation}
		\vec{\nabla}_1=\frac{m_1}{m_1+m_2}\vec{\nabla}_R+\vec{\nabla}_r;
	\end{equation}
analogamente si trova
	\begin{equation}
		\vec{\nabla}_2=\frac{m_2}{m_1+m_2}\vec{\nabla}_R-\vec{\nabla}_r.
	\end{equation}
Prendendo il quadrato di queste espressioni otteniamo per i laplaciani:
	\begin{equation}
		\vec{\nabla}_1^2=\frac{m_1^2}{\left(m_1+m_2\right)^2}\vec{\nabla}_R^2+\vec{\nabla}_r^2+\frac{2m_1}{m_1+m_2}\vec{\nabla}_R\cdot\vec{\nabla}_r
	\end{equation}
\begin{equation}
\vec{\nabla}_2^2=\frac{m_2^2}{\left(m_1+m_2\right)^2}\vec{\nabla}_R^2+\vec{\nabla}_r^2-\frac{2m_2}{m_1+m_2}\vec{\nabla}_R\cdot\vec{\nabla}_r.
\end{equation}
Allora
\begin{equation}
\frac{1}{m_1}\nabla_1^2+\frac{1}{m_2}\nabla_2^2=\frac{1}{m_1+m_2}\nabla_R^2+\left(\frac{1}{m_1}+\frac{1}{m_2}\right)\nabla_r^2 .
\end{equation}\\

\textbf{L'hamiltoniana delle due particelle prende} allora, in termini delle variabili del centro di massa e del moto relativo, la forma:
	\begin{equation}
		\tcboxmath[enhanced, sharp corners=downhill, colframe=black, colback=white, borderline={2pt}{-3pt}{black}]{
			H=\frac{\vec{P}^{2}}{2M}+\frac{\vec{p}^{\ 2}}{2m}+V\left(r\right) , 
			}
	\end{equation}
dove
	\begin{equation}
		\tcboxmath[sharp corners=downhill, colback=white, colframe=black]{
			\vec{P}=-i\hbar\vec{\nabla}_R,
			} \qquad
		\tcboxmath[sharp corners=downhill, colback=white, colframe=black]{
			\vec{p}=-i\hbar\vec{\nabla}_r ,
			}
\end{equation}
ed abbiamo introdotto la \textbf{massa totale} del sistema
	\begin{equation}
		\tcboxmath[sharp corners=downhill, colback=white, colframe=black]{
			M=m_1+m_2
			}
	\end{equation}
e la \textbf{massa ridotta}\footnote{Nel caso di protone ed elettrone, per i quali $m_p \gg m_e$ ($m_p \simeq$ 938MeV, $m_e\simeq$ 0.51MeV) si ha $M\simeq m_p$ ed $m\simeq m_e$.}
	\begin{equation}
		\tcboxmath[sharp corners=downhill, colback=white, colframe=black]{
			m=\left(\frac{1}{m_1}+\frac{1}{m_2}\right)^{-1}=\frac{m_1m_2}{m_1+m_2}.
			}
	\end{equation}\\
	
L'hamiltoniano si scompone quindi nella somma di due parti indipendenti. Partendo da questo fatto \textbf{si può cercare la soluzione dell'equazione di Schr\"{o}dinger del sistema nella forma}:
	\begin{equation}
		\tcboxmath[enhanced, sharp corners=downhill, colframe=black, colback=white, borderline={2pt}{-3pt}{black}]{
			\psi\left(\vec{r}_1,\vec{r}_2\right)=\phi(\vec{R})\psi(\vec{r}) ,
			}
	\end{equation}
\textbf{dove la funzione $\psi (\vec{R} )$ descrive il moto del centro di massa, come moto libero di una particella di massa $M=m_1+m_2$, e $\psi\left(\vec{r}\right)$ descrive il moto relativo delle particelle come moto di una particella di massa $m$ in un campo a simmetria centrale $V=V\left(r\right)$}.\\

L'equazione di Schr\"{o}dinger del moto di una particella nel campo a simmetria centrale ha la forma
	\begin{equation}
		\tcboxmath[enhanced, sharp corners=downhill, colframe=black, colback=white, borderline={2pt}{-3pt}{black}]{
			\left[-\frac{\hbar^2}{2m}\nabla^2+V\left( r \right) \right]\psi\left(r\right)=E\psi\left(r\right).
			}
	\label{21.1}
	\end{equation} 
In questa equazione \textbf{$E=E_{tot}-\frac{\vec{P}^2}{2M}$} è l'energia interna del sistema delle due particelle, ossia l'energia restante a seguito della sottrazione dell'energia cinetica associata al moto traslatorio del sistema nel suo insieme.\\ 

Risulta conveniente studiare l'equazione (\ref{21.1}) in coordinate polari. A tale scopo deriviamo in primo luogo una \textbf{relazione tra l'operatore laplaciano ed il quadrato $L^2$ del momento angolare orbitale}. Si ha:
	\begin{align} 
	\label{21.2}
		L^2 & = \left(\vec{r}\wedge\vec{p}\right)^2=-\hbar^2\left(\vec{r}\wedge\vec{\nabla}\right)^2=-\hbar^2\left(\vec{r}\wedge\vec{\nabla}\right)_i\left(\vec{r}\wedge\vec{\nabla}\right)_i= \nonumber \\
		 & =  -\hbar^2\varepsilon_{ijk}\,\varepsilon_{ilm}\, r_j\partial_k r_l\partial_m= -\hbar^2\left(\delta_{jl}\delta_{km}-\delta_{jm}\delta_{kl}\right)r_j\partial_k r_l\partial_m=\nonumber \\
		 & =-\hbar^2\left(r_j\partial_kr_j\partial_k-r_j\partial_kr_k\partial_j\right) ,
\end{align}
facendo uso della seguente relazione
	\begin{equation}
		\partial_kr_j=r_j\partial_k+\delta_{kj}
	\end{equation}
è possibile scrivere la (\ref{21.2}) come:
	\begin{align}
	\label{21.3}
		L^2&=-\hbar^2\left[r_j\left(r_j\partial_k+\delta_{kj}\right)\partial_k-\left(\partial_kr_j-\delta_{kj}\right)r_k\partial_j\right]= \nonumber\\
		&=-\hbar^2\left[r_jr_j\partial_k\partial_k+r_k\partial_j-\partial_kr_kr_j\partial_j+r_j\partial_j\right]= \nonumber \\
		&=-\hbar^2\left[r^2\nabla^2+2\vec{r}\cdot\vec{\nabla}-\left(r_k\partial_k+\delta_{kk}\right)r_j\partial_j\right]= \nonumber\\
		&=-\hbar^2\left[r^2\nabla^2-\left(\vec{r}\cdot\vec{\nabla}\right)^2-\vec{r}\cdot\vec{\nabla}\right].
	\end{align}\\
%Ossia
%\begin{equation}
%L^2=r^2p^2-\left(\vec{r}\cdot\vec{p}\right)^2+i\hbar\vec{r}\cdot\vec{p} ,
%\end{equation}
%o, equivalentemente
%\begin{equation}
%\label{21.3}
%p^2=\frac{\left(\vec{r}\cdot\vec{p}\right)^2}{r^2}-i\hbar\frac{\vec{r}\cdot\vec{p}}{r^2}+\frac{L^2}{r^2}.
%\end{equation}
%Si noti che in meccanica classica vale la stessa relazione con $\hbar=0$ ($\left[p_i,r_j\right]\rightarrow0$)\\
D'altra parte il prodotto scalare $\vec{r}\cdot\vec{p}$ coincide, a meno di un fattore moltiplicativo, con la proiezione dell'operatore gradiente nella direzione del raggio vettore $\vec{r}$, ossia con la derivata rispetto ad $r$. Si ha infatti esplicitamente:
	\begin{align} 
		\frac{\partial}{\partial r}&= \frac{\partial x}{\partial r} \frac{\partial}{\partial x}+\frac{\partial y}{\partial r} \frac{\partial}{\partial y}+\frac{\partial z}{\partial r} \frac{\partial}{\partial z}= \nonumber \\
		&=  \frac{1}{r}\left(x\frac{\partial}{\partial x}+y\frac{\partial}{\partial y}+z\frac{\partial}{\partial z}\right)=\frac{1}{r}\vec{r}\cdot\vec{\nabla} ,
	\end{align}
ossia
	\begin{equation}
		\tcboxmath[sharp corners=downhill, colback=white, colframe=black]{
			\left(\vec{r}\cdot\vec{p}\right)=-i\hbar\left(\vec{r}\cdot\vec{\nabla}\right)=-i\hbar r \frac{\partial}{\partial r} .
			}
	\end{equation}\\
	
Sostituendo questa equazione nella relazione (\ref{21.3}) giungiamo infine all'espressione del laplaciano, in coordinate polari, in termini dell'operatore $L^2$
	\begin{align}
		\vec{p}^{\ 2}&= -\hbar^2\nabla^2= -\frac{\hbar ^2}{r^2}\left[ \left( \vec{r}\cdot \vec{\nabla} \right) ^2 - \left( \vec{r}\cdot \vec{\nabla} \right)\right] +\frac{L^2}{r^2} = \nonumber \\
		&= -\frac{\hbar^2}{r^2}\left[\left(r\frac{\partial}{\partial r}\right)\left(r\frac{\partial}{\partial r}\right)+\left(r\frac{\partial}{\partial r}\right)\right]+\frac{L^2}{r^2}= -\frac{\hbar}{r^2}\left[r^2\frac{\partial^2}{\partial r^2}+2r\frac{\partial}{\partial r}\right]+\frac{L^2}{r^2} ,
	\end{align}
o, equivalentemente,
	\begin{equation}
		\tcboxmath[sharp corners=downhill, colback=white, colframe=black]{
			\vec{p}^{\ 2}=-\hbar^2\nabla^2=\frac{\hbar^2}{r^2}\frac{\partial}{\partial r}\left(r^2 \frac{\partial}{\partial r}\right)+\frac{L^2}{r^2} .
			}
	\end{equation}\\
	
\textbf{L'equazione di Schr\"{o}dinger del moto di una particella nel campo a simmetria centrale si scrive allora nella forma:}
	\begin{equation}
		\tcboxmath[enhanced, sharp corners=downhill, colframe=black, colback=white, borderline={2pt}{-3pt}{black}]{
			-\frac{\hbar^2}{2mr^2}\frac{\partial}{\partial r}\left(r^2\frac{\partial}{\partial r}\right)\psi+\left[\frac{L^2}{2mr^2}+V\left(r\right)\right]\psi=E\psi .
			}
	\end{equation}
Tutta la dipendenza dagli angoli delle coordinate polari, in questa equazione, è \textbf{contenuta nell'operatore $L^2$}, che dunque \textbf{commuta con l'hamiltoniano}. Ne segue pertanto che\textbf{ nel moto in un campo a simmetria centrale il momento angolare orbitale si conserva}. \textbf{L'hamiltoniano commuta anche con le componenti $L_x$, $L_y$ , $L_z$ che dunque sono separatamente conservate.}\\

Consideriamo ora le autofunzioni simultanee dell'hamiltoniano e dell'operatore $L^2$, ossia le f.d.o.~degli stati stazionari del sistema con valori determinati del momento angolare $l$ e della sua proiezione lungo l'asse $z$. Queste funzioni sono della forma
	\begin{equation}
		\tcboxmath[enhanced, sharp corners=downhill, colframe=black, colback=white, borderline={2pt}{-3pt}{black}]{
			\psi\left(\vec{r}\right)=R\left(r\right)Y_{lm}\left(\theta,\phi\right) ,
			}
	\end{equation}
dove $Y_{lm}$ sono le funzioni armoniche sferiche.\\

Poiché $L^2Y_{lm}=\hbar^2l\left(l+1\right)Y_{lm}$, per la funzione d'onda radiale $R\left(r\right)$ si ottiene l'equazione:
	\begin{equation}
		\tcboxmath[enhanced, sharp corners=downhill, colframe=black, colback=white, borderline={2pt}{-3pt}{black}]{
			\left[-\frac{\hbar^2}{2mr^2}\frac{\partial}{\partial r}\left(r^2\frac{\partial}{\partial r}\right)+\frac{\hbar^2l\left(l+1\right)}{2mr^2}+V\left(r\right)-E\right]R\left(r\right)=0 .
			}
	\end{equation}\\
	
Questa equazione non contiene affatto il valore di $L_z=m$, da cui segue che \textbf{i livelli di energia sono $\left(2l+1\right)$ volte degeneri rispetto alle direzioni del momento angolare}.\\

Effettuiamo nella equazione d'onda per il moto radiale la sostituzione:
	\begin{equation}
		\tcboxmath[sharp corners=downhill, colback=white, colframe=black]{
			R\left(r\right)=\frac{1}{r}\chi\left(r\right) .
			}
	\end{equation}
Poiché
	\begin{align}
		\frac{1}{r^2}\frac{\partial}{\partial r}\left(r^2\frac{\partial R}{\partial r}\right)&=\frac{1}{r^2}\frac{\partial}{\partial r}\left(r^2\frac{\partial}{\partial r}\frac{\chi}{r}\right)=\frac{1}{r^2}\frac{\partial}{\partial r}\left(r \frac{\partial\chi}{\partial r}-\chi\right)= \nonumber \\
		&= \frac{1}{r^2}\left(r\frac{\partial^2\chi}{\partial r^2}+\frac{\partial\chi}{\partial r}-\frac{\partial\chi}{\partial r}\right) = \frac{1}{r}\frac{\partial^2 \chi}{\partial r^2}
\end{align}
l'equazione radiale si riduce a:
	\begin{equation}
		\tcboxmath[sharp corners=downhill, colback=white, colframe=black]{
			\left[-\frac{\hbar^2}{2m}\frac{\partial^2}{\partial r^2}+\frac{\hbar^2l\left(l+1\right)}{2mr^2}+V\left(r\right)-E\right]\chi\left(r\right)=0 .
			}
	\end{equation}\\

Questa equazione coincide formalmente con l'\textbf{equazione di Schr\"{o}dinger unidimensionale} per il moto in un campo con energia potenziale
	\begin{equation}
		\tcboxmath[sharp corners=downhill, colback=white, colframe=black]{
			V_l\left(r\right)=V\left(r\right)+\frac{\hbar l\left(l+1\right)}{2mr^2} ,
			}
	\end{equation}
uguale alla somma dell'energia $V\left(r\right)$ e del termine
\begin{equation}
\frac{\hbar l\left(l+1\right)}{2mr^2}=\frac{L^2}{2mr^2} ,
\end{equation}
che si chiama energia centrifuga. IL problema del moto in un campo a simmetria centrale si riduce, quindi, al problema unidimensionale in una regione semilimitata, $r>0$.\\

Carattere unidimensionale ha anche la condizione di normalizzazione della funzione $\chi\left(r\right)$, che è definita dall'integrale:
	\begin{equation}
		\tcboxmath[sharp corners=downhill, colback=white, colframe=black]{
			\int_0^\infty dr\ r^2|R\left(r\right)|^2=\int_0^\infty dr\ |\chi\left(r\right)|^2 .
			}
	\end{equation}\\

É possibile dimostrare che nel moto unidimensionale in una regione semilimitata \textbf{i livelli energetici non sono degeneri}. Si può allora affermare che l'assegnazione del valore dell'energia determina completamente la parte radiale della funzione d'onda. Tenendo anche conto che la parte angolare della funzione d'onda è data completamente dai valori di l ed m, concludiamo che \textbf{nel moto in un campo a simmetrica centrale la f.d.o.~è definita completamente da valori i $E$, $l$, $m$}. In altri termini l'energia, il quadrato del momento angolare e la sua proiezione costituiscono un sistema completo di grandezze fisiche per tale moto.

\section{Campo Coulombiano}
Un caso molto importante di  moto in un campo a simmetria centrale è quello del moto in un campo coulombiano
	\begin{equation}
		\tcboxmath[enhanced, sharp corners=downhill, colframe=black, colback=white, borderline={2pt}{-3pt}{black}]{
			V\left(r\right)=-\frac{Ze^2}{r} 
			}
	\end{equation}
($Z=1$ per l'atomo di idrogeno).\\

Dalle considerazioni fatte sappiamo che il moto si riduce formalmente ad un moto unidimensionale con energia potenziale efficace
	\begin{equation}
		\tcboxmath[sharp corners=downhill, colback=white, colframe=black]{
V_l\left(R\right)=-\frac{Xe^2}{r}+\frac{\hbar^2l\left(l+1\right)}{2mr^2} .
}
	\end{equation}
Riportiamo qui di seguito un grafico di tale potenziale\\
\begin{figure}[!htbp]
\begin{center}
\includegraphics[width=8.5cm]{immagini/cap_21/fig_21_1.png}
\end{center}
\end{figure}\\
Si vede allora che lo \textbf{spettro degli autovalori negativi dell'energia è discreto e corrisponde agli stati legati del sistema, mentre quello delle energie positive è continuo ed il moto corrispondente si estende da zero all'infinito}.\\

Consideriamo qui in particolare il caso dello \textbf{spettro discreto} ossia degli stati legati degli atomi idrogenoidi.\\

L'equazione di Schr\"{o}dinger per le funzioni radiali si scrive:
	\begin{equation}
		\begin{aligned}
		\tcboxmath[enhanced, sharp corners=downhill, colframe=black, colback=white, borderline={2pt}{-3pt}{black}]{
			\left[-\frac{\hbar^2}{2m}\frac{1}{r^2}\frac{d}{d r}\left(r^2\frac{d}{d r}\right)+\frac{\hbar^2l\left(l+1\right)}{2mr^2}\right]R\left(r\right)-\left(\frac{Ze^2}{r}+E\right)R\left(r\right)=0 .
		}\\[-0.3cm]			
			\quad
		\end{aligned}
	\label{21.4}
	\end{equation}\\

Per risolvere questa equazione risulta conveniente introdurre in primo luogo delle \textbf{variabili adimensionali}. Dalla precedente equazione risultano evidenti le seguenti uguaglianze "dimensionali":
	\begin{equation}
		\left[E\right]=\left[\frac{Ze^2}{r}\right]=\left[\frac{\hbar^2}{mr^2}\right] ,
	\end{equation}
ossia
	\begin{equation}
		\tcboxmath[sharp corners=downhill, colback=white, colframe=black]{
			\left[r\right]=\left[\frac{\hbar^2}{mZe^2}\right] ,
			} \qquad
		\tcboxmath[sharp corners=downhill, colback=white, colframe=black]{
			\left[E\right]=\left[\frac{m\left(Ze^2\right)^2}{\hbar^2}\right] .
			}
	\end{equation}

Introduciamo allora, in luogo dell'energia, una nuova variabile definita da
	\begin{equation}
		\tcboxmath[enhanced, sharp corners=downhill, colframe=black, colback=white, borderline={2pt}{-3pt}{black}]{
			E=-\frac{1}{2n^2}\frac{m\left(Ze^2\right)^2}{\hbar^2} ;
			}
	\label{21.5}
	\end{equation}
per energie negative $n$ è un numero reale positivo.\\

In luogo del raggio $r$ introduciamo poi la variabile adimensionale $\rho$ definita da 
	\begin{equation}
		\tcboxmath[enhanced, sharp corners=downhill, colframe=black, colback=white, borderline={2pt}{-3pt}{black}]{
			r=\frac{n}{2}\frac{\hbar^2}{mze^2}\, \rho .
			}
	\label{21.6}
	\end{equation}
Sostituendovi le (\ref{21.5}) e (\ref{21.6}), l'equazione (\ref{21.4}) diventa:
	\begin{align}
		\frac{\hbar^2}{2m}\left(\frac{2mZe^2}{n\hbar^2}\right) &\left[\frac{1}{\rho^2}\frac{d}{d\rho}\left(\rho^2\frac{d}{d\rho}\right)   -\frac{l\left(l+1\right)}{\rho^2}\right]R +\nonumber \\
		& \qquad + \left[\frac{2m\left(Ze^2\right)^2}{n\hbar^2}\frac{1}{\rho}-\frac{m\left(Ze^2\right)^2}{2n^2\hbar^2}\right]R=0 ,
	\end{align}
ossia, dividendo per $2m\left(Ze^2\right)^2/\left(n^2\hbar^2\right)$ ed esplicitando le derivate:
	\begin{equation}
		\tcboxmath[enhanced, sharp corners=downhill, colframe=black, colback=white, borderline={2pt}{-3pt}{black}]{
			\frac{d^2R}{d\rho^2}+\frac{2}{\rho}\frac{dR}{d\rho}+\left[-\frac{l\left(l+1\right)}{\rho^2}+\frac{n}{\rho}-\frac{1}{4}\right]R=0 .
			}
	\label{21.7}
	\end{equation}\\

Consideriamo dapprima le \textbf{soluzioni asintotiche} dell'equazione (\ref{21.7}) valide per $\rho\rightarrow0$ (che è un punto singolare) e $\rho\rightarrow\infty$.\\

Nel limite \textbf{$\rho\rightarrow0$} l'equazione (\ref{21.7}) si riduce a:
	\begin{equation}
		\tcboxmath[sharp corners=downhill, colback=white, colframe=black]{
			\frac{d^2R}{d\rho^2}+\frac{2}{\rho}\frac{dR}{d\rho}-\frac{l\left(l+1\right)}{\rho^2}R=0 .
			}
	\label{21.8}
	\end{equation}\\
	
Cerchiamo una soluzione di questa equazione della forma
	\begin{equation}
		\tcboxmath[sharp corners=downhill, colback=white, colframe=black]{
			R\left(\rho\right)=\mbox{costante}\cdot \rho^s .
			}
	\end{equation}\\

Sostituendo questa espressione nell'equazione (\ref{21.8}) si ottiene:
	\begin{equation}
		s\left(s-1\right)+2s-l\left(l+1\right)=0 ,
	\end{equation}
ossia
	\begin{equation}
		\tcboxmath[sharp corners=downhill, colback=white, colframe=black]{
		s\left(s+1\right)=l\left(l+1\right) ,
		}
	\end{equation}
che ha come soluzione
\begin{equation}
s=\frac{-1\pm\sqrt{1+4l\left(l+1\right)}}{2}=\frac{-1\pm\left(2l+1\right)}{2}=\begin{cases}l, \\-\left(l+1\right) . \end{cases}
\end{equation}\\

La soluzione $s=-\left(l+1\right)$ deve essere scartata perché  conduce ad una f.d.o.~divergente nell'origine. \textbf{Nell'intorno dell'origine} si ha quindi
	\begin{equation}
		\tcboxmath[enhanced, sharp corners=downhill, colframe=black, colback=white, borderline={2pt}{-3pt}{black}]{
			R\left(\rho\right)\simeq\rho^l ,
			} \qquad \left(\rho\rightarrow0\right) .
	\end{equation}\\

Osserviamo che \textbf{questo risultato rimane valido per ogni potenziale che nell'origine diverge più lentamente del potenziale centrifugo}, ossia più lentamente di $1/r^2$. Il suo significato è che quanto più grande è il valore del momento angolare, tanto più piccolo è la probabilità di trovare la particella nell'origine.
Questo risultato è in accordo anche con le previsioni classiche.\\

Studiando ora l'equazione per grandi $\rho$, ossia nel limite \textbf{$\rho\rightarrow\infty$}. In tale approssimazione l'equazione (\ref{21.7}) si riduce a:
	\begin{equation}
		\tcboxmath[sharp corners=downhill, colback=white, colframe=black]{
			\frac{d^2R}{d\rho^2}-\frac{1}{4}R=0 .
			}
	\end{equation}
La cui soluzione è
	\begin{equation}
		R\left(\rho\right)=e^{\pm\rho/2} . 
	\end{equation}
La soluzione che si annulla all'infinito, la sola fisicamente accettabile, è:
	\begin{equation}
		\tcboxmath[enhanced, sharp corners=downhill, colframe=black, colback=white, borderline={2pt}{-3pt}{black}]{
			R\left(\rho\right)=e^{-\rho/2} .
			}
	\end{equation}\\
	
In definitiva concludiamo che la soluzione cercata è della forma
	\begin{equation}
		\tcboxmath[enhanced, sharp corners=downhill, colframe=black, colback=white, borderline={2pt}{-3pt}{black}]{
			R\left(\rho\right)=\rho^le^{-\rho/2}w\left(\rho\right) ,
			}
	\label{21.9}
	\end{equation}
dove $w$ è una funzione da determinare che deve divergere all'infinito non più rapidamente di una potenza finita di $\rho$ e deve essere finita per $\rho=0$.\\

Sostituendo la f.d.o.~(\ref{21.9}) nell'equazione radiale (\ref{21.7}), e considerando che 
	\begin{equation}
		\frac{dR}{d\rho}=\rho^{l-1}e^{-\rho/2}[lw-\frac{1}{2}\rho w+\rho w^\prime] ,
	\end{equation}
e
	\begin{equation}
		\tcboxmath[sharp corners=downhill, colback=white, colframe=black]{
			\frac{d^2R}{d\rho^2}=\rho^{l-2}e^{-\rho/2}\left[\rho^2w^{\prime\prime}+\left(2l-\rho\right)\rho w^\prime+\left(l\left(l-1\right)-l\rho+\frac{1}{4}\rho^2\right)w\right] ,
			}
	\end{equation}
otteniamo per $w$ l'equazione
	\begin{equation}
		\tcboxmath[enhanced, sharp corners=downhill, colframe=black, colback=white, borderline={2pt}{-3pt}{black}]{
			\rho w^{\prime\prime}+\left(2l+2-\rho\right)w^\prime+\left(n-l-1\right)w=0 .
			}
	\end{equation}\\
	
Cerchiamo per la soluzione $w\left(\rho\right)$ un'espressione per serie, poniamo cioè 
	\begin{equation}
		\tcboxmath[enhanced, sharp corners=downhill, colframe=black, colback=white, borderline={2pt}{-3pt}{black}]{
			w\left(\rho\right)=\sum_{k=0}^\infty a_k \rho^k .
			}
	\end{equation}
Sostituendo nell'equazione di $w$ otteniamo:
	\begin{equation}
		\tcboxmath[sharp corners=downhill, colback=white, colframe=black]{
			\sum_{k=0}^\infty\left[a_{k+1}k\left(k+1\right)+\left(2l+2\right)\left(k+1\right)a_{k+1}-ka_k+\left(n-l-1\right)a_k\right]\rho^k=0 .
			}
	\end{equation}
Poiché la serie sia nulla per ogni valore di $\rho$ devono essere separatamente nulli i coefficienti di ogni potenza di $\rho$, si deve cioè avere
	\begin{equation}
		\tcboxmath[enhanced, sharp corners=downhill, colframe=black, colback=white, borderline={2pt}{-3pt}{black}]{
			a_{k+1}=\frac{k-n+l+1}{\left(k+1\right)\left(k+2l+2\right)}a_k .
			}
	\end{equation}
L'andamento asintotico dei coefficienti della serie, per grandi valori di $k$, risulta tale che
	\begin{equation}
		\frac{a_{k+1}}{a_k}\simeq \frac{1}{k},
	\end{equation}
pertanto
	\begin{equation}
		a_{k+1}=\frac{a_{k+1}}{a_{k}}\frac{a_{k}}{a_{k-1}}\frac{a_{k-1}}{a_{k-2}}\dots \frac{a_{1}}{a_{0}}a_0  \simeq \frac{1}{k} \frac{1}{k-1}\frac{1}{k-2}\dots a_0\simeq \frac{a_0}{k!}  .
	\end{equation}
Ma in questo caso
	\begin{equation}
		w(\rho) =\sum _{k=0} ^{\infty} a_n\rho ^k \approx a_0 \sum _k \frac{\rho ^k}{k!} \approx a_0\ e^{\rho}.
	\end{equation}\\

La funzione $w (\rho )$ così trovata non soddisfa la condizione al contorno all'infinito. Perché $w(\rho )$ diverga all'infinito come una potenza finita di $\rho$ deve essere $(n-l-1)$ un numero intero positivo o nullo. In tal caso al serie viene interrotta e $w (\rho )$ si riduce ad un polinomio di grado $(n-l-1)$. Siamo così giunti alla conclusione che \textbf{il numero $n$ deve essere un intero positivo e, per $l$ dato, si deve avere:}
	\begin{equation}
		\tcboxmath[enhanced, sharp corners=downhill, colframe=black, colback=white, borderline={2pt}{-3pt}{black}]{
			n \geq l+1.
			}
	\end{equation}\\
	
Ricordando la definizione del parametro $n$ troviamo
	\begin{equation}
		\tcboxmath[enhanced, sharp corners=downhill, colframe=black, colback=white, borderline={2pt}{-3pt}{black}]{
			E_n= -\frac{m\left(Ze^2\right) ^2}{2\hbar ^2 n^2}
			}\qquad n=1,2,\dots
	\end{equation}
Lo spettro discreto in un campo coulombiano è costituito dunque da un'infinità di livelli compresi tra il livello fondamentale
	\begin{equation}
		\tcboxmath[enhanced, sharp corners=downhill, colframe=black, colback=white, borderline={2pt}{-3pt}{black}]{
			E_1-\frac{m\left(Ze^2\right) ^2}{2\hbar ^2 }\simeq Z^2\cdot \left(13.6 \textrm{eV}\right)
			}
	\end{equation}
e zero. Gli intervalli tra due livelli consecutivi diminuiscono al crescere di $n$. I livelli si infittiscono man mano che ci si avvicina al valore $E=0$, dove lo spettro discreto si connette con quello continuo.\\

Il numero intero $n$ è detto \textbf{numero quantico principale}. Per un numero quantico principale dato, il numero $l$ può assumere i valori
	\begin{equation}
		\tcboxmath[enhanced, sharp corners=downhill, colframe=black, colback=white, borderline={2pt}{-3pt}{black}]{
			l=0,1,\dots,n-1,
			}
	\end{equation}
in totale $n$ valori diversi.\\

Nell'espressione dell'energia entra solo il numero $n$. pertanto tutti gli stati con $l$ diversi, ma con uguali $n$, hanno la stessa energia. Ogni autovalore è quindi degenere non soltanto rispetto al numero quantico $m$ (come per qualsiasi moto in un campo a simmetria centrale) ma anche rispetto al numero $l$. Quest'ultima \textbf{degenerazione}, detta \textbf{accidentale}, è specifica del campo coulombiano. Ad ogni valore di $l$ dato corrispondono $2l+1$ valori differenti di $m$. pertanto \textbf{l'ordine di degenerazione dell' n-esimo livello energetico è}, considerando anche la degenerazione di spin:
	\begin{equation}
		2 \left[ \sum _{l=0} ^{n-1} \left(2l+1 \right)\right]=2 \left[ 2\frac{\left( n-1 \right) n}{2}+n\right],
	\end{equation}
ossia
	\begin{equation}
		\tcboxmath[enhanced, sharp corners=downhill, colframe=black, colback=white, borderline={2pt}{-3pt}{black}]{
			2 \left[ \sum _{l=0} ^{n-1} \left(2l+1 \right)\right]= 2n^2.
			}
	\end{equation}\\
	
I polinomi che si ottengono interrompendo la serie che esprime $w(\rho)$ sono i cosiddetti \textbf{polinomi generalizzati di Laguerre.} Le f.d.o.~radiali complete normalizzate con la condizione
	\begin{equation}
		\tcboxmath[enhanced, sharp corners=downhill, colframe=black, colback=white]{
			\int _0 ^{\infty} dr\, r^2 {R_{nl} (r)}^2=1,
			}
	\end{equation}
hanno la forma
	\begin{equation}
		\tcboxmath[enhanced, sharp corners=downhill, colframe=black, colback=white, borderline={2pt}{-3pt}{black}]{
			R_{nl}(r) =\frac{2}{n^2}\sqrt{\frac{\left(n-l-1\right) !}{\left(n+l\right) !}} e^{-\frac{\bar{r}}{n}}\left(\frac{2\bar{r}}{n}\right) ^l L_{n-l-1} ^{2l+1} \left(\frac{2\bar{r}}{n}\right) ,
			}
	\end{equation}
dove $L_k ^a (x)$ sono i polinomi generalizzati di Laguerre e
	\begin{align}
		&\tcboxmath[enhanced, sharp corners=downhill, colframe=black, colback=white, borderline={2pt}{-3pt}{black}]{
			\bar{r}= Z\frac{r}{a_0},
			} \\[0.3cm]
		&\tcboxmath[enhanced, sharp corners=downhill, colframe=black, colback=white, borderline={2pt}{-3pt}{black}]{
			a_0= \frac{\hbar ^2}{me^2}.
			}
	\end{align}\\
	
La grandezza $a_0$ è detta \textbf{raggio di Bohr} e vale
	\begin{equation}
		\tcboxmath[sharp corners=downhill, colback=white, colframe=black]{
			a_0 \simeq 0.529 \times 10^{-8} \textrm{cm}.
			}
	\end{equation}
La decrescita esponenziale delle f.d.o. radiali indica che $n a_0/Z$ rappresenta la tipica dimensione radiale delle orbite stazionarie per un dato valore del numero quantico principale $n$. Inoltre le orbite sono tanto più vicine al nucleo quanto più è alta la carica elettrica $Z$ di quest'ultimo.\\

A partire dalla dimensione tipica del raggio delle orbite, $r \backsim a_0$, deriviamo la tipica \textbf{velocità del moto dell'elettrone}, utilizzando il principio di indeterminazione. Si ha:
	\begin{equation}
		r   \backsim   a_0 =\frac{\hbar ^2}{me^2}  \Rightarrow   v=\frac{p}{m} \backsim \frac{1}{m} \frac{\hbar}{r}\backsim \frac{\hbar}{m}\frac{me^2}{\hbar ^2}=\frac{e^2}{\hbar}, 
	\end{equation}
da cui
	\begin{equation}
		\tcboxmath[sharp corners=downhill, colback=white, colframe=black]{
			\frac{v}{c}\backsim \frac{e^2}{\hbar c}=\alpha \simeq \frac{1}{137}.
			}
	\end{equation}
%\section{Modello di Bohr dell'atomo di idrogeno (1913)}
%Tre ipotesi:
%\begin{enumerate}
%\item L'elettrone ruota attorno al nucleo su orbita stabile (senza emettere radiazione);
%\item Le sole orbite consentite sono quelle per le quali il momento angolare risulta essere un multiplo intero di $\hbar$:
%\begin{equation}
%L=mvr=h\hbar;
%\label{eq:21.10}
%\end{equation}
%\item L'elettrone può effettuare transizioni discontinue tra due orbite consentite. Quando ciò accade viene emessa o assorbita radiazione di frequenza
%\begin{equation}
%\hbar \omega = E-E'
%\end{equation}
%dove $\Delta E = E-E'$ è la variazione di energia dell'elettrone tra le due orbite.
%\end{enumerate}
%\textbf{Conseguenze:} la stabilità di un'orbita è determinata dall'equilibrio tra la forza coulombiana e la forza centrifuga:
%\begin{eqnarray}
%&\displaystyle{\frac{e^2}{r^2}=\frac{mv^2}{r}.}&\\
%&\textrm{N.B. Forza centrifuga: } F=m\omega ^2 r = \frac{mv^2}{r}. \qquad \qquad (v=\omega r)& \nonumber
%\end{eqnarray}
%Questa condizione, unita alla (\ref{eq:21.10}), fornisce
%\begin{equation}
%\begin{cases}
%mv^2r=e^2\\
%mvr=n\hbar
%\end{cases}
%\Rightarrow \ v=\frac{e^2}{n\hbar}, \quad r=\frac{n\hbar}{mv}=\frac{n^2\hbar ^2}{me^2},
%\end{equation}
%dunque
%\begin{eqnarray}
%&r=n^2\frac{\hbar^2}{me^2}=n^2a_0, \qquad \left[\textrm{In meccanica quantistica } \langle \frac{1}{r}\rangle =\frac{1}{n^2 a_0}\right]& \\
%&\frac{v}{c}=\frac{1}{n}\frac{e^2}{\hbar c}=\frac{\alpha}{n}. \qquad \qquad \left[\textrm{In meccanica quantistica } \langle \frac{v^2}{c^2}\rangle =\frac{\alpha ^2}{n^2}\right]&
%\end{eqnarray}
%Calcoliamo l'energia delle orbite:
%\begin{eqnarray}
%&\displaystyle{\frac{p^2}{2m}=\frac{mv^2}{2}=\frac{me^4}{2n^2\hbar ^2}=\frac{mc^2 \alpha ^2}{2n^2}} ,& \\
%\nonumber \\
%&\displaystyle{\frac{e^2}{r}=\frac{me^4}{n^2\hbar ^2}=\frac{mc^2 \alpha ^2}{n^2}} ,&\\
%& \left[\textrm{N.B. In M.Q. valgono questi stessi risultati per i valori medi di T e V.}\right]& \nonumber
%\end{eqnarray}
%da cui
%\begin{equation}
%E=\frac{p^2}{2m}-\frac{e^2}{r}=-\frac{mc^2\alpha ^2}{2n^2}.
%\end{equation}
%Le righe di emissione e assorbimento dell'atomo hanno pertanto frequenze della forma
%\begin{equation}
%\hbar \omega =E-E' = \frac{mc^2\alpha ^2}{2} \left( \frac{1}{n'^2}-\frac{1}{n^2}\right).
%\end{equation}
\section{Trucchi con le costanti fondamentali}
In luogo di $c$, $\hbar$, $e$, $m_e$ utilizzare:
\begin{itemize}
\item $\displaystyle{\alpha =\frac{e^2}{\hbar c}}= \frac{1}{137}$, costante di struttura fine;
\item $\displaystyle{r_0 =\frac{e^2}{m_e c^2}=2.82 \times 10^{-13}}$ cm ($=2.82$ fm), raggio classico dell'elettrone;
\item $\displaystyle{m_ec^2=0.511}$ MeV, massa a riposo dell'elettrone;
\item $\displaystyle{c=3\times 10^{10}}$ cm/s, velocità della luce nel vuoto.
\end{itemize}
Esempi:
\begin{itemize}
\item[-]Raggio di Bohr:
\begin{eqnarray}
a_o &=&\frac{\hbar ^2}{me^2}=\left( \frac{\hbar c}{e^2}\right)\frac{e ^2}{mc^2}=\frac{r_0}{\alpha ^2} \nonumber \\
&\simeq &(137)^2\cdot 2.82 \cdot 10^{-13}=0.529\cdot10^{-10} \textrm{ cm}\ (=0.529 \textrm{ Å})\nonumber
\end{eqnarray}
\item[-]Lunghezza d'onda Compton dell'elettrone:
\begin{eqnarray}
\lambda _e &=&\frac{\hbar}{mc}=\frac{\hbar}{e^2}\frac{e^2}{mc}= \frac{r_0}{\alpha}= \nonumber \\
&\simeq & 137\cdot 2.82\cdot 10^{-13} \textrm{ cm}= 3.86 \cdot 10^{-11} \textrm{ cm} \nonumber
\end{eqnarray}
\item[ ]Livelli di energia dell'atomo di idrogeno:
\begin{eqnarray}
E_n &=& -\frac{me^4}{2n^2\hbar^2}=-\frac{1}{2n^2}mc^2 \left(\frac{e^2}{\hbar c^2}\right)^2= -\frac{mc^2 \alpha ^2}{2n^2}= \nonumber \\
&\simeq & -\frac{1}{n^2}\frac{0.511 \textrm{ MeV}}{2\cdot (137)^2}=-\frac{1}{n^2}13.6 \textrm{ eV}. \nonumber
\end{eqnarray}
\item[-]Costante di Planck:
\begin{eqnarray}
\hbar =\frac{\hbar c}{e^2}\frac{e^2}{mc^2}\frac{mc^2}{c}=\frac{1}{\alpha}\frac{r_0 mc^2}{c}\simeq 6.58\cdot10^{-22} \textrm{ MeV}\cdot \textrm{s} \nonumber 
\end{eqnarray}
\end{itemize}
\section[Raggio classico dell'elettrone]{Raggio classico dell'elettrone: significato fisico}
Schematizziamo l'elettrone come una sfera di raggio $a$ sulla cui superficie è distribuita una carica $e$.\\

Il campo elettrico generato dalla carica è:
\begin{equation}
E=\begin{cases}
0 \qquad \textrm{per }r<a;\\
\\
\displaystyle{\frac{e}{r^2}} \qquad \textrm{per }r>a.
\end{cases}
\end{equation}
La densità associata al campo è:
\begin{equation}
U=\frac{E^2}{8\pi}=\begin{cases}
0 \qquad \textrm{per }r<a;\\
\\
\displaystyle{\frac{e^2}{8\pi r^4}} \qquad \textrm{per }r>a.
\end{cases}
\end{equation}
L'energia totale è dunque
\begin{equation}
E_{el}=\int U dV = \int _a ^{\infty} \frac{e^2}{8\pi r^4} 4\pi r^2 dr = \frac{e^2}{2} \int _a ^{\infty} \frac{dr}{r^2}=\frac{e^2}{2a}. 
\end{equation}
Uguagliando questa energia all'energia di riposo dell'elettrone, $E=m_e c^2$, ricaviamo il raggio dell'elettrone:
\begin{equation}
a=\frac{1}{2}\frac{e^2}{mc^2}=\frac{1}{2} r_0,
\end{equation}
dove $r_0$ è il raggio classico dell'elettrone. il fattore $\frac{1}{2}$, ottenuto nella precedente espressione, è una conseguenza dei dettagli del modello e segue in particolare dall'aver scelto la carica distribuita solo sulla superficie della sfera (e non, ad esempio, all'interno della sfera stessa). Notiamo anche che l'energia associata ad una carica puntiforme ($a=0$) risulta infinita.
\begin{figure}[!htbp]
\begin{center}
\includegraphics[width= \textwidth]{immagini/cap_21/fig_21_2.png}\\
\end{center}
\end{figure}
\begin{figure}[!htbp]
\begin{center}
\includegraphics[width= \textwidth]{immagini/cap_21/fig_21_3.png}\\
\end{center}
\end{figure}
\begin{figure}[!htbp]
\begin{center}
\includegraphics[width= \textwidth]{immagini/cap_21/fig_21_4.png}\\
\end{center}
\end{figure}
\begin{figure}[!htbp]
\begin{center}
\includegraphics[width= \textwidth]{immagini/cap_21/fig_21_5.png}
\end{center}
\end{figure}
%DEFINITIVO
\input{capitolo_22_(Marco)/capitolo22.tex}%DEFINITIVO
\chapter[Effetto Stark]{Effetto Stark}
Se un atomo viene sottoposto ad un campo elettrico esterno, i suoi livelli energetici cambiano. Questo fenomeno è detto \textbf{effetto Stark}.\\

Supporremo che il campo elettrico sia sufficientemente debole, perché l'energia addizionale ad esso dovuta sia piccola rispetto alle distanze fra i livelli energetici vicini dell'atomo. Allora per calcolare gli spostamenti dei livelli un un campo elettrico, si può ricorrere alla \textbf{teoria delle perturbazioni}.\\

Ci proponiamo di calcolare, facendo uso della teoria delle perturbazioni, le correzioni al primo ordine da apportare ai livelli energetici dell'\textbf{atomo di idrogeno}.\\

Scegliendo la direzione ed il verso dell'asse $z$ parallelo al campo elettrico $\mathcal{E}$ possiamo scrivere l'Hamiltoniano del sistema perturbato come
	\begin{equation}
		\tcboxmath[sharp corners=downhill, colback=white, colframe=black]{
			H=H_0+V, 
			}
	\end{equation}
dove
	\begin{equation}
		\tcboxmath[sharp corners=downhill, colback=white, colframe=black]{
			H_0=\frac{p^2}{2m}-\frac{Ze^2}{r},
			}
	\end{equation}
($Z=1$ per l'atomo di idrogeno) è l'Hamiltoniano imperturbato e
	\begin{equation}
		\tcboxmath[sharp corners=downhill, colback=white, colframe=black]{
			V=+e\mathcal{E}z
			}
	\end{equation}
è la perturbazione introdotta.\\

In assenza della perturbazione lo stato dell'elettrone nell'atomo di idrogeno è, negli stati stazionari, individuato da tre numeri quantici $n$, $l$, $m$. Indichiamo tali stati con $\mathbf{|n,l,m\rangle}$.\\

Consideriamo inizialmente la \textbf{correzione} da apportare \textbf{al livello energetico dello stato fondamentale}. Tale stato non è degenere e possiamo allora scrivere
	\begin{equation}
		\tcboxmath[sharp corners=downhill, colback=white, colframe=black]{
			E^{(1)}_{100}=V_{11}=\langle100| V |100\rangle =+e\mathcal{E}\langle 100 | z |100 \rangle,
			}
	\end{equation}
Questa correzione è nulla. Essa può essere infatti scritta in termini di un integrale della forma
	\begin{equation}
		\tcboxmath[sharp corners=downhill, colback=white, colframe=black]{
			E^{(1)}_{100}=e\mathcal{E} \int d^3r \left|\phi_{100}(\vec{r})\right|^2 z=0,
			}
	\end{equation}
in virtù della simmetria sferica della funzione d'onda nella stato fondamentale. \textbf{In prima approssimazione, allora, il campo elettrico non altera il livello energetico fondamentale}.\\

\textbf{La correzione} al livello energetico dello stato fondamentale \textbf{risulta non nulla al secondo ordine della teoria delle perturbazioni}. Questa correzione è espressa dalla sommatoria
	\begin{equation}
	\label{eq:cap23_1}
		\tcboxmath[sharp corners=downhill, colback=white, colframe=black]{
			E^{(2)}_{100}=e^2 \mathcal{E}^2 \sum_{k\neq(100)}\frac{\left| \langle k^{(0)} | z |100 \rangle \right|^2}{E^{(0)}_{100}-E^{(0)}_{k}},
			}
	\end{equation}
estesa non solo agli stati legati $|n,l,m \rangle$ (con $n>1$) ma anche agli stati del continuo con energia positiva dell'atomo di idrogeno.\\

La sommatoria che compare nell'espressione ($\ref{eq:cap23_1}$) può essere calcolata esattamente e si trova:
	\begin{equation}
		\tcboxmath[sharp corners=downhill, colback=white, colframe=black]{
			E^{(2)}_{100}=-\frac{9}{4} \mathcal{E}^2 \left( \frac{a_0}{z} \right)^3,
			}
	\end{equation}
dove $a_0$ è il raggio di Bohr. (Osserviamo che $\int d^3r \mathcal{E}^2$ è un'energia, cosicché un'analisi dimensionale implica che comunque $E^{(2)}_{100}\sim\mathcal{E}^2\left( a_0/z \right)^3 $.\\

Poiché lo spostamento del livello energetico fondamentale dell'atomo di idrogeno risulta proporzionale al quadrato del campo elettrico esterno, tale effetto viene indicato con il nome di \textbf{effetto Stark quadratico}.\\

Consideriamo ora l'\textbf{effetto del campo elettrico sugli stati corrispondenti al primo livello eccitato dell'atomo di idrogeno ($\boldsymbol{n=2}$)}.\\

In questo caso, come si sa, il livello energetico è \textbf{quattro volte degenere}. I possibili valori dei numeri quantici sono: \\ \\
	\begin{equation}
		\tcboxmath[sharp corners=downhill, colback=white, colframe=black]{
		\begin{matrix} 
			& &  &  &  & & \\
 			 & &  &  &  & m=1 & \cdot  \\
			 &  &  &  & \iddots & &  \\
			 &  &  & l=1 & \cdots & m=0 & \cdot \\
			 &  & \iddots &  & \ddots & & \\
			&  n=2 &  &  &  & m=-1 & \cdot \\
			 &  & \ddots  & &  &  & \\
			 &   &  & l=0 & \cdots & m=0 & \cdot \\
			 & &  &  &  & & 
		 \end{matrix}
		 }
	\end{equation} \\

Gli spostamenti del livello energetico sono allora determinati, in accordo con le formule della teoria delle perturbazioni nel caso degenere, dagli autovalori della matrice della perturbazione $V$ nel sottospazio degli autostati imperturbati degeneri.  Ordiniamo gli elementi di questa matrice secondo il seguente schema:
	\begin{equation}  
		V=
		\bordermatrix{~&200&210&211&21\textrm{-}1\cr
		& \cdot & \cdot & \cdot & \cdot \cr
		& \cdot & \cdot & \cdot & \cdot \cr
		& \cdot & \cdot & \cdot & \cdot \cr
		& \cdot & \cdot & \cdot & \cdot \cr}
		\end{equation}
Osserviamo, innanzitutto, che \textbf{l'operatore $\mathbf{V=e\mathcal{E}z}$ è invariante per rotazioni attorno all'asse $\boldsymbol{z}$}, ossia \textbf{la perturbazione commuta con l'operatore proiezione sull'asse $\boldsymbol{z}$ del momento angolare, $\boldsymbol{L_z=xp_y-yp_x}$}:
	\begin{equation} 
		\tcboxmath[sharp corners=downhill, colback=white, colframe=black]{
			\left[V,L_z\right]=0.
			}
	\end{equation}
Ne segue che \textbf{gli elementi della matrice $\boldsymbol{V}$ tra stati con diverso valore di $\boldsymbol{m}$ sono nulli}. Infatti 
	\begin{align} 
		0 & =  \langle n,l,m| \left[V,L_z\right] |n,l',m'\rangle = \langle n,l,m | \left(VL_z-L_zV\right)|n,l',m'\rangle= \nonumber \\
		 & = \left( m-m' \right) \langle n,l,m |V |n,l',m'\rangle
	\end{align}
da cui
	\begin{equation}
		\tcboxmath[sharp corners=downhill, colback=white, colframe=black]{
			\langle n,l,m | V |n,l',m'\rangle=0 \qquad per \quad m \neq m'.
			}
	\end{equation}\\
	
Nel sottospazio degli stati degeneri corrispondenti ad $n=2$ la matrice $V$ ha allora la forma
	\begin{equation} 
		V=
		\bordermatrix{~&200&210&211&21\textrm{-}1\cr
		& \cdot & \cdot^{*} & 0 & 0 \cr
		& \cdot^{*} & \cdot & 0 & 0 \cr
		& 0 & 0 & \cdot & 0 \cr
		& 0 & 0 & 0 & \cdot \cr} ;
	\end{equation}
gli elementi di matrice indicati con $*$ hanno lo stesso $m$.\\

È semplice dimostrare, inoltre, che \textbf{la perturbazione $\boldsymbol{V}$ anticommuta con l'operatore di parità}. Utilizzando per gli operatori la loro espressione nella rappresentazione delle coordinate, si ha infatti:
	\begin{equation}
		PV\phi(\vec{r})=e\mathcal{E}Pz\phi(\vec{r})=-e\mathcal{E}z\phi(-\vec{r})=-VP\phi(\vec{r}),
	\end{equation}
ossia:
	\begin{equation} 
		\tcboxmath[sharp corners=downhill, colback=white, colframe=black]{
			\{P,V\}=0.
			}
	\end{equation}\\
	
Ne segue che \textbf{gli elementi di matrice della perturbazione tra stati con uguale parità sono nulli}. Considerando infatti due stati con parità $p_1$ e $p_2$ si ha:
	\begin{align}
		0 & =  \langle p_1 | \{ P,V \} |p_2\rangle= \langle p_1 | \left( PV+VP \right) |p_2 \rangle = \nonumber \\
		 & =  (p_1+p_2)\langle p_1|V|p_2\rangle 
	\end{align}
da cui 
	\begin{equation}
	\label{eq:cap23_2}
		\tcboxmath[sharp corners=downhill, colback=white, colframe=black]{
		\langle p_1| V |p_2\rangle=0 \qquad \textrm{per} \quad p_1=p_2.
		}
	\end{equation}\\
	
Discutiamo allora le \textbf{proprietà di simmetria degli autostati $\boldsymbol{|n,l,m \rangle}$ sotto operazione di parità}.\\

Cominciamo con l'osservare che l'operatore di parità $P$ commuta con l'operatore momento angolare orbitale $\vec{L}$:
	\begin{equation}
		\tcboxmath[sharp corners=downhill, colback=white, colframe=black]{
			[ P, \vec{L} ]=0.
			}
	\end{equation}
Infatti $\vec{L}=\vec{r}\wedge \vec{p}$ e sia $\vec{r}$ sia $\vec{p}$ sono dispari per parità. Così:
	\begin{align}
		P\vec{L}\, \phi(\vec{r)} & =  P(\vec{r}\wedge\vec{p})\, \phi(\vec{r})= (-\vec{r}) \wedge (-\vec{p})\,\phi(-\vec{r})= \nonumber \\
		& =  (\vec{r} \wedge \vec{p})\,\phi( -\vec{r})=\vec{L}P\phi(\vec{r}).
	\end{align}\\

Ne segue anche che l'operatore di parità commuta con il quadrato del momento angolare orbitale
	\begin{equation}
		\tcboxmath[sharp corners=downhill, colback=white, colframe=black]{
				\left[ P,L^2\right]=0.
				}
	\end{equation}\\

Allora gli autostati di $L^2$ ed $L_z$ sono anche autostati dell'operatore di parità e si deve avere
	\begin{equation}
		\tcboxmath[sharp corners=downhill, colback=white, colframe=black]{
			P|l,m\rangle=\lambda_{l,m}|l,m\rangle.
			}
	\end{equation}\\

Inoltre, in virtù della commutatività tra l'operatore di parità e gli operatori a scala $L_{\pm}$, gli stati con stesso valore di $l$ e diverso valore di $m$ devono avere stessa parità. Infatti
	\begin{align}
		 PL_-|l,m\rangle &=   cP|l,m-1\rangle=c\lambda_{l,m-1}|l,m-1\rangle=  \nonumber \\
		&=   L_-P|l,m\rangle=L_- \lambda_{l,m}|l,m\rangle=c\lambda_{l,m}|l,m-1\rangle ,
	\end{align}
ossia
	\begin{equation} 
		\tcboxmath[sharp corners=downhill, colback=white, colframe=black]{
			\lambda_{l,m}=\lambda_{l,m-1},
			}
	\end{equation}
e possiamo scrivere allora:
	\begin{equation} 
		\tcboxmath[sharp corners=downhill, colback=white, colframe=black]{
			P|l,m\rangle=\lambda_l|l,m \rangle.
			}
	\end{equation}\\
	
Per determinare la parità $\lambda_l$ osserviamo che una trasformazione di parità in coordinate polari è realizzata dalla trasformazione 

\begin{center}
\begin{minipage}[c]{0.35\textwidth}
\centering
\begin{equation}
\begin{cases} \nonumber
r \to r \\
\theta \to \pi-\theta \\
\varphi \to \varphi-\pi
\end{cases}
\end{equation}
\end{minipage}
\begin{minipage}{0.50\textwidth}
\centering
\tdplotsetmaincoords{60}{110}
%
\pgfmathsetmacro{\rvec}{.8}
\pgfmathsetmacro{\thetavec}{30}
\pgfmathsetmacro{\phivec}{60}
%
\begin{tikzpicture}[scale=5,tdplot_main_coords]
    \coordinate (O) at (0,0,0);
    \draw[thick,->] (0,0,0) -- (1,0,0) node[anchor=north east]{$x$};
    \draw[thick,->] (0,0,0) -- (0,1,0) node[anchor=north west]{$y$};
    \draw[thick,->] (0,0,0) -- (0,0,1) node[anchor=south]{$z$};
    \tdplotsetcoord{P}{\rvec}{\thetavec}{\phivec}
    \draw[-stealth,color=red] (O) -- (P) node[above right] {$\vec{r}$};
    \draw[dashed, color=red] (O) -- (Pxy);
    \draw[dashed, color=red] (P) -- (Pxy);
    \tdplotdrawarc{(O)}{0.2}{0}{\phivec}{anchor=north}{$\varphi$}
    \tdplotsetthetaplanecoords{\phivec}
    \tdplotdrawarc[tdplot_rotated_coords]{(0,0,0)}{0.5}{0}%
        {\thetavec}{anchor=south west}{$\theta$}
\end{tikzpicture}
\end{minipage}
\end{center}

L'effetto di una trasformazione di parità è facilmente calcolabile sugli stati con $m=l$, giacché in questo caso le corrispondenti autofunzioni hanno una forma particolarmente semplice:
	\begin{equation} 
		Y_{l,l}(\theta, \varphi)=A_l \left( \sin \theta \right)^l e^{il\varphi}.
	\end{equation}
Si ha allora:
	\begin{align} 
		PY_{l,l}(\theta, \varphi) & =   A_l \left( \sin (\pi -\theta) \right)^l e^{il(\varphi+\pi)}= \nonumber \\
		 & =   A_l \left( \sin \theta \right)^l e^{il\varphi} e^{i \pi l}= (-1)^l Y_{l,l}(\theta, \varphi),
	\end{align}
da cui, in definitiva:
	\begin{equation} 
		\tcboxmath[enhanced, sharp corners=downhill, colframe=black, colback=white, borderline={2pt}{-3pt}{black}]{
			P|l,m\rangle= (-1)^l |l,m \rangle.
			}
	\end{equation}\\
	
Ricordando che la relazione ($\ref{eq:cap23_2}$), siamo portati a concludere che \textbf{gli elementi di matrice della perturbazione $\boldsymbol{V}$ tra due stati per i quali $\boldsymbol{l}$ è sempre pari o sempre dispari sono nulli}.\\

Siamo dunque giunti, per la matrice $V$, alla seguente espressione:
	\begin{equation}  
		V=
			\bordermatrix{~&200&210&211&21\textrm{-}1\cr
			& 0 & \cdot & 0 & 0 \cr
			& \cdot & 0 & 0 & 0 \cr
			& 0 & 0 & 0 & 0 \cr
			& 0 & 0 & 0 & 0 \cr}
	\end{equation}\\

Poiché la matrice è hermitiana, ci resta solo da calcolare l'elemento 
	\begin{equation}
	\label{eq:cap23_3}
		\tcboxmath[sharp corners=downhill, colback=white, colframe=black]{
			\langle 210| V |200 \rangle =  \langle 200| V |210 \rangle^* = \int d\Omega \ r^2 dr \ \phi_{210}^*(\vec{r}) V \phi_{200}(\vec{r}),
			}
	\end{equation}
dove le funzioni d'onda rilevanti sono date dalle espressioni
	\begin{align} 
		\phi_{210}& = R_{21}(r)Y_{10}(\theta, \varphi)=  \frac{1}{\sqrt{3}}\left( \frac{1}{2a_0} \right)^{3/2} \left( \frac{r}{a_0} \right) e^{-\frac{r}{2a_0}} \sqrt{\frac{3}{4 \pi}} \cos \theta , \\
		\phi_{200}& =  R_{20}(r)Y_{00}(\theta, \varphi)= 2\left( \frac{1}{2a_0} \right)^{3/2} \left( 1-\frac{r}{2a_0} \right) e^{-\frac{r}{2a_0}} \sqrt{\frac{1}{4 \pi}} ,
	\end{align}
(per gli atomi idrogenoidi si deve sostituire $a_0 \to \frac{a_0}{Z}$).\\

Esprimiamo il potenziale $V$ in coordinate sferiche 
\begin{equation}
V=e \mathcal{E}z=e \mathcal{E} r \cos\theta
\end{equation}
e calcoliamo l'integrale ($\ref{eq:cap23_3}$) 
	\begin{align}
		 &  \int r^2 dr\, d\Omega \   e \mathcal{E} r \cos\theta \cdot    \frac{2}{\sqrt{3}} \left( \frac{1}{2a_0} \right)^{3}   \left( \frac{r}{a_0} \right)     \left( 1-\frac{r}{2a_0} \right)  e^{-\frac{r}{a_0}}\sqrt{\frac{3}{4 \pi}} \cos \theta \sqrt{\frac{1}{4 \pi}}=  \nonumber \\ 
		 & =  e \mathcal{E}  \frac{2}{3}  \left( \frac{1}{2a_0} \right)^{3} \int_0^{\infty} dr \, r^3 \left( \frac{r}{a_0} \right)     \left( 1-\frac{r}{2a_0} \right)  e^{-\frac{r}{a_0}}  \int d\Omega \left| Y_{10}(\theta, \varphi) \right|^2=  \nonumber \\
		 & =  e \mathcal{E} \, \frac{a_0}{12} \int_0^{\infty} ds \, s^4 \left(1-\frac{1}{2}s \right) e^{-s} = \frac{1}{12} e \mathcal{E} \, a_0 \left( 4!-\frac{1}{2}5! \right)= \frac{1}{12}  e \mathcal{E} \, a_0 (24-60),
	\end{align}
ossia 
	\begin{equation} 
		\tcboxmath[sharp corners=downhill, colback=white, colframe=black]{
			\langle 210|V|200\rangle= \langle 200 |V |210 \rangle^*=-3e\mathcal{E}a_0. 
			}
	\end{equation}\\
	
\textbf{La matrice della perturbazione $\boldsymbol{V}$ nel sottospazio degli autostati degeneri corrispondenti agli stati con $\boldsymbol{n=2}$ ha dunque la forma}:
	\begin{equation}  
		\tcboxmath[enhanced, sharp corners=downhill, colframe=black, colback=white, borderline={2pt}{-3pt}{black}]{
			V=
			\bordermatrix{~&200&210&211&21\textrm{-}1\cr
			& 0 & -3e\mathcal{E}a_0 & 0 & 0 \cr
			& -3e\mathcal{E}a_0 & 0 & 0 & 0 \cr
			& 0 & 0 & 0 & 0 \cr
			& 0 & 0 & 0 & 0 \cr}
			}
	\end{equation}\\
	
Gli autovalori di questa matrice rappresentano le correzioni, al primo ordine nella perturbazione, ai livelli energetici imperturbati corrispondenti agli stati con $n=2$. Questi \textbf{autovalori} sono:
	\begin{equation} 
		\tcboxmath[enhanced, sharp corners=downhill, colframe=black, colback=white, borderline={2pt}{-3pt}{black}]{
			E^{(1)}=0, \ \pm 3e\mathcal{E}a_0,
			}
	\end{equation}
con l'autovalore nullo avente molteplicità di 2. (Si osservi come la sottomatrice $2\times2$ da diagonalizzare è proporzionale alla matrice di Pauli $\sigma_1$).\\

Gli \textbf{autostati} corrispondenti agli autovalori $E^{(1)}=\pm 3e\mathcal{E}a_0$, nel sottospazio di dimensione  di interesse, sono dati da
	\begin{equation} 
		\frac{1}{\sqrt{2}} 
		\begin{pmatrix}
		1 \\
		-1 \\
		\end{pmatrix}
		\qquad \textrm{e} \qquad 
		\frac{1}{\sqrt{2}}
		\begin{pmatrix}
		1 \\
		1 \\
		\end{pmatrix}
	\end{equation}
e corrispondono dunque alle combinazioni lineari 
\begin{align}
& \tcboxmath[enhanced, sharp corners=downhill, colframe=black, colback=white, borderline={2pt}{-3pt}{black}]{
\frac{1}{\sqrt{2}} \left( \phi_{200}-\phi_{210} \right) 
} \qquad 
\left( E^{(1)}=3e\mathcal{E}a_0 \right) ;\\[0.3cm]
&\tcboxmath[enhanced, sharp corners=downhill, colframe=black, colback=white, borderline={2pt}{-3pt}{black}]{
\frac{1}{\sqrt{2}} \left( \phi_{200}+\phi_{210} \right)
}\qquad
\left( E^{(1)}=-3e\mathcal{E}a_0 \right) .
\end{align}\\

In conclusione, i livelli corrispondenti ad $n=2$ si separano, per effetto del campo elettrico, come indicato nello schema sottostante
\begin{center}
\includegraphics[width=10cm]{immagini/cap_23/fig23_1.png}
\end{center}
Poiché lo spostamento dei livelli, in prima approssimazione, è proporzionale al campo elettrico esterno $\mathcal{E}$, si parla in questo caso di $\textbf{effetto Stark lineare}$.\\

Osserviamo che, \textbf{in presenza del campo elettrico, gli autostati dell'Hamiltoniano non sono più autostati di $\boldsymbol{L^2}$}. Ad esempio, nel sottospazio degli stati con $n=2$, abbiamo ottenuto combinazioni lineari di stati corrispondenti ad $l=0$ ed $l=1$. La ragione è che, in presenza del campo elettrico esterno, \textbf{il sistema non è più invariante per rotazioni arbitrarie, e l'Hamiltoniano non commuta più con l'operatore momento angolare $\boldsymbol{L^2}$}.\\
Tuttavia, \textbf{il sistema è ancora invariante per rotazione attorno all'asse $\boldsymbol{z}$, che definisce la direzione del campo esterno. L'Hamiltoniano perturbato commuta con la proiezione $\boldsymbol{L_z}$ del momento angolare orbitale e gli autostati di $\boldsymbol{H}$ sono simultaneamente autostati di $\boldsymbol{L_z}$}.
 %DEFINITIVO
\input{Capitolo_24_(Fuego&Irene)/Capitolo24.tex} %DEFINITIVO
\chapter[Correzioni relativistiche all'atomo di idrogeno]{Correzioni relativistiche\\ all'hamiltoniano\\ dell'atomo di idrogeno\footnote{G17, S5.3}}

La trattazione fatta dell'atomo di idrogeno era basata sull'hamiltoniana
\begin{equation} \label{eq:cap25_1}
H_0=\frac{\vec{p^2}}{2m}-\frac{\mathcal{Z} e^2}{r}
\end{equation}
($\mathcal{Z}=1$ per l'atomo di idrogeno). In una trattazione più realistica è necessario prendere in considerazione diverse \textbf{correzioni}.

\section{Termine cinetico}

L'espressione dell'\textbf{energia cinetica} dell'elettrone si modifica quando si considerano \textbf{correzioni relativistiche }. Nella meccanica relativistica l'energia cinetica è data da:

\begin{eqnarray} 
E & = &\sqrt{m^2c^4+c^2\vec{p^2}}= mc^2 \sqrt{1+ \frac{\vec{p^2}}{m^2c^2}} \simeq \nonumber  \\
& \simeq & mc^2 \left( 1+ \frac{\vec{p^2}}{2m^2c^2}-\frac{(\vec{p^2})^2}{8m^4c^4}+\dots \right) = \nonumber  \\
& = & mc^2+\frac{\vec{p^2}}{2m}-\frac{(\vec{p^2})^2}{8m^3c^2}+ \dots 
\end{eqnarray}
Il termine della massa a riposo rappresenta una costante additiva irrilevante nella definizione dell'energia. La prima correzione relativistica è allora data dal termine 
\begin{equation} \label{eq:cap25_2}
H_1=-\frac{(\vec{p^2})^2}{8m^3c^2} ,
\end{equation}
che deve essere aggiunto all'hamiltoniana $H_0$. \\
Una stima dell'effetto di questa correzione sui livelli di energia dell'atomo può essere ottenuto utilizzando il principio di indeterminazione ed assumendo come valore approssimato del raggio dell'orbita \textbf{elettronica il valore $\boldsymbol{a_0\mathcal{Z}}$}. Si ottiene in tal modo
\begin{equation} 
\frac{\left< H_1 \right>}{\langle H_0 \rangle} \approx \frac{\vec{p^2}}{m^2c^2} \simeq \frac{\hbar^2 \mathcal{Z}^2}{m^2c^2a_0^2}=\frac{\mathcal{Z}^2 \hbar^2}{m^2c^2} \left( \frac{me^2}{\hbar^2}\right)^2=\frac{\mathcal{Z}^2 e^4}{\hbar^2 c^2} ,
\end{equation}
ossia
\begin{equation} 
\frac{\left< H_1 \right>}{\langle H_0 \rangle} \approx \left( \frac{v_e}{c} \right)^2 \approx \left(  \mathcal{Z} \alpha \right)^2 ,
\end{equation}
dove
\begin{equation} 
\alpha=\frac{e^2}{\hbar c} \simeq \frac{1}{137}
\end{equation}
è la cosiddetta costante di struttura fine. \\ Così, per l'atomo di idrogeno, questa correzione relativistica è dell'ordine di grandezza relativo di $\sim10^{-4}$.
\section{Accoppiamento spin-orbita}

L'\textbf{esistenza dello spin} dell'elettrone implica un'altra \textbf{correzione dello stesso ordine di grandezza}. Questa può essere qualitativamente compresa col seguente ragionamento: se l'elettrone fosse in quiete rispetto al protone, risentirebbe solo di un campo elettrico generato dalla carica del protone. Questo è il \textbf{termine del potenziale di Coulomb che appare in $\boldsymbol{H_0}$}. Poiché l'elettrone è però in movimento vi sono effetti addizionali. \\
\textbf{Nel sistema di riferimento dell'elettrone, il protone è in moto così che è presente una corrente e l'elettrone risente di un campo magnetico. Questo campo magnetico interagisce con lo spin dell'elettrone, o, più precisamente, con il momento magnetico dell'elettrone}. \\
Se il moto relativo del protone rispetto all'elettrone fosse rettilineo, il campo magnetico visto dall'elettrone sarebbe dato da:
\begin{equation} 
\vec{B}^{'}=-\gamma \frac{\vec{v}}{c} \wedge \vec{E} \ \simeq -\frac{\vec{v}}{c} \wedge \vec{E} ,
\end{equation}
dove $\vec{E}$ è il campo elettrico nel sistema di quiete del protone. \\
Poiché l'elettrone ha un momento magnetico intrinseco proporzionale al suo spin, dalla forma
\begin{equation} 
\vec{\mu}=-\frac{e}{mc}\vec{S} 
\end{equation}
ci aspettiamo che l'interazione con il campo magnetico effettivo risulti data da
\begin{eqnarray}
H_2 & =& -\vec{\mu} \cdot \vec{B}^{'}=\frac{e}{mc} \vec{S} \cdot \vec{B}^{'}= -\frac{e}{mc^2}\vec{S} \cdot \left( \vec{v} \wedge \vec{E} \right)= \nonumber \\
& =& + \frac{e}{m^2c^2}\vec{S} \cdot \vec{p} \wedge \vec{\nabla}\phi= \frac{e}{m^2c^2}\vec{S} \cdot \vec{p} \wedge \frac{\vec{r}}{r} \ \frac{d\phi}{dr}= \nonumber \\
& = &-\frac{e}{m^2c^2}\frac{1}{r} \ \frac{d\phi}{dr}\vec{S} \cdot \vec{L}= \frac{1}{m^2c^2} \left( \frac{1}{r} \ \frac{dV}{dr} \right) \vec{S} \cdot \vec{L} ,
\end{eqnarray}

dove $V=-e\phi=-\mathcal{Z}e^2/r$ è il potenziale cui è soggetto l'elettrone. \\
In realtà il moto dell'elettrone non è rettilineo uniforme, ed il risultato ottenuto risulta troppo grande di un fattore $2$ (\textbf{questo effetto è noto come precessione di Thomas}. Il termine di interazione corretto ha dunque la forma
\begin{equation} \label{eq:cap25_3}
H_2=\frac{1}{2m^2c^2} \left( \frac{1}{r} \ \frac{dV}{dr} \right) \vec{S} \cdot \vec{L} ,
\end{equation}
con
\begin{equation} 
V=-\frac{\mathcal{Z}e^2}{r} .
\end{equation}
Anche in questo caso è semplice ottenere una \textbf{stima della correzione} indotta dal termine aggiuntivo nell'hamiltoniana:
\begin{eqnarray}
\frac{\left< H_2 \right>}{\langle H_0 \rangle} & \approx & \frac{1}{m^2c^2} \left( \frac{1}{r} \ \frac{\mathcal{Z}e^2}{r^2} \right) \vec{S} \cdot \vec{L} \cdot \left( \frac{\mathcal{Z}e^2}{r} \right)^{-1} \approx \nonumber  \\
& \approx &\frac{\vec{S} \cdot \vec{L}}{m^2c^2r^2} \approx  \frac{\hbar^2 \mathcal{Z}^2}{m^2c^2a_0^2}=  \frac{\mathcal{Z}^2 \hbar^2}{m^2c^2}\left( \frac{me^2}{\hbar^2} \right)^2= \frac{\mathcal{Z}^2e^4}{\hbar^2c^2} ,
\end{eqnarray}
ossia
\begin{equation} 
\frac{\left< H_2 \right>}{\langle H_0 \rangle} \approx \left( \mathcal{Z}\alpha \right)^2 .
\end{equation}
L'interazione spin-orbita induce pertanto una correzione relativa ai livelli energetici dell'atomo che è dello stesso ordine di grandezza della correzione relativistica dovuta al termine cinematico ($\sim10^{-4}$ per $\mathcal{Z}=1$).
\section{Calcolo perturbativo delle correzioni}

L'effetto stimato delle correzioni ai livelli energetici degli atomo idrogenoidi, indotto dalla correzione relativistica all'energia cinetica e dell'accoppiamento spin-orbita, è sufficientemente piccolo da poter essere trattato, con ottima approssimazione, nella \textbf{teoria delle perturbazioni}. \\
Consideriamo come hamiltoniana imperturbata l'hamiltoniana $H_0$ dell'equazione (\ref{eq:cap25_1}) e come perturbazione $V$ la somma delle hamiltoniane $H_1$ e $H_2$ definite in equazione (\ref{eq:cap25_2}) e (\ref{eq:cap25_3}):
\begin{equation} 
V=H_1+H_2=-\frac{p^4}{8m^3c^2}+\frac{1}{2m^2c^2} \left( \frac{1}{r} \ \frac{dV}{dr} \right) \vec{S} \cdot \vec{L} .
\end{equation}
Proponiamoci qui di calcolare le \textbf{correzioni ai livelli energetici imperturbati, al primo ordine nella perturbazione $\boldsymbol{V}$}. \\
I livelli di energia dell'hamiltoniana imperturbata $H_0$, corrispondenti ad un determinato valore del numero quantico principale $n$, hanno un grado di degenerazione pari a $2n^2$. Gli stati degeneri differiscono per i diversi valori dei numeri quantici $l,m_l$ ed $m_s$ definiti dagli autovalori di $L^2$, $L_z$ ed $S_z$. Per tutti gli stati con $n>1$ risulta quindi necessario applicare la \textbf{teoria delle perturbazioni nel caso degenere}. \\
Il calcolo risulta estremamente semplificato se, per quanto concerne la dipendenza delle variabili angolari e di spin delle funzioni d'onda imperturbate, si considerano gli \textbf{autostati di}
\begin{equation} \label{eq:cap25_4}
J^2, \ J_x, \ L^2, \ S^2 ,
\end{equation}
in luogo degli autostati di $L^2, L_z, S^2, S_z$. \\
Le perturbazioni $H_1$ ed $H_2$ possono infatti essere scritte convenientemente nella forma
\begin{equation} \label{eq:cap25_5}
H_1=-\frac{1}{2mc^2} \left( \frac{\vec{p^2}}{2m} \right)^2 = -\frac{1}{2mc^2} \left( H_0+\frac{\mathcal{Z}e^2}{r} \right)^2 ,
\end{equation}
e
\begin{equation} \label{eq:cap25_6}
H_2=\frac{1}{4m^2c^2} \left( \frac{1}{r} \ \frac{dV}{dr} \right) \left(\vec{J}^2-\vec{L}^2-\vec{S}^2 \right) .
\end{equation}
Da queste espressioni risulta infatti evidente che \textbf{gli operatori $\boldsymbol{H_1}$ ed $\boldsymbol{H_2}$ commutano con gli operatori in equazione (\ref{eq:cap25_4}), cosicché, nella corrispondente base, la perturbazione risulta già diagonale}. \\ (\'E bene sottolineare, tuttavia, che le perturbazioni $H_1$ ed $H_2$ non commutano, ovviamente, con l'hamiltoniano imperturbato $H_0$. Pertanto $H_1$ ed $H_2$ risultano diagonali solo nel sottospazio degli autostati degeneri di $H_0$ corrispondenti a diversi valori di $J^2$, $J_z$ ed $L^2$). \\
\textbf{Le correzioni al primo ordine ai livelli energetici dell'atomo si ottengono allora direttamente calcolando i valori di aspettazione della perturbazione $\boldsymbol{V}$ sugli autostati di $\boldsymbol{H_0}$, $\boldsymbol{J^2}$, $\boldsymbol{J_z}$, $\boldsymbol{L^2}$, $\boldsymbol{S^2}$}. \textbf{Indichiamo questi autostati con}
\begin{equation} 
| n,l,j,m_j \rangle .
\end{equation}
Per un determinato valore del numero quantico orbitale $l$, il numero quantico $j$ può assumere i valori
\begin{equation} 
j=l\pm1/2 .
\end{equation}
Le corrispondenti autofunzioni sono della forma
\begin{equation} 
\psi_{n,l,j,m_j}=R_{nl}(r)Y_{j=l\pm1/2,m_j} .
\end{equation}
Cominciamo con il calcolare, su questi stati, \textbf{i valori medi della perturbazione $\boldsymbol{H_1}$} indotta dalle correzioni relativistiche all'energia cinetica. Utilizziamo l'equazione ($\ref{eq:cap25_5}$) troviamo:
\begin{eqnarray} \label{eq:cap25_7} 
& &\langle n,l,j,m_j | H_1 |n,l,j,m_j \rangle = \nonumber \\
& &=-\frac{1}{2mc^2}\langle n,l,j,m_j |\left( H_0+\frac{\mathcal{Z}e^2}{r} \right) \left( H_0+\frac{\mathcal{Z}e^2}{r} \right)  |n,l,j,m_j \rangle = \nonumber \\
& &=-\frac{1}{2mc^2}\langle n,l,j,m_j | \left( E_n+\frac{\mathcal{Z}e^2}{r} \right)^2  | n,l,j,m_j\rangle = \nonumber \\
& &=-\frac{1}{2mc^2} \left( E_n^2+2E_n\mathcal{Z}e^2 \left< \frac{1}{r} \right>_{nl} +\left( \mathcal{Z}e^2 \right)^2\left< \frac{1}{r^2} \right>_{nl} \right)  ,
\end{eqnarray}
dove si sono definiti i valori medi
\begin{eqnarray} 
\left< \frac{1}{r^k} \right>_{nl} & = \langle n,l,j,m_j | \frac{1}{r^k} | n,l,j,m_j \rangle = \nonumber\\
& = \int_0^{\infty} dr \ r^2 \ \frac{1}{r^k} \left( R_{nl}(r) \right)^2 .
\end{eqnarray}
Per questi valori medi sono note delle formule generali, che qui riportiamo:
\begin{equation} \label{eq:cap25_8}
\begin{split}
\left< \frac{1}{r} \right>_{nl} & = \left( \frac{\mathcal{Z}}{a_0} \right) \frac{1}{n^2} , \\ 
\left< \frac{1}{r^2} \right>_{nl} & = \left( \frac{\mathcal{Z}}{a_0} \right)^2 \frac{1}{n^3(l+1/2)}  ,\\ 
\left< \frac{1}{r^3} \right>_{nl} &= \left( \frac{\mathcal{Z}}{a_0} \right)^3 \frac{1}{n^3 \ l \ (l+1/2)(l+1)} \qquad (l \neq0) .
\end{split}
\end{equation}
(Il valor medio $\left<1/r^3\right>_{nl}$ risulterà utile nel calcolo della correzione indotta dall'accoppiamento spin-orbita). \\
Sostituendo i valori medi (\ref{eq:cap25_8}) nell'equazione (\ref{eq:cap25_7}) ed utilizzando le espressioni
\begin{equation} 
E_n=-\frac{mc^2 \left(\mathcal{Z}\alpha\right)^2}{2n^2} 
\end{equation}
per l'energia dei livelli imperturbati e 
\begin{equation} 
a_0=\frac{\hbar^2}{me^2}=\frac{\hbar^2c^2}{e^4}\frac{e^2}{mc^2}=\frac{e^2}{mc^2\alpha^2}
\end{equation}
per il raggio di Bohr, otteniamo:

\begin{eqnarray} 
 \left< H_1 \right>_{nljm_j} & =& -\frac{1}{2mc^2} \left[ \frac{m^2c^4 \left( \mathcal{Z} \alpha \right)^4}{4n^4}-\frac{mc^2 \left( \mathcal{Z} \alpha \right)^2 \mathcal{Z} e^2}{n^4} \frac{mc^2\alpha^2 \mathcal{Z}}{e^2}+ \right. \nonumber \\
& &\left. + \left( \mathcal{Z} e^2 \right)^2 \frac{1}{n^3(l+1/2)} \left(\frac{mc^2\alpha^2}{e^2} \right)^2 \mathcal{Z}^2 \right] = \nonumber  \\
& = & -\frac{1}{2mc^2} \left( m^2c^4 \right) \left( \mathcal{Z} \alpha\right)^4 \left[ \frac{1}{4n^4}-\frac{1}{n^4}+\frac{1}{n^3(l+1/2)}  \right] ,
\end{eqnarray}
ossia
\begin{equation} \label{eq:cap25_9}
\left< H_1 \right>_{nljm_j} = -\frac{1}{2}mc^2 \left( \mathcal{Z} \alpha \right)^4 \left[ \frac{1}{n^3(l+1/2)}-\frac{3}{4n^4}\right]  .
\end{equation}
Questa correzione risulta dall'ordine di grandezza aspettato: più piccola di circa un fattore $\left( \mathcal{Z} \alpha \right)^2$ rispetto ai livelli di energia imperturbati. \\
Calcoliamo ora la \textbf{correzione}, al primo ordine dello sviluppo perturbativo, \textbf{indotto} sui livelli di energia imperturbati \textbf{dall'interazione spin-orbita}. \\ 
Utilizzando l'equazione (\ref{eq:cap25_6}) otteniamo:
\begin{eqnarray} \label{eq:cap25_10}
\left< H_2 \right>_{nljm_j} &=& \langle n,l,j,m_j | H_2 | n,l,j,m_j \rangle = \nonumber  \\
& = &\frac{1}{4m^2c^2} \langle n,l,j,m_j | \left( \frac{\mathcal{Z}e^2}{r^3} \right) \left( \vec{J}^2-\vec{L}^2-\vec{S}^2 \right)      | n,l,j,m_j \rangle= \nonumber  \\
& = & \frac{\hbar^2\mathcal{Z}e^2}{4m^2c^2} \left< \frac{1}{r^3} \right>_{nl} \left[ j(j+1)-l(l+1)-\frac{3}{4}  \right] .
\end{eqnarray}
Il fattore tra parentesi quadre che entra in questa espressione, nei due casi $j=l\pm1/2$, vale 
\begin{eqnarray} 
& &\left. \left[ j(j+1)-l(l+1)-\frac{3}{4}  \right]  \right|_{j=l+1/2} = \nonumber \\ 
& &\qquad = \left( l+\frac{1}{2} \right) \left( l+\frac{3}{2}\right)-l\left(l+1 \right) -\frac{3}{4}=l
\end{eqnarray}
e
\begin{eqnarray} 
& &\left. \left[ j(j+1)-l(l+1)-\frac{3}{4}  \right] \right|_{j=l-1/2} = \nonumber  \\ 
& &\qquad = \left( l-\frac{1}{2} \right) \left(l+\frac{1}{2} \right)-l\left( l+1 \right) -\frac{3}{4}=-l-1 .
\end{eqnarray}
Sostituendo questi risultati nell'equazione (\ref{eq:cap25_10}), insieme al valore medio $\left< 1/r^3 \right>_{nl}$ dato nell'equazione (\ref{eq:cap25_8}), si ottiene:
\begin{equation} \label{eq:cap25_11}
\left< H_2 \right>_{nljm_j}=\frac{1}{4}mc^2\left( \mathcal{Z}\alpha \right)^4 \frac{\begin{Bmatrix}
      l  \\ 
      -l-1 
    \end{Bmatrix}
    \begin{matrix}
    \leftarrow j=l+1/2 \\
    \leftarrow j=l-1/2
    \end{matrix} }{n^3 \ l \ \left(l+1/2 \right) \left( l+1 \right)} ,
\end{equation}
valida per $l\neq0$. La correzione è nulla per $l=0$. \\
Le due correzioni, fornite dalle equazioni (\ref{eq:cap25_9}) e (\ref{eq:cap25_11}), possono essere infine combinate per ottenere la \textbf{correzione totale}, al primo ordine nella perturbazione, ai livelli di energia degli atomi idrogenoidi. Nei due casi, corrispondenti a $j=l\pm1/2$ si ottiene:

\begin{eqnarray} 
\left( \Delta E \right)_{j=l+1/2} &= & \left( \left< H_1\right> +\left<H_2 \right>  \right)_{j=l+1/2}= \nonumber \\
& =& -\frac{1}{2} mc^2 \left( \mathcal{Z}\alpha \right)^4 \frac{1}{n^3} \left[ \frac{1}{l+1/2}-\frac{3}{4n}-\frac{1}{2(l+1/2)(l+1)}\right]=  \nonumber \\
& = & -\frac{1}{2} mc^2 \left( \mathcal{Z}\alpha \right)^4 \frac{1}{n^3} \left[ \frac{l+1-1/2}{(l+1/2)(l+1)}-\frac{3}{4n}\right]= \nonumber  \\
& =v& -\frac{1}{2} mc^2 \left( \mathcal{Z}\alpha \right)^4 \frac{1}{n^3} \left[ \frac{1}{l+1}-\frac{3}{4n} \right]= \nonumber \\
& = &-\frac{1}{2} mc^2 \left( \mathcal{Z}\alpha \right)^4 \frac{1}{n^3} \left[ \frac{1}{j+1/2}-\frac{3}{4n} \right]
\end{eqnarray}
e
\begin{eqnarray} 
 \left( \Delta E \right)_{j=l-1/2}&=&\left( \left< H_1\right> +\left<H_2 \right>  \right)_{j=l-1/2}= \nonumber \\
& = &-\frac{1}{2} mc^2 \left( \mathcal{Z}\alpha \right)^4 \frac{1}{n^3} \left[ \frac{1}{l+1/2}-\frac{3}{4n}-\frac{1}{2l(l+1/2)}\right]= \nonumber \\
& =& -\frac{1}{2} mc^2 \left( \mathcal{Z}\alpha \right)^4 \frac{1}{n^3} \left[ \frac{l+1/2}{l(l+1/2)}-\frac{3}{4n}\right]= \nonumber \\
& = &-\frac{1}{2} mc^2 \left( \mathcal{Z}\alpha \right)^4 \frac{1}{n^3} \left[ \frac{1}{l}-\frac{3}{4n} \right]= \nonumber  \\
& = & -\frac{1}{2} mc^2 \left( \mathcal{Z}\alpha \right)^4 \frac{1}{n^3} \left[ \frac{1}{j+1/2}-\frac{3}{4n} \right] .
\end{eqnarray}
In conclusione, possiamo scrivere:
\begin{eqnarray}
\Delta E &=& \langle H_1+H_2 \rangle_{nljm_j}= \nonumber\\
& = &-\frac{1}{2}mc^2 \left( \mathcal{Z}\alpha \right)^4 \frac{1}{n^3} \left[ \frac{1}{j+1/2}-\frac{3}{4n}\right] ,
\end{eqnarray}
valida per entrambi i valori $j=l\pm1/2$. \\
Utilizzando l'equazione relativistica di Dirac è possibile mostrare che il risultato ottenuto è corretto anche nel caso \textbf{$\boldsymbol{l=0}$}. \\
Lo splitting dei livelli è rappresentato, per il caso $n=2$, nello schema sottostante\\
\begin{center}
\includegraphics[width=10cm]{immagini/cap_25/fig25_1.png}
\end{center}
Gli stati con $l=1$ (stati $p$) possono avere $j=1/2$ e $j=3/2$ mentre gli stati con $l=0$ (stati $s$) corrispondono necessariamente a $j=1/2$. \'E interessante osservare le correzioni dovute allo spin-orbita e al termine cinetico si sommano in modo tale da rendere degeneri gli stati $^2S_{1/2}$ e $^2P_{1/2}$.
Una trattazione più accurata, basata sull'equazione relativistica di Dirac, non altera questo risultato. Tuttavia, nel $1947$, un accurato esperimento condotto da Lamb e Rutherford ha mostrato una sottile separazione tra i due livelli $^2S_{1/2}$ e $^2P_{1/2}$ (\textbf{Lamb-shift}). Questo effetto è spiegabile soltanto nel contesto della completa teoria quantistica relativistica ed è originato dalle fluttuazioni quantistiche del campo dell'elettrone. %DEFINITIVO
\end{document}