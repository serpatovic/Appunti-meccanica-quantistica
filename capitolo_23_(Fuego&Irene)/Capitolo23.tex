\chapter[Effetto Stark]{Effetto Stark}
Se un atomo viene sottoposto ad un campo elettrico esterno, i suoi livelli energetici cambiano. Questo fenomeno è detto \textbf{effetto Stark}.\\

Supporremo che il campo elettrico sia sufficientemente debole, perché l'energia addizionale ad esso dovuta sia piccola rispetto alle distanze fra i livelli energetici vicini dell'atomo. Allora per calcolare gli spostamenti dei livelli un un campo elettrico, si può ricorrere alla \textbf{teoria delle perturbazioni}.\\

Ci proponiamo di calcolare, facendo uso della teoria delle perturbazioni, le correzioni al primo ordine da apportare ai livelli energetici dell'\textbf{atomo di idrogeno}.\\

Scegliendo la direzione ed il verso dell'asse $z$ parallelo al campo elettrico $\mathcal{E}$ possiamo scrivere l'Hamiltoniano del sistema perturbato come
	\begin{equation}
		\tcboxmath[sharp corners=downhill, colback=white, colframe=black]{
			H=H_0+V, 
			}
	\end{equation}
dove
	\begin{equation}
		\tcboxmath[sharp corners=downhill, colback=white, colframe=black]{
			H_0=\frac{p^2}{2m}-\frac{Ze^2}{r},
			}
	\end{equation}
($Z=1$ per l'atomo di idrogeno) è l'Hamiltoniano imperturbato e
	\begin{equation}
		\tcboxmath[sharp corners=downhill, colback=white, colframe=black]{
			V=+e\mathcal{E}z
			}
	\end{equation}
è la perturbazione introdotta.\\

In assenza della perturbazione lo stato dell'elettrone nell'atomo di idrogeno è, negli stati stazionari, individuato da tre numeri quantici $n$, $l$, $m$. Indichiamo tali stati con $\mathbf{|n,l,m\rangle}$.\\

Consideriamo inizialmente la \textbf{correzione} da apportare \textbf{al livello energetico dello stato fondamentale}. Tale stato non è degenere e possiamo allora scrivere
	\begin{equation}
		\tcboxmath[sharp corners=downhill, colback=white, colframe=black]{
			E^{(1)}_{100}=V_{11}=\langle100| V |100\rangle =+e\mathcal{E}\langle 100 | z |100 \rangle,
			}
	\end{equation}
Questa correzione è nulla. Essa può essere infatti scritta in termini di un integrale della forma
	\begin{equation}
		\tcboxmath[sharp corners=downhill, colback=white, colframe=black]{
			E^{(1)}_{100}=e\mathcal{E} \int d^3r \left|\phi_{100}(\vec{r})\right|^2 z=0,
			}
	\end{equation}
in virtù della simmetria sferica della funzione d'onda nella stato fondamentale. \textbf{In prima approssimazione, allora, il campo elettrico non altera il livello energetico fondamentale}.\\

\textbf{La correzione} al livello energetico dello stato fondamentale \textbf{risulta non nulla al secondo ordine della teoria delle perturbazioni}. Questa correzione è espressa dalla sommatoria
	\begin{equation}
	\label{eq:cap23_1}
		\tcboxmath[sharp corners=downhill, colback=white, colframe=black]{
			E^{(2)}_{100}=e^2 \mathcal{E}^2 \sum_{k\neq(100)}\frac{\left| \langle k^{(0)} | z |100 \rangle \right|^2}{E^{(0)}_{100}-E^{(0)}_{k}},
			}
	\end{equation}
estesa non solo agli stati legati $|n,l,m \rangle$ (con $n>1$) ma anche agli stati del continuo con energia positiva dell'atomo di idrogeno.\\

La sommatoria che compare nell'espressione ($\ref{eq:cap23_1}$) può essere calcolata esattamente e si trova:
	\begin{equation}
		\tcboxmath[sharp corners=downhill, colback=white, colframe=black]{
			E^{(2)}_{100}=-\frac{9}{4} \mathcal{E}^2 \left( \frac{a_0}{z} \right)^3,
			}
	\end{equation}
dove $a_0$ è il raggio di Bohr. (Osserviamo che $\int d^3r \mathcal{E}^2$ è un'energia, cosicché un'analisi dimensionale implica che comunque $E^{(2)}_{100}\sim\mathcal{E}^2\left( a_0/z \right)^3 $.\\

Poiché lo spostamento del livello energetico fondamentale dell'atomo di idrogeno risulta proporzionale al quadrato del campo elettrico esterno, tale effetto viene indicato con il nome di \textbf{effetto Stark quadratico}.\\

Consideriamo ora l'\textbf{effetto del campo elettrico sugli stati corrispondenti al primo livello eccitato dell'atomo di idrogeno ($\boldsymbol{n=2}$)}.\\

In questo caso, come si sa, il livello energetico è \textbf{quattro volte degenere}. I possibili valori dei numeri quantici sono: \\ \\
	\begin{equation}
		\tcboxmath[sharp corners=downhill, colback=white, colframe=black]{
		\begin{matrix} 
			& &  &  &  & & \\
 			 & &  &  &  & m=1 & \cdot  \\
			 &  &  &  & \iddots & &  \\
			 &  &  & l=1 & \cdots & m=0 & \cdot \\
			 &  & \iddots &  & \ddots & & \\
			&  n=2 &  &  &  & m=-1 & \cdot \\
			 &  & \ddots  & &  &  & \\
			 &   &  & l=0 & \cdots & m=0 & \cdot \\
			 & &  &  &  & & 
		 \end{matrix}
		 }
	\end{equation} \\

Gli spostamenti del livello energetico sono allora determinati, in accordo con le formule della teoria delle perturbazioni nel caso degenere, dagli autovalori della matrice della perturbazione $V$ nel sottospazio degli autostati imperturbati degeneri.  Ordiniamo gli elementi di questa matrice secondo il seguente schema:
	\begin{equation}  
		V=
		\bordermatrix{~&200&210&211&21\textrm{-}1\cr
		& \cdot & \cdot & \cdot & \cdot \cr
		& \cdot & \cdot & \cdot & \cdot \cr
		& \cdot & \cdot & \cdot & \cdot \cr
		& \cdot & \cdot & \cdot & \cdot \cr}
		\end{equation}
Osserviamo, innanzitutto, che \textbf{l'operatore $\mathbf{V=e\mathcal{E}z}$ è invariante per rotazioni attorno all'asse $\boldsymbol{z}$}, ossia \textbf{la perturbazione commuta con l'operatore proiezione sull'asse $\boldsymbol{z}$ del momento angolare, $\boldsymbol{L_z=xp_y-yp_x}$}:
	\begin{equation} 
		\tcboxmath[sharp corners=downhill, colback=white, colframe=black]{
			\left[V,L_z\right]=0.
			}
	\end{equation}
Ne segue che \textbf{gli elementi della matrice $\boldsymbol{V}$ tra stati con diverso valore di $\boldsymbol{m}$ sono nulli}. Infatti 
	\begin{align} 
		0 & =  \langle n,l,m| \left[V,L_z\right] |n,l',m'\rangle = \langle n,l,m | \left(VL_z-L_zV\right)|n,l',m'\rangle= \nonumber \\
		 & = \left( m-m' \right) \langle n,l,m |V |n,l',m'\rangle
	\end{align}
da cui
	\begin{equation}
		\tcboxmath[sharp corners=downhill, colback=white, colframe=black]{
			\langle n,l,m | V |n,l',m'\rangle=0 \qquad per \quad m \neq m'.
			}
	\end{equation}\\
	
Nel sottospazio degli stati degeneri corrispondenti ad $n=2$ la matrice $V$ ha allora la forma
	\begin{equation} 
		V=
		\bordermatrix{~&200&210&211&21\textrm{-}1\cr
		& \cdot & \cdot^{*} & 0 & 0 \cr
		& \cdot^{*} & \cdot & 0 & 0 \cr
		& 0 & 0 & \cdot & 0 \cr
		& 0 & 0 & 0 & \cdot \cr} ;
	\end{equation}
gli elementi di matrice indicati con $*$ hanno lo stesso $m$.\\

È semplice dimostrare, inoltre, che \textbf{la perturbazione $\boldsymbol{V}$ anticommuta con l'operatore di parità}. Utilizzando per gli operatori la loro espressione nella rappresentazione delle coordinate, si ha infatti:
	\begin{equation}
		PV\phi(\vec{r})=e\mathcal{E}Pz\phi(\vec{r})=-e\mathcal{E}z\phi(-\vec{r})=-VP\phi(\vec{r}),
	\end{equation}
ossia:
	\begin{equation} 
		\tcboxmath[sharp corners=downhill, colback=white, colframe=black]{
			\{P,V\}=0.
			}
	\end{equation}\\
	
Ne segue che \textbf{gli elementi di matrice della perturbazione tra stati con uguale parità sono nulli}. Considerando infatti due stati con parità $p_1$ e $p_2$ si ha:
	\begin{align}
		0 & =  \langle p_1 | \{ P,V \} |p_2\rangle= \langle p_1 | \left( PV+VP \right) |p_2 \rangle = \nonumber \\
		 & =  (p_1+p_2)\langle p_1|V|p_2\rangle 
	\end{align}
da cui 
	\begin{equation}
	\label{eq:cap23_2}
		\tcboxmath[sharp corners=downhill, colback=white, colframe=black]{
		\langle p_1| V |p_2\rangle=0 \qquad \textrm{per} \quad p_1=p_2.
		}
	\end{equation}\\
	
Discutiamo allora le \textbf{proprietà di simmetria degli autostati $\boldsymbol{|n,l,m \rangle}$ sotto operazione di parità}.\\

Cominciamo con l'osservare che l'operatore di parità $P$ commuta con l'operatore momento angolare orbitale $\vec{L}$:
	\begin{equation}
		\tcboxmath[sharp corners=downhill, colback=white, colframe=black]{
			[ P, \vec{L} ]=0.
			}
	\end{equation}
Infatti $\vec{L}=\vec{r}\wedge \vec{p}$ e sia $\vec{r}$ sia $\vec{p}$ sono dispari per parità. Così:
	\begin{align}
		P\vec{L}\, \phi(\vec{r)} & =  P(\vec{r}\wedge\vec{p})\, \phi(\vec{r})= (-\vec{r}) \wedge (-\vec{p})\,\phi(-\vec{r})= \nonumber \\
		& =  (\vec{r} \wedge \vec{p})\,\phi( -\vec{r})=\vec{L}P\phi(\vec{r}).
	\end{align}\\

Ne segue anche che l'operatore di parità commuta con il quadrato del momento angolare orbitale
	\begin{equation}
		\tcboxmath[sharp corners=downhill, colback=white, colframe=black]{
				\left[ P,L^2\right]=0.
				}
	\end{equation}\\

Allora gli autostati di $L^2$ ed $L_z$ sono anche autostati dell'operatore di parità e si deve avere
	\begin{equation}
		\tcboxmath[sharp corners=downhill, colback=white, colframe=black]{
			P|l,m\rangle=\lambda_{l,m}|l,m\rangle.
			}
	\end{equation}\\

Inoltre, in virtù della commutatività tra l'operatore di parità e gli operatori a scala $L_{\pm}$, gli stati con stesso valore di $l$ e diverso valore di $m$ devono avere stessa parità. Infatti
	\begin{align}
		 PL_-|l,m\rangle &=   cP|l,m-1\rangle=c\lambda_{l,m-1}|l,m-1\rangle=  \nonumber \\
		&=   L_-P|l,m\rangle=L_- \lambda_{l,m}|l,m\rangle=c\lambda_{l,m}|l,m-1\rangle ,
	\end{align}
ossia
	\begin{equation} 
		\tcboxmath[sharp corners=downhill, colback=white, colframe=black]{
			\lambda_{l,m}=\lambda_{l,m-1},
			}
	\end{equation}
e possiamo scrivere allora:
	\begin{equation} 
		\tcboxmath[sharp corners=downhill, colback=white, colframe=black]{
			P|l,m\rangle=\lambda_l|l,m \rangle.
			}
	\end{equation}\\
	
Per determinare la parità $\lambda_l$ osserviamo che una trasformazione di parità in coordinate polari è realizzata dalla trasformazione 

\begin{center}
\begin{minipage}[c]{0.35\textwidth}
\centering
\begin{equation}
\begin{cases} \nonumber
r \to r \\
\theta \to \pi-\theta \\
\varphi \to \varphi-\pi
\end{cases}
\end{equation}
\end{minipage}
\begin{minipage}{0.50\textwidth}
\centering
\tdplotsetmaincoords{60}{110}
%
\pgfmathsetmacro{\rvec}{.8}
\pgfmathsetmacro{\thetavec}{30}
\pgfmathsetmacro{\phivec}{60}
%
\begin{tikzpicture}[scale=5,tdplot_main_coords]
    \coordinate (O) at (0,0,0);
    \draw[thick,->] (0,0,0) -- (1,0,0) node[anchor=north east]{$x$};
    \draw[thick,->] (0,0,0) -- (0,1,0) node[anchor=north west]{$y$};
    \draw[thick,->] (0,0,0) -- (0,0,1) node[anchor=south]{$z$};
    \tdplotsetcoord{P}{\rvec}{\thetavec}{\phivec}
    \draw[-stealth,color=red] (O) -- (P) node[above right] {$\vec{r}$};
    \draw[dashed, color=red] (O) -- (Pxy);
    \draw[dashed, color=red] (P) -- (Pxy);
    \tdplotdrawarc{(O)}{0.2}{0}{\phivec}{anchor=north}{$\varphi$}
    \tdplotsetthetaplanecoords{\phivec}
    \tdplotdrawarc[tdplot_rotated_coords]{(0,0,0)}{0.5}{0}%
        {\thetavec}{anchor=south west}{$\theta$}
\end{tikzpicture}
\end{minipage}
\end{center}

L'effetto di una trasformazione di parità è facilmente calcolabile sugli stati con $m=l$, giacché in questo caso le corrispondenti autofunzioni hanno una forma particolarmente semplice:
	\begin{equation} 
		Y_{l,l}(\theta, \varphi)=A_l \left( \sin \theta \right)^l e^{il\varphi}.
	\end{equation}
Si ha allora:
	\begin{align} 
		PY_{l,l}(\theta, \varphi) & =   A_l \left( \sin (\pi -\theta) \right)^l e^{il(\varphi+\pi)}= \nonumber \\
		 & =   A_l \left( \sin \theta \right)^l e^{il\varphi} e^{i \pi l}= (-1)^l Y_{l,l}(\theta, \varphi),
	\end{align}
da cui, in definitiva:
	\begin{equation} 
		\tcboxmath[enhanced, sharp corners=downhill, colframe=black, colback=white, borderline={2pt}{-3pt}{black}]{
			P|l,m\rangle= (-1)^l |l,m \rangle.
			}
	\end{equation}\\
	
Ricordando che la relazione ($\ref{eq:cap23_2}$), siamo portati a concludere che \textbf{gli elementi di matrice della perturbazione $\boldsymbol{V}$ tra due stati per i quali $\boldsymbol{l}$ è sempre pari o sempre dispari sono nulli}.\\

Siamo dunque giunti, per la matrice $V$, alla seguente espressione:
	\begin{equation}  
		V=
			\bordermatrix{~&200&210&211&21\textrm{-}1\cr
			& 0 & \cdot & 0 & 0 \cr
			& \cdot & 0 & 0 & 0 \cr
			& 0 & 0 & 0 & 0 \cr
			& 0 & 0 & 0 & 0 \cr}
	\end{equation}\\

Poiché la matrice è hermitiana, ci resta solo da calcolare l'elemento 
	\begin{equation}
	\label{eq:cap23_3}
		\tcboxmath[sharp corners=downhill, colback=white, colframe=black]{
			\langle 210| V |200 \rangle =  \langle 200| V |210 \rangle^* = \int d\Omega \ r^2 dr \ \phi_{210}^*(\vec{r}) V \phi_{200}(\vec{r}),
			}
	\end{equation}
dove le funzioni d'onda rilevanti sono date dalle espressioni
	\begin{align} 
		\phi_{210}& = R_{21}(r)Y_{10}(\theta, \varphi)=  \frac{1}{\sqrt{3}}\left( \frac{1}{2a_0} \right)^{3/2} \left( \frac{r}{a_0} \right) e^{-\frac{r}{2a_0}} \sqrt{\frac{3}{4 \pi}} \cos \theta , \\
		\phi_{200}& =  R_{20}(r)Y_{00}(\theta, \varphi)= 2\left( \frac{1}{2a_0} \right)^{3/2} \left( 1-\frac{r}{2a_0} \right) e^{-\frac{r}{2a_0}} \sqrt{\frac{1}{4 \pi}} ,
	\end{align}
(per gli atomi idrogenoidi si deve sostituire $a_0 \to \frac{a_0}{Z}$).\\

Esprimiamo il potenziale $V$ in coordinate sferiche 
\begin{equation}
V=e \mathcal{E}z=e \mathcal{E} r \cos\theta
\end{equation}
e calcoliamo l'integrale ($\ref{eq:cap23_3}$) 
	\begin{align}
		 &  \int r^2 dr\, d\Omega \   e \mathcal{E} r \cos\theta \cdot    \frac{2}{\sqrt{3}} \left( \frac{1}{2a_0} \right)^{3}   \left( \frac{r}{a_0} \right)     \left( 1-\frac{r}{2a_0} \right)  e^{-\frac{r}{a_0}}\sqrt{\frac{3}{4 \pi}} \cos \theta \sqrt{\frac{1}{4 \pi}}=  \nonumber \\ 
		 & =  e \mathcal{E}  \frac{2}{3}  \left( \frac{1}{2a_0} \right)^{3} \int_0^{\infty} dr \, r^3 \left( \frac{r}{a_0} \right)     \left( 1-\frac{r}{2a_0} \right)  e^{-\frac{r}{a_0}}  \int d\Omega \left| Y_{10}(\theta, \varphi) \right|^2=  \nonumber \\
		 & =  e \mathcal{E} \, \frac{a_0}{12} \int_0^{\infty} ds \, s^4 \left(1-\frac{1}{2}s \right) e^{-s} = \frac{1}{12} e \mathcal{E} \, a_0 \left( 4!-\frac{1}{2}5! \right)= \frac{1}{12}  e \mathcal{E} \, a_0 (24-60),
	\end{align}
ossia 
	\begin{equation} 
		\tcboxmath[sharp corners=downhill, colback=white, colframe=black]{
			\langle 210|V|200\rangle= \langle 200 |V |210 \rangle^*=-3e\mathcal{E}a_0. 
			}
	\end{equation}\\
	
\textbf{La matrice della perturbazione $\boldsymbol{V}$ nel sottospazio degli autostati degeneri corrispondenti agli stati con $\boldsymbol{n=2}$ ha dunque la forma}:
	\begin{equation}  
		\tcboxmath[enhanced, sharp corners=downhill, colframe=black, colback=white, borderline={2pt}{-3pt}{black}]{
			V=
			\bordermatrix{~&200&210&211&21\textrm{-}1\cr
			& 0 & -3e\mathcal{E}a_0 & 0 & 0 \cr
			& -3e\mathcal{E}a_0 & 0 & 0 & 0 \cr
			& 0 & 0 & 0 & 0 \cr
			& 0 & 0 & 0 & 0 \cr}
			}
	\end{equation}\\
	
Gli autovalori di questa matrice rappresentano le correzioni, al primo ordine nella perturbazione, ai livelli energetici imperturbati corrispondenti agli stati con $n=2$. Questi \textbf{autovalori} sono:
	\begin{equation} 
		\tcboxmath[enhanced, sharp corners=downhill, colframe=black, colback=white, borderline={2pt}{-3pt}{black}]{
			E^{(1)}=0, \ \pm 3e\mathcal{E}a_0,
			}
	\end{equation}
con l'autovalore nullo avente molteplicità di 2. (Si osservi come la sottomatrice $2\times2$ da diagonalizzare è proporzionale alla matrice di Pauli $\sigma_1$).\\

Gli \textbf{autostati} corrispondenti agli autovalori $E^{(1)}=\pm 3e\mathcal{E}a_0$, nel sottospazio di dimensione  di interesse, sono dati da
	\begin{equation} 
		\frac{1}{\sqrt{2}} 
		\begin{pmatrix}
		1 \\
		-1 \\
		\end{pmatrix}
		\qquad \textrm{e} \qquad 
		\frac{1}{\sqrt{2}}
		\begin{pmatrix}
		1 \\
		1 \\
		\end{pmatrix}
	\end{equation}
e corrispondono dunque alle combinazioni lineari 
\begin{align}
& \tcboxmath[enhanced, sharp corners=downhill, colframe=black, colback=white, borderline={2pt}{-3pt}{black}]{
\frac{1}{\sqrt{2}} \left( \phi_{200}-\phi_{210} \right) 
} \qquad 
\left( E^{(1)}=3e\mathcal{E}a_0 \right) ;\\[0.3cm]
&\tcboxmath[enhanced, sharp corners=downhill, colframe=black, colback=white, borderline={2pt}{-3pt}{black}]{
\frac{1}{\sqrt{2}} \left( \phi_{200}+\phi_{210} \right)
}\qquad
\left( E^{(1)}=-3e\mathcal{E}a_0 \right) .
\end{align}\\

In conclusione, i livelli corrispondenti ad $n=2$ si separano, per effetto del campo elettrico, come indicato nello schema sottostante
\begin{center}
\includegraphics[width=10cm]{immagini/cap_23/fig23_1.png}
\end{center}
Poiché lo spostamento dei livelli, in prima approssimazione, è proporzionale al campo elettrico esterno $\mathcal{E}$, si parla in questo caso di $\textbf{effetto Stark lineare}$.\\

Osserviamo che, \textbf{in presenza del campo elettrico, gli autostati dell'Hamiltoniano non sono più autostati di $\boldsymbol{L^2}$}. Ad esempio, nel sottospazio degli stati con $n=2$, abbiamo ottenuto combinazioni lineari di stati corrispondenti ad $l=0$ ed $l=1$. La ragione è che, in presenza del campo elettrico esterno, \textbf{il sistema non è più invariante per rotazioni arbitrarie, e l'Hamiltoniano non commuta più con l'operatore momento angolare $\boldsymbol{L^2}$}.\\
Tuttavia, \textbf{il sistema è ancora invariante per rotazione attorno all'asse $\boldsymbol{z}$, che definisce la direzione del campo esterno. L'Hamiltoniano perturbato commuta con la proiezione $\boldsymbol{L_z}$ del momento angolare orbitale e gli autostati di $\boldsymbol{H}$ sono simultaneamente autostati di $\boldsymbol{L_z}$}.
