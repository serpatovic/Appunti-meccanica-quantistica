\pagenumbering{Alph}
\thispagestyle{empty}
\setcounter{page}{3}
 \begin{tikzpicture}[remember picture,overlay]
   \node at (+6.75cm,-9.5cm) {\includegraphics[width=\pdfpagewidth,height=\pdfpageheight]{immagini/copertina/Sfondo.png}};
 \end{tikzpicture}
\clearpage
\setcounter{page}{3}
\facciatabianca
\begin{titlepage}
\setcounter{page}{3}
\begin{center}
% Upper part of the page. The '~' is needed because \\
% only works if a paragraph has started.
\includegraphics[width=0.40\textwidth]{immagini/copertina/logo.pdf}~\\[1cm]

\textsc{\LARGE \bfseries Università degli studi  ``Roma Tre''}\\[0.4cm]

\textsc{\large Dipartimento di Matematica e Fisica}\\[1.5cm]


% Title
\hrule 
\vspace{0.1cm}
\hrule 
\vspace{0.4cm}
{ \Huge \bfseries \textsc{\myfont{Appunti di Meccanica Quantistica}} \\[0.4cm] }
\hrule 
\vspace{0.1cm}
\hrule 
\vspace{0.5cm}
\hfill
\begin{minipage}[t]{0.47\textwidth}\raggedleft
{\Large{\myfont{\textit{Prof. ~Vittorio Lubicz}}}}\\

\vspace{55mm}
\par
\noindent





\end{minipage}

\vspace{1mm}

\begin{center}
\hrule height 0.6mm
{\large{\textbf \quad}\\
\vspace{3mm}}
\includegraphics[width=0.20\textwidth]{immagini/copertina/cc.png}\\
\small{Quest'opera è stata rilasciata sotto la licenza Creative Commons Attribuzione-Non Commerciale-Non opere derivate 2.5 Italia, vedi http://creativecommons.org//licenses/by-nc-nd/2.5/it/}
\end{center}


\end{center}
\end{titlepage}
\setcounter{page}{3}
\facciatabianca
\thispagestyle{empty}
\setcounter{page}{3}
\begin{center}
\myfont{\Large{\textbf{\textit{Premessa}}}}
\end{center}
Gli \textit{Appunti di Meccanica Quantistica} sono da anni il testo di riferimento per il corso di Istituzioni di Fisica Teorica del CdL in Fisica a Roma Tre e rappresentano un ottimo punto di partenza per chi si affaccia, per la prima volta, al vasto e complesso mondo della Meccanica Quantistica. Tutti noi ne abbiamo apprezzato la semplicità, la chiarezza e l'esaustività con cui gli argomenti del corso venivano affrontati, eppure sapevamo che la chiarezza e la fruibilità degli \textit{Appunti} potevano  essere, in qualche modo, affinate. L'idea di base era quella di realizzare una sorta di ``nuova edizione'',  in cui venissero curati maggiormente gli aspetti grafici (coinvolgendo l'indice, le figure e l'impaginazione), il tutto senza perdere il minimo dettaglio per ciò che riguarda le dimostrazioni e, in generale, gli argomenti trattati, che sono rimasti immutati.\\
Questa idea, dato il vasto numero di argomenti trattati negli \textit{Appunti}, ha richiesto parecchio tempo nella realizzazione ed ha coinvolto un team molto ampio di persone. Speriamo che il risultato vi sia utile e vi possa piacere: è il nostro piccolo regalo. Da studenti, per gli studenti.\\
\begin{table}[!htbp]
\begin{center}
\begin{tabular}{cc}
Matteo Altorio & Ilaria Carlomagno \\
Paolo Costantini & Marco De Cicco \\
Eleonora Diociaiuti & Emilia Margoni \\
Carmen Monaco & Emanuele Navisse \\
Irene Schiesaro & Valerio Serpente \\
Giulio Settanta & Carlotta Trigila \\
Valentina Vecchio & Antonio Vigilante
\end{tabular}
\end{center}
\end{table}
\begin{table}[!htbp]
\begin{tabular}{c}
Un ringraziamento particolare a Marco Petrucci per la copertina.
\end{tabular}
\end{table}
\newpage
\setcounter{page}{3}
\facciatabianca
\thispagestyle{empty}
\setcounter{page}{3}
\begin{center}
\myfont{\Large{\textbf{\textit{Nota per il lettore}}}}
\end{center}
All'interno del testo viene citata la seguente bibliografia:
\begin{itemize}
\item J.J. Sakurai, Jim Napolitano, \textit{Meccanica Quantistica Moderna}, Seconda edizione, Zanichelli.
\item R.P. Feynman et al., \textit{La Fisica di Feynman}, Volume III, Masson;
\item L. Landau e E. Lifschitz, \textit{Meccanica Quantistica}, Editori Riuniti;
\item S. Gasiorowicz, \textit{Quantum Physics}, J.Wiley \& Sons.
\end{itemize}
Nel proseguo dei capitoli si fa esplicito riferimento ai testi indicando, ad esempio, con S1.6 il paragrafo
1.6 del libro di Sakurai, con F1.1 il paragrafo 1.1 del libro di Feynman, eccetera.
\newpage
\setcounter{page}{3}
\facciatabianca
